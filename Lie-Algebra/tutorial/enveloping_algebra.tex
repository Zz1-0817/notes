\section{包络代数}

\begin{theorem}
  \label{tutorial-theorem-universal-enveloping-algebra}
  对于任意李代数 \( \mathfrak{g} \), 存在一个带单位的结合代数, 记作 \(
  U(\mathfrak{g}) \), 以及线性映射 \( i: \mathfrak{g} \to U(\mathfrak{g}) \)
  具有下面性质.
  \begin{enumerate}
    \item 对任意 \( X, Y \in \mathfrak{g} \), 我们有
      \[
        i([X, Y]) = i(X) i(Y) - i(Y) i(X).
      \]
    \item 代数 \( U(\mathfrak{g}) \) 有形如 \( i(X) \) 的元素生成, 其中 \( X \in
      \mathfrak{g} \). 换句话说, 包含所有形如 \( i(X) \)
      元素带单位的最小结合代数就是 \( U(\mathfrak{g}) \).
    \item 假设 \( \mathcal{A} \) 是一个带单位的集合代数, \( j: \mathfrak{g} \to
      \mathcal{A}\) 为线性映射, 使得 \( j([X, Y]) = j(X) j(Y) - j(Y) j(X) \)
      对所有 \( X, Y \in \mathfrak{g} \) 成立.
      那么存在一个唯一的代数同态 \( \phi: U(\mathfrak{g}) \to \mathcal{A} \)
      使得 \( \phi(1) = 1 \) 且使得 \( \phi(i(X)) = X \) 对所有 \( X \in
      \mathfrak{g} \) 成立.
      % https://q.uiver.app/#q=WzAsMyxbMSwxLCJcXG1hdGhjYWx7QX0iXSxbMCwxLCJcXG1hdGhmcmFre2d9Il0sWzEsMCwiVShcXG1hdGhmcmFre2d9KSJdLFsxLDAsImoiLDJdLFsxLDIsImkiXSxbMiwwLCJcXHBoaSIsMCx7InN0eWxlIjp7ImJvZHkiOnsibmFtZSI6ImRhc2hlZCJ9fX1dXQ==
      \[\begin{tikzcd}
        & {U(\mathfrak{g})} \\
        {\mathfrak{g}} & {\mathcal{A}}
        \arrow["\phi", dashed, from=1-2, to=2-2]
        \arrow["i", from=2-1, to=1-2]
        \arrow["j"', from=2-1, to=2-2]
      \end{tikzcd}\]
  \end{enumerate}
  我们称满足上述性质的对 \( (U(\mathfrak{g}), i) \) 称为 \( \mathfrak{g} \)
  的 \emph{泛包络代数}.
\end{theorem}
\begin{proof}
  考虑张量代数 \( T(\mathfrak{g}) = \bigoplus_{k =
  0}^{\infty}\mathfrak{g}^{\otimes k} \). 我们定义下面运算, 并直接验证知道此时
  \( T(\mathfrak{g}) \) 为结合代数
  \[
    (u_1 \otimes u_2 \otimes \cdots \otimes u_k) \cdot (v_1 \otimes v_2 \otimes
    \cdots \otimes v_l) = u_1 \otimes u_2 \otimes \cdots \otimes u_k \otimes v_1 \otimes v_2 \otimes
    \cdots \otimes v_l
  \]
  我们断言 \( T(\mathfrak{g}) \) 具有如下性质: 如果 \( \mathcal{A} \)
  任意带单位的结合代数, \( j: \mathfrak{g} \to \mathcal{A} \) 是任意线性映射,
  那么存在一个代数同态 \( \psi: T(\mathfrak{g}) \to \mathcal{A} \) 使得 \(
  \psi(1) = 1 \) 且 \( \psi(X) = j(X) \). 验证只需要构造如下 \( \psi \)
  并直接验证即可:
  \[
    \psi(X_1 \otimes \cdots \otimes X_k) = j(X_1) \cdots j(X_k), \text{其中}
    X_1, \cdots, X_k \in \mathfrak{g}.
  \]
  我们定义 \( T(\mathfrak{g}) \) 的一个双边理想 \( J \): 我们考虑所有包含元素
  \[
    X \otimes Y - Y \otimes X - [X, Y],\quad X, Y \in \mathfrak{g}
  \]
  的最小双边理想 \( J \). 更具体地, \( J \) 由形如
  \[
    \sum_{j = 1}^N \alpha_j\left(X_j \otimes Y_j - Y_j \otimes X_j - [X_j, Y_j]\right)
    \beta_j
  \]
  的元素构成, 其中 \( X_j, Y_j \in \mathfrak{g} \) 且 \( \alpha_j, \beta_j \in
  T_{\mathfrak{g}} \). \( T(\mathfrak{g}) / J \) 即为所求.
\end{proof}

\begin{corollary}
  如果 \( \rho: \mathfrak{g} \to \operatorname{End}(V) \) 是李代数 \(
  \mathfrak{g} \) 的表示, 那么存在唯一的代数同态 \( \widetilde{\pi}:
  U(\mathfrak{g}) \to \operatorname{End}(V) \) 使得 \( \widetilde{\pi}(1) =
  \operatorname{Id} \) 且 \( \widetilde{\pi}(X) = \pi(X) \) 对所有 \( X \in
  \mathfrak{g} \subseteq U(\mathfrak{g}) \).
\end{corollary}

\begin{theorem}[Poincare-Birkhoff-Witt]
  \label{tutorial-theorem-PBW}
  如果 \( \mathfrak{g} \) 是基为 \( X_1, \ldots, X_k \) 的有限维李代数,
  那么形如
  \[
    i(X_1)^{n_1}i(X_2)^{n_2}\cdots i(X_k)^{n_k}, \text{其中} n_k
    \text{是非负整数},
  \]
  张成了 \( U(\mathfrak{g}) \). 特别地, \( i(X_1), \ldots, i(X_k) \)
  是线性独立的, 换句话说, \( i: \mathfrak{g} \to U(\mathfrak{g}) \) 是单射.
\end{theorem}
\begin{proof}
  \ref{theorem-Poincare-Birkhoff-Witt}.
\end{proof}
\begin{corollary}
  如果 \( \mathfrak{g} \) 是有限维李代数且 \( \mathfrak{h} \) 是 \( \mathfrak{g}
  \) 的子代数, 那么将 \( U(\mathfrak{h}) \) 中乘积 \( X_1X_2 \cdots X_N \)
  映射到 \( U(\mathfrak{g}) \) 中同样的乘积是一个自然的单态射.
\end{corollary}

\section{Verma模}

\subsection{Verma 模的构造}

对于 \( U(\mathfrak{g}) \) 的向量集 \( \left\lbrace \alpha_j \right\rbrace \),
我们先来回顾一下由这些向量生成的左理想 \( I \): \( I \) 由所有形如
\[
  \sum_j \beta_j \alpha_j
\]
的元素组成, 其中 \( \beta_j \in U(\mathfrak{g}) \) 任意. 取定 \( \mu \in
\mathfrak{h} \), 令 \( I_{\mu} \) 为由
\[
  H - \left\langle \mu, H \right\rangle 1,\quad H \in \mathfrak{h}
\]
以及
\[
  X \in \mathfrak{g}_\alpha,\quad \alpha \in R^+.
\]
下面元素生成的左理想. 我们令 \( W_{\mu} \) 为\textbf{商向量空间}
\[
  W_{\mu} = U(\mathfrak{g}) / I_{\mu},
\]
并且用 \( [\alpha] \) 记 \( \alpha \in U(\mathfrak{g}) \) 在此商中的像.
我们定义一个 \( U(\mathfrak{g}) \) 在 \( W_{\mu} \) 上的表示
\[
  \alpha.([\beta]) = [\alpha \beta],\quad \alpha, \beta \in U(\mathfrak{g}).
\]
这时 \( W_{\mu} \) 称为最高权为 \( \mu \) 的\emph{Verma 模}.


\subsection{Verma 模的性质}

令 \( \mathfrak{n}^+ \) 为根向量 \( X_\alpha \in \mathfrak{g}_\alpha \)
长成的子空间, 其中 \( \alpha \in R^+ \); 令 \( \mathfrak{n}^- \) 为根向量 \(
X_\alpha \in \mathfrak{g}_{-\alpha} \) 长成的子空间, 其中 \( \alpha \in R^- \);
令 \( \mathfrak{b} = \mathfrak{h} \oplus \mathfrak{n}^+ \) 为 Borel 子代数.

\begin{lemma}
  \label{tutorial-lemma-Verma-properties}
  如果 \( J_{\mu} \) 为在 \( U(\mathfrak{b}) \) 中由元素
  \[
    H - \left\langle \mu, H \right\rangle 1, H \in \mathfrak{h} \text{ 以及 } X
    \in \mathfrak{g}_\alpha, \alpha \in R^+
  \]
  生成的左理想, 那么 \( 1 \notin J_{\mu} \).
\end{lemma}
\begin{proof}
  考虑 \( \mathfrak{b} \) 的一维表示 \( (\mathbb{C}, \rho_{\mu}) \): \( \rho(X + H) =
  \left\langle \mu, H \right\rangle\), 其中 \( X \in \mathfrak{n}^+, H \in
  \mathfrak{h} \). 注意到 \( \mathfrak{h} \) 交换, 容易证明这确实是一个表示.

  由定理 \ref{tutorial-theorem-universal-enveloping-algebra}, \( \mathfrak{b} \)
  的表示 \( \rho_{\mu} \) 可以扩张为 \( U(\mathfrak{b}) \) 的表示 \(
  \widetilde{\rho}_{\mu} \) 满足 \( \widetilde{\rho}_{\mu}(1) = 1 \).
  因此 \( \operatorname{ker} \widetilde{\rho}_{\mu} \) 可以视为 \(
  U(\mathfrak{b}) \) 的左理想, 且 \( 1 \notin \operatorname{ker}
  \widetilde{\rho}_{\mu} \), 但 \( \operatorname{ker} \widetilde{\rho}_{\mu} \)
  包含所有形如
  \[
    H - \left\langle \mu, H \right\rangle 1, H \in \mathfrak{h} \text{ 以及 } X
    \in \mathfrak{g}_\alpha, \alpha \in R^+
  \]
  的元素, 也就是 \( \operatorname{ker} \widetilde{\rho}_{\mu} \supseteq J_{\mu}
  \).
\end{proof}

\begin{theorem}
  \label{tutorial-theorem-Verma-properties}
  \begin{enumerate}
    \item 向量 \( v_0 := [1] \) 是 \( (W_{\mu}, \rho_{\mu}) \) 的一个非零向量.
    \item \( (W_{\mu}, \rho_{\mu}) \) 是一个最高权为 \( \mu \) 的最高权循环表示, 并且 \( v_0 \)
      为其最高权向量.
    \item 如果 \( Y_1,\ldots,Y_k \) 形成了 \( \mathfrak{n}^- \) 的一组基, 那么元素
      \[
        Y_1^{n_1}.Y_2^{n_2}\cdots Y_k^{n_k}.v_0,\quad \text{其中每个} n_j
        \text{是非负整数},
      \]
      形成了 \( W_{\mu} \) 的一组基.
  \end{enumerate}
\end{theorem}
\begin{proof}
  %TODO: 重写这段, 利用重排引理之类的?
  对任意 \( H \in \mathfrak{h} \), 我们有
  \[
    (H - \left\langle \mu, H \right\rangle 1).v_0 = [H - \left\langle
    \mu, H\right\rangle 1] = 0,
  \]
  因此 \( H.v_0 = \left\langle \mu, H \right\rangle v_0 \). 类似地, \(
  X_\alpha.v_0 = [X_\alpha] = 0 \). 假设 \( U \) 是 \( W_{\mu} \) 包含 \( v_0 \)
  的任意 \( \mathfrak{g} \)-不变子空间, 那么对任意 \( [\alpha] \in W_{\mu} \),
  \( \alpha v_0 = [\alpha] \), 也就是 \( U = W_{\mu} \). 对于前两条, 只需要知道
  \( v_0 \neq 0 \) 即可.

  由 PBW 定理 \ref{tutorial-theorem-PBW}, 每个 \( \alpha \in U(\mathfrak{g}) \)
  可以唯一写成
  \begin{equation}
    \alpha = \sum_{n_1,\ldots,n_k = 0}^{\infty} Y^{n_1}_1 Y^{n_2}_2 \cdots
    Y^{n_k}_k a_{n_1,\ldots, n_k}, \label{tutorial-equation-Verma-properties}
  \end{equation}
  其中 \( a_{n_1,\ldots,n_k} \in U(\mathfrak{b}) \subseteq U(\mathfrak{g}) \)
  只有有限个不为零.

  假设 \( \alpha \in I_{\mu} \), 那么 \( \alpha \) 为元素形如 \( \beta(H -
  \left\langle \mu, H \right\rangle 1) \) 以及 \( \beta X_\alpha \) 的线性组合,
  其中 \( \beta \in U(\mathfrak{g}) \), \( H \in \mathfrak{h} \), \( X_\alpha
  \in \mathfrak{g} \) 以及 \( \alpha \in R^+ \). 由
  \eqref{tutorial-equation-Verma-properties}, 其形如
  \[
    Y^{n_1}_1 Y^{n_2}_2 \cdots Y^{n_k}_k b_{n_1,\ldots, b_k}(H - \left\langle
    \mu, H \right\rangle 1) \text{ 以及 } Y^{n_1}_1 Y^{n_2}_2 \cdots Y^{n_k}_k
    b_{n_1, \ldots, n_k} X_\alpha
  \]
  的线性组合. 而 \( b_{n_1,\ldots, n_k}(H - \left\langle \mu, H \right\rangle 1)
  \) 以及 \( b_{n_1,\ldots, n_k}X_\alpha \) 属于引理
  \ref{tutorial-lemma-Verma-properties} 中的左理想 \( J_{\mu} \subseteq
  U(\mathfrak{b}) \), 因此, 由唯一性, 如果 \( \alpha \in I_{\mu} \), \( \alpha
  \) 在 \eqref{tutorial-equation-Verma-properties} 中的每个 \(
  a_{n_1,\ldots,n_k} \) 都属于这样的 \( J_{\mu} \). 又由唯一性 \( \alpha \in
  J_{\mu} \) 要等于 \( 1 \), \( \alpha \)只能在上段表达式中 \( n_1,\ldots, n_k =
  0 \) 的项 \( a_{n_1,\ldots, n_k} = 1 \), 其它项均为 \( 0 \). 但是引理
  \ref{tutorial-lemma-Verma-properties} 告诉我们 \( 1 \notin J_{\mu} \),
  这是不可能的.

  最后, 如果对系数 \( c_{n_1,\ldots,n_k} \) 有
  \[
    \alpha:= \sum_{n_1,\ldots,n_k = 0}^{\infty} Y^{n_1}Y^{n_2}_2\cdots Y^{n_k}_k
    c_{n_1,\ldots,n_k} \in I_{\mu}
  \]
  只能 \( c_{n_1,\ldots,n_k} \in J_{\mu} \). 再次由引理
  \ref{tutorial-lemma-Verma-properties}, 它们都是零.

\end{proof}
