\chapter{预备知识}

本章涉及的环均是带单位的交换环


%TODO: 和不可约的关系?


\section{域上有限生成代数定理若干}

\subsection{Noether 正规定理}

\begin{lemma}
  假设 \( f \in k[X_1, \ldots, X_d, T] \). 对恰当的 \( m \in \mathbb{N} \),
  \[
    f\left(X_1 + T^m, X_2 + T^{m^2}, \ldots, X_d + T^{m^d}, T\right)
  \]
  形如 \( c_0 T^r + c_1 T^{r - 1} + \cdots + c_r \) 其中 \( c_0 \in k^{\times}
  \).
\end{lemma}

\begin{lemma}
  假设 \( A = k[x_1, \ldots, x_n] \) 是一个有限生成 \( k \)-代数, 令 \(
  \left\lbrace x_1, \ldots, x_d \right\rbrace \) 为 \( \left\lbrace x_1, \ldots,
  x_n \right\rbrace \) 的极大代数独立子集. 如果 \( n > d \), 那么存在一个 \( m
  \in \mathbb{N} \) 使得 \( A \) 在子代数 \( k\left[x_1 - x_n^m,\ldots, x_d -
  x_n^{m^d}, x_{d + 1}, \ldots, x_{n - 1}\right] \).
\end{lemma}

\begin{theorem}[Noether正规定理]
  每个域 \( k \) 上的有限生成代数包含一个多项式代数 \( R \), 使得 \( A \) 是一个
  \( R \)-代数. 换句话说, 存在 \( A \) 的元素 \( y_1, \ldots, y_r \) 在 \( k \)
  中代数独立, 使得 \( A \) 在 \( k[y_1, \ldots, y_r] \) 上有限.
\end{theorem}

\subsection{Nakayama 引理}

\begin{theorem}[Zariski 引理]
  \label{theorem-Zariski-lemma}
  假设 \( K/k \) 为域扩张.
  如果 \( K \) 是有限生成 \( k \)-代数, 那么 \( K \) 在 \( k \) 上代数.
  特别地, \( K \) 在 \( k \) 上有限.
\end{theorem}
\begin{proof}
  设 \( K = k[x_1, \ldots, x_r] \), 其中 \( x_1, \ldots, x_r \) 为 \( K \)
  的一组 \( k \)-最小生成元, 下面对 \( r \) 作归纳.
  \( r = 0 \) 的情况平凡.

  考虑 \( r \geq 1 \).
  如果 \( K \) 不在 \( k \) 上代数, 那么存在至少一个 \( x_i \) 不在 \( k \)
  上代数, 通过一个置换不妨设其为 \( x_1 \).
  于是 \( k[x_1] \) 是带一个未定元的 \( k \) 多项式环, 且 \( K \) 是一个有限生成
  \( k(x_1) \)-代数, 其一组生成元为 \( x_2, \ldots, x_r \).
  于是由归纳假设 \( x_2, \ldots, x_r \) 在 \( k(x_1) \) 上代数.
  因此存在 \( c \in k[x_1] \) 使得 \( cx_2, \ldots, cx_r \) 在 \( k[x_1] \)
  上整.

  对任意 \( f \in k(x_1) \subseteq K = k[x_1, \ldots, x_r] \), 存在足够大的 \( N
  \) 使得 \( c^N f \in k[x_1, cx_2, \ldots, cx_r] \).
  于是 \( c^N f \) 在 \( k[x_1] \) 上整.
  而 \( k[x_1] \) 作为域上多项式环是一个唯一因子分解整环, 是整闭的, 于是 \( c^N
  f \in k[x_1] \) 这与 \( k[x_1] \) 有无穷多个不可约元矛盾.
\end{proof}
%TODO: 分别考虑是不是代数闭域

\begin{corollary}
  假设 \( A \) 是一个有限生成 \( k \)-代数.
  \( A \) 的每个极大理想 \( \mathfrak{m} \) 都是 \( A \) 到某个 \( k
  \)-有限生成扩域的同态核.
\end{corollary}

\begin{corollary}
  假设 \( k \subseteq K \subseteq A \) 为 \( k \)-代数, \( K \) 是一个域, \( A
  \) 为一个 \( k \)-上有限生成代数.
  那么 \( K \) 在 \( k \) 上代数.
\end{corollary}
\begin{proof}
  假设 \( \mathfrak{m} \) 为 \( A \) 的一个极大理想, 那么 \( K \cap \mathfrak{m}
  = (0) \) 于是 \( k \subseteq K \subseteq A/\mathfrak{m} \).
  由\cref{theorem-Zariski-lemma} \( A/\mathfrak{m} \) 在 \( k \) 上代数, 所以 \(
  K \) 在 \( k \) 上代数.
\end{proof}

\subsection{零点定理}

\begin{theorem}[Nullstellensatz]
  \label{theorem-Nullstellensatz}
  假设 \( k \) 为一个域, \( \overline{k} \) 为其代数闭包.
  \( k[X_1, \ldots, X_n] \) 中的每个真理想都在 \( \overline{k}^n \)
  中有公共零点.
  换句话说, 存在 \( (a_1, \ldots, a_n) \in \overline{k}^n \) 使得 \( f(a_1,
  \ldots, a_n) = 0 \) 对所有 \( f \in \mathfrak{a} \) 成立.
\end{theorem}
\begin{proof}
  考虑找到一个 \( k \)-代数同态 \( \varphi: k[X_1, \ldots, X_n] \to \overline{k}
  \), 其核包含 \( \mathfrak{a} \).
  如果能找到这样的 \( \varphi \), 设 \( \alpha_i = \varphi(X_i) \).
  那么对任意 \( f \in \mathfrak{a} \), 则 \( 0 = \varphi f(X_1, \ldots, X_n) =
  f(\alpha_1, \ldots, \alpha_n) \).

  假设 \( \mathfrak{m} \) 是 \( k[X_1, \ldots, X_n] \) 的一个极大理想, 那么域 \( K
  = k[X_1, \ldots, X_n]/ \mathfrak{m} \) 是一个有限生成 \( k \)-代数.
  由\cref{theorem-Zariski-lemma}, \( K \) 在 \( k \) 上代数, 于是有嵌入 \( K
  \hookrightarrow \overline{k} \).
  考虑复合 \( k \to K \hookrightarrow \overline{k} \) 即可.
\end{proof}

\begin{corollary}
  如果 \( k \) 代数闭, 那么 \( k[X_1, \ldots, X_n] \) 的所有极大理想即
  \[
    \left\lbrace (X_1 - a_1,\ldots, X_n - a_n): (a_1, \ldots, a_n) \in k^n
    \right\rbrace.
  \]
\end{corollary}
\begin{proof}
  \( (X_1 - a_1,\ldots, X_n - a_n) \) 为 \( k[X_1, \ldots, X_n] \)
  极大理想是因为 \( k[X_1, \ldots, X_n]/(X_1 - a_1,\ldots, X_n - a_n) \simeq k
  \).

  假设 \( \mathfrak{m} \) 是 \( K[X_1, \ldots, X_n] \) 的一个极大理想.
  由\cref{theorem-Nullstellensatz}, \( \mathfrak{m} \) 有公共零点 \( (a_1,
  \ldots, a_n) \in k^n \).
  将 \( f \in K[X_1, \ldots, X_n] \) 写成 \( X_i - a_i \) 的多项式, 其常数项即
  \( f(a_1, \ldots, a_n) \).
  于是对 \( f \in \mathfrak{m} \) 有 \( f(a_1, \ldots, a_n) = 0 \), 换句话说 \(
  \mathfrak{m} \subseteq (X_1 - a_1, \ldots, X_n - a_n) \), 结合极大性,
  只能相等.
\end{proof}

\begin{theorem}[强 Nullstellensatz]
  \label{theorem-strong-Nullstellensatz}
  假设 \( \mathfrak{a} \) 为 \( k[X_1, \ldots, X_n] \) 的任意理想.
  假设 \( Z(\mathfrak{a}) \) 为 \( \mathfrak{a} \) 在 \( \overline{k}^n \)
  的所有公共零点集.
  如果一个多项式 \( h \in k[X_1, \ldots, X_n] \) 在 \( Z(\mathfrak{a}) \)
  中取值均为零, 那么 \( h \in \operatorname{rad}(\mathfrak{a}) \).
\end{theorem}
\begin{proof}
  只需证明 \( h \neq 0 \) 的情况.
  由\cref{theorem-Hilbert-basis}, \( k[X_1, \ldots, X_n] \) noetherian, 于是 \(
  \mathfrak{a} \) 有限生成, 设其由 \( g_1, \ldots, g_m \) 生成.
  考虑下面关于变元 \( X_1, \ldots, X_n, Y \) 的方程组
  \[
    \begin{split}
      g_i(X_1, \ldots, X_n) &= 0, i = 1, \ldots, m\\
      1 - Yh(X_1, \ldots, X_n) &= 0.
    \end{split}
  \]
  这样的方程组是没有解的: 如果 \( (a_1, \ldots, a_n, b) \) 满足前 \( m \) 个,
  那么 \( (a_1, \ldots, a_n) \in \mathfrak{a} \), 于是 \( h(a_1, \ldots, a_n) =
  0 \), 最后一个方程变成 \( 1 = 0 \), 这是不可能的.
  因此, 由\cref{theorem-Nullstellensatz}, \( (g_1, \ldots, g_m, 1 - Yh) = k[X_1,
  \ldots, X_n, Y] \).
  换句话说, 存在 \( f_i \in k[X_1, \ldots, X_n, Y] \) 使得
  \[
    1 = \sum_{i = 1}^mf_i g_i + f_{m + 1}(1 - Yh).
  \]
  考虑态射
  \[
    k[X_1, \ldots, X_n, Y] \to k(X_1, \ldots, X_n),\quad X_i \mapsto X_i, Y
    \mapsto h^{-1},
  \]
  其将 \( \sum_{i = 1}^mf_i g_i + f_{m + 1}(1 - Yh) \) 映到
  \[
    1 = \sum_i f_i(X_1, \ldots, X_n, h^{-1})g_i(X_1, \ldots, X_n).
  \]
  对足够大的 \( N \), 两边乘以 \( h^N \) 知道 \( h^N \in \mathfrak{a} \).
\end{proof}

\begin{proposition}
  \label{proposition-radical-as-maximal-intersection-in-finite-generated-algebra-over-field}
  假设 \( A \) 是一个有限生成 \( k \)-代数, \( \mathfrak{a} \) 为 \( A \)
  的理想.
  那么
  \[
    \operatorname{rad}(\mathfrak{a}) = \bigcap_{\mathfrak{m} \supseteq
    \mathfrak{a} \text{极大}} \mathfrak{m}.
  \]
  特别地, 如果 \( A \) 约化, 那么 \( \bigcap_{\mathfrak{m} \text{极大}}
  \mathfrak{m} = 0 \).
\end{proposition}
\begin{proof}
  只需对 \( A = k[X_1, \ldots, X_n] \) 的情况证明.
  由\cref{proposition-radical-as-prime-intersection}包含关系 \( \operatorname{rad}(\mathfrak{a}) \subseteq \bigcap_{\mathfrak{m}
  \supseteq \mathfrak{a}} \mathfrak{m} \) 是已知的.
  下面证明\( \operatorname{rad}(\mathfrak{a}) \subseteq \bigcap_{\mathfrak{m}
  \supseteq \mathfrak{a}} \mathfrak{m} \).

  \cref{theorem-Nullstellensatz} 保证了 \( Z(\mathfrak{a}) \neq \varnothing \),
  取任意 \( (a_1, \ldots, a_n) \in Z(\mathfrak{a}) \).
  假设 \( h \in \bigcap_{\mathfrak{m} \supseteq \mathfrak{a}} \mathfrak{m} \),
  考虑赋值映射
  \[
    \operatorname{ev}: k[X_1, \ldots, X_n] \to \overline{k},\quad f \mapsto f(a_1, \ldots, a_n),
  \]
  其中 \( \overline{k} \) 为 \( k \) 的代数闭包.
  \( \operatorname{Im} \operatorname{ev} \) 在 \( k \) 上代数, \(
  \operatorname{Im} \operatorname{ev} \) 是一个域.
  因此 \( \operatorname{ker} \operatorname{ev} \) 是一个包含 \( \mathfrak{a} \)
  的极大理想.
  由 \( h \) 定义, \( h \in \operatorname{ker} \operatorname{ev} \), 换句话说 \(
  h(a_1, \ldots, a_n) = 0 \).
  由\cref{theorem-strong-Nullstellensatz} \( h \in
  \operatorname{rad}(\mathfrak{a}) \).
\end{proof}

\section{有向极限与逆极限}

\subsection{基本定义以及构造}

\paragraph{有向极限} 给定一个预序集范畴 \( (I, \leq) \), 给定范畴 \( \mathcal{M}
\) 其对象集是模 \( (M_i)_{i \in I} \), 态射集为 \( \left\lbrace \alpha^i_j: M_i
\to M_j: i \leq j \right\rbrace \), 我们称模 \( \varinjlim M_k \) 为逗号范畴 \(
(\alpha / \Delta) := (j_\alpha / \Delta) \)
\[
  \mathbf{1} \xrightarrow{\alpha} \mathcal{M}^I = \operatorname{Fct}(I,
  \mathcal{M}) \xleftarrow{\Delta} \mathcal{M}
\]
的始对象为 \emph{有向极限}, 其中 \( \mathbf{1} \)
为恰有一个对象和一个态射的范畴, \( \alpha \) 为函子 \( 1 \mapsto \left( (i
\mapsto j) \mapsto (M_i \to M_j) \right) \) , \( \Delta \)
为\href{https://en.wikipedia.org/wiki/Diagonal_functor}{对角函子}. 换句话说, \(
\varinjlim M_k \) 满足下面的交换图.
% https://q.uiver.app/#q=WzAsNCxbMCwwLCJNX2kiXSxbMiwwLCJNX2oiXSxbMSwxLCJcXHZhcmluamxpbSBNX2siXSxbMSwyLCJOIl0sWzAsMywiXFxiZXRhX2kiLDJdLFswLDEsIlxcYWxwaGFeaV9qIl0sWzAsMiwiXFxhbHBoYV9pIl0sWzEsMiwiXFxhbHBoYV9qIiwyXSxbMSwzLCJcXGJldGFfaiJdLFsyLDMsIlxcZXhpc3RzISIsMSx7InN0eWxlIjp7ImJvZHkiOnsibmFtZSI6ImRhc2hlZCJ9fX1dXQ==
\[\begin{tikzcd}
	{M_i} && {M_j} \\
	& {\varinjlim M_k} \\
	& N
	\arrow["{\alpha^i_j}", from=1-1, to=1-3]
	\arrow["{\alpha_i}", from=1-1, to=2-2]
	\arrow["{\beta_i}"', from=1-1, to=3-2]
	\arrow["{\alpha_j}"', from=1-3, to=2-2]
	\arrow["{\beta_j}", from=1-3, to=3-2]
	\arrow["{\exists!}"{description}, dashed, from=2-2, to=3-2]
\end{tikzcd}\]

\paragraph{有向极限的构造} 考察 \( M_i \) 的直积 \( \bigoplus_{i \in I} M_i \),
那么可以将 \( M_{i_0} \) 视为 \( \bigoplus_{i \in I} M_i \) 的子模, 后者对所有
\( i \neq i_0 \) 都满足 \( m_i = 0 \). 取 \( M \) 为 \( \bigoplus_{i \in I}
M_i \) 商去元素
\[
  m_i - \alpha^i_j(m_i),\quad m_i \in M_i,\quad i < j
\]
生成的子模即为所求.

\paragraph{逆极限}
类似地, 沿用上面的 \( I \), \( \mathcal{M} \) 对象集为 \( (M_i)_{i \in I} \),
态射集 \( \left\lbrace p^i_j: M_j \to M_i: i \leq j \right\rbrace \), 我们称模
\( \varprojlim M_k \) 为逗号范畴 \( (\Delta / \beta) := (\Delta / j_\beta) \)
\[
  \mathcal{M} \xrightarrow{\Delta} \mathcal{M}^{I^{\operatorname{op}}}
  \xleftarrow{j_\beta} \mathbf{1}
\]
的终对象为 \emph{逆极限}. 换句话说, \( \varprojlim M_k \) 满足下面交换图.
% https://q.uiver.app/#q=WzAsNCxbMCwwLCJNX2kiXSxbMiwwLCJNX2oiXSxbMSwxLCJcXHZhcnByb2psaW0gTV9rIl0sWzEsMiwiTiJdLFszLDIsIlxcZXhpc3RzISIsMSx7InN0eWxlIjp7ImJvZHkiOnsibmFtZSI6ImRhc2hlZCJ9fX1dLFszLDFdLFszLDBdLFsxLDAsInBeaV9qIiwyXSxbMiwwLCJwX2kiLDJdLFsyLDEsInBfaiJdXQ==
\[\begin{tikzcd}
	{M_i} && {M_j} \\
	& {\varprojlim M_k} \\
	& N
	\arrow["{p^i_j}"', from=1-3, to=1-1]
	\arrow["{p_i}"', from=2-2, to=1-1]
	\arrow["{p_j}", from=2-2, to=1-3]
	\arrow[from=3-2, to=1-1]
	\arrow[from=3-2, to=1-3]
	\arrow["{\exists!}"{description}, dashed, from=3-2, to=2-2]
\end{tikzcd}\]
特别地, 我们常取 \( I = \mathbb{N} \).

\paragraph{逆极限的构造} 给定义 \( A \)-模逆向系统 \( (M_n, \alpha_n) \),
我们定义 \( \varprojlim M_n \) 和 \( \varprojlim^1 M_n \) 为 \( A
\)-模同态的核与余核
\[
  \prod M_n \to \prod M_n,\quad (\ldots, m_n, \ldots) \mapsto (\ldots, m_n -
  \alpha_n(m_{n + 1}), \ldots)
\]

\subsection{基本性质}

\begin{proposition}
  对环 \( A \) 的每个乘性子集 \( S \), 有 \( S^{-1} A \simeq \varinjlim A_h \),
  其中 \( h \) 跑遍 \( S \), 偏序关系为整除关系.
\end{proposition}
\begin{proof}
  如果在 \( A \) 中有 \( h' = hq \), 那么 \( h \) 在 \( A_{h'} \) 中为单位,
  由局部化的泛性质存在唯一的同态 \( A_h \to A_{h'},\quad \frac{a}{h} \mapsto
  \frac{aq}{h'} \), 由此可以构造 \( \varinjlim A_h \). 又由有自然的 \( A_h \to
  S^{-1} A \) 以及有向极限的泛性质知道存在 \( \varinjlim A_h \to S^{-1}A \).
  又由有向极限的构造知道存在 \( S^{-1} A \to \varinjlim A_h \).
  对比这两个箭头知道, 它们互逆.
\end{proof}

\begin{proposition}
  假设 \( \left(M_i, \alpha^i_j\right), \left(N_i, \beta^i_j\right) \) 和 \(
  \left(P_i, \gamma^i_j\right) \) 为相对于有序集 \( I \) 的有向系统, 令
  \[
    \left(M_i, \alpha^i_j\right) \xrightarrow{(a_i)} \left(N_i, \beta^i_j\right)
    \xrightarrow{(b_i)} \left(P_i, \gamma^i_j\right)
  \]
  为有向系统列. 如果列
  \[
    M_i \xrightarrow{a_i} N_i \xrightarrow{b_i} P_i
  \]
  对所有 \( i \) 均正合, 那么
  \[
    \varinjlim M_i \xrightarrow{\varinjlim a_i} \varinjlim N_i
    \xrightarrow{\varinjlim b_i} \varinjlim P_i
  \]
  正合.
\end{proposition}
\begin{proof}
  %TODO: 完成此证明.
\end{proof}

\begin{proposition}
  假设对每个逆向系统 \( (M_n, \alpha_n) \) 和 \( A \)-模 \( N \)
  \[
    \operatorname{Hom}(\varprojlim M_n, N) \simeq \varprojlim
    \operatorname{Hom}(M_n, N)
  \]
\end{proposition}
\begin{proof}
  函子 \( \operatorname{Hom}(\bullet, N) \) 将逆向系统 \( (M_n, \alpha_n) \)
  映到逆向系统 \( \left(\operatorname{Hom}(M_n, N), (\alpha_n)_*\right) \). 映射
  \( \varprojlim M_n \to M_i \) 诱导了 \( \operatorname{Hom}(\varprojlim M_n, N)
  \to \operatorname{Hom}(M_i, N) \), 进而由逆极限的泛性质, 有态射
  \[
    \operatorname{Hom}(\varprojlim M_n, N) \to \varprojlim
    \operatorname{Hom}(M_n, N)
  \]
  反过来, 取 \( (f_i) \in \varprojlim \operatorname{Hom}(M_n, N) \), 取 \(
  (m_i) \in \varprojlim M_n \), \( (f_i) \) 诱导了 \( (m_i) \mapsto f_1(m_1) \).
  验证两者互逆即可.
\end{proof}

\begin{proposition}
  逆向系统列
  \[
    0 \to (M_n, \alpha_n) \to (N_n, \beta_n) \to (P_n, \gamma_n) \to 0
  \]
  诱导了正合列
  \[
    0 \to \varprojlim M_n \to \varprojlim N_n \to \varprojlim P_n \to
    \varprojlim\nolimits^1 M_n \to \varprojlim\nolimits^1 N_n \to
    \varprojlim\nolimits^1 P_n \to 0
  \]
\end{proposition}
\begin{proof}
  考察
  % https://q.uiver.app/#q=WzAsMTQsWzAsMiwiMCJdLFsxLDIsIlxccHJvZCBNX24iXSxbMSwxLCJcXHByb2QgTV9uIl0sWzIsMSwiXFxwcm9kIE5fbiJdLFsyLDIsIlxccHJvZCBOX24iXSxbMywxLCJcXHByb2QgUF9uIl0sWzMsMiwiXFxwcm9kIFBfbiJdLFs0LDEsIjAiXSxbMSwwLCJcXHZhcnByb2psaW0gTV9uIl0sWzEsMywiXFx2YXJwcm9qbGltXjEgTV9uIl0sWzIsMywiXFx2YXJwcm9qbGltXjEgTl9uIl0sWzMsMywiXFx2YXJwcm9qbGltXjEgTl9uIl0sWzIsMCwiXFx2YXJwcm9qbGltIE5fbiJdLFszLDAsIlxcdmFycHJvamxpbSBQX24iXSxbMCwxXSxbMiwxXSxbMiwzXSxbMSw0XSxbMyw0XSxbMyw1XSxbNCw2XSxbNSw2XSxbNSw3XSxbOCwyXSxbMTIsM10sWzEzLDVdLFs4LDEyLCIiLDEseyJzdHlsZSI6eyJib2R5Ijp7Im5hbWUiOiJkYXNoZWQifX19XSxbMTIsMTMsIiIsMSx7InN0eWxlIjp7ImJvZHkiOnsibmFtZSI6ImRhc2hlZCJ9fX1dLFs5LDEwLCIiLDEseyJzdHlsZSI6eyJib2R5Ijp7Im5hbWUiOiJkYXNoZWQifX19XSxbMTAsMTEsIiIsMSx7InN0eWxlIjp7ImJvZHkiOnsibmFtZSI6ImRhc2hlZCJ9fX1dLFsxMyw5LCIiLDEseyJzdHlsZSI6eyJib2R5Ijp7Im5hbWUiOiJkYXNoZWQifX19XSxbMSw5XSxbNCwxMF0sWzYsMTFdXQ==
  \[\begin{tikzcd}
    & {\varprojlim M_n} & {\varprojlim N_n} & {\varprojlim P_n} \\
    & {\prod M_n} & {\prod N_n} & {\prod P_n} & 0 \\
    0 & {\prod M_n} & {\prod N_n} & {\prod P_n} \\
    & {\varprojlim^1 M_n} & {\varprojlim^1 N_n} & {\varprojlim^1 N_n}
    \arrow[dashed, from=1-2, to=1-3]
    \arrow[from=1-2, to=2-2]
    \arrow[dashed, from=1-3, to=1-4]
    \arrow[from=1-3, to=2-3]
    \arrow[from=1-4, to=2-4]
    \arrow[dashed, from=1-4, to=4-2]
    \arrow[from=2-2, to=2-3]
    \arrow[from=2-2, to=3-2]
    \arrow[from=2-3, to=2-4]
    \arrow[from=2-3, to=3-3]
    \arrow[from=2-4, to=2-5]
    \arrow[from=2-4, to=3-4]
    \arrow[from=3-1, to=3-2]
    \arrow[from=3-2, to=3-3]
    \arrow[from=3-2, to=4-2]
    \arrow[from=3-3, to=3-4]
    \arrow[from=3-3, to=4-3]
    \arrow[from=3-4, to=4-4]
    \arrow[dashed, from=4-2, to=4-3]
    \arrow[dashed, from=4-3, to=4-4]
  \end{tikzcd}\]
  利用 \href{https://en.wikipedia.org/wiki/Snake_lemma}{蛇形引理} 即可.
\end{proof}

\begin{corollary}
  如果映射 \( \alpha_n: M_{n + 1} \to M_n \) 都是满射的, 那么 \( \varprojlim^1
  M_n = 0 \).
\end{corollary}
\begin{proof}
  %TODO: 证明.
\end{proof}

\section{张量积}

\paragraph{张量积} 假设 \( A \) 是一个环, \( M, N \) 和 \( P \) 都是 \( A \)-模.
态射 \( \phi: M \times N \to P \) 称为 \emph{\( A \)-模双线性}, 如果
\[
  \begin{aligned}
    &\phi(x + x', y) = \phi(x, y) + \phi(x', y),     &x, x' \in M,& y \in N\\
    &\phi(x, y + y') = \phi(x, y) + \phi(x, y + y'), &x \in M,& y, y' \in N\\
    &\phi(ax, y) = a \phi(x, y), &a \in A, &x \in M, &y \in N\\
    &\phi(x, ay) = a \phi(x, y), &a \in A, &x \in M, &y \in N.
  \end{aligned}
\]
一个 \( A \)-模 \( T \) 带 \( A \)-模双线性映射
\[
  \phi: M \times N \to T
\]
称为 \emph{\( M \) 和 \( N \) 的张量积}, 如果它由下述泛性质: 对每个 \( A
\)-双线性映射
\[
  \phi': M \times N \to T'
\]
能被 \( \phi \) 唯一分解.
% https://q.uiver.app/#q=WzAsMyxbMCwwLCJNIFxcdGltZXMgTiJdLFsxLDAsIlQiXSxbMSwxLCJUJyJdLFswLDIsIlxccGhpJyIsMl0sWzAsMSwiXFxwaGkiXSxbMSwyLCJcXGV4aXN0cyAhIFxcdGV4dHvnur/mgKd9IiwwLHsic3R5bGUiOnsiYm9keSI6eyJuYW1lIjoiZGFzaGVkIn19fV1d
\[\begin{tikzcd}
	{M \times N} & T \\
	& {T'}
	\arrow["\phi", from=1-1, to=1-2]
	\arrow["{\phi'}"', from=1-1, to=2-2]
	\arrow["{\exists ! \text{线性}}", dashed, from=1-2, to=2-2]
\end{tikzcd}\]
我们将其记作 \( M \otimes_A N \), 注意到
\[
  \operatorname{Hom}_{A \text{线性}} (M \times N, T) \simeq
  \operatorname{Hom}_{A \text{线性}} (M \otimes_A N, T).
\]

\paragraph{张量积的构造} 假设 \( M \) 和 \( N \) 为 \( A \)-模, 令 \( A^{(M
\times N)} \) 为自由 \( A \)-模, 其基为 \( M \times N \), 因此每个 \( A^{(M
\times N)} \) 可以唯一表示为有限和
\[
  \sum a_i (x_i, y_i),\quad a_i \in A,\quad x_i \in M,\quad y_i \in N.
\]
假设 \( A^{(M \times N)} \) 的子模 \( P \) 由
\[
  \begin{aligned}
    &(x + x', y) - (x, y) - (x', y), &x, x' \in M, &y \in N\\
    &(x, y + y') - (x, y) - (x, y'), &x \in M, &y, y' \in N\\
    &(ax, y) - a(x, y), &a \in A, &x \in M, &y \in N\\
    &(x, ay) - a(x, y), &a \in A, &x \in M, &y \in N.
  \end{aligned}
\]
\( A^{(M \times N)} / P \) 即为 \( M \otimes_A N \).

\subsection{基本性质}

\begin{proposition}
  假设 \( M, N, P \) 为 \( A \)-模.
  \begin{enumerate}
    \item 存在唯一的同构
      \[
        \lambda: A \otimes M \to M
      \]
      对所有 \( a \in A, m \in M \) 有 \( \lambda(a \otimes m) = am  \).
    \item 存在唯一的同构
      \[
        \alpha: M \otimes (N \otimes P) \to (M \otimes N) \otimes P
      \]
      对所有 \( m \in M, n \in N, p \in P \) 有 \( \alpha(m \otimes (n \otimes
      p)) = (m \otimes n) \otimes p \).
    \item 存在唯一的同构
      \[
        \gamma: M \otimes N \to N \otimes M
      \]
      对所有 \( m \in M, n \in N \) 有 \( \gamma(m \otimes N) = n \otimes m \).
  \end{enumerate}
\end{proposition}

\paragraph{标量扩张} 假设 \( A \) 为交换环, \( B \) 为一个(未必交换的) \( A
\)-代数, 使得 \( A \to B \) 结构同态的像落于 \( B \) 的中心里. 因此
\[
  M \leadsto B \otimes_A M
\]
是一个左 \( A \)-模到左 \( B \)-模的函子. 假设 \( M \) 为 \( A \)-模, \( N \) 为
\( B \)-模, 那么
\[
  \begin{aligned}
    \operatorname{Hom}_{A-\text{线性}} (M, N) &\simeq
    \operatorname{Hom}_{B-\text{线性}} (B \otimes_A M, N)\\ \alpha &\mapsto
    \left(b \otimes m \mapsto b \cdot \alpha(m)\right)\\ \beta \circ \iota
    &\mapsfrom \beta.
  \end{aligned}
\]
如果 \( (e_{\alpha})_{\alpha \in I} \) 是 \( A \)-模 \( M \) 的一族生成元, 那么
\( (1 \otimes e_{\alpha})_{\alpha \in I} \) 为 \( B \)-模 \( B \otimes_A M \)
的一组生成元.

函子 \( M \leadsto M_B := B \otimes_A M \) 与张量积交换
\[
  (M \otimes_A N)_B \simeq M_B \otimes_B N_B.
\]
这是因为 %TODO: 证明结合律
\begin{align*}
  M_B &= (B \otimes_A M) \otimes_B (B \otimes_A N)\\ &\simeq((B \otimes_A M)
  \otimes_B B) \otimes_A N\\ &\simeq(B \otimes_A M) \otimes_A N\\ &\simeq B
  \otimes_A (M \otimes_A N) \\ &\simeq (M \otimes_A N)_B.
\end{align*}

\paragraph{有向极限}
\begin{proposition}
  有向极限和张量积是交换的
  \[
    \varinjlim_{i \in I} M_i \otimes_A \varinjlim_{j \in J} N_j \simeq
    \varinjlim_{(i, j) \in I \times J} M_i \otimes_A N_j.
  \]
\end{proposition}
\begin{proof}
  %TODO: 完成此证明
  % 由张量积的泛性质, \( \left((m_i), (n_j)\right) \mapsto \left(m_i \otimes_A
  % n_j \right) \) 诱导了 \( (m_i) \otimes_A (n_j) \mapsto \left(m_i \otimes_A
  % n_j \right) \). 反过来, 在由张量积的泛性质 \( (m_i, n_j) \mapsto m_i
  % \otimes_A n_j \), 其中
\end{proof}

\subsection{代数张量积, 张量代数以及对称代数}
\paragraph{代数张量积} 假设 \( k \) 是一个环, \( A \) 和 \( B \) 为 \( k
\)-代数. 我们称 \( k \) 代数 \( C \) 和同态 \( i: A \to C \) 和 \( j: B \to C \)
称为 \emph{\( A \) 和 \( B \) 的张量积}, 如果其满足下述泛性质: 对每一对 \( k
\)-代数同态 \( f: A \to R \) 和 \( g: B \to R \), 存在唯一的同态 \( (f, g): C
\to R \) 使得 \( (f, g) \circ i = \alpha \) 和 \( (f, g) \circ j = \beta \).
% https://q.uiver.app/#q=WzAsNCxbMCwwLCJBIl0sWzEsMCwiQyJdLFsxLDEsIlIiXSxbMiwwLCJCIl0sWzMsMiwiZyJdLFswLDIsImYiLDJdLFswLDEsImkiXSxbMywxLCJqIiwyXSxbMSwyLCJcXGV4aXN0cyAhIChmLCBnKSIsMSx7InN0eWxlIjp7ImJvZHkiOnsibmFtZSI6ImRhc2hlZCJ9fX1dXQ==
\[\begin{tikzcd}
	A & C & B \\
	& R
	\arrow["i", from=1-1, to=1-2]
	\arrow["f"', from=1-1, to=2-2]
	\arrow["{\exists ! (f, g)}"{description}, dashed, from=1-2, to=2-2]
	\arrow["j"', from=1-3, to=1-2]
	\arrow["g", from=1-3, to=2-2]
\end{tikzcd}\]
其同构意义上是唯一的, 记作 \( A \otimes_k B \). 这个泛性质告诉我们
\[
  \operatorname{Hom}(A \otimes_k B, R) \simeq \operatorname{Hom}(A, R) \times
  \operatorname{Hom}(B, R).
\]

\paragraph{张量积的构造} 首先将 \( A \) 和 \( B \) 视为 \( k \)-模,
形成了模张量积 \( A \otimes_k B \), 那么存在乘法 \( A \otimes_k B \times A
\otimes_k B \to A \otimes_k B \) 使得
\[
  (a \otimes b)(a' \otimes b') = a a' \otimes aa' \otimes aa',\quad a, a \in
  A,\quad b, b' \in B.
\]
这使得 \( A \otimes_k B \) 成为一个环, 态射
\[
  c \mapsto c (1 \otimes 1) = c \otimes 1 = 1 \otimes c
\]
使得其变为一个 \( k \)-代数. 映射 \( i, j \) 构造如下.
\[
  i: A \to A \otimes_k B, a \mapsto a \otimes 1 \text{ 和 } j: B \to A \otimes_k
  B \to B, b \mapsto 1 \otimes b
\]

\paragraph{张量代数} 假设 \( M \) 是 \( A \)-模. 对每一个 \( A \neq 0 \), 置
\[
  T^r M = \underbrace{M \otimes_A \cdots \otimes_A M}_{r \text{个}}
\]
特别地, \( T^0 M = A \) 以及 \( T^1 M = M \). 定义
\[
  TM = \bigoplus_{r \geq 0} T^r M.
\]
这个模可以赋予一个非交换 \( A \)-代数, 称为 \emph{\( M \) 的张量代数},
其乘法如:
\begin{align*}
  T^r M \times T^s M &\to T^{r + s}M\\ (m_1 \otimes \cdots \otimes m_r, m_{r +
  1} \otimes \cdots \otimes m_{r + s}) &\mapsto m_1 \otimes \cdots \otimes m_{r
  + s}.
\end{align*}
对 \( (TM, M \to TM) \) 由下述泛性质: 由 \( M \) 到(未必交换的) \( A \)-代数 \(
R \) 的 \( A \)-线性映射能唯一分解为 \( A \)-代数同态 \( TM \to R \).
% https://q.uiver.app/#q=WzAsMyxbMCwwLCJNIl0sWzEsMCwiVE0iXSxbMSwxLCJSIl0sWzAsMiwiQVxcdGV4dHst57q/5oCnfSIsMl0sWzAsMV0sWzEsMiwiXFxleGlzdHMhQVxcdGV4dHst5Luj5pWwfSIsMCx7InN0eWxlIjp7ImJvZHkiOnsibmFtZSI6ImRhc2hlZCJ9fX1dXQ==
\[\begin{tikzcd}
	M & TM \\
	& R
	\arrow[from=1-1, to=1-2]
	\arrow["{A\text{-线性}}"', from=1-1, to=2-2]
	\arrow["{\exists!A\text{-代数}}", dashed, from=1-2, to=2-2]
\end{tikzcd}\]
如果 \( M \) 是一个基为 \( x_1, \ldots, x_n \) 自由 \( A \)-模, 那么 \( TM \)
是一个 \( A \) 的非交换记号 \( x_i \) 上的非交换多项式环.

\paragraph{对称代数} 一个 \( A \)-模 \( M \) 称为 \emph{对称代数 \(
\operatorname{Sym}(M) \)} 为 \( TM \) 商去由下面 \( T^2 M \) 元素生成的理想
\[
  m \otimes n - n \otimes m, \quad m, n \in M.
\]
这是分次代数 \( \operatorname{Sym}(M) = \bigoplus_{r \geq 0}
\operatorname{Sym}^r (M) \), 其中 \( \operatorname{Sym}^{r}(M) \) 为 \(
M^{\otimes r} \) 商去下面元素生成的理想
\[
  m_1 \otimes \cdots \otimes m_r - m_{\sigma(1)} \otimes \cdots \otimes
  m_{\sigma(r)},\quad m_i \in M,\quad \sigma \in \operatorname{Sym}(r)
\]
对 \( (\operatorname{Sym}(M), M \to \operatorname{Sym}(M)) \) 有下述泛性质:
对每个由 \( M \) 到一个交换的 \( A \)-代数 \( M \) 的 \( A \)-线性映射 \( M \to
R \) 能唯一分解为 \( A \)-代数同态 \( \operatorname{Sym}(M) \to R \).
% https://q.uiver.app/#q=WzAsMyxbMCwwLCJNIl0sWzEsMCwiXFxvcGVyYXRvcm5hbWV7U3ltfShNKSJdLFsxLDEsIlIiXSxbMCwyLCJBXFx0ZXh0ey3nur/mgKd9IiwyXSxbMCwxXSxbMSwyLCJcXGV4aXN0cyFBXFx0ZXh0ey3ku6PmlbB9IiwwLHsic3R5bGUiOnsiYm9keSI6eyJuYW1lIjoiZGFzaGVkIn19fV1d
\[\begin{tikzcd}
	M & {\operatorname{Sym}(M)} \\
	& R
	\arrow[from=1-1, to=1-2]
	\arrow["{A\text{-线性}}"', from=1-1, to=2-2]
	\arrow["{\exists!A\text{-代数}}", dashed, from=1-2, to=2-2]
\end{tikzcd}\]
如果 \( M \) 是一个基为 \( x_1, \ldots, x_n \) 自由 \( A \)-模, 那么 \(
\operatorname{Sym}M \) 是一个 \( A \) 的交换记号 \( x_i \) 上的交换多项式环.
