\usepackage[leqno]{amsmath}
\usepackage{amssymb}
\usepackage[centercolon]{mathtools}
\usepackage{stmaryrd}
\usepackage{wasysym}
\usepackage{amsthm}
\usepackage{mathrsfs}
\usepackage{bm}

\usepackage{hyperref}
\hypersetup{
  colorlinks = true,
  linkcolor = blue,
  citecolor = red,
  urlcolor = teal
}
\usepackage{cleveref}

\usepackage{graphicx}
\usepackage{float}

\usepackage{tikz}
\usepackage{tikz-cd}
\usepackage{quiver}

\usepackage{xeCJK}
\usepackage{fontspec}

\usepackage{graphicx}
\usepackage{float}

\usepackage{geometry}

\geometry{
  paper=a4paper,
  top=3cm,
  inner=2.54cm,
  outer=2.54cm,
  bottom=3cm,
  headheight=6ex,
  headsep=6ex,
  twoside,
  asymmetric
}{\relax}

\usepackage[style=alphabetic]{biblatex}

\usepackage{fancyhdr}
\pagestyle{fancy}
\renewcommand{\sectionmark}[1]{\markright{#1}}
\fancyhf{}
\fancyhead[EC]{\footnotesize{\leftmark}\vspace{1mm}} %页眉部分偶数页显示章
\fancyhead[OC]{\footnotesize{\rightmark}\vspace{1mm}} %页眉部分奇数页显示节
\fancyhead[LE,RO]{{\footnotesize \thepage}\vspace{1mm}} %奇数页右边, 偶数页左边显示页码
\fancyhead[RE,LO]{}
% \fancyfoot[C]{\NTdraftstring}
\renewcommand{\headrulewidth}{0pt} %删去页眉横线
\renewcommand{\footrulewidth}{0pt} %删去页脚横线
\addtolength{\headheight}{0.5pt}

\numberwithin{equation}{section}

\usepackage{enumitem}
\setlist[enumerate,1]{
    label=(\textup{\arabic*}),
    % itemsep=0.2ex plus 0.1ex,
    % parsep=0.1ex,
    % leftmargin=1.5em,
    % topsep=0.5em,
    % align=left,
    font=\normalfont
}

\theoremstyle{plain}
\newtheorem{theorem}{定理}[section]
\newtheorem{lemma}[theorem]{引理}
\newtheorem{proposition}[theorem]{命题}
\newtheorem{corollary}[theorem]{推论}

\theoremstyle{definition}
\newtheorem{definition}[theorem]{定义}
\newtheorem{example}[theorem]{例}

\theoremstyle{remark}
\newtheorem{remark}[theorem]{注}

\crefname{theorem}{定理}{定理}
\crefname{proposition}{命题}{命题}
\crefname{lemma}{引理}{引理}
\crefname{corollary}{推论}{推论}
\crefname{definition}{定义}{定义}
\crefname{example}{例}{例}
\crefname{remark}{注}{注}

\newcommand{\dif}{\mathop{}\!\mathrm{d}}

\renewcommand{\proofname}{证明}
