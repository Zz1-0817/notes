\chapter{基本知识}

\section{环与理想}

一个\emph{环}指的是一个配备结合乘法的加性交换群.
在不加说明的情况下, 本笔记中提到的环一般要求其乘法交换且有乘法单位元.

\subsection{根式理想}

假设给定一个环 \( A \).
称 \( f \in A \) \emph{幂零}, 如果存在 \( r \geq 1 \) 使得 \( f^r = 0 \).
称 \( A \) \emph{约化的}, 如果 \( A \) 有没有非零的幂零元.
给定 \( A \) 的任意理想 \( \mathfrak{a} \), 其对应了 \( A \) 的一个理想
\begin{equation}
  \left\lbrace f \in A : f^r \in \mathfrak{a}, \text{对某个} r \geq 1
  \text{成立} \right\rbrace.
  \label{equation-radical-ideal}
\end{equation}
称由\eqref{equation-radical-ideal}定义的理想为 \( \mathfrak{a} \) 的根,
将其记作 \( \operatorname{rad}(\mathfrak{a}) \).
特别地, \( (0) \) 的根称为 \( A \) 的\emph{幂零根}, 记作\(
\operatorname{nil}(A) \).
此外, 如果 \( \operatorname{rad}(\mathfrak{a}) = \mathfrak{a} \), 那么称 \(
\mathfrak{a} \) 是 \emph{根式的}.

\begin{proposition}
  假设 \( \mathfrak{a} \) 是环 \( A \) 的一个理想.
  \begin{enumerate}
    \item \( \operatorname{rad}(A) \) 亦是 \( A \) 的一个理想.
    \item \( \operatorname{rad}(\operatorname{rad}(\mathfrak{a})) =
      \operatorname{rad}(\mathfrak{a}) \).
  \end{enumerate}
\end{proposition}

\subsection{素理想}

\paragraph{乘性子集}
环 \( A \) 的 \emph{乘性子集} 是一个子集 \( S \) 满足如下性质
\[
  1 \in S,\quad a, b \in S \implies ab \in S.
\]

\begin{proposition}
  \label{proposition-prime-not-intersect-multiplicative-set}
  假设 \( S \) 是一个环 \( A \) 的子集, \( \mathfrak{a} \) 是 \( A \) 的不交于
  \( S \) 一个理想.  \( A \) 中包含 \( \mathfrak{a} \) 但不交于 \( S \)
  的理想集族 \( \Sigma \) 有一个极大元, 如果 \( S \) 是乘性子集,
  那么每个这样的极大元都是素的.
\end{proposition}
\begin{proof}
  \( \Sigma \) 的每个上升链 \( \mathfrak{a}_1 \subseteq \mathfrak{a}_2 \subseteq
  \cdots \) 都有上界 \( \sum \mathfrak{a}_i \), 由
  \href{https://en.wikipedia.org/wiki/Zorn%27s_lemma}{Zorn 引理}, \( \Sigma \)
  有一个极大元 \( \mathfrak{c} \).
  现在说明 \( \mathfrak{c} \) 是一个素理想.
  假设不然, 如果 \( b b' \in \mathfrak{c} \), 且 \( b, b' \notin \mathfrak{c}
  \), 那么 \( \mathfrak{c} + (b) \) 和 \( \mathfrak{c} + (b') \) 真包含 \(
  \mathfrak{c} \), 因此它们不落于 \( \Sigma \) 中.
  因此 \( \mathfrak{c} + (b) \) 和 \( \mathfrak{c} + (b') \) 各包含 \( S \)
  中的一个元素, 我们分别记其为
  \[
    f = c + ab,\quad c \in \mathfrak{c},\quad a' \in A \text{ 以及 } f' = c' +
    a'b,\quad c' \in \mathfrak{c},\quad a' \in A
  \]
  因此
  \[
    f f' = c c' + abc' + a'b'c + aa'bb' \in \mathfrak{c},
  \]
  这与 \( f f' \in S \) 矛盾.
\end{proof}
\begin{remark}
  证明理想集中极大元素的方法后面还会遇到若干次.
\end{remark}

\begin{proposition}
  \label{proposition-radical-as-prime-intersection}
  给定 \( A \) 的一个理想 \( \mathfrak{a} \), 那么所有包含 \( \mathfrak{a} \)
  的素理想的交即\( \mathfrak{a} \) 的根式理想, 也就是
  \[
    \operatorname{rad}(\mathfrak{a}) = \bigcap_{\mathfrak{p}\supseteq a}
    \mathfrak{p}.
  \]
  特别地, \( \operatorname{rad}(0) = \bigcap \mathfrak{p} \).
\end{proposition}
\begin{proof}
  \( \operatorname{rad}(\mathfrak{a}) \supset \bigcap_{\mathfrak{p}\supseteq a}
  \mathfrak{p} \).
  如果 \( x \notin \operatorname{rad} \mathfrak{a} \), 也就是乘性子集 \(
  \left\lbrace 1, x, x^2, \ldots \right\rbrace \not\subset \mathfrak{a} \),
  那么由\cref{proposition-prime-not-intersect-multiplicative-set}, 总存在素理想
  \(  \mathfrak{p} \), 使得 \( x \notin \mathfrak{p} \), 故 \( x \notin \bigcap
  \mathfrak{p} \).
\end{proof}

\paragraph{素避引理}

\begin{proposition}[素避]
  假设 \( \mathfrak{p}_1, \ldots, \mathfrak{p}_r, r \geq 1 \) 为 \( A \) 的理想,
  其中 \( \mathfrak{p}_2, \ldots, \mathfrak{p}_r \) 为素理想. 如果理想 \(
  \mathfrak{a} \) 不包含在任何一个 \( \mathfrak{p}_i \), 那么 \( \mathfrak{a} \)
  不包含在 \( \mathfrak{p}_i \) 的并中.
\end{proposition}
\begin{proof}
  对 \( r \) 应用归纳假设.
  \( r = 1 \) 时结论显然.
  假设 \( r > 1 \) 且 \( \mathfrak{a} \subseteq \bigcup_{1 \leq j \leq r}
  \mathfrak{p}_j \), 但对任意 \( i \), \( \mathfrak{a} \not\subseteq \bigcup_{j
  \neq i}\mathfrak{p}_j \).
  由假设知道,
  存在 \( a_i \in \mathfrak{a} \setminus \bigcup_{j \neq i} \mathfrak{p}_j \),
  那么 \( a_i \in \mathfrak{p}_i \). 考察
  \[
    a = a_1 \cdots a_{r - 1} + a_r \in \mathfrak{a}.
  \]
  因为 \( \mathfrak{p}_r \) 素, \( a_1 \cdots a_{r - 1} \notin \mathfrak{p}_r
  \). 而 \( a_{r} \in \mathfrak{p}_r \), 故 \( a \notin \mathfrak{p}_r \).
  对 \( i \leq r - 1 \), 则 \( a_1 \cdots a_{r - 1} \in \mathfrak{p}_{i} \), 但
  \( a_r \in \mathfrak{p}_i \), 从而 \( a \notin \mathfrak{p}_i \). 这与 \( a
  \subseteq \cup_{1 \leq j \leq r} \mathfrak{p}_j \) 矛盾.
  因此 \( \mathfrak{a} \subseteq \cup_{j \neq i} \mathfrak{p}_j \) 对某个 \( i
  \) 成立, 应用归纳假设即可.
\end{proof}

\subsection{极大理想与 Jacobson 根}

环 \( A \) 的 \emph{Jacobson 根} 是 \( A \) 所有极大理想的交
\[
  \mathfrak{J} = \bigcap \left\lbrace \mathfrak{m}: \mathfrak{m} \text{ 为 } A
  \text{的极大理想} \right\rbrace.
\]
如果 \( A \) 只有一个极大理想, 那么环 \( A \) 称为是 \emph{局部} 的

\begin{proposition}
  假设 \( A \) 的 Jacobson 根为 \( \mathfrak{J} \), 那么 \( c \in \mathfrak{J}
  \) 当且仅当对所有 \( a \in A \), \( 1 - ac \) 是单位.
\end{proposition}
\begin{proof}
  给定 \( c \in A \).
  \( c \notin \mathfrak{J} \iff \) 存在 \( A \) 的极大理想 \( \mathfrak{m} \)
  使得 \( c \notin \mathfrak{m} \iff \) 存在极大理想 \( \mathfrak{m} \) 以及 \(
  a \in A, m \in \mathfrak{m} \) 使得 \( ac = 1 + m \), 也就是 \( 1 - ac \)
  并非单位.
\end{proof}

\paragraph{Nakayama引理}

\begin{theorem}[Nakayama引理]
  \label{theorem-Nakayama-lemma}
  假设 \( A \) 是一个环, \( \mathfrak{a} \) 是 \( A \) 的理想, \( M \) 是一个 \(
  A\)-模. 如果 \( \mathfrak{a} \) 包含于所有 \( A \) 的极大理想中, 且 \( M \)
  有限生成, 那么
  \begin{enumerate}
    \item 如果 \( M = \mathfrak{a} M \), 那么 \( M = 0 \).
    \item 如果 \( N \) 是 \( M \) 的子模, 使得 \( M = N + \mathfrak{a} M \),
      那么 \( M = N \).
  \end{enumerate}
\end{theorem}
\begin{proof}
  (i) 如果 \( M \neq 0 \), 由有限生成性可以取一族 \( M \) 数量最少的生成元 \(
  \left\lbrace e_1, \ldots, e_n \right\rbrace, n \geq 1 \), 记
  \[
    e_1 = a_1 e_1 + \cdots + a_n e_n,\quad a_i \in \mathfrak{a}.
  \]
  那么
  \[
    (1 - a_1) e_1 = a_2 e_2 + \cdots + a_n e_n.
  \]
  \( \mathfrak{a} \) 包含于 \( A \) 的 Jacobson 根中, 从而 \( 1 - a_1 \) 可逆.
  因此 \( e_2, \ldots, e_n \) 生成 \( M_2 \), 与选取矛盾. (ii) 取商, 利用 (i)
  立刻得到结果.
\end{proof}

\section{环态射, 理想扩张与收缩}

\subsection{中国剩余定理}

\( A \) 的两个理想 \( \mathfrak{a} \) 和 \( \mathfrak{b} \) 是 \emph{互素} 的,
如果 \( \mathfrak{a} + \mathfrak{b} = A \).
\begin{proposition}
  如果 \( \mathfrak{a}_1,\ldots, \mathfrak{a}_n \) 互素, 那么 \( \bigcap
  \mathfrak{a}_i = \prod \mathfrak{a}_i \).
\end{proposition}
\begin{proof}
  \( \bigcap \mathfrak{a}_i \subset \prod \mathfrak{a}_i \).
  这里只证明 \( n = 2 \) 的情况, 一般情况由归纳法保证.
  由假设可以找到 \( a_1 \in \mathfrak{a}_1 \) 以及 \( a_2 \in \mathfrak{a}_2 \)
  使得 \( a_1 + a_2 = 1 \). 如果 \( a \in \mathfrak{a}_1 \cap \mathfrak{a}_2 \),
  那么 \( a = (a_1 + a_2)a \in \mathfrak{a}_1 \mathfrak{a}_2 \).
\end{proof}

\begin{theorem}[中国剩余定理]
  假设 \( \mathfrak{a}_1, \ldots, \mathfrak{a}_n \) 是环 \( A \) 的理想. 并且在
  \( i \neq j \) 时,  \( \mathfrak{a}_i \) 和 \( \mathfrak{a}_j \) 互素,
  那么映射
  \[
    A \to A / \mathfrak{a}_1 \times \cdots \times A / \mathfrak{a}_n,\quad a
    \mapsto (\ldots, a + \mathfrak{a}_i, \ldots)
  \]
  是满射, 且其核为 \( \prod \mathfrak{a}_i = \bigcap \mathfrak{a}_i \).
\end{theorem}
\begin{proof}
  对 \( n \) 做归纳假设.
  \( n = 2 \) 时.
  因为 \( \mathfrak{a}_1 + \mathfrak{a}_2 = A \), 存在 \( a_i \in \mathfrak{a}_i
  \) 使得 \( a_1 + a_2 = 1 \).
  对任意 \( x_1, x_2 \in A \), \( a_1 x_2 + a_2 x_1 \) 映射到 \( (x_1
  \mod{\mathfrak{a}_1}, x_2 \mod{\mathfrak{a}_2}) \), 所以此映射是满射.
  其核显然是 \( \mathfrak{a}_1 \cap \mathfrak{a}_2 \).

  现假设 \( n > 2 \), 对于 \( i \geq 2 \), 存在 \( a_i \in \mathfrak{a}_1 \)
  以及 \( b_i \in \mathfrak{a}_i \) 使得
  \[
    a_i + b_i = 1.
  \]
  积 \( \prod_{i \geq 2}(a_i + b_i) = 1 \) 落于 \( \mathfrak{a}_1 +
  \mathfrak{a}_2 \cdots \mathfrak{a}_n \) 中, 换句话说
  \[
    \mathfrak{a}_1 + \mathfrak{a}_2\cdots \mathfrak{a}_n = A.
  \]
  因此,
  \begin{align*}
    A / \mathfrak{a}_1 \cdots \mathfrak{a}_n &= A / \mathfrak{a}_1 \cdot
    (\mathfrak{a}_2 \cdots \mathfrak{a}_n)\\ &\simeq A / \mathfrak{a}_1 \times A
    / \mathfrak{a}_2 \cdots \mathfrak{a}_n\\ &\simeq A/ \mathfrak{a}_1 \times A
    / \mathfrak{a}_2 \times \cdots \times A / \mathfrak{a}_n,
  \end{align*}
  其中, 最后一个同构由归纳法保证.
\end{proof}


\subsection{理想的扩张与收缩}

假设 \( \varphi: A \to B \) 是一个环同态.
\begin{itemize}
  \item 对 \( B \) 的理想 \( \mathfrak{b} \), \( \varphi^{-1}(\mathfrak{b}) \)
    是\( A \) 的理想, 称为 \( \mathfrak{b} \) 在 \( A \) 上的\emph{收缩},
    记作 \( \mathfrak{b}^{c} \).
  \item 对于 \( A \) 的理想 \( \mathfrak{a} \), \( B \) 中由 \(
    \varphi(\mathfrak{a}) \) 生成的理想, 称为 \( \mathfrak{a} \) 在 \( B \) 中的
    \emph{扩张}, 记作 \( \mathfrak{a}^{e} \).
\end{itemize}
特别地, 如果 \( \varphi \) 满, 那么 \( \varphi(\mathfrak{a}) = \mathfrak{a}^e \);
如果 \( A \) 是 \( B \) 的子环, 那么 \( \mathfrak{b}^{c} = \mathfrak{b} \cap A
\).

\begin{proposition}
  假设\( \mathfrak{a}, \mathfrak{a}' \) 为 \( A \) 的理想, \( \mathfrak{b},
  \mathfrak{b}' \) 为 \( B \) 的理想
  \begin{enumerate}
    \item 我们有下面关系
      \[
        (\mathfrak{a} + \mathfrak{a}')^{e} = \mathfrak{a}^e +
        \mathfrak{a}'^e,\quad (\mathfrak{a} \mathfrak{a}')^e = \mathfrak{a}^e
        \mathfrak{a}'^{e},\quad (\mathfrak{b} \cap \mathfrak{b}')^c =
        \mathfrak{b}^c \cap \mathfrak{b}'^c,\quad
        \operatorname{rad}(\mathfrak{b})^c = \operatorname{rad}(\mathfrak{b}^c).
      \]
    \item \( \mathfrak{a} \subseteq \mathfrak{a}^{ec} \) 且 \( \mathfrak{b}^{ce}
      \subseteq \mathfrak{b} \). 特别地, \( \mathfrak{a}^{e} \subseteq
      \mathfrak{a}^{ece} \) 以及 \( \mathfrak{b}^{cec} = \mathfrak{b}^c \),
      从而我们有双射
      \[
        \left\lbrace \mathfrak{b}^c \subseteq A: \mathfrak{b} \text{为}B
        \text{的理想} \right\rbrace \xleftrightarrow[\mathfrak{b}^c \mapsfrom
        \mathfrak{b}]{\mathfrak{a} \mapsto \mathfrak{a}^e} \left\lbrace
        \mathfrak{a}^e \subseteq B: \mathfrak{a} \text{是} A \text{中的理想}
        \right\rbrace
      \]
  \end{enumerate}
\end{proposition}
