\chapter{基本知识}

\section{环与理想}

一个\emph{环}指的是一个配备结合乘法的加性交换群.
在不加说明的情况下, 本笔记中提到的环一般要求其乘法交换且有乘法单位元.

\subsection{根式理想}

假设给定一个环 \( A \).
称 \( f \in A \) \emph{幂零}, 如果存在 \( r \geq 1 \) 使得 \( f^r = 0 \).
称 \( A \) \emph{约化的}, 如果 \( A \) 有没有非零的幂零元.
给定 \( A \) 的任意理想 \( \mathfrak{a} \), 其对应了 \( A \) 的一个理想
\begin{equation}
  \left\lbrace f \in A : f^r \in \mathfrak{a}, \text{对某个} r \geq 1
  \text{成立} \right\rbrace.
  \label{equation-radical-ideal}
\end{equation}
称由\eqref{equation-radical-ideal}定义的理想为 \( \mathfrak{a} \) 的根,
将其记作 \( \operatorname{rad}(\mathfrak{a}) \).
特别地, \( (0) \) 的根称为 \( A \) 的\emph{幂零根}, 记作\(
\operatorname{nil}(A) \).
此外, 如果 \( \operatorname{rad}(\mathfrak{a}) = \mathfrak{a} \), 那么称 \(
\mathfrak{a} \) 是 \emph{根式的}.

\begin{proposition}
  假设 \( \mathfrak{a} \) 是环 \( A \) 的一个理想.
  \begin{enumerate}
    \item \( \operatorname{rad}(A) \) 亦是 \( A \) 的一个理想.
    \item \( \operatorname{rad}(\operatorname{rad}(\mathfrak{a})) =
      \operatorname{rad}(\mathfrak{a}) \).
  \end{enumerate}
\end{proposition}

\subsection{素理想}

\paragraph{乘性子集}
环 \( A \) 的 \emph{乘性子集} 是一个子集 \( S \) 满足如下性质
\[
  1 \in S,\quad a, b \in S \implies ab \in S.
\]

\begin{proposition}
  \label{proposition-prime-not-intersect-multiplicative-set}
  假设 \( S \) 是一个环 \( A \) 的子集, \( \mathfrak{a} \) 是 \( A \) 的不交于
  \( S \) 一个理想.  \( A \) 中包含 \( \mathfrak{a} \) 但不交于 \( S \)
  的理想集族 \( \Sigma \) 有一个极大元, 如果 \( S \) 是乘性子集,
  那么每个这样的极大元都是素的.
\end{proposition}
\begin{proof}
  \( \Sigma \) 的每个上升链 \( \mathfrak{a}_1 \subseteq \mathfrak{a}_2 \subseteq
  \cdots \) 都有上界 \( \sum \mathfrak{a}_i \), 由
  \href{https://en.wikipedia.org/wiki/Zorn%27s_lemma}{Zorn 引理}, \( \Sigma \)
  有一个极大元 \( \mathfrak{c} \).
  现在说明 \( \mathfrak{c} \) 是一个素理想.
  假设不然, 如果 \( b b' \in \mathfrak{c} \), 且 \( b, b' \notin \mathfrak{c}
  \), 那么 \( \mathfrak{c} + (b) \) 和 \( \mathfrak{c} + (b') \) 真包含 \(
  \mathfrak{c} \), 因此它们不落于 \( \Sigma \) 中.
  因此 \( \mathfrak{c} + (b) \) 和 \( \mathfrak{c} + (b') \) 各包含 \( S \)
  中的一个元素, 我们分别记其为
  \[
    f = c + ab,\quad c \in \mathfrak{c},\quad a' \in A \text{ 以及 } f' = c' +
    a'b,\quad c' \in \mathfrak{c},\quad a' \in A
  \]
  因此
  \[
    f f' = c c' + abc' + a'b'c + aa'bb' \in \mathfrak{c},
  \]
  这与 \( f f' \in S \) 矛盾.
\end{proof}
\begin{remark}
  \label{remark-maximal-element-prime}
  这种证明理想集中极大元素的方法后面还会遇到若干次.
\end{remark}

\begin{proposition}
  \label{proposition-radical-as-prime-intersection}
  给定 \( A \) 的一个理想 \( \mathfrak{a} \), 那么所有包含 \( \mathfrak{a} \)
  的素理想的交即\( \mathfrak{a} \) 的根式理想, 也就是
  \[
    \operatorname{rad}(\mathfrak{a}) = \bigcap_{\mathfrak{p}\supseteq a}
    \mathfrak{p}.
  \]
  特别地, \( \operatorname{rad}(0) = \bigcap \mathfrak{p} \).
\end{proposition}
\begin{proof}
  \( \operatorname{rad}(\mathfrak{a}) \supset \bigcap_{\mathfrak{p}\supseteq a}
  \mathfrak{p} \).
  如果 \( x \notin \operatorname{rad} \mathfrak{a} \), 也就是乘性子集 \(
  \left\lbrace 1, x, x^2, \ldots \right\rbrace \not\subset \mathfrak{a} \),
  那么由\cref{proposition-prime-not-intersect-multiplicative-set}, 总存在素理想
  \(  \mathfrak{p} \), 使得 \( x \notin \mathfrak{p} \), 故 \( x \notin \bigcap
  \mathfrak{p} \).
\end{proof}

\paragraph{素避引理}

\begin{proposition}[素避]
  假设 \( \mathfrak{p}_1, \ldots, \mathfrak{p}_r, r \geq 1 \) 为 \( A \) 的理想,
  其中 \( \mathfrak{p}_2, \ldots, \mathfrak{p}_r \) 为素理想. 如果理想 \(
  \mathfrak{a} \) 不包含在任何一个 \( \mathfrak{p}_i \), 那么 \( \mathfrak{a} \)
  不包含在 \( \mathfrak{p}_i \) 的并中.
\end{proposition}
\begin{proof}
  对 \( r \) 应用归纳假设.
  \( r = 1 \) 时结论显然.
  假设 \( r > 1 \) 且 \( \mathfrak{a} \subseteq \bigcup_{1 \leq j \leq r}
  \mathfrak{p}_j \), 但对任意 \( i \), \( \mathfrak{a} \not\subseteq \bigcup_{j
  \neq i}\mathfrak{p}_j \).
  由假设知道,
  存在 \( a_i \in \mathfrak{a} \setminus \bigcup_{j \neq i} \mathfrak{p}_j \),
  那么 \( a_i \in \mathfrak{p}_i \). 考察
  \[
    a = a_1 \cdots a_{r - 1} + a_r \in \mathfrak{a}.
  \]
  因为 \( \mathfrak{p}_r \) 素, \( a_1 \cdots a_{r - 1} \notin \mathfrak{p}_r
  \). 而 \( a_{r} \in \mathfrak{p}_r \), 故 \( a \notin \mathfrak{p}_r \).
  对 \( i \leq r - 1 \), 则 \( a_1 \cdots a_{r - 1} \in \mathfrak{p}_{i} \), 但
  \( a_r \in \mathfrak{p}_i \), 从而 \( a \notin \mathfrak{p}_i \). 这与 \( a
  \subseteq \cup_{1 \leq j \leq r} \mathfrak{p}_j \) 矛盾.
  因此 \( \mathfrak{a} \subseteq \cup_{j \neq i} \mathfrak{p}_j \) 对某个 \( i
  \) 成立, 应用归纳假设即可.
\end{proof}

\subsection{极大理想与 Jacobson 根}

环 \( A \) 的 \emph{Jacobson 根} 是 \( A \) 所有极大理想的交
\[
  \mathfrak{J} = \bigcap \left\lbrace \mathfrak{m}: \mathfrak{m} \text{ 为 } A
  \text{的极大理想} \right\rbrace.
\]
如果 \( A \) 只有一个极大理想, 那么环 \( A \) 称为是 \emph{局部} 的

\begin{proposition}
  \label{proposition-Jacobson-root-iff-condition}
  假设 \( A \) 的 Jacobson 根为 \( \mathfrak{J} \), 那么 \( c \in \mathfrak{J}
  \) 当且仅当对所有 \( a \in A \), \( 1 - ac \) 是单位.
\end{proposition}
\begin{proof}
  给定 \( c \in A \).
  \( c \notin \mathfrak{J} \iff \) 存在 \( A \) 的极大理想 \( \mathfrak{m} \)
  使得 \( c \notin \mathfrak{m} \iff \) 存在极大理想 \( \mathfrak{m} \) 以及 \(
  a \in A, m \in \mathfrak{m} \) 使得 \( ac = 1 + m \), 也就是 \( 1 - ac \)
  并非单位.
\end{proof}

\paragraph{Nakayama引理}

\begin{theorem}[Nakayama引理]
  \label{theorem-Nakayama-lemma}
  假设 \( A \) 是一个环, \( \mathfrak{a} \) 是 \( A \) 的理想, \( M \) 是一个 \(
  A\)-模. 如果 \( \mathfrak{a} \) 包含于所有 \( A \) 的极大理想中, 且 \( M \)
  有限生成, 那么
  \begin{enumerate}
    \item 如果 \( M = \mathfrak{a} M \), 那么 \( M = 0 \).
    \item 如果 \( N \) 是 \( M \) 的子模, 使得 \( M = N + \mathfrak{a} M \),
      那么 \( M = N \).
  \end{enumerate}
\end{theorem}
\begin{proof}
  (1) 如果 \( M \neq 0 \), 由有限生成性可以取一族 \( M \) 的极小生成元 \(
  \left\lbrace e_1, \ldots, e_n \right\rbrace, n \geq 1 \).
  记
  \[
    e_1 = a_1 e_1 + \cdots + a_n e_n,\quad a_i \in \mathfrak{a}.
  \]
  那么
  \[
    (1 - a_1) e_1 = a_2 e_2 + \cdots + a_n e_n.
  \]
  \( \mathfrak{a} \) 包含于 \( A \) 的 Jacobson 根中, 从而 \( 1 - a_1 \) 可逆.
  因此 \( e_2, \ldots, e_n \) 生成 \( M_2 \), 与选取矛盾.
  立刻得到结果.
\end{proof}

\section{环态射, 理想扩张与收缩}

\subsection{中国剩余定理}

\( A \) 的两个理想 \( \mathfrak{a} \) 和 \( \mathfrak{b} \) 是 \emph{互素} 的,
如果 \( \mathfrak{a} + \mathfrak{b} = A \).
\begin{proposition}
  如果 \( \mathfrak{a}_1,\ldots, \mathfrak{a}_n \) 互素, 那么 \( \bigcap
  \mathfrak{a}_i = \prod \mathfrak{a}_i \).
\end{proposition}
\begin{proof}
  \( \bigcap \mathfrak{a}_i \subset \prod \mathfrak{a}_i \).
  这里只证明 \( n = 2 \) 的情况, 一般情况由归纳法保证.
  由假设可以找到 \( a_1 \in \mathfrak{a}_1 \) 以及 \( a_2 \in \mathfrak{a}_2 \)
  使得 \( a_1 + a_2 = 1 \). 如果 \( a \in \mathfrak{a}_1 \cap \mathfrak{a}_2 \),
  那么 \( a = (a_1 + a_2)a \in \mathfrak{a}_1 \mathfrak{a}_2 \).
\end{proof}

\begin{theorem}[中国剩余定理]
  假设 \( \mathfrak{a}_1, \ldots, \mathfrak{a}_n \) 是环 \( A \) 的理想. 并且在
  \( i \neq j \) 时,  \( \mathfrak{a}_i \) 和 \( \mathfrak{a}_j \) 互素,
  那么映射
  \[
    A \to A / \mathfrak{a}_1 \times \cdots \times A / \mathfrak{a}_n,\quad a
    \mapsto (\ldots, a + \mathfrak{a}_i, \ldots)
  \]
  是满射, 且其核为 \( \prod \mathfrak{a}_i = \bigcap \mathfrak{a}_i \).
\end{theorem}
\begin{proof}
  对 \( n \) 做归纳假设.
  \( n = 2 \) 时.
  因为 \( \mathfrak{a}_1 + \mathfrak{a}_2 = A \), 存在 \( a_i \in \mathfrak{a}_i
  \) 使得 \( a_1 + a_2 = 1 \).
  对任意 \( x_1, x_2 \in A \), \( a_1 x_2 + a_2 x_1 \) 映射到 \( (x_1
  \mod{\mathfrak{a}_1}, x_2 \mod{\mathfrak{a}_2}) \), 所以此映射是满射.
  其核显然是 \( \mathfrak{a}_1 \cap \mathfrak{a}_2 \).

  现假设 \( n > 2 \), 对于 \( i \geq 2 \), 存在 \( a_i \in \mathfrak{a}_1 \)
  以及 \( b_i \in \mathfrak{a}_i \) 使得
  \[
    a_i + b_i = 1.
  \]
  积 \( \prod_{i \geq 2}(a_i + b_i) = 1 \) 落于 \( \mathfrak{a}_1 +
  \mathfrak{a}_2 \cdots \mathfrak{a}_n \) 中, 换句话说
  \[
    \mathfrak{a}_1 + \mathfrak{a}_2\cdots \mathfrak{a}_n = A.
  \]
  因此,
  \begin{align*}
    A / \mathfrak{a}_1 \cdots \mathfrak{a}_n &= A / \mathfrak{a}_1 \cdot
    (\mathfrak{a}_2 \cdots \mathfrak{a}_n)\\ &\simeq A / \mathfrak{a}_1 \times A
    / \mathfrak{a}_2 \cdots \mathfrak{a}_n\\ &\simeq A/ \mathfrak{a}_1 \times A
    / \mathfrak{a}_2 \times \cdots \times A / \mathfrak{a}_n,
  \end{align*}
  其中, 最后一个同构由归纳法保证.
\end{proof}


\subsection{理想的扩张与收缩}

假设 \( \varphi: A \to B \) 是一个环同态.
\begin{itemize}
  \item 对 \( B \) 的理想 \( \mathfrak{b} \), \( \varphi^{-1}(\mathfrak{b}) \)
    是\( A \) 的理想, 称为 \( \mathfrak{b} \) 在 \( A \) 上的\emph{收缩},
    记作 \( \mathfrak{b}^{c} \).
  \item 对于 \( A \) 的理想 \( \mathfrak{a} \), \( B \) 中由 \(
    \varphi(\mathfrak{a}) \) 生成的理想, 称为 \( \mathfrak{a} \) 在 \( B \) 中的
    \emph{扩张}, 记作 \( \mathfrak{a}^{e} \).
\end{itemize}
特别地, 如果 \( \varphi \) 满, 那么 \( \varphi(\mathfrak{a}) = \mathfrak{a}^e \);
如果 \( A \) 是 \( B \) 的子环, 那么 \( \mathfrak{b}^{c} = \mathfrak{b} \cap A
\).

\begin{proposition}
  假设\( \mathfrak{a}, \mathfrak{a}' \) 为 \( A \) 的理想, \( \mathfrak{b},
  \mathfrak{b}' \) 为 \( B \) 的理想
  \begin{enumerate}
    \item 我们有下面关系
      \[
        (\mathfrak{a} + \mathfrak{a}')^{e} = \mathfrak{a}^e +
        \mathfrak{a}'^e,\quad (\mathfrak{a} \mathfrak{a}')^e = \mathfrak{a}^e
        \mathfrak{a}'^{e},\quad (\mathfrak{b} \cap \mathfrak{b}')^c =
        \mathfrak{b}^c \cap \mathfrak{b}'^c,\quad
        \operatorname{rad}(\mathfrak{b})^c = \operatorname{rad}(\mathfrak{b}^c).
      \]
    \item \( \mathfrak{a} \subseteq \mathfrak{a}^{ec} \) 且 \( \mathfrak{b}^{ce}
      \subseteq \mathfrak{b} \). 特别地, \( \mathfrak{a}^{e} \subseteq
      \mathfrak{a}^{ece} \) 以及 \( \mathfrak{b}^{cec} = \mathfrak{b}^c \),
      从而我们有双射
      \[
        \left\lbrace \mathfrak{b}^c \subseteq A: \mathfrak{b} \text{为}B
        \text{的理想} \right\rbrace \xleftrightarrow[\mathfrak{b}^c \mapsfrom
        \mathfrak{b}]{\mathfrak{a} \mapsto \mathfrak{a}^e} \left\lbrace
        \mathfrak{a}^e \subseteq B: \mathfrak{a} \text{是} A \text{中的理想}
        \right\rbrace
      \]
  \end{enumerate}
\end{proposition}

\begin{proposition}
  \label{proposition-prime-as-contraction-iff-homomorphism-condition}
  \( A \) 的素理想 \( \mathfrak{p} \)
  是 \( B \) 某个素理想的收缩当且仅当 \( \mathfrak{p} = \mathfrak{p}^{ec} \).
\end{proposition}
\begin{proof}
  \( \impliedby \) 假设 \( S = A \setminus \mathfrak{p} \), \( s \in S \).
  如果 \( \varphi(s) \in \mathfrak{p}^e \), 那么 \( s \in \mathfrak{p}^{ec} =
  \mathfrak{p} \), 与 \( s \) 定义矛盾, 因此 \( \varphi(S) \cap \mathfrak{p}^e =
  \varnothing \).
  由\cref{proposition-prime-not-intersect-multiplicative-set}, 存在 \(
  B \) 的一个素理想 \( \mathfrak{q} \) 包含 \( \mathfrak{p}^e \) 且不交于 \(
  \varphi(S) \).
  结合 \( S \) 的构造 \( \mathfrak{q}^e = \mathfrak{p} \).
\end{proof}

\section{局部化}

\subsection{分式环}

假设 \( S \) 是环 \( A \) 的一个乘性子集. 定义 \( A \times S \) 的一个关系
\[
  (a, s) \sim (b, t) \iff \text{存在} u \in S, \text{使得} u(at - bs) = 0.
\]
可以验证, 这是一个等价关系, 将其中代表元 \( (a, s) \) 记作 \(
\frac{a}{s} \), 那么我们得到了一个环
\[
  S^{-1}A = \left\lbrace \frac{a}{s}: a \in A, s \in S \right\rbrace
\]
以及一个环同态 \( i_S: A \to S^{-1} A, a \mapsto \frac{a}{1} \), 其核为
\[
  \left\lbrace a \in A: \text{存在} s \in S, \text{使得} sa = 0 \right\rbrace.
\]
\begin{remark}
  \label{remark-fraction-ring}
  \begin{enumerate}
    \item \( S \) 没有零除子当且仅当 \( i_S \) 就是单射
    \item 如果 \( 0 \in S \), 那么 \( S^{-1} A = 0 \).
  \end{enumerate}
\end{remark}
\begin{proposition}
  \label{proposition-universal-property-localization}
  局部化 \( (S^{-1}A, i_S) \) 具有下述泛性质: 每个 \( S \) 中的元素映到 \(
  S^{-1} A \) 中的单位, 并且每个将 \( S \) 映到单位的环同态 \( \alpha: A \to B
  \) 能通过 \( i_S \) 唯一分解.
  % https://q.uiver.app/#q=WzAsMyxbMCwwLCJBIl0sWzEsMCwiU157LTF9QSJdLFsxLDEsIkIiXSxbMCwyLCJcXGFscGhhIiwyXSxbMSwyLCJcXGV4aXN0cyAhIiwwLHsic3R5bGUiOnsiYm9keSI6eyJuYW1lIjoiZGFzaGVkIn19fV0sWzAsMSwiaV9TIl1d
\[\begin{tikzcd}
	A & {S^{-1}A} \\
	& B
	\arrow["{i_S}", from=1-1, to=1-2]
	\arrow["\alpha"', from=1-1, to=2-2]
	\arrow["{\exists !}", dashed, from=1-2, to=2-2]
\end{tikzcd}\]
特别地, 局部化是同构意义下唯一的.
\end{proposition}

设 \( h \in A \), 那么 \( S_h = \left\lbrace 1, h, h^2, \ldots \right\rbrace \)
是 \( A \) 的一个乘性子集. 我们简记 \( S_h^{-1}A \) 为 \( A_h \).

\begin{proposition}
  给定一个环 \( A \) 和 \( h \in A \), 映射
  \[
    \sum a_i X^i \mapsto \sum \frac{a_i}{h^i}
  \]
  给出了同构
  \[
    A[X] / (1 - hX) \to A_h.
  \]
\end{proposition}
\begin{proof}
  假设 \( h \in \operatorname{nil} A \), 那么 \( hX \in \operatorname{nil} A[X]
  \).
  而 \( \operatorname{nil}A[X] \) 包含于 \( A[X] \) 的 Jacobson 根中,
  于是由\cref{proposition-Jacobson-root-iff-condition} 得到 \( (1 - hX) = A[X]
  \), 即 \( A[X] / (1 - hX) = 0 \).
  结合\cref{remark-fraction-ring} 知道 \( A_h = 0 \), 故这是一个同构.

  假设 \( h \notin \operatorname{nil} A \).
  记 \( A[x] = A[X] / (1 - hX)\), 那么 \( 1 = hx \), 即 \( h \) 在 \( A[x] \)
  中可逆.
  由\cref{proposition-universal-property-localization}, 自然态射 \( A \to A[x]
  \) 可以分解为 \( A \xrightarrow{i} A_h \xrightarrow{\alpha} A[x]  \), 可以验证
  \( \alpha \) 和命题中映射互逆.
\end{proof}

\paragraph{局部化下的理想对应} 假设 \( S \) 是一个 \( A \) 的乘性子集, \( S^{-1}
A \) 为对应局部化. \( A \) 的理想 \( \mathfrak{a} \) 在 \( S^{-1}A \)
中生成的理想为
\[
  S^{-1} \mathfrak{a} = \left\lbrace \frac{a}{s}: a \in \mathfrak{a}, s \in S
  \right\rbrace.
\]
\begin{proposition}
  \label{proposition-local-ideal-contraction-and-extension}
  假设 \( S \) 是 \( A \) 的乘性子集, 自然态射 \( i_S: A \to  S^{-1} A \)
  具有下面性质:
  \begin{enumerate}
    \item 对所有 \( S^{-1} A \) 的理想 \( \mathfrak{b} \), 有 \(
      \mathfrak{b}^{ce} = \mathfrak{b} \).
    \item 如果 \( \mathfrak{p} \) 是 \( A \) 的一个不交于 \( S \) 的素理想, 那么
      \( \mathfrak{p}^{ec} = \mathfrak{p} \).
      特别地, 每个这样的 \( \mathfrak{p} \) 都是 \( S^{-1} A \)
      中某个素理想的收缩.
  \end{enumerate}
\end{proposition}
\begin{proof}
  \( \mathfrak{b}^{ce} \supset \mathfrak{b} \).
  对任意 \( \frac{a}{s} \in \mathfrak{b} \), \( s\frac{a}{s} = \frac{a}{1} \in
  \mathfrak{b} \).
  因此 \( a \in \mathfrak{b}^c \). 而 \( \frac{a}{s} = a \frac{1}{s}
  \in \mathfrak{b}^{ce} \).

  \( \mathfrak{a}^{ec} \subset \mathfrak{p} \).
  对任意 \( a \in \mathfrak{p}^{ec} \), 存在 \( a' \in \mathfrak{p}, s \in S \)
  使得 \( \frac{a}{1} = \frac{a'}{s} \).
  由局部化定义 \( t \in S \) 使得 \( tsa = ta' \in \mathfrak{p} \).
  由条件 \( S \cap \mathfrak{p} = \varnothing \) 以及 \( ts \in S \), 只能 \( a
  \in \mathfrak{p} \).
  最后一个论断由\cref{proposition-prime-as-contraction-iff-homomorphism-condition}保证.
\end{proof}

\begin{proposition}
  下面映射互为双射
  \[
    \begin{split}
      \left\lbrace A \text{中不交} S \text{的素理想} \right\rbrace
      &\leftrightarrow \left\lbrace S^{-1} A \text{的素理想} \right\rbrace\\
      \mathfrak{p} &\mapsto \mathfrak{p}^e\\
      \mathfrak{q}^c &\mapsfrom \mathfrak{q}
    \end{split}
  \]
\end{proposition}
\begin{proof}
  由\cref{proposition-local-ideal-contraction-and-extension}, 只需再验证对 \(
  S^{-1}A \) 的任意素理想 \( \mathfrak{q} \), \( \mathfrak{q}^c \) 是不交 \( S
  \) 的素理想.
  假设不然, 存在 \( s \in \mathfrak{q}^c \cap S \), 那么 \( \frac{s}{1} \in
  \mathfrak{q}^{ce} = \mathfrak{q} \), 也就是 \( \mathfrak{q} = S^{-1} A \),
  这与 \( \mathfrak{q} \) 素矛盾.
\end{proof}

\paragraph{局部化与商交换}
\begin{proposition}
  假设 \( \mathfrak{p} \) 是一个不交于 \( S \) 的 \( A \) 的素理想, 令 \(
  \overline{S} \) 为 \( S \) 在 \( A / \mathfrak{p} \) 中的像.
  那么
  \[
    (S^{-1} A) / \mathfrak{p}^e \simeq \overline{S}^{-1} (A / \mathfrak{p}),
  \]
  也就是说, 此时局部化和商是可交换的.
\end{proposition}
\begin{proof}
  由条件, 合成映射 \( A \to A / \mathfrak{p} \to \overline{S}^{-1}(A /
  \mathfrak{p}) \) 将 \( S \) 中元素映到可逆元,
  故可通过\cref{proposition-universal-property-localization}提升为
  \( S^{-1} A \to \overline{S}^{-1}(A / \mathfrak{p}) \);
  而 \( \mathfrak{p}^e \) 在此态射中像为零, 故可以进一步提升 \( (S^{-1} A)
  /\mathfrak{p}^e \to \overline{S}^{-1} (A/ \mathfrak{p})\).

  合成态射 \( A \to S^{-1} A \to S^{-1} A / \mathfrak{p}^e \) 将 \( \mathfrak{p}
  \) 映到零, 故可提升为 \( A / \mathfrak{p} \to S^{-1} A / \mathfrak{p}^e \), \(
  \overline{S} \) 中元素均被此映射映到可逆元,
  故又由\cref{proposition-universal-property-localization} 可提升为 \(
  \overline{S}^{-1}(A / \mathfrak{p}) \to S^{-1} A / \mathfrak{p}^e  \).

  可以验证两个映射互逆.
\end{proof}


\begin{corollary}
  如果 \( A \) 是 noetherian 的, \( S \) 为 \( A \) 的一个乘性子集, 那么 \(
  S^{-1} A \) 亦是 noetherian 的.
\end{corollary}
\begin{proof}
  \( \mathfrak{b}^c \) 有限生成, \( \mathfrak{b} = \mathfrak{b}^{ce} \)
  有限生成.
\end{proof}

令 \( \mathfrak{p} \) 为 \( A \) 的素理想, 那么 \( S_{\mathfrak{p}} := A
\setminus \mathfrak{p} \) 为 \( A \) 的乘性子集, 那么每个 \( A_{\mathfrak{p}} \)
中的元素都能写成 \( \frac{a}{c}, c \notin \mathfrak{p} \), 且
\[
  \frac{a}{c} = \frac{b}{d} \iff \text{存在} s \notin \mathfrak{p} \text{使得}
  s(ad - bc) = 0.
\]
一个不交于 \( S_{\mathfrak{p}} \) 的素理想当且仅当其包含 \( \mathfrak{p} \),
换句话说,
\[
  \operatorname{Spec}(A_\mathfrak{p}) \simeq \left\lbrace \mathfrak{q} \in
  \operatorname{Spec} A: \mathfrak{q} \subseteq \mathfrak{p} \right\rbrace.
\]
因此 \( A_{\mathfrak{p}} \) 是一个极大理想为 \( \mathfrak{p}^{e} \) 的局部环.

\begin{proposition}
  假设 \( \mathfrak{m} \) 是环 \( A \) 的极大理想, 即 \( \mathfrak{n} =
  \mathfrak{m} A_{\mathfrak{m}} \) 为 \( A_{\mathfrak{m}} \) 的极大理想.
  那么对所有自然数 \( n \), 映射
  \[
    a + \mathfrak{m}^n \mapsto a + \mathfrak{n}^n: A / \mathfrak{m}^n \to
    A_{\mathfrak{m}} / \mathfrak{n}^n
  \]
  是一个同构. 此外, 对所有 \( r \leq n \), 其诱导了同构
  \[
    \mathfrak{m}^r / \mathfrak{m}^n \to \mathfrak{n}^r / \mathfrak{n}^n.
  \]
\end{proposition}
\begin{proof}
  由下面交换图, 第二个是第一个论断的直接推论
% https://q.uiver.app/#q=WzAsMTAsWzAsMCwiMCJdLFswLDEsIjAiXSxbNCwwLCIwIl0sWzQsMSwiMCJdLFsxLDAsIlxcbWF0aGZyYWt7bX1eciAvIFxcbWF0aGZyYWt7bX1ebiJdLFsxLDEsIlxcbWF0aGZyYWt7bn1eciAvIFxcbWF0aGZyYWt7bn1ebiJdLFsyLDAsIkEgLyBcXG1hdGhmcmFre219Xm4iXSxbMywwLCJBIC8gXFxtYXRoZnJha3ttfV5yIl0sWzIsMSwiQV97XFxtYXRoZnJha3ttfX0gLyBcXG1hdGhmcmFre259Xm4iXSxbMywxLCJBX3tcXG1hdGhmcmFre219fSAvIFxcbWF0aGZyYWt7bn1eciJdLFsxLDVdLFswLDRdLFs0LDVdLFs0LDZdLFs1LDhdLFs2LDgsIlxcc2ltZXEiLDFdLFs2LDddLFs4LDldLFs3LDJdLFs5LDNdLFs3LDksIlxcc2ltZXEiLDFdXQ==
\[\begin{tikzcd}
	0 & {\mathfrak{m}^r / \mathfrak{m}^n} & {A / \mathfrak{m}^n} & {A / \mathfrak{m}^r} & 0 \\
	0 & {\mathfrak{n}^r / \mathfrak{n}^n} & {A_{\mathfrak{m}} / \mathfrak{n}^n} & {A_{\mathfrak{m}} / \mathfrak{n}^r} & 0
	\arrow[from=1-1, to=1-2]
	\arrow[from=1-2, to=1-3]
	\arrow[from=1-2, to=2-2]
	\arrow[from=1-3, to=1-4]
	\arrow["\simeq"{description}, from=1-3, to=2-3]
	\arrow[from=1-4, to=1-5]
	\arrow["\simeq"{description}, from=1-4, to=2-4]
	\arrow[from=2-1, to=2-2]
	\arrow[from=2-2, to=2-3]
	\arrow[from=2-3, to=2-4]
	\arrow[from=2-4, to=2-5]
\end{tikzcd}\]

映射是单射: 考虑自然映射 \( a \mapsto \frac{a}{1} \) 的扩张与收缩. 因为 \(
\mathfrak{n}^n = (\mathfrak{m}^n)^e \), 所以 \( A / \mathfrak{m}^n \to
A_{\mathfrak{m}} / \mathfrak{n}^n \) 的核为 \( (\mathfrak{m}^n)^{ec} /
\mathfrak{m}^n \). 如果 \( a \in (\mathfrak{m}^n)^{ec} \), 那么对某些 \( b \in
\mathfrak{m}^n \) 以及 \( s \in S\), 有 \( \frac{a}{1} = \frac{b}{s} \),
因此存在 \( t \in S_{\mathfrak{m}} \), 使得 \( tsa \in \mathfrak{m}^n \). 每个
\( A \) 中包含 \( \mathfrak{m}^n \) 的极大理想 包含 \(
\operatorname{rad}(\mathfrak{m}^n) = \mathfrak{m} \), 只能是 \( \mathfrak{m} \).
因此 \( A/ \mathfrak{m}^n \) 的极大理想为唯一且为 \( \mathfrak{m} \). \( ts
\notin \mathfrak{m} \), 它们是 \( A / \mathfrak{m}^n \) 中的单位, 只能 \( a \in
\mathfrak{m}^n \).

映射是满射: 假设 \( \frac{a}{s} \in A_{\mathfrak{m}}, a \in A, s \in S_{\mathfrak{m}}
\mathfrak{m}\). 由上段, \( \mathfrak{m} \) 是 \( A / \mathfrak{m}^n \)
唯一的极大理想, 从而没有极大理想同时包含 \( s \) 和 \( \mathfrak{m}^n \),
换句话说 \( (s) + \mathfrak{m}^n = A \). 因此存在 \( b \in A \) 和 \( q \in
\mathfrak{m}^n \) 使得 \( sb + q = 1 \). 因此
\[
  s(ba) = a(1 - q).
\]
因此在 \( A_{\mathfrak{m}} \) 上
\[
  \frac{ba}{1} = \frac{a}{s} - \frac{aq}{s}.
\]
\end{proof}

\begin{proposition}
  假设 \( A \) 是一个 noetherian 环, 那么
  \[
    \bigcap \left\lbrace \mathfrak{m}^n: \mathfrak{m} \text{极大}, n \in
    \mathbb{N} \right\rbrace = \left\lbrace 0 \right\rbrace.
  \]
\end{proposition}
\begin{proof}
  如果 \( 0 \neq  a \in A \) , 那么 \( \operatorname{ann}(a) \) 是 \( A \)
  的一个真理想, 其包含在某个极大理想 \( \mathfrak{m} \) 中. 因此在 \(
  A_{\mathfrak{m}} \) 中, \( \frac{a}{1} \neq 0 \).
  % \cref{theorem-Krull-intersection} 告诉我们, 存在 \( n \) 使得 \( \frac{a}{1}
  % \notin (\mathfrak{m} A_{\mathfrak{m}})^n \), 换句话说 \( a \notin
  % \mathfrak{m}^n \).
\end{proof}

\subsection{分式模}

\begin{proposition}
  函子 \( M \to S^{-1} M \) 是正合函子.
\end{proposition}
\begin{proof}
  假设给定了正合列
  \[
    M' \xrightarrow{\alpha} M \xrightarrow{\beta} M''
  \]
  如果 \( \beta \circ \alpha = 0 \), 那么 \( 0 = S^{-1}(\beta \circ \alpha) =
  S^{-1} \beta \circ S^{-1} \alpha \), 因此 \( \operatorname{Im} (S^{-1} \alpha)
  \subseteq \operatorname{Ker} (S^{-1}\beta) \). 反过来, 如果给定了 \(
  \frac{m}{s} \in \operatorname{Ker}(S^{-1}\beta) \), 其中 \( m \in M, s \in S
  \). 那么存在 \( t \in S \) 使得 \( t(\beta(m)) = \beta(tm) = 0 \).
  因此由原列的正合性, 存在 \( m' \in M' \) 使得 \( tm = \alpha(m') \). 现在
  \[
    \frac{m}{s} = \frac{tm}{ts} = \frac{\alpha(m')}{ts} \in
    \operatorname{Im}(S^{-1}\alpha).
  \]
\end{proof}

\begin{proposition}
  假设 \( M \) 是一个有限生成 \( A \)-模. 如果 \( S^{-1} M = 0 \), 那么存在 \( h
  \in S \) 使得 \( M_h = 0 \).
\end{proposition}

\begin{proposition}
  假设 \( M \) 是一个有限生成 \( A \)-模. 那么典范映射
  \[
    M \to \prod \left\lbrace M_{\mathfrak{m}}: \mathfrak{m} \text{是} A
    \text{的一个极大理想} \right\rbrace
  \]
  是一个单射.
\end{proposition}
\begin{proof}
  如果 \( 0 \neq  x \) 在此映射下像为零, 那么存在极大理想 \( \mathfrak{m} \)
  包含 \( \operatorname{ann}(x) \). \( x \) 在 \( M \to M_{\mathfrak{m}} \)
  下像非零, 与假设矛盾, 从而 \( x = 0 \).
\end{proof}

\begin{corollary}
  \label{corollary-module-zero-iff-condition}
  假设 \( M \) 是一个 \( A \)-模. 如果 \( M_{\mathfrak{m}} = 0 \) 对所有 \( A \)
  的极大理想 \( \mathfrak{m} \) 成立, 那么 \( M = 0 \).
\end{corollary}

\begin{proposition}
  列(或复形链)
  \[
    M' \xrightarrow{\alpha} M \xrightarrow{\beta} M''
  \]
  正合当且仅当对 \( A \) 的所有极大理想, 列
  \[
    M'_{\mathfrak{m}} \xrightarrow{\alpha_{\mathfrak{m}}} M_{\mathfrak{m}}
    \xrightarrow{\beta_{\mathfrak{m}}} M''_{\mathfrak{m}}
  \]
  正合.
\end{proposition}
\begin{proof}
%TODO: 修正此证明
  \( \implies \) 是我们已经证明的. 反过来, 首先注意到 \(
  (\operatorname{ker}\beta)_{\mathfrak{m}} \subseteq
  \operatorname{ker}\beta_{\mathfrak{m}} \), 反过来, 如果 \( \frac{m}{s} \in
  \operatorname{ker} \beta_{\mathfrak{m}} \), 那么存在 \( s' \notin \mathfrak{m}
  \) 使得 \( s'\beta(m) = 0 \) 也就是 \( s'm \in \operatorname{ker}\beta \).
  因此 \( \frac{m}{s} = \frac{s'm}{s's} \in M_{\mathfrak{m}} \), 换句话说 \(
  (\operatorname{ker}\beta)_{\mathfrak{m}} =
  \operatorname{ker}\beta_{\mathfrak{m}} \). 同理 \(
  (\operatorname{im}\alpha)_{\mathfrak{m}} = \operatorname{im}
  \alpha_{\mathfrak{m}} \).

  于是 \( \left(\frac{\operatorname{ker} \beta +
    \operatorname{im}\alpha}{\operatorname{im} \alpha}\right)_{\mathfrak{m}} =
    \frac{\operatorname{ker}\beta_{\mathfrak{m}} +
      \operatorname{im}\alpha_{\mathfrak{m}}}{\operatorname{im}
    \alpha_{\mathfrak{m}}}  = 0 \) 对任意 \( \mathfrak{m} \) 成立. \cref{corollary-module-zero-iff-condition}
    告诉我们, \( \frac{\operatorname{ker}\beta +
    \operatorname{im}\alpha}{\operatorname{im}\alpha} = 0 \) 成立, 换句话说 \(
    \operatorname{ker} \beta \subseteq \operatorname{im} \alpha \). 类似地, 考虑
    \( \frac{\operatorname{ker} \beta + \operatorname{im}
    \alpha}{\operatorname{ker}\beta} \) 就能得到 \( \operatorname{im} \alpha
    \subseteq \operatorname{ker} \beta \).
\end{proof}

\begin{corollary}
  一个 \( A \)-模态射 \( M \to N \) 单(resp. 满, 零) 当且仅当 \(
  M_{\mathfrak{m}} \to N_{\mathfrak{m}} \) 单(resp. 满, 零).
\end{corollary}
\begin{proof}
  考虑 \( 0 \to M \to N \)(resp. \( M \to N \to 0, M
  \xrightarrow{\operatorname{id}} M \to N \)).
\end{proof}

\begin{proposition}
  假设 \( \mathfrak{N} \) 是 \( A \) 的幂零根, \( S \) 为 \( A \)
  的一个乘性子集, 那么 \( S^{-1}\mathfrak{N} \) 是 \( S^{-1}A \) 的幂零根.
\end{proposition}
\begin{proof}
  如果 \( (\frac{a}{s})^n = 0 \) 对某个 \( n \) 成立, 其中 \( a \in A, s \in S \).
  那么存在 \( t \in S \) 使得 \( ta^n = 0 \), 因此 \( \frac{a}{s} =
  \frac{ta}{ts} \in S^{-1} \mathfrak{N} \).
\end{proof}

\begin{corollary}
  环 \( A \) 约化当且仅当 \( A_{\mathfrak{m}} \) 均约化.
\end{corollary}


