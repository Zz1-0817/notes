\chapter{基本知识}

\section{环与理想}

一个\emph{环}指的是一个配备结合乘法的加性交换群.
在不加说明的情况下, 本笔记中提到的环一般要求其乘法交换且有乘法单位元.

\subsection{根式理想}

假设给定一个环 \( A \).
称 \( f \in A \) \emph{幂零}, 如果存在 \( r \geq 1 \) 使得 \( f^r = 0 \).
称 \( A \) \emph{约化的}, 如果 \( A \) 有没有非零的幂零元.
给定 \( A \) 的任意理想 \( \mathfrak{a} \), 其对应了 \( A \) 的一个理想
\begin{equation}
  \left\lbrace f \in A : f^r \in \mathfrak{a}, \text{对某个} r \geq 1
  \text{成立} \right\rbrace.
  \label{equation-radical-ideal}
\end{equation}
称由\eqref{equation-radical-ideal}定义的理想为 \( \mathfrak{a} \) 的根,
将其记作 \( \operatorname{rad}(\mathfrak{a}) \).
特别地, \( (0) \) 的根称为 \( A \) 的\emph{幂零根}, 记作\(
\operatorname{nil}(A) \).
此外, 如果 \( \operatorname{rad}(\mathfrak{a}) = \mathfrak{a} \), 那么称 \(
\mathfrak{a} \) 是 \emph{根式的}.

\begin{proposition}
  假设 \( \mathfrak{a} \) 是环 \( A \) 的一个理想.
  \begin{enumerate}
    \item \( \operatorname{rad}(A) \) 亦是 \( A \) 的一个理想.
    \item \( \operatorname{rad}(\operatorname{rad}(\mathfrak{a})) =
      \operatorname{rad}(\mathfrak{a}) \).
  \end{enumerate}
\end{proposition}

\subsection{素理想}

\paragraph{乘性子集}
环 \( A \) 的 \emph{乘性子集} 是一个子集 \( S \) 满足如下性质
\[
  1 \in S,\quad a, b \in S \implies ab \in S.
\]

\begin{proposition}
  \label{proposition-prime-not-intersect-multiplicative-set}
  假设 \( S \) 是一个环 \( A \) 的子集, \( \mathfrak{a} \) 是 \( A \) 的不交于
  \( S \) 一个理想.  \( A \) 中包含 \( \mathfrak{a} \) 但不交于 \( S \)
  的理想集族 \( \Sigma \) 有一个极大元, 如果 \( S \) 是乘性子集,
  那么每个这样的极大元都是素的.
\end{proposition}
\begin{proof}
  \( \Sigma \) 的每个上升链 \( \mathfrak{a}_1 \subseteq \mathfrak{a}_2 \subseteq
  \cdots \) 都有上界 \( \sum \mathfrak{a}_i \), 由
  \href{https://en.wikipedia.org/wiki/Zorn%27s_lemma}{Zorn 引理}, \( \Sigma \)
  有一个极大元 \( \mathfrak{c} \).
  现在说明 \( \mathfrak{c} \) 是一个素理想.
  假设不然, 如果 \( b b' \in \mathfrak{c} \), 且 \( b, b' \notin \mathfrak{c}
  \), 那么 \( \mathfrak{c} + (b) \) 和 \( \mathfrak{c} + (b') \) 真包含 \(
  \mathfrak{c} \), 因此它们不落于 \( \Sigma \) 中.
  因此 \( \mathfrak{c} + (b) \) 和 \( \mathfrak{c} + (b') \) 各包含 \( S \)
  中的一个元素, 我们分别记其为
  \[
    f = c + ab,\quad c \in \mathfrak{c},\quad a' \in A \text{ 以及 } f' = c' +
    a'b,\quad c' \in \mathfrak{c},\quad a' \in A
  \]
  因此
  \[
    f f' = c c' + abc' + a'b'c + aa'bb' \in \mathfrak{c},
  \]
  这与 \( f f' \in S \) 矛盾.
\end{proof}
\begin{remark}
  \label{remark-maximal-element-prime}
  这种证明理想集中极大元素的方法后面还会遇到若干次.
\end{remark}

\begin{proposition}
  \label{proposition-radical-as-prime-intersection}
  给定 \( A \) 的一个理想 \( \mathfrak{a} \), 那么所有包含 \( \mathfrak{a} \)
  的素理想的交即\( \mathfrak{a} \) 的根式理想, 也就是
  \[
    \operatorname{rad}(\mathfrak{a}) = \bigcap_{\mathfrak{p}\supseteq a}
    \mathfrak{p}.
  \]
  特别地, \( \operatorname{rad}(0) = \bigcap \mathfrak{p} \).
\end{proposition}
\begin{proof}
  \( \operatorname{rad}(\mathfrak{a}) \supset \bigcap_{\mathfrak{p}\supseteq a}
  \mathfrak{p} \).
  如果 \( x \notin \operatorname{rad} \mathfrak{a} \), 也就是乘性子集 \(
  \left\lbrace 1, x, x^2, \ldots \right\rbrace \not\subset \mathfrak{a} \),
  那么由\cref{proposition-prime-not-intersect-multiplicative-set}, 总存在素理想
  \(  \mathfrak{p} \), 使得 \( x \notin \mathfrak{p} \), 故 \( x \notin \bigcap
  \mathfrak{p} \).
\end{proof}

\paragraph{素避引理}

\begin{proposition}[素避]
  假设 \( \mathfrak{p}_1, \ldots, \mathfrak{p}_r, r \geq 1 \) 为 \( A \) 的理想,
  其中 \( \mathfrak{p}_2, \ldots, \mathfrak{p}_r \) 为素理想. 如果理想 \(
  \mathfrak{a} \) 不包含在任何一个 \( \mathfrak{p}_i \), 那么 \( \mathfrak{a} \)
  不包含在 \( \mathfrak{p}_i \) 的并中.
\end{proposition}
\begin{proof}
  对 \( r \) 应用归纳假设.
  \( r = 1 \) 时结论显然.
  假设 \( r > 1 \) 且 \( \mathfrak{a} \subseteq \bigcup_{1 \leq j \leq r}
  \mathfrak{p}_j \), 但对任意 \( i \), \( \mathfrak{a} \not\subseteq \bigcup_{j
  \neq i}\mathfrak{p}_j \).
  由假设知道,
  存在 \( a_i \in \mathfrak{a} \setminus \bigcup_{j \neq i} \mathfrak{p}_j \),
  那么 \( a_i \in \mathfrak{p}_i \). 考察
  \[
    a = a_1 \cdots a_{r - 1} + a_r \in \mathfrak{a}.
  \]
  因为 \( \mathfrak{p}_r \) 素, \( a_1 \cdots a_{r - 1} \notin \mathfrak{p}_r
  \). 而 \( a_{r} \in \mathfrak{p}_r \), 故 \( a \notin \mathfrak{p}_r \).
  对 \( i \leq r - 1 \), 则 \( a_1 \cdots a_{r - 1} \in \mathfrak{p}_{i} \), 但
  \( a_r \in \mathfrak{p}_i \), 从而 \( a \notin \mathfrak{p}_i \). 这与 \( a
  \subseteq \cup_{1 \leq j \leq r} \mathfrak{p}_j \) 矛盾.
  因此 \( \mathfrak{a} \subseteq \cup_{j \neq i} \mathfrak{p}_j \) 对某个 \( i
  \) 成立, 应用归纳假设即可.
\end{proof}

\subsection{极大理想与 Jacobson 根}

环 \( A \) 的 \emph{Jacobson 根} 是 \( A \) 所有极大理想的交
\[
  \mathfrak{J} = \bigcap \left\lbrace \mathfrak{m}: \mathfrak{m} \text{ 为 } A
  \text{的极大理想} \right\rbrace.
\]
如果 \( A \) 只有一个极大理想, 那么环 \( A \) 称为是 \emph{局部} 的

\begin{proposition}
  \label{proposition-Jacobson-root-iff-condition}
  假设 \( A \) 的 Jacobson 根为 \( \mathfrak{J} \), 那么 \( c \in \mathfrak{J}
  \) 当且仅当对所有 \( a \in A \), \( 1 - ac \) 是单位.
\end{proposition}
\begin{proof}
  给定 \( c \in A \).
  \( c \notin \mathfrak{J} \iff \) 存在 \( A \) 的极大理想 \( \mathfrak{m} \)
  使得 \( c \notin \mathfrak{m} \iff \) 存在极大理想 \( \mathfrak{m} \) 以及 \(
  a \in A, m \in \mathfrak{m} \) 使得 \( ac = 1 + m \), 也就是 \( 1 - ac \)
  并非单位.
\end{proof}

\paragraph{Nakayama引理}

\begin{theorem}[Nakayama引理]
  \label{theorem-Nakayama-lemma}
  假设 \( A \) 是一个环, \( \mathfrak{a} \) 是 \( A \) 的理想, \( M \) 是一个 \(
  A\)-模. 如果 \( \mathfrak{a} \) 包含于所有 \( A \) 的极大理想中, 且 \( M \)
  有限生成, 那么
  \begin{enumerate}
    \item 如果 \( M = \mathfrak{a} M \), 那么 \( M = 0 \).
    \item 如果 \( N \) 是 \( M \) 的子模, 使得 \( M = N + \mathfrak{a} M \),
      那么 \( M = N \).
  \end{enumerate}
\end{theorem}
\begin{proof}
  (1) 如果 \( M \neq 0 \), 由有限生成性可以取一族 \( M \) 的极小生成元 \(
  \left\lbrace e_1, \ldots, e_n \right\rbrace, n \geq 1 \).
  记
  \[
    e_1 = a_1 e_1 + \cdots + a_n e_n,\quad a_i \in \mathfrak{a}.
  \]
  那么
  \[
    (1 - a_1) e_1 = a_2 e_2 + \cdots + a_n e_n.
  \]
  \( \mathfrak{a} \) 包含于 \( A \) 的 Jacobson 根中, 从而 \( 1 - a_1 \) 可逆.
  因此 \( e_2, \ldots, e_n \) 生成 \( M_2 \), 与选取矛盾.
  立刻得到结果.
\end{proof}

\section{环态射, 理想扩张与收缩}

\subsection{中国剩余定理}

\( A \) 的两个理想 \( \mathfrak{a} \) 和 \( \mathfrak{b} \) 是 \emph{互素} 的,
如果 \( \mathfrak{a} + \mathfrak{b} = A \).
\begin{proposition}
  如果 \( \mathfrak{a}_1,\ldots, \mathfrak{a}_n \) 互素, 那么 \( \bigcap
  \mathfrak{a}_i = \prod \mathfrak{a}_i \).
\end{proposition}
\begin{proof}
  \( \bigcap \mathfrak{a}_i \subset \prod \mathfrak{a}_i \).
  这里只证明 \( n = 2 \) 的情况, 一般情况由归纳法保证.
  由假设可以找到 \( a_1 \in \mathfrak{a}_1 \) 以及 \( a_2 \in \mathfrak{a}_2 \)
  使得 \( a_1 + a_2 = 1 \). 如果 \( a \in \mathfrak{a}_1 \cap \mathfrak{a}_2 \),
  那么 \( a = (a_1 + a_2)a \in \mathfrak{a}_1 \mathfrak{a}_2 \).
\end{proof}

\begin{theorem}[中国剩余定理]
  假设 \( \mathfrak{a}_1, \ldots, \mathfrak{a}_n \) 是环 \( A \) 的理想. 并且在
  \( i \neq j \) 时,  \( \mathfrak{a}_i \) 和 \( \mathfrak{a}_j \) 互素,
  那么映射
  \[
    A \to A / \mathfrak{a}_1 \times \cdots \times A / \mathfrak{a}_n,\quad a
    \mapsto (\ldots, a + \mathfrak{a}_i, \ldots)
  \]
  是满射, 且其核为 \( \prod \mathfrak{a}_i = \bigcap \mathfrak{a}_i \).
\end{theorem}
\begin{proof}
  对 \( n \) 做归纳假设.
  \( n = 2 \) 时.
  因为 \( \mathfrak{a}_1 + \mathfrak{a}_2 = A \), 存在 \( a_i \in \mathfrak{a}_i
  \) 使得 \( a_1 + a_2 = 1 \).
  对任意 \( x_1, x_2 \in A \), \( a_1 x_2 + a_2 x_1 \) 映射到 \( (x_1
  \mod{\mathfrak{a}_1}, x_2 \mod{\mathfrak{a}_2}) \), 所以此映射是满射.
  其核显然是 \( \mathfrak{a}_1 \cap \mathfrak{a}_2 \).

  现假设 \( n > 2 \), 对于 \( i \geq 2 \), 存在 \( a_i \in \mathfrak{a}_1 \)
  以及 \( b_i \in \mathfrak{a}_i \) 使得
  \[
    a_i + b_i = 1.
  \]
  积 \( \prod_{i \geq 2}(a_i + b_i) = 1 \) 落于 \( \mathfrak{a}_1 +
  \mathfrak{a}_2 \cdots \mathfrak{a}_n \) 中, 换句话说
  \[
    \mathfrak{a}_1 + \mathfrak{a}_2\cdots \mathfrak{a}_n = A.
  \]
  因此,
  \begin{align*}
    A / \mathfrak{a}_1 \cdots \mathfrak{a}_n &= A / \mathfrak{a}_1 \cdot
    (\mathfrak{a}_2 \cdots \mathfrak{a}_n)\\ &\simeq A / \mathfrak{a}_1 \times A
    / \mathfrak{a}_2 \cdots \mathfrak{a}_n\\ &\simeq A/ \mathfrak{a}_1 \times A
    / \mathfrak{a}_2 \times \cdots \times A / \mathfrak{a}_n,
  \end{align*}
  其中, 最后一个同构由归纳法保证.
\end{proof}


\subsection{理想的扩张与收缩}

假设 \( \varphi: A \to B \) 是一个环同态.
\begin{itemize}
  \item 对 \( B \) 的理想 \( \mathfrak{b} \), \( \varphi^{-1}(\mathfrak{b}) \)
    是\( A \) 的理想, 称为 \( \mathfrak{b} \) 在 \( A \) 上的\emph{收缩},
    记作 \( \mathfrak{b}^{c} \).
  \item 对于 \( A \) 的理想 \( \mathfrak{a} \), \( B \) 中由 \(
    \varphi(\mathfrak{a}) \) 生成的理想, 称为 \( \mathfrak{a} \) 在 \( B \) 中的
    \emph{扩张}, 记作 \( \mathfrak{a}^{e} \).
\end{itemize}
特别地, 如果 \( \varphi \) 满, 那么 \( \varphi(\mathfrak{a}) = \mathfrak{a}^e \);
如果 \( A \) 是 \( B \) 的子环, 那么 \( \mathfrak{b}^{c} = \mathfrak{b} \cap A
\).

\begin{proposition}
  假设\( \mathfrak{a}, \mathfrak{a}' \) 为 \( A \) 的理想, \( \mathfrak{b},
  \mathfrak{b}' \) 为 \( B \) 的理想
  \begin{enumerate}
    \item 我们有下面关系
      \[
        (\mathfrak{a} + \mathfrak{a}')^{e} = \mathfrak{a}^e +
        \mathfrak{a}'^e,\quad (\mathfrak{a} \mathfrak{a}')^e = \mathfrak{a}^e
        \mathfrak{a}'^{e},\quad (\mathfrak{b} \cap \mathfrak{b}')^c =
        \mathfrak{b}^c \cap \mathfrak{b}'^c,\quad
        \operatorname{rad}(\mathfrak{b})^c = \operatorname{rad}(\mathfrak{b}^c).
      \]
    \item \( \mathfrak{a} \subseteq \mathfrak{a}^{ec} \) 且 \( \mathfrak{b}^{ce}
      \subseteq \mathfrak{b} \). 特别地, \( \mathfrak{a}^{e} \subseteq
      \mathfrak{a}^{ece} \) 以及 \( \mathfrak{b}^{cec} = \mathfrak{b}^c \),
      从而我们有双射
      \[
        \left\lbrace \mathfrak{b}^c \subseteq A: \mathfrak{b} \text{为}B
        \text{的理想} \right\rbrace \xleftrightarrow[\mathfrak{b}^c \mapsfrom
        \mathfrak{b}]{\mathfrak{a} \mapsto \mathfrak{a}^e} \left\lbrace
        \mathfrak{a}^e \subseteq B: \mathfrak{a} \text{是} A \text{中的理想}
        \right\rbrace
      \]
  \end{enumerate}
\end{proposition}

\begin{proposition}
  \label{proposition-prime-as-contraction-iff-homomorphism-condition}
  \( A \) 的素理想 \( \mathfrak{p} \)
  是 \( B \) 某个素理想的收缩当且仅当 \( \mathfrak{p} = \mathfrak{p}^{ec} \).
\end{proposition}
\begin{proof}
  \( \impliedby \) 假设 \( S = A \setminus \mathfrak{p} \), \( s \in S \).
  如果 \( \varphi(s) \in \mathfrak{p}^e \), 那么 \( s \in \mathfrak{p}^{ec} =
  \mathfrak{p} \), 与 \( s \) 定义矛盾, 因此 \( \varphi(S) \cap \mathfrak{p}^e =
  \varnothing \).
  由\cref{proposition-prime-not-intersect-multiplicative-set}, 存在 \(
  B \) 的一个素理想 \( \mathfrak{q} \) 包含 \( \mathfrak{p}^e \) 且不交于 \(
  \varphi(S) \).
  结合 \( S \) 的构造 \( \mathfrak{q}^e = \mathfrak{p} \).
\end{proof}

\section{多项式环}

\begin{proposition}[带余除法]
  \label{proposition-monic-polynomial-division-algorithm}
  假设 \( f, g \in A[X] \), 其中 \( g \) 是首一多项式, 那么存在 \( q, r \in A[X]
  \) 使得 \( f = gq + r \), 且要么 \( \deg r < \deg g \) 要么 \( \deg r = 0 \).
  特别地, 如果 \( \alpha \in A \) 是 \( f \in A[X] \) 的一个根, 那么 \( X -
  \alpha \mid f \).
\end{proposition}

\begin{proposition}
  \label{proposition-splitting-polynomial-ring}
  假设 \( A \) 是一个环, \( B \) 是一个 \( A \)-代数, \( f \in B[T] \)
  是一个首一多项式, 那么
  \begin{enumerate}
    \item 存在一个包含 \( B \) 的环 \( B' \) 使得 \( f \) 在 \( B'[T] \)
      上可以分解为一次多项式的积.
    \item \( f \) 首一因子的系数是 \( f \) 一些根的代数组合.
  \end{enumerate}
\end{proposition}
\begin{proof}
  (1) 对 \( n = \deg f \) 做归纳, 只需对 \( n > 1 \) 的情况证明.
  设 \( B_1 = B[T] / (f) \).
  因为 \( f \) 首一, 对任意 \( g \in B[T] \), \( \deg fg = \deg f + \deg g \),
  也就是 \( (f) \cap B = 0 \).
  因此, 同态 \( B \to B_1 \) 单, 也就是说 \( B[X] \to B_1[X] \) 单, \(
  f(X) \) 可视为多项式环 \( B_1[X] \) 中有一个根 \( b = X + (f) \) 的元素.
  因此由\cref{proposition-monic-polynomial-division-algorithm} \( f \) 可以写为
  \( f = (X - b) f_1 \), 其中 \( f_1 \in b_1[X] \) 首一且 \( \deg f_1 < \deg f
  \).
  对 \( f_1 \) 应用归纳假设, 使得 \( f_1 \) 分裂的 \( B_1 \)-代数 \( B' \) 存在,
  其作为 \( A \)-代数使得 \( f \) 分裂.
\end{proof}

\section{环的素谱}

假设 \( A \) 是一个环, \( V \) 为 \( A \) 的所有素理想.
对 \( A \) 的一个理想 \( \mathfrak{a} \), 记
\[
  V(\mathfrak{a}) := \left\lbrace \mathfrak{p} \in V: \mathfrak{p} \supset
  \mathfrak{a} \right\rbrace
\]

\begin{proposition}
  \( V(\mathfrak{a}) \) 具有如下性质
  \begin{enumerate}
    \item \( \mathfrak{a} \subset \mathfrak{b} \implies V(\mathfrak{a}) \supset
      V(\mathfrak{b}) \).
    \item \( V(0) = V \), \( V(A) = \varnothing \).
    \item \( V(\mathfrak{ab}) = V(\mathfrak{a} \cap \mathfrak{b}) =
      V(\mathfrak{a}) \cap V(\mathfrak{b}) \).
    \item \( V(\sum_{i \in I} \mathfrak{a}_i) = \bigcap_{i \in I}
      V(\mathfrak{a}_i) \).
  \end{enumerate}
\end{proposition}

赋予 \( V \) 一个拓扑结构, 使得 \( \left\lbrace V(\mathfrak{a}): \mathfrak{a}
\text{为} A \text{的所有理想} \right\rbrace \) 为 \( V \) 的所有闭集,
称此拓扑结构为 \( A \) 的\emph{Zariski}拓扑, 并将赋予此拓扑的 \( V \) 记作 \(
\operatorname{Spec} A \).
此外, 我们常用 \( V(f) \) 来记 \( V((f)) \).

\paragraph{闭集对应理想}
给定一个子集 \( Y \subset \operatorname{Spec} A \), 记
\[
  I(Y) := \bigcap{\mathfrak{p} \in Y} \mathfrak{p}
\]
\begin{proposition}
  假设 \( A \) 是一个环, \( \mathfrak{a} \) 是一个理想, \( Y \) 是 \(
  \operatorname{Spec} A \) 的一个子集, 那么
  \begin{enumerate}
    \item \( \operatorname{rad}(I(Y)) = I(Y) \).
    \item \( IV(\mathfrak{a}) = \operatorname{rad}(\mathfrak{a}), VI(Y) =
      \overline{Y} \).
    \item \( V, I \) 给出了双射
      \[
        \begin{split}
          \left\lbrace A \text{的根式理想} \right\rbrace &\leftrightarrow
          \left\lbrace \operatorname{Spec} A \text{的闭子集} \right\rbrace\\
          \mathfrak{a} &\mapsto V(\mathfrak{a})\\
          I(Y) &\mapsfrom Y
        \end{split}
      \]
  \end{enumerate}
\end{proposition}

\paragraph{主开集} 记
\[
  D(f) := \operatorname{Spec} A \setminus V(f)
\]
那么 \( D(f) \) 是 \( \operatorname{Spec}A \) 的一个开集, 其中元素由 \( A \)
中不含 \( f \) 的素元素组成, 称 \( \operatorname{Spec} A \)
中这样的开集为\emph{主开集}.

\begin{proposition}
  \( D(f) \) 具有如下性质
  \begin{enumerate}
    \item 如果 \( u \in \operatorname{nil} A \), 那么 \( D(u) = 0 \).
      如果 \( u \) 是 \( A \) 的单位, 那么 \( D(u) = \operatorname{Spec} A \).
      特别地, \( D(1) = \operatorname{Spec} A, D(0) = \varnothing \).
    \item \( D(f) \cap D(g) = D(fg) \).
  \end{enumerate}
\end{proposition}

\begin{lemma}
  假设 \( f_i \) 为 \( A \) 的一组元素, \( g \in A \), 那么 \( D(g) \subset
  \bigcup_i D(f_i) \iff g \in \operatorname{rad}(\sum (f_i)) \)
\end{lemma}
\begin{proof}
  \( D(g) \subset \bigcup_i D(f_i) \iff V(g) \supset \bigcap_i V(f_i) =  V(\sum
  (f_i)) \iff \operatorname{rad}(g) \subset \operatorname{rad}(\sum (f_i) ) \).
\end{proof}

\begin{proposition}
  假设 \( A \) 是一个环, 那么
  \begin{enumerate}
    \item \( \operatorname{Spec} A \) 的主开集组成 \( \operatorname{Spec} A \)
      的一个开集基.
    \item \( D(f) \) 是紧的, 特别地, \( \operatorname{Spec} A \) 是紧的.
  \end{enumerate}
\end{proposition}

\paragraph{不可约集}

\begin{proposition}
  \( \operatorname{Spec} A \) 的子集 \( Y \) 不可约当且仅当 \( \mathfrak{p} =
  I(Y) \) 是一个素理想.
  此时 \( \left\lbrace \mathfrak{p} \right\rbrace \) 在 \( \overline{Y} \)
  中稠密.
\end{proposition}
\begin{proof}
  \( \implies \).
  假设不然, 存在 \( a, b \notin \mathfrak{p} \) 但 \( ab \in \mathfrak{p} \).
  那么 \( Y \subset V(ab) = V(a) \cup V(b) \).
  由 \( Y \) 的不可约性, 只能 \( Y \subset V(a) \) 或 \( Y \subset V(b) \),
  假设前者成立.
  那么 \( \mathfrak{p} \supset IV(a) = \operatorname{rad} (a) \ni a \),
  与假设矛盾.

  \( \impliedby \) 是因为闭包不可约的集合自身不可约.
\end{proof}

\begin{corollary}
  \( I, V \) 给出了双射
  \[
    \begin{split}
      \operatorname{Spec} A &\leftrightarrow \left\lbrace \operatorname{Spec} A
      \text{的不可约闭子集}\right\rbrace\\
        \mathfrak{p} &\mapsto V(\mathfrak{p})\\
        I(Y) &\mapsfrom Y
    \end{split}.
  \]
  特别地, 极小素理想对应到一个极大不可约分支.
\end{corollary}

假设 \( X \) 为任意拓扑空间
\begin{enumerate}
  \item 称 \( x \in X \) \emph{闭}, 如果 \( \left\lbrace x \right\rbrace \)
    是闭集.
  \item 称 \( \eta \in X \) 为\emph{泛点}, 如果 \( \overline{\left\lbrace \eta
    \right\rbrace} = X \).
  \item 设 \( x, x' \in X \), 称 \( x \) 为 \( x' \) 的\emph{泛化}, 或 \( x' \)
    为 \( x \) 的\emph{特化}, 如果 \( x' \in \overline{\left\lbrace x
    \right\rbrace} \).
  \item 称 \( x \in X \) 为一个\emph{极大点}, 如果闭包 \( \overline{\left\lbrace
    x \right\rbrace} \) 是 \( X \) 的一个不可约分支.
\end{enumerate}

\paragraph{函子 \( A \to \operatorname{Spec} A \)}

假设 \( \varphi: A \to B \) 是一个环同态, 那么记
\[
  {}^a \varphi: \operatorname{Spec} B \to \operatorname{Spec} A,\quad
  \mathfrak{q} \mapsto \mathfrak{q}^c.
\]

\begin{proposition}
  \( {}^a \varphi \) 具有下面性质:
  \begin{enumerate}
    \item 对 \( A \) 的理想 \( \mathfrak{a} \), \( {}^a
      \varphi^{-1}(V(\mathfrak{a})) = V(\varphi(\mathfrak{a}))
      \).
      特别地, \( {}^a\varphi \) 连续.
    \item 对 \( f \in A \), \( {}^a \varphi^{-1}(D(f)) = D(\varphi(f)) \).
    \item 对 \( B \) 素理想 \( \mathfrak{b} \), \(
      V(\varphi^{-1}(\mathfrak{b})) = \overline{{}^a \varphi(V(\mathfrak{b}))}
      \).
  \end{enumerate}
\end{proposition}
\begin{proof}
  (1) \( {}^a \varphi^{-1}(V(\mathfrak{a})) \subset V(\varphi(\mathfrak{a})) \).
  假设 \( \mathfrak{q} \in {}^a \varphi^{-1}(V(\mathfrak{a})) \), 那么 \(
   \varphi^{-1}(\mathfrak{q}) = {}^a\varphi(\mathfrak{q}) \supset \mathfrak{a}
  \).
  因此 \( \mathfrak{q} \supset \varphi \varphi^{-1}(\mathfrak{q}) \supset
  \varphi(\mathfrak{a}) \).

  \( {}^a \varphi^{-1}(V(\mathfrak{a})) \supset V(\varphi(\mathfrak{a})) \).
  假设 \( \mathfrak{q} \in V(\varphi(\mathfrak{a})) \), 那么 \( \mathfrak{q}
  \supset \varphi(\mathfrak{a}) \).
  因此 \( {}^a \varphi(\mathfrak{q}) = \varphi^{-1}(\mathfrak{q}) \supset
  \varphi^{-1} \varphi(\mathfrak{a}) \supset \mathfrak{a} \), 换句话说 \(
  \mathfrak{q} \in {}^a \varphi^{-1}(V(\mathfrak{a})) \).

  (3)
  \[
    \begin{split}
      \overline{{}^a \varphi(V(\mathfrak{b}))} &= VI({}^a
      \varphi(V(\mathfrak{b}))) = V \left( \bigcap_{\mathfrak{p} \in {}^a
      \varphi(V(\mathfrak{b})} \mathfrak{p}\right)\\
      &= V \left( \bigcap_{\mathfrak{q} \in \varphi(V(\mathfrak{b})}
      \varphi^{-1}\mathfrak{q}\right) = V \left(
      \varphi^{-1}\bigcap_{\mathfrak{q} \in \varphi(V(\mathfrak{b})} \mathfrak{q}
      \right)\\
      &= V\left(\varphi^{-1} \operatorname{rad} (\mathfrak{b})\right) =
      V\left( \operatorname{rad} \varphi^{-1}(\mathfrak{b})\right)\\
      &= V(\varphi^{-1}(\mathfrak{b})).
    \end{split}
  \]
\end{proof}

\begin{proposition}
  假设 \( \varphi: A \to B \) 是一个核为 \( \mathfrak{a} \) 的满同态, 那么 \(
  {}^a \varphi \) 是 \( \operatorname{Spec} B \) 到 \( V(\mathfrak{a}) \)
  的同胚.
\end{proposition}

\section{局部化}

\subsection{分式环}

假设 \( S \) 是环 \( A \) 的一个乘性子集. 定义 \( A \times S \) 的一个关系
\[
  (a, s) \sim (b, t) \iff \text{存在} u \in S, \text{使得} u(at - bs) = 0.
\]
可以验证, 这是一个等价关系, 将其中代表元 \( (a, s) \) 记作 \(
\frac{a}{s} \), 那么我们得到了一个环
\[
  S^{-1}A = \left\lbrace \frac{a}{s}: a \in A, s \in S \right\rbrace
\]
以及一个环同态 \( i_S: A \to S^{-1} A, a \mapsto \frac{a}{1} \), 其核为
\[
  \left\lbrace a \in A: \text{存在} s \in S, \text{使得} sa = 0 \right\rbrace.
\]
\begin{remark}
  \label{remark-fraction-ring}
  \begin{enumerate}
    \item \( S \) 没有零除子当且仅当 \( i_S \) 就是单射
    \item 如果 \( 0 \in S \), 那么 \( S^{-1} A = 0 \).
  \end{enumerate}
\end{remark}
\begin{proposition}
  \label{proposition-universal-property-localization}
  局部化 \( (S^{-1}A, i_S) \) 具有下述泛性质: 每个 \( S \) 中的元素映到 \(
  S^{-1} A \) 中的单位, 并且每个将 \( S \) 映到单位的环同态 \( \alpha: A \to B
  \) 能通过 \( i_S \) 唯一分解.
  % https://q.uiver.app/#q=WzAsMyxbMCwwLCJBIl0sWzEsMCwiU157LTF9QSJdLFsxLDEsIkIiXSxbMCwyLCJcXGFscGhhIiwyXSxbMSwyLCJcXGV4aXN0cyAhIiwwLHsic3R5bGUiOnsiYm9keSI6eyJuYW1lIjoiZGFzaGVkIn19fV0sWzAsMSwiaV9TIl1d
\[\begin{tikzcd}
	A & {S^{-1}A} \\
	& B
	\arrow["{i_S}", from=1-1, to=1-2]
	\arrow["\alpha"', from=1-1, to=2-2]
	\arrow["{\exists !}", dashed, from=1-2, to=2-2]
\end{tikzcd}\]
特别地, 局部化是同构意义下唯一的.
\end{proposition}

设 \( h \in A \), 那么 \( S_h = \left\lbrace 1, h, h^2, \ldots \right\rbrace \)
是 \( A \) 的一个乘性子集. 我们简记 \( S_h^{-1}A \) 为 \( A_h \).

\begin{proposition}
  给定一个环 \( A \) 和 \( h \in A \), 映射
  \[
    \sum a_i X^i \mapsto \sum \frac{a_i}{h^i}
  \]
  给出了同构
  \[
    A[X] / (1 - hX) \to A_h.
  \]
\end{proposition}
\begin{proof}
  假设 \( h \in \operatorname{nil} A \), 那么 \( hX \in \operatorname{nil} A[X]
  \).
  而 \( \operatorname{nil}A[X] \) 包含于 \( A[X] \) 的 Jacobson 根中,
  于是由\cref{proposition-Jacobson-root-iff-condition} 得到 \( (1 - hX) = A[X]
  \), 即 \( A[X] / (1 - hX) = 0 \).
  结合\cref{remark-fraction-ring} 知道 \( A_h = 0 \), 故这是一个同构.

  假设 \( h \notin \operatorname{nil} A \).
  记 \( A[x] = A[X] / (1 - hX)\), 那么 \( 1 = hx \), 即 \( h \) 在 \( A[x] \)
  中可逆.
  由\cref{proposition-universal-property-localization}, 自然态射 \( A \to A[x]
  \) 可以分解为 \( A \xrightarrow{i} A_h \xrightarrow{\alpha} A[x]  \), 可以验证
  \( \alpha \) 和命题中映射互逆.
\end{proof}

\paragraph{局部化下的理想对应} 假设 \( S \) 是一个 \( A \) 的乘性子集, \( S^{-1}
A \) 为对应局部化. \( A \) 的理想 \( \mathfrak{a} \) 在 \( S^{-1}A \)
中生成的理想为
\[
  S^{-1} \mathfrak{a} = \left\lbrace \frac{a}{s}: a \in \mathfrak{a}, s \in S
  \right\rbrace.
\]
\begin{proposition}
  \label{proposition-local-ideal-contraction-and-extension}
  假设 \( S \) 是 \( A \) 的乘性子集, 自然态射 \( i_S: A \to  S^{-1} A \)
  具有下面性质:
  \begin{enumerate}
    \item 对所有 \( S^{-1} A \) 的理想 \( \mathfrak{b} \), 有 \(
      \mathfrak{b}^{ce} = \mathfrak{b} \).
    \item 如果 \( \mathfrak{p} \) 是 \( A \) 的一个不交于 \( S \) 的素理想, 那么
      \( \mathfrak{p}^{ec} = \mathfrak{p} \).
      特别地, 每个这样的 \( \mathfrak{p} \) 都是 \( S^{-1} A \)
      中某个素理想的收缩.
  \end{enumerate}
\end{proposition}
\begin{proof}
  \( \mathfrak{b}^{ce} \supset \mathfrak{b} \).
  对任意 \( \frac{a}{s} \in \mathfrak{b} \), \( s\frac{a}{s} = \frac{a}{1} \in
  \mathfrak{b} \).
  因此 \( a \in \mathfrak{b}^c \). 而 \( \frac{a}{s} = a \frac{1}{s}
  \in \mathfrak{b}^{ce} \).

  \( \mathfrak{a}^{ec} \subset \mathfrak{p} \).
  对任意 \( a \in \mathfrak{p}^{ec} \), 存在 \( a' \in \mathfrak{p}, s \in S \)
  使得 \( \frac{a}{1} = \frac{a'}{s} \).
  由局部化定义 \( t \in S \) 使得 \( tsa = ta' \in \mathfrak{p} \).
  由条件 \( S \cap \mathfrak{p} = \varnothing \) 以及 \( ts \in S \), 只能 \( a
  \in \mathfrak{p} \).
  最后一个论断由\cref{proposition-prime-as-contraction-iff-homomorphism-condition}保证.
\end{proof}

\begin{proposition}
  \label{proposition-local-prime-ideal-correspondence}
  下面映射互为双射
  \[
    \begin{split}
      \left\lbrace A \text{中不交} S \text{的素理想} \right\rbrace
      &\leftrightarrow \left\lbrace S^{-1} A \text{的素理想} \right\rbrace\\
      \mathfrak{p} &\mapsto \mathfrak{p}^e\\
      \mathfrak{q}^c &\mapsfrom \mathfrak{q}
    \end{split}
  \]
\end{proposition}
\begin{proof}
  由\cref{proposition-local-ideal-contraction-and-extension}, 只需再验证对 \(
  S^{-1}A \) 的任意素理想 \( \mathfrak{q} \), \( \mathfrak{q}^c \) 是不交 \( S
  \) 的素理想.
  假设不然, 存在 \( s \in \mathfrak{q}^c \cap S \), 那么 \( \frac{s}{1} \in
  \mathfrak{q}^{ce} = \mathfrak{q} \), 也就是 \( \mathfrak{q} = S^{-1} A \),
  这与 \( \mathfrak{q} \) 素矛盾.
\end{proof}

\begin{proposition}
  假设 \( S \) 是 \( A \) 的一个乘性子集, \( \varphi: A \to S^{-1} A \)
  为典范映射, 那么 \( {}^a \varphi \) 是 \( \operatorname{Spec} S^{-1} A \) 到
  \( \operatorname{Spec} A \) 中由与 \( S \) 不交的素理想组成的子空间同胚.
\end{proposition}

\paragraph{局部化与商交换}
\begin{proposition}
  假设 \( \mathfrak{p} \) 是一个不交于 \( S \) 的 \( A \) 的素理想, 令 \(
  \overline{S} \) 为 \( S \) 在 \( A / \mathfrak{p} \) 中的像.
  那么
  \[
    (S^{-1} A) / \mathfrak{p}^e \simeq \overline{S}^{-1} (A / \mathfrak{p}),
  \]
  也就是说, 此时局部化和商是可交换的.
\end{proposition}
\begin{proof}
  由条件, 合成映射 \( A \to A / \mathfrak{p} \to \overline{S}^{-1}(A /
  \mathfrak{p}) \) 将 \( S \) 中元素映到可逆元,
  故可通过\cref{proposition-universal-property-localization}提升为
  \( S^{-1} A \to \overline{S}^{-1}(A / \mathfrak{p}) \);
  而 \( \mathfrak{p}^e \) 在此态射中像为零, 故可以进一步提升 \( (S^{-1} A)
  /\mathfrak{p}^e \to \overline{S}^{-1} (A/ \mathfrak{p})\).

  合成态射 \( A \to S^{-1} A \to S^{-1} A / \mathfrak{p}^e \) 将 \( \mathfrak{p}
  \) 映到零, 故可提升为 \( A / \mathfrak{p} \to S^{-1} A / \mathfrak{p}^e \), \(
  \overline{S} \) 中元素均被此映射映到可逆元,
  故又由\cref{proposition-universal-property-localization} 可提升为 \(
  \overline{S}^{-1}(A / \mathfrak{p}) \to S^{-1} A / \mathfrak{p}^e  \).

  可以验证两个映射互逆.
\end{proof}

\paragraph{在素理想处的局部化}
令 \( \mathfrak{p} \) 为 \( A \) 的素理想, 那么 \( S_{\mathfrak{p}} := A
\setminus \mathfrak{p} \) 为 \( A \) 的乘性子集, 此时记 \( A_{\mathfrak{p}} =
S_{\mathfrak{p}}^{-1} A \),
由\cref{proposition-local-prime-ideal-correspondence}有
\[
  \operatorname{Spec}(A_\mathfrak{p}) \simeq \left\lbrace \mathfrak{q} \in
  \operatorname{Spec} A: \mathfrak{q} \subseteq \mathfrak{p} \right\rbrace.
\]
因此 \( A_{\mathfrak{p}} \) 是一个极大理想为 \( \mathfrak{p}^{e} \) 的局部环.

\begin{proposition}
  假设 \( \mathfrak{m} \) 是环 \( A \) 的极大理想, \( \mathfrak{n} =
  \mathfrak{m} A_{\mathfrak{m}} \), 那么对所有自然数 \( n \),
  \begin{enumerate}
    \item \( A/\mathfrak{m}^n \) 是局部环, 其极大理想即 \(
      \mathfrak{m}/\mathfrak{m}^n \).
    \item 映射
      \[
        \varphi: A / \mathfrak{m}^n \to A_{\mathfrak{m}} / \mathfrak{n}^n \quad a +
        \mathfrak{m}^n \mapsto a + \mathfrak{n}^n
      \]
      是同构.
    \item 对所有 \( r \leq n \), 下面同构成立
      \[
        \mathfrak{m}^r / \mathfrak{m}^n \to \mathfrak{n}^r / \mathfrak{n}^n.
      \]
  \end{enumerate}
\end{proposition}
\begin{proof}
  (1) \( A \) 中包含 \( \mathfrak{m}^n \) 的极大理想包含 \( \operatorname{rad}
  (\mathfrak{m}^n) = \mathfrak{m} \), 只能等于 \( \mathfrak{m} \).

  (2)
  \( \varphi \) 单.
  因为 \( \mathfrak{n}^n = (\mathfrak{m}^n)^e \), 所以 \( \operatorname{ker}
  \varphi = (\mathfrak{m}^n)^{ec} / \mathfrak{m}^n \).
  设 \( a \in (\mathfrak{m}^n)^{ec} \), 那么存在 \( b \in \mathfrak{m}^n, s
  \notin \mathfrak{m} \), 使得 \( \frac{a}{1} = \frac{b}{s} \), 于是存在某个 \(
  t \notin \mathfrak{m} \), 使得 \( tsa \in \mathfrak{m}^n \).
  而  \( ts \notin \mathfrak{m} \),
  由(1)和\cref{proposition-Jacobson-root-iff-condition}, \( ts \) 是 \( A /
  \mathfrak{m}^n \) 中的单位, 故 \( a \in \mathfrak{m}^n \).

  \( \varphi \) 满.
  设 \( a/s \in A_{\mathfrak{m}} \), 其中 \( a \in A, s \notin
  \mathfrak{m}\).
  再由(1)和\cref{proposition-Jacobson-root-iff-condition}, \( s \) 是 \(
  A/\mathfrak{m}^n \) 中的单位, 即\( (s) + \mathfrak{m}^n = A \).
  因此存在 \( b \in A \) 和 \( q \in
  \mathfrak{m}^n \) 使得 \( sb + q = 1 \). 因此
  \[
    s(ba) = a(1 - q).
  \]
  因此在 \( A_{\mathfrak{m}} \) 上
  \[
    \frac{ba}{1} = \frac{a}{s} - \frac{aq}{s}.
  \]
  换句话说 \( \varphi(\overline{ba}) = \overline{a/s} \).

  (3) 由下图, (3)是(2)和五引理的直接推论
% https://q.uiver.app/#q=WzAsMTAsWzAsMCwiMCJdLFswLDEsIjAiXSxbNCwwLCIwIl0sWzQsMSwiMCJdLFsxLDAsIlxcbWF0aGZyYWt7bX1eciAvIFxcbWF0aGZyYWt7bX1ebiJdLFsxLDEsIlxcbWF0aGZyYWt7bn1eciAvIFxcbWF0aGZyYWt7bn1ebiJdLFsyLDAsIkEgLyBcXG1hdGhmcmFre219Xm4iXSxbMywwLCJBIC8gXFxtYXRoZnJha3ttfV5yIl0sWzIsMSwiQV97XFxtYXRoZnJha3ttfX0gLyBcXG1hdGhmcmFre259Xm4iXSxbMywxLCJBX3tcXG1hdGhmcmFre219fSAvIFxcbWF0aGZyYWt7bn1eciJdLFsxLDVdLFswLDRdLFs0LDVdLFs0LDZdLFs1LDhdLFs2LDgsIlxcc2ltZXEiLDFdLFs2LDddLFs4LDldLFs3LDJdLFs5LDNdLFs3LDksIlxcc2ltZXEiLDFdXQ==
\[\begin{tikzcd}
	0 & {\mathfrak{m}^r / \mathfrak{m}^n} & {A / \mathfrak{m}^n} & {A / \mathfrak{m}^r} & 0 \\
	0 & {\mathfrak{n}^r / \mathfrak{n}^n} & {A_{\mathfrak{m}} / \mathfrak{n}^n} & {A_{\mathfrak{m}} / \mathfrak{n}^r} & 0
	\arrow[from=1-1, to=1-2]
	\arrow[from=1-2, to=1-3]
	\arrow[from=1-2, to=2-2]
	\arrow[from=1-3, to=1-4]
	\arrow["\simeq"{description}, from=1-3, to=2-3]
	\arrow[from=1-4, to=1-5]
	\arrow["\simeq"{description}, from=1-4, to=2-4]
	\arrow[from=2-1, to=2-2]
	\arrow[from=2-2, to=2-3]
	\arrow[from=2-3, to=2-4]
	\arrow[from=2-4, to=2-5]
\end{tikzcd}\]
\end{proof}

\subsection{分式模}

\begin{proposition}
  函子 \( M \to S^{-1} M \) 是正合函子.
\end{proposition}
\begin{proof}
  假设给定了正合列
  \[
    M' \xrightarrow{\alpha} M \xrightarrow{\beta} M''
  \]

  \( \operatorname{Im} (S^{-1} \alpha) \subset \operatorname{Ker} (S^{-1}\beta)
  \) 是因为 \( 0 = S^{-1}(\beta \circ \alpha) = S^{-1} \beta \circ S^{-1} \alpha
  \).

  \(  \operatorname{Im} (S^{-1} \alpha) \supset \operatorname{Ker} (S^{-1}\beta)
   \).
  设 \( \frac{m}{s} \in \operatorname{Ker}(S^{-1}\beta) \), 其中 \( m \in M, s
  \in S \).
  那么存在 \( t \in S \) 使得 \( t(\beta(m)) = \beta(tm) = 0 \).
  由原列的正合性, 存在 \( m' \in M' \) 使得 \( tm = \alpha(m') \).
  因此
  \[
    \frac{m}{s} = \frac{tm}{ts} = \frac{\alpha(m')}{ts} \in
    \operatorname{Im}(S^{-1}\alpha).
  \]
\end{proof}

\paragraph{模为零的充要条件}
\begin{proposition}
  假设 \( M \) 是一个\( A \)-模. 那么典范映射
  \[
    M \to \prod \left\lbrace M_{\mathfrak{m}}: \mathfrak{m} \text{是} A
    \text{的一个极大理想} \right\rbrace
  \]
  是一个单射.
\end{proposition}
\begin{proof}
  假设不然, 存在\( 0 \neq  x \) 在此映射下像为零, 那么存在极大理想 \( \mathfrak{m} \)
  包含 \( \operatorname{ann}(x) \).
  于是 \( x \) 在 \( M \to M_{\mathfrak{m}} \)
  下像非零, 矛盾.
\end{proof}

\begin{corollary}
  \label{corollary-module-zero-iff-condition}
  假设 \( M \) 是一个 \( A \)-模. 如果 \( M_{\mathfrak{m}} = 0 \) 对所有 \( A \)
  的极大理想 \( \mathfrak{m} \) 成立, 那么 \( M = 0 \).
\end{corollary}

\paragraph{模复形链正合的充要条件}

\begin{proposition}
  复形链
  \[
    M' \xrightarrow{\alpha} M \xrightarrow{\beta} M''
  \]
  正合当且仅当对 \( A \) 的所有极大理想 \( \mathfrak{m} \) , 列
  \[
    M'_{\mathfrak{m}} \xrightarrow{\alpha_{\mathfrak{m}}} M_{\mathfrak{m}}
    \xrightarrow{\beta_{\mathfrak{m}}} M''_{\mathfrak{m}}
  \]
  正合.
\end{proposition}
\begin{proof}
  只需证明 \( \impliedby \).
\end{proof}

\begin{corollary}
  \label{corollary-local-global-module-morphism}
  一个 \( A \)-模态射 \( M \to N \) 单(resp. 满, 零) 当且仅当 \(
  M_{\mathfrak{m}} \to N_{\mathfrak{m}} \) 单(resp. 满, 零).
\end{corollary}
\begin{proof}
  考虑 \( 0 \to M \to N \)(resp. \( M \to N \to 0, M
  \xrightarrow{\operatorname{id}} M \to N \)).
\end{proof}

\paragraph{环约化充要条件}

\begin{proposition}
  假设 \( S \) 为 \( A \) 的一个乘性子集, 那么
  \[
    S^{-1} \operatorname{nil}(A) = \operatorname{nil}(S^{-1}A).
  \]
\end{proposition}
\begin{proof}
  如果 \( (\frac{a}{s})^n = 0 \) 对某个 \( n \) 成立, 其中 \( a \in A, s \in S \).
  那么存在 \( t \in S \) 使得 \( ta^n = 0 \), 因此 \( \frac{a}{s} =
  \frac{ta}{ts} \in S^{-1} \operatorname{nil} (A) \).
\end{proof}

\begin{corollary}
  环 \( A \) 约化当且仅当 \( A_{\mathfrak{m}} \) 均约化.
\end{corollary}

\section{整性}

\subsection{基本性质}

假设 \( A \) 是环 \( B \) 的子环. 一个 \( B \) 的元素 \( \alpha \) 称为在 \( A
\) 上\emph{整}, 如果 \( \alpha \) 是 \( A \) 的一个首一系数多项式的根,
也就是它满足某个形如下面的方程
\[
  x^n + a_1 x^{n - 1} + \cdots + a_n = 0,\quad a_i \in A.
\]
更一般地, 一个 \( A \)-代数 \( B \) 的元素称为在 \( A \) 上\emph{整}, 如果它在
\( A \) 在 \( B \) 的像中整. 如果 \( B \) 的每个元素都在 \( A \) 中整, 那么称 \( B
\) 在 \( A \) 中\emph{整}.

\begin{lemma}[Cramer]
  \label{lemma-Cramer-varient}
  假设 \( x_j \) 是 \( A \)-系数线性方程
  \[
    \sum_{j = 1}^m c_{ij} x_j = 0,\quad i = 1, \ldots, m,
  \]
  的解, 那么
  \[
    \det(C) \cdot x_j = 0,\quad j = 1,\ldots, m.
  \]
\end{lemma}
\begin{proof}
  考察
  \[
    \det \begin{pmatrix}
      c_{11} &\cdots &c_{1, j - 1} &\sum_{i} &c_{1i} x_{i} &c_{1, j + 1} &\cdots
      &c_{1m}\\ \vdots &&\vdots &\vdots &\vdots &&\vdots\\ c_{m1} &\cdots
      &c_{m, j - 1} &\sum_{i} &c_{mi} x_{i} &c_{m, j + 1} &\cdots &c_{mm}
    \end{pmatrix} = 0
  \]
\end{proof}

\begin{proposition}
  \label{proposition-integral-iff-condition}
  假设环 \( B \) 是 \( A \) 的扩张. \( B \) 的一个元素 \( \alpha \) 在 \( A \)
  上整当且仅当存在 \( B \) 的忠实 \( A[\alpha] \)-子模作为 \( A \)-模有限.
\end{proposition}
\begin{proof}
  \( \implies \) 考虑 \( A[\alpha] \) 即可. \( \impliedby \) 假设 \( C \)
  是一个由极小生成元 \( c_1,\ldots, c_n \) 生成的忠实\( A[\alpha] \)-子模, 那么
  \[
    \alpha c_i = a_{i1}c_1 + a_{i2}c_2 + \cdots + a_{in}c_n
  \]
  记
  \[
    C = \begin{pmatrix}
      \alpha - a_{11} &-a_{12} &\cdots &-a_{1n}\\
      -a_{21} &\alpha - a_{12} &\cdots &-a_{2n}\\
      \vdots &\vdots &\ddots &\vdots\\
      -a_{n1} &-a_{n2} &\cdots &\alpha-a_{nn}\\
    \end{pmatrix}
  \]
  那么由\cref{lemma-Cramer-varient}有 \( \det(C) e_i = 0 \).
  由忠实性, \( \det(C) = 0 \), 展开 \( \det(C) \), 可以找到 \( b \) 满足一个首一
  \( A \)-系数多项式.
\end{proof}


\begin{proposition}
  \label{proposition-finite-generated-integral-algebra-as-finite-module}
  如果 \( A \)-代数 \( B \) 由有限个在 \( A \) 上整的元素生成, 那么 \( B \) 作为
  \( A \)-模有限.
\end{proposition}
\begin{proof}
  假设 \( B \) 由 \( \alpha_1, \ldots, \alpha_m \) 生成, 并且
  \[
    \alpha_i^{n_i} + a_{i1}\alpha_{i}^{n_i - 1} + \cdots + a_{in_i} =
    0,\quad a_{ij} \in A,\quad i = 1,\ldots, m.
  \]
  那么 \( B \) 由 \( \alpha_1^{r_i}\cdots \alpha_m^{r_m}, 1 \leq r_i < n_i
  \) 生成.
\end{proof}

\begin{corollary}
  考虑环 \( A \subseteq B \subseteq C \).  \( B \) 在 \( A \) 上整并且 \( C \)
  在 \( B \) 上整当且仅当 \( C \) 在 \( A \) 上整.
\end{corollary}
\begin{proof}
  \( \implies \) 假设 \( c \in C \) 满足
  \[
    c^{n} + b_{1}c_{i}^{n - 1} + \cdots + b_{n} = 0,\quad b_{i} \in B,\quad i =
    1,\ldots, n.
  \]
  那么 \( A[b_1, \ldots, b_n, c] \) 在 \( A[b_1, \ldots, b_n] \) 上有限,
  而由\cref{proposition-finite-generated-module-over-noetherian-ring} \( A[b_1,
  \ldots, b_n] \) 在 \( A \) 上有限, 于是 \( A[b_1, \ldots, b_n, c] \) 在 \( A
  \) 上有限, 于是由\cref{proposition-integral-iff-condition}, \( c \) 在 \( A \)
  上整.
\end{proof}

\subsection{整闭包}

\begin{theorem}
  \label{theorem-integral-closure}
  假设 \( A \) 为 \( B \) 的子环, 那么 \( B \) 在 \( A \) 上的整的元素构成一个
  \( B \) 的 \( A \)-子代数.
  我们称 \( A \) 在 \( B \) 的\emph{整闭包}为 \( B \) 在 \( A \) 中的所有整元素.
  特别地, 在没有歧义的情况下, 如果 \( A \) 是一个整环, \( A \)
  在其分式域的整闭包称为 \( A \) 的\emph{整闭包}.
\end{theorem}

\begin{proposition}
  假设 \( A \) 是一个整环, 其分式域为 \( F \), \( E \) 是一个包含 \( F \) 的域.
  如果 \( \alpha \in E \) 在 \( F \) 上代数, 那么存在\( d \in A \) 使得 \( d
  \alpha \) 在 \( A \) 上整.
\end{proposition}

\begin{corollary}
  如果 \( A \) 是一个整环, \( E \) 是 \( A \) 的分式域的整扩张, 那么 \( E \)
  是 \( A \) 在 \( E \) 中整闭包的分式域.
\end{corollary}

\paragraph{整闭整环} 一个整环 \( A \) 称为是\emph{整闭的} 或者\emph{正规的},
如果它等于它在自身分式域上的整闭包, 换句话说
\[
  \alpha \in F,\quad \alpha \text{在} A \text{中整} \implies \alpha \in A.
\]

假设 \( F \subseteq E \) 均为域, \( \alpha \in E \) 在 \( F \) 上代数. \( \alpha
\) 在 \( F \) 上的\emph{极小多项式}是首一的 \( \alpha \) 的最小多项式.

\begin{proposition}
  \label{proposition-integral-element-minimal-polynomial-coefficients}
  假设 \( A \) 是一个正规整环, \( E \) 是 \( A \) 的分式域 \( F \)
  的一个有限扩张. \( E \) 的一个元素 \( \alpha \) 在 \( A \) 上整当且仅当其在 \(
  F \) 上的极小多项式系数均在 \( A \) 上.
\end{proposition}
\begin{proof}
  只需验证 \( \implies \).
  假设 \( \alpha \) 在 \( A \) 上整, \( f \) 为 \( \alpha \) 在 \( F \)
  上的极小多项式, 那么 \( f \) 的所有根都在 \( A \) 上整.
  由\cref{proposition-splitting-polynomial-ring} \( f \) 的系数是 \( f \)
  根的代数组合, 因此由\cref{theorem-integral-closure}, 这些系数在 \( A \) 上整,
  又由 \( A \) 整闭, 这些系数都在 \( A \) 上.
\end{proof}

\begin{corollary}
  假设 \( A \) 是一个分式域为 \( F \) 的整闭整环, \( f \) 是 \( A[X] \)
  的一个首一系数多项式, 那么 \( f \) 在 \( F[X] \) 中的每个首一系数因子系数都在
  \( A \) 中.
\end{corollary}

\paragraph{整闭的局部性质}

\begin{proposition}
  \label{proposition-integral-imples-local-integral}
  假设 \( B \) 为一个 \( A \) 代数, \( A' \) 为 \( A \) 在 \( B \) 的整闭包.
  对每个 \( A \) 的在结构同态下像不含零的乘性子集 \( S \), \( S^{-1}A' \)
  为 \( S^{-1} A \) 在 \( S^{-1} B \) 的整闭包.
\end{proposition}
\begin{proof}
  假设 \( b/s \in S^{-1}A' \) 其中 \( b \in A' \) 和 \( s \in S \), 那么有
  \[
    b^n + a_1 b^{n - 1} + \cdots + a_n = 0,\quad a_i \in A,
  \]
  因此
  \[
    \left(\frac{b}{s}\right)^n + \frac{a_1}{s} \left( \frac{b}{s} \right)^{n -
    1} + \cdots + \frac{a_n}{s^n} = 0
  \]
  \( b/s \) 在 \( S^{-1}A \) 上整.

  反过来, 假设 \( b / s \) 在 \( S^{-1}A \) 上整, 其中 \( b \in B, s \in S \), 那么有
  \[
    \left( \frac{b}{s} \right)^n + \frac{a_1}{s_1} \left( \frac{b}{s} \right)^{n
    - 1} + \cdots + \frac{a_n}{s_n} = 0,\quad a_i \in A,\quad s_i S.
  \]
  同乘 \( s^n s_1^n \cdots s_n^n \), 知道 \( s_1 \cdots s_n b \in A' \), 于是 \(
  b/s \in S^{-1} A \).
\end{proof}

\begin{corollary}
  假设 \( B \) 为一个 \( A \) 代数, \( S \) 是 \( A \)
  的一个结构同态像不含零的乘性子集.
  如果 \( A \) 在 \( B \) 中整闭, 那么 \( S^{-1}A \) 在 \( S^{-1} B \) 中整闭.
\end{corollary}

\begin{proposition}
  假设 \( A \) 是一个整环, 那么下面条件等价
  \begin{enumerate}
    \item \( A \) 整闭.
    \item 对所有素理想 \( \mathfrak{p} \), \( A_{\mathfrak{p}} \) 整闭.
    \item 对所有极大理想 \( \mathfrak{m} \), \( A_{\mathfrak{m}} \) 整闭.
  \end{enumerate}
\end{proposition}
\begin{proof}
  (3) \( \implies \) (1).
  假设 \( A' \) 为 \( A \) 的整闭包, 那么由 (3) 对所有极大理想 \( \mathfrak{m} \), \(
  A_{\mathfrak{m}} \to (A')_{\mathfrak{m}} \) 满,
  因此由\cref{corollary-local-global-module-morphism} \( A \to A' \) 满.
\end{proof}

\paragraph{整闭与多项式}

\begin{lemma}
  \label{lemma-integral-monic-factor}
  如果 \( A \) 是一个环, \( B \) 是一个 \( A \)-代数, \( f, g \in B[T] \)
  为首一多项式使得 \( f \) 的系数在 \( A \) 上整且 \( g \mid f \), 那么 \( g \)
  的系数也在 \( A \) 上整.
\end{lemma}
\begin{proof}
  \cref{proposition-splitting-polynomial-ring}.
\end{proof}

\begin{proposition}
  \label{proposition-polynomial-integral-implies-coefficient-integral}
  假设 \( B \) 是一个 \( A \)-代数. 如果 \( P \in B[T] \) 在 \( A[T] \) 中整,
  那么 \( P \) 的系数都在 \( A \) 中整.
\end{proposition}
\begin{proof}
  假设 \( P \in B[T] \) 为多项式
  \[
    q(X) = X^n + f_1 X^{n - 1} + \cdots + f_n,\quad f_i \in A[T]
  \]
  的根.
  令 \( r \) 为一个大于 \( \deg f_1,\ldots, \deg f_n \) 的整数, \( P_1 = P - T^r
  \), 以及
  \[
    q_1(X) = q(X + T^r) = X^n + g_1 X^{n - 1} + \cdots + g_n,\quad g_i \in A[T].
  \]
  那么 \( P_1 \) 为 \( q_1(X) \) 的一个根, 因此
  \[
    g_n = -P_1 \cdot (P^{n - 1}_1 + g_1 P^{n - 2}_1 + \cdots + g_{n - 1}).
  \]
  由 \( r \) 的选择知道, \( g_n \) 或 \( -g_n \) 是首一的.
  结合 \( g_n \) 系数在 \( A \) 中以及\cref{lemma-integral-monic-factor}, 知道
  \( P_1 \) 系数在 \( A \) 中整, 因此 \( P \) 系数在 \( A \) 中整.
\end{proof}

\begin{proposition}
  整闭整环的多项式环都是整闭整环.
\end{proposition}
\begin{proof}
  只需证明, 如果 \( A \) 是一个整闭整环, 那么 \( A[T] \) 亦是一个整闭整环.
  假设 \( F \) 是 \( A \) 的分式域.
  如果 \( A[T] \) 的分式域 \( F(T) \) 的一个元素 \( f \) 在 \( A[T] \) 中整,
  那么它在 \( F[T] \) 中整.
  而 \( F[T] \) 是一个唯一因子分解整环, 其分式域亦为 \( F(T) \), 因此 \( f \in
  F[T] \).
  因此由\cref{proposition-polynomial-integral-implies-coefficient-integral}, \(
  f \) 系数在 \( A \) 中整, 结合 \( A \) 整闭, \( f \in A[T] \).
\end{proof}

\subsection{理想整}

假设环 \( A, B \) 满足 \( A \subset B \), \( \mathfrak{a} \) 为
\( A \) 的一个理想.
\( B \) 的一个元素 \( b \) 称为在 \( \mathfrak{a} \) 上\emph{整},
如果其满足某个方程
\[
  b^n + a_1 b^{n - 1} + \cdots + a_n = 0,
\]
其中 \( a_i \in \mathfrak{a} \).
\( B \) 在 \( \mathfrak{a} \) 上的\emph{整闭包}为 \( B \) 在 \( \mathfrak{a} \)
上整的元素.

类似于\cref{proposition-integral-iff-condition}, 我们可以知道
\begin{proposition}
  \label{proposition-integral-over-ideal-condition}
  如果一个 \( B \) 的忠实 \( A[b] \)-子模 \( M \), 其作为 \( A \)-模有限, 使得
  \( b M \subseteq \mathfrak{a} M \), 那么 \( b \in B \) 在 \( \mathfrak{a} \)
  上整.
\end{proposition}

\begin{lemma}
  \label{lemma-integral-closure-over-ideal}
  如果 \( A \) 在 \( B \) 的整闭包为 \( A' \), 那么 \( \mathfrak{a} \) 在 \( B
  \) 中的整闭包是 \( \operatorname{rad}(\mathfrak{a}A') \).
\end{lemma}
\begin{proof}
  如果 \( b \in B \) 在 \( \mathfrak{a} \) 上整, 那么 \( b \in A' \) 且 \(
  b^n \in \mathfrak{a} A' \), 因此 \( b \in \operatorname{rad}(\mathfrak{a}A')
  \).

  反过来, 如果 \( b \in \operatorname{rad}(\mathfrak{a} A') \), 使得
  \[
    b^m = \sum_i a_i x_i,\quad \text{其中} m > 0,\quad a_i \in
    \mathfrak{a},\quad x_i \in A'.
  \]
  设 \( M = A[x_1, \ldots, x_n] \),
  由\cref{proposition-finite-generated-integral-algebra-as-finite-module}, \( M
  \) 作为 \( A \)-模有限. 结合 \( b^m M \subseteq \mathfrak{a} M \)
  与\cref{proposition-integral-over-ideal-condition}, \( b^m \) 在 \(
  \mathfrak{a} \) 上整, 因此 \( b \) 在 \( \mathfrak{a} \) 上整.
\end{proof}

\begin{proposition}
  \label{proposition-minimal-polynomial-coefficients-element-integral-over-ideal}
  假设 \( A \) 是一个整闭整环, \( E \) 是 \( A \) 的分式域的扩张, \( \alpha \in
  E \).
  如果  \( \alpha \) 在 \( A \) 的理想 \( \mathfrak{a} \) 上整,
  那么 \( \alpha \) 在 \( F \) 的极小多项式系数都在 \(
  \operatorname{rad}(\mathfrak{a}) \) 上.
\end{proposition}

\subsection{上升下降定理}

\paragraph{整扩张下极大理想的收缩}

\begin{proposition}
  如果 \( A \subseteq B \) 为整环, \( B \) 在 \( A \) 上整, 那么 \( B \)
  是一个域当且仅当 \( A \) 是一个域.
\end{proposition}
\begin{proof}
  \( \implies \). 假设 \( 0 \neq \alpha \in A \), 那么 \( \alpha^{-1} \in B \)
  在 \( A \) 上整. 因此存在 \( \alpha_i \in A \) 使得
  \[
    (\alpha^{-1})^n + a_1(\alpha^{-1})^{n - 1} + \cdots + a_n = 0.
  \]
  两边同乘 \( a^{n - 1} \) 立刻知道 \( \alpha^{-1} \in A \).

  \( \impliedby \). 假设 \( 0 \neq \alpha \in B \) 在 \( A \) 上整,
  那么能找到一个 \( \alpha \) 次数最低的方程
  \[
    \alpha^n + a_1\alpha^{n - 1} + \cdots + a_n = 0,\quad a_i \in A.
  \]
  显然 \( a_n \neq 0 \), 因此
  \[
    (-a_n)^{-1}(\alpha^{n - 1} + a_1\alpha^{n - 2} + \cdots + a_{n - 1})\alpha =
    1,
  \]
  也就是 \( \alpha \) 在 \( B \) 中有逆.
\end{proof}

\begin{corollary}
  \label{corollary-maximal-integral-iff-condition}
  假设 \( B \) 是一个 \( A \)-代数, \( B \) 在 \( A \) 上整, \( \mathfrak{q} \)
  为\( B \) 的一个素理想, \( \mathfrak{p} = \mathfrak{q} \cap A \).
  那么 \( \mathfrak{q} \) 极大当且仅当 \( \mathfrak{p} \) 极大.
\end{corollary}

\begin{corollary}[不相容性]
  假设 \( \varphi: A \to B \) 为环同态, \( B \) 在 \( A \) 上整, 并且 \(
  \mathfrak{q} \subset \mathfrak{q}' \) 为 \( B \) 的素理想.
  如果 \( \mathfrak{q} \cap A = \mathfrak{q}' \cap A \), 那么 \( \mathfrak{q} =
  \mathfrak{q}' \).
  % 换句话说, 素谱态射 \( {}^a \varphi: \operatorname{Spec} B \to \operatorname{Spec} A \)
  % 单.
\end{corollary}
\begin{proof}
  设 \( \mathfrak{p} = \mathfrak{q} \cap A = \mathfrak{q}' \cap A \), 那么 \(
  A_{\mathfrak{p}} \subset B_{\mathfrak{p}} \)
  且由\cref{proposition-integral-imples-local-integral} \( B_{\mathfrak{p}} \) 在
  \( A_{\mathfrak{p}} \) 上整.
  \( B_{\mathfrak{p}} \) 的理想 \( \mathfrak{q} B_{\mathfrak{p}} \subset \mathfrak{q}' B_{\mathfrak{p}}
  \) 收缩到 \( A_{\mathfrak{p}}\) 上均为极大理想 \(
  \mathfrak{p}A_{\mathfrak{p}} \).
  由\cref{proposition-integral-iff-condition}以及极大性 \( \mathfrak{q} B_{\mathfrak{p}} =
  \mathfrak{q}' B_{\mathfrak{p}} \).
  因此
  \[
    \mathfrak{q} = (\mathfrak{q}B_{\mathfrak{p}})^c =
    (\mathfrak{q}'B_{\mathfrak{p}})^c = \mathfrak{q}'.
  \]
\end{proof}

\paragraph{上升定理}

\begin{proposition}
  \label{proposition-integral-imply-surjective-spectrum-morphism}
  假设 \( B \) 为 \( A \) 代数, \( B \) 在 \( A \) 上整, \( \mathfrak{p} \) 为
  \( A \) 的素理想, 且 \( A \setminus \mathfrak{p} \) 结构同态像不含零.
  那么存在 \( B \) 的素理想 \( \mathfrak{q} \) 使得 \( \mathfrak{p} =
  \mathfrak{q} \cap A \).
\end{proposition}
\begin{proof}
  由假设 \( B_{\mathfrak{p}} \) 作为 \( A_{\mathfrak{p}} \) 代数整.
  设 \( \mathfrak{n} \) 为 \( B_{\mathfrak{p}} \) 的一个极大理想.
  由下面交换图 \( \mathfrak{n} \) 在 \( B \) 上的收缩在 \( A \) 上收缩为 \(
  \mathfrak{p} \).
  % https://q.uiver.app/#q=WzAsNCxbMCwxLCJBIl0sWzEsMSwiQV97XFxtYXRoZnJha3twfX0iXSxbMSwwLCJCX3tcXG1hdGhmcmFre3B9fSJdLFswLDAsIkIiXSxbMCwzXSxbMCwxXSxbMSwyXSxbMywyXV0=
  \[\begin{tikzcd}
    B & {B_{\mathfrak{p}}} \\
    A & {A_{\mathfrak{p}}}
    \arrow[from=1-1, to=1-2]
    \arrow[from=2-1, to=1-1]
    \arrow[from=2-1, to=2-2]
    \arrow[from=2-2, to=1-2]
  \end{tikzcd}\]
\end{proof}

\begin{theorem}[上升定理]
  假设 \( A \subset B \) 为环并且 \( B \) 在 \( A \) 上整.
  如果 \( \mathfrak{p} \subset \mathfrak{p}' \) 为 \( A \) 的素理想且 \(
  \mathfrak{q} \) 为 \( B \) 的素理想使得 \( \mathfrak{q} \cap A = \mathfrak{p}
  \), 那么存在 \( B \) 的素理想 \( \mathfrak{q}' \) 使得 \( \mathfrak{q}' \cap A
  = \mathfrak{p}' \).
  % https://q.uiver.app/#q=WzAsNixbMCwwLCJCIl0sWzAsMSwiQSJdLFsxLDAsIlxcbWF0aGZyYWt7cX0iXSxbMSwxLCJcXG1hdGhmcmFre3B9Il0sWzIsMCwiXFxtYXRoZnJha3txfSciXSxbMiwxLCJcXG1hdGhmcmFre3B9JyJdLFszLDUsIlxcc3Vic2V0IiwxLHsic3R5bGUiOnsiYm9keSI6eyJuYW1lIjoibm9uZSJ9LCJoZWFkIjp7Im5hbWUiOiJub25lIn19fV0sWzIsNCwiXFxzdWJzZXQiLDEseyJzdHlsZSI6eyJib2R5Ijp7Im5hbWUiOiJub25lIn0sImhlYWQiOnsibmFtZSI6Im5vbmUifX19XSxbMywyLCIiLDEseyJzdHlsZSI6eyJoZWFkIjp7Im5hbWUiOiJub25lIn19fV0sWzUsNCwiIiwxLHsic3R5bGUiOnsiYm9keSI6eyJuYW1lIjoiZGFzaGVkIn0sImhlYWQiOnsibmFtZSI6Im5vbmUifX19XSxbMSwwLCIiLDEseyJzdHlsZSI6eyJoZWFkIjp7Im5hbWUiOiJub25lIn19fV1d
  \[\begin{tikzcd}
    B & {\mathfrak{q}} & {\mathfrak{q}'} \\
    A & {\mathfrak{p}} & {\mathfrak{p}'}
    \arrow["\subset"{description}, draw=none, from=1-2, to=1-3]
    \arrow[no head, from=2-1, to=1-1]
    \arrow[no head, from=2-2, to=1-2]
    \arrow["\subset"{description}, draw=none, from=2-2, to=2-3]
    \arrow[dashed, no head, from=2-3, to=1-3]
  \end{tikzcd}\]
\end{theorem}
\begin{proof}
  结合\cref{proposition-integral-imply-surjective-spectrum-morphism}考虑下面的交换图即可.
  % https://q.uiver.app/#q=WzAsNCxbMCwxLCJBIl0sWzEsMSwiQS9cXG1hdGhmcmFre3B9Il0sWzEsMCwiQi9cXG1hdGhmcmFre3F9Il0sWzAsMCwiQiJdLFswLDNdLFswLDFdLFsxLDJdLFszLDJdXQ==
  \[\begin{tikzcd}
    B & {B/\mathfrak{q}} \\
    A & {A/\mathfrak{p}}
    \arrow[from=1-1, to=1-2]
    \arrow[from=2-1, to=1-1]
    \arrow[from=2-1, to=2-2]
    \arrow[from=2-2, to=1-2]
  \end{tikzcd}\]
\end{proof}

\begin{corollary}
  假设 \( A \subset B \) 为环且 \( B \) 在 \( A \) 上整, 且 \( \mathfrak{p}_1
  \subset \cdots \subset \mathfrak{p}_n \) 为 \( A \) 的素理想.
  令素理想包含链
  \[
    \mathfrak{q}_1 \subset \cdots \subset \mathfrak{q}_m \quad (m < n)
  \]
  为 \( B \) 的素理想使得 \( \mathfrak{q}_i \cap A = \mathfrak{p} \) 对所有 \( i
  \leq m \) 成立, 那么此素理想包含链可以扩张为
  \[
    \mathfrak{q}_1 \subset \cdots \subset \mathfrak{q}_n
  \]
  使得 \( \mathfrak{q}_i \cap A = \mathfrak{p} \), 其中 \( i \leq n \).
\end{corollary}

\paragraph{下降定理}

\begin{theorem}[下降定理]
  假设 \( A \subset B \) 为整环, 且 \( A \) 整闭, \( B \) 在 \( A \) 上整.
  如果 \( \mathfrak{p} \supset \mathfrak{p}' \) 为 \( A \) 的素理想, \(
  \mathfrak{q} \) 为 \( B \) 的素理想使得 \( \mathfrak{q} \cap A = \mathfrak{p}
  \). 那么 \( \mathfrak{q} \) 包含一个 \( B \) 的素理想 \( \mathfrak{q}' \) 使得
  \( \mathfrak{q}' \cap A = \mathfrak{p}' \).
  % https://q.uiver.app/#q=WzAsNixbMCwwLCJCIl0sWzAsMSwiQSJdLFsxLDAsIlxcbWF0aGZyYWt7cX0iXSxbMSwxLCJcXG1hdGhmcmFre3B9Il0sWzIsMCwiXFxtYXRoZnJha3txfSciXSxbMiwxLCJcXG1hdGhmcmFre3B9JyJdLFszLDUsIlxcc3Vwc2V0IiwxLHsic3R5bGUiOnsiYm9keSI6eyJuYW1lIjoibm9uZSJ9LCJoZWFkIjp7Im5hbWUiOiJub25lIn19fV0sWzIsNCwiXFxzdXBzZXQiLDEseyJzdHlsZSI6eyJib2R5Ijp7Im5hbWUiOiJub25lIn0sImhlYWQiOnsibmFtZSI6Im5vbmUifX19XSxbMywyLCIiLDEseyJzdHlsZSI6eyJoZWFkIjp7Im5hbWUiOiJub25lIn19fV0sWzUsNCwiIiwxLHsic3R5bGUiOnsiYm9keSI6eyJuYW1lIjoiZGFzaGVkIn0sImhlYWQiOnsibmFtZSI6Im5vbmUifX19XSxbMSwwLCIiLDEseyJzdHlsZSI6eyJoZWFkIjp7Im5hbWUiOiJub25lIn19fV1d
  \[\begin{tikzcd}
    B & {\mathfrak{q}} & {\mathfrak{q}'} \\
    A & {\mathfrak{p}} & {\mathfrak{p}'}
    \arrow["\supset"{description}, draw=none, from=1-2, to=1-3]
    \arrow[no head, from=2-1, to=1-1]
    \arrow[no head, from=2-2, to=1-2]
    \arrow["\supset"{description}, draw=none, from=2-2, to=2-3]
    \arrow[dashed, no head, from=2-3, to=1-3]
  \end{tikzcd}\]
\end{theorem}
\begin{proof}
  \( B \) 的包含在 \( \mathfrak{q} \) 的素理想与 \( B_{\mathfrak{q}} \)
  的素理想一一对应, 因此只需要证明 \( \mathfrak{p}' \) 为 \( B_{\mathfrak{q}} \)
  的一个素理想的收缩.
  由\cref{proposition-integral-iff-condition} 这等价于证明
  \[
    A \cap \mathfrak{p}' B_{\mathfrak{q}} = \mathfrak{p}'.
  \]

  只需证明 \( A \cap \mathfrak{p}' B_{\mathfrak{q}} \subset \mathfrak{p}' \).
  设 \( b = y/s \in A \cap \mathfrak{p}' B_{\mathfrak{q}} \), 其中 \( y \in
  \mathfrak{p}'B, s \in B \setminus \mathfrak{q} \).
  由\cref{lemma-integral-closure-over-ideal} \( y \) 在 \( \mathfrak{p}' \) 整,
  由\cref{proposition-minimal-polynomial-coefficients-element-integral-over-ideal},
  \( y \) 极小多项式在 \( y \) 处赋值有
  \[
    y^m + a_1 y^{m - 1} + \cdots + a_m = 0,
  \]
  其中 \( a_i \in \mathfrak{p}' \).
  代入 \( y = bs \), 有
  \[
    s^m + (a_1 / b) s^{m - 1} + \cdots + (a_m / b^m) = 0,
  \]
  这也对应 \( s \) 极小多项式在 \( s \) 处赋值, 结合 \( s \) 在 \( A \)
  中整以及\cref{proposition-integral-element-minimal-polynomial-coefficients},
  知道 \( a_i / b^i \in A \).
  假设 \( b \notin \mathfrak{p}' \), 系数 \( a_i / b^i \in \mathfrak{p}' \)(否则
  \( a_i \notin \mathfrak{p}' \)), 因而 \( s^m \in \mathfrak{p}' B \subseteq
  \mathfrak{p} B \subseteq \mathfrak{q} \), 因此 \( s \in \mathfrak{q} \), 矛盾.
\end{proof}

\begin{corollary}
  假设 \( A \subseteq B \) 为整环, \( A \) 整闭且 \( B \) 在 \( A \) 上整. 如果
  \( \mathfrak{p}_1 \supset \cdots \supset \mathfrak{p}_n \) 为 \( A \)
  的素理想, 且令
  \[
    \mathfrak{q}_1 \supset \cdots \supset \mathfrak{q}_m \quad (m < n)
  \]
  为 \( B \) 的素理想使得对所有 \( i \) 都有 \( \mathfrak{q}_i \cap A =
  \mathfrak{p}_i \), 那么存在素理想链扩张
  \[
    \mathfrak{q}_1 \supseteq \cdots \supseteq \mathfrak{q}_n
  \]
  使得 \( \mathfrak{q}_i \cap A = \mathfrak{p} \) 对所有 \( i \) 成立.
\end{corollary}


\section{Krull 维数}

假设 \( A \) 是一个环.
\begin{enumerate}
  \item 给定 \( A \) 的一个素理想 \( \mathfrak{p} \).
    \( \mathfrak{p} \)
    的\emph{高度 \( \operatorname{ht}(\mathfrak{p}) \)} 定义为

    \[
      \sup \left\lbrace d: \mathfrak{p} = \mathfrak{p}_d \supsetneq
      \mathfrak{p}_{d - 1} \supsetneq \cdots \supsetneq \mathfrak{p}_0,
    \mathfrak{p}_i \text{为素理想} \right\rbrace
    \]
  \item \( A \) 的 \emph{Krull 维数} 定义为
    \[
      \sup \left\lbrace \operatorname{ht}(\mathfrak{p}): \mathfrak{p} \subseteq
      A, \text{其中} \mathfrak{p} \text{素} \right\rbrace.
    \]
\end{enumerate}

\begin{remark}
如果 \( A \) 是一个整环, 那么
\[
  \operatorname{dim} (A) = 0 \iff (0) \text{极大} \iff A \text{是一个域}.
\]
\end{remark}

\section{张量积}

\paragraph{张量积} 假设 \( A \) 是一个环, \( M, N \) 和 \( P \) 都是 \( A \)-模.
态射 \( \phi: M \times N \to P \) 称为 \emph{\( A \)-模双线性}, 如果
\[
  \begin{aligned}
    &\phi(x + x', y) = \phi(x, y) + \phi(x', y),     &x, x' \in M,& y \in N\\
    &\phi(x, y + y') = \phi(x, y) + \phi(x, y + y'), &x \in M,& y, y' \in N\\
    &\phi(ax, y) = a \phi(x, y), &a \in A, &x \in M, &y \in N\\
    &\phi(x, ay) = a \phi(x, y), &a \in A, &x \in M, &y \in N.
  \end{aligned}
\]
一个 \( A \)-模 \( T \) 带 \( A \)-模双线性映射
\[
  \phi: M \times N \to T
\]
称为 \emph{\( M \) 和 \( N \) 的张量积}, 如果它由下述泛性质: 对每个 \( A
\)-双线性映射
\[
  \phi': M \times N \to T'
\]
能被 \( \phi \) 唯一分解.
% https://q.uiver.app/#q=WzAsMyxbMCwwLCJNIFxcdGltZXMgTiJdLFsxLDAsIlQiXSxbMSwxLCJUJyJdLFswLDIsIlxccGhpJyIsMl0sWzAsMSwiXFxwaGkiXSxbMSwyLCJcXGV4aXN0cyAhIFxcdGV4dHvnur/mgKd9IiwwLHsic3R5bGUiOnsiYm9keSI6eyJuYW1lIjoiZGFzaGVkIn19fV1d
\[\begin{tikzcd}
	{M \times N} & T \\
	& {T'}
	\arrow["\phi", from=1-1, to=1-2]
	\arrow["{\phi'}"', from=1-1, to=2-2]
	\arrow["{\exists ! \text{线性}}", dashed, from=1-2, to=2-2]
\end{tikzcd}\]
我们将其记作 \( M \otimes_A N \), 注意到
\[
  \operatorname{Hom}_{A \text{线性}} (M \times N, T) \simeq
  \operatorname{Hom}_{A \text{线性}} (M \otimes_A N, T).
\]

\paragraph{张量积的构造} 假设 \( M \) 和 \( N \) 为 \( A \)-模, 令 \( A^{(M
\times N)} \) 为自由 \( A \)-模, 其基为 \( M \times N \), 因此每个 \( A^{(M
\times N)} \) 可以唯一表示为有限和
\[
  \sum a_i (x_i, y_i),\quad a_i \in A,\quad x_i \in M,\quad y_i \in N.
\]
假设 \( A^{(M \times N)} \) 的子模 \( P \) 由
\[
  \begin{aligned}
    &(x + x', y) - (x, y) - (x', y), &x, x' \in M, &y \in N\\
    &(x, y + y') - (x, y) - (x, y'), &x \in M, &y, y' \in N\\
    &(ax, y) - a(x, y), &a \in A, &x \in M, &y \in N\\
    &(x, ay) - a(x, y), &a \in A, &x \in M, &y \in N.
  \end{aligned}
\]
\( A^{(M \times N)} / P \) 即为 \( M \otimes_A N \).

\subsection{基本性质}

\begin{proposition}
  假设 \( M, N, P \) 为 \( A \)-模.
  \begin{enumerate}
    \item 存在唯一的同构
      \[
        \lambda: A \otimes M \to M
      \]
      对所有 \( a \in A, m \in M \) 有 \( \lambda(a \otimes m) = am  \).
    \item 存在唯一的同构
      \[
        \alpha: M \otimes (N \otimes P) \to (M \otimes N) \otimes P
      \]
      对所有 \( m \in M, n \in N, p \in P \) 有 \( \alpha(m \otimes (n \otimes
      p)) = (m \otimes n) \otimes p \).
    \item 存在唯一的同构
      \[
        \gamma: M \otimes N \to N \otimes M
      \]
      对所有 \( m \in M, n \in N \) 有 \( \gamma(m \otimes N) = n \otimes m \).
  \end{enumerate}
\end{proposition}

\paragraph{标量扩张} 假设 \( A \) 为交换环, \( B \) 为一个(未必交换的) \( A
\)-代数, 使得 \( A \to B \) 结构同态的像落于 \( B \) 的中心里. 因此
\[
  M \leadsto B \otimes_A M
\]
是一个左 \( A \)-模到左 \( B \)-模的函子. 假设 \( M \) 为 \( A \)-模, \( N \) 为
\( B \)-模, 那么
\[
  \begin{aligned}
    \operatorname{Hom}_{A-\text{线性}} (M, N) &\simeq
    \operatorname{Hom}_{B-\text{线性}} (B \otimes_A M, N)\\ \alpha &\mapsto
    \left(b \otimes m \mapsto b \cdot \alpha(m)\right)\\ \beta \circ \iota
    &\mapsfrom \beta.
  \end{aligned}
\]
如果 \( (e_{\alpha})_{\alpha \in I} \) 是 \( A \)-模 \( M \) 的一族生成元, 那么
\( (1 \otimes e_{\alpha})_{\alpha \in I} \) 为 \( B \)-模 \( B \otimes_A M \)
的一组生成元.

函子 \( M \leadsto M_B := B \otimes_A M \) 与张量积交换
\[
  (M \otimes_A N)_B \simeq M_B \otimes_B N_B.
\]
这是因为 %TODO: 证明结合律
\begin{align*}
  M_B &= (B \otimes_A M) \otimes_B (B \otimes_A N)\\ &\simeq((B \otimes_A M)
  \otimes_B B) \otimes_A N\\ &\simeq(B \otimes_A M) \otimes_A N\\ &\simeq B
  \otimes_A (M \otimes_A N) \\ &\simeq (M \otimes_A N)_B.
\end{align*}

\section{平坦性}
\begin{proposition}
  如果 \( M \) 是一个 \( A \)-模, \( 0 \to N' \xrightarrow{\alpha} N
  \xrightarrow{\beta} N'' \to 0 \) 是一个正合的 \( A \)-模列, 那么列
  \[
    M \otimes_A N' \xrightarrow{1 \otimes \alpha} M \otimes_A N \xrightarrow{1
    \otimes \beta} M \otimes_A N'' \to 0
  \]
  正合.
\end{proposition}
\begin{proof}
  %TODO: 证明
\end{proof}

\subsection{平坦与忠实平坦}

\begin{example}
  单射 \( N \to N' \) 诱导的映射 \( M \otimes_A N' \to M \otimes_A N \)
  \textbf{未必}是单射. 考虑正合列
  \[
    0 \to \mathbb{Z} \xrightarrow{\times m} \mathbb{Z} \to \mathbb{Z} / m
    \mathbb{Z} \to 0.
  \]
  函子 \( \bullet \otimes_{\mathbb{Z}} \mathbb{Z} / m \mathbb{Z} \) 下
  \( \mathbb{Z} / m \mathbb{Z} \xrightarrow{x \mapsto mx = 0} \mathbb{Z} / m
  \mathbb{Z} \) 为零态射, 并非单射.
\end{example}

\begin{example}
  \( M \) 和 \( N \) 均非零时, \( M \otimes_A N \) \textbf{未必} 非零. 例如
  \[
    \mathbb{Z} / 2 \mathbb{Z} \otimes_{\mathbb{Z}} \mathbb{Z} / 3 \mathbb{Z} =
    0,
  \]
  因此 \( a \otimes b = (3 - 2) a \otimes b = -2a \otimes b + a \otimes 3b = 0
  \). 实际上, 更特别地, \( M = N \neq 0 \) 时 \( M \otimes_A N \) 亦有可能为零,
  例如
  \[
    \mathbb{Q} / \mathbb{Z} \otimes_{\mathbb{Z}} \mathbb{Q} / \mathbb{Z} = 0.
  \]
\end{example}

一个 \( A \)-模 \( M \) 称为
\begin{enumerate}
  \item \emph{平坦}, 如果对任意 \( A \)-模 \( N, N' \)
    \[
      N' \to N \text{单} \implies M \otimes_A N' \to M \otimes_A N \text{单}.
    \]
  \item \emph{忠实平坦}, 如果对任意 \( A \)-模 \( N \)
    \[
      M \otimes_A N = 0 \implies N = 0.
    \]
\end{enumerate}
一个环同态 \( A \to B \) 称为\emph{平坦}(resp. \emph{忠实平坦}), 如果 \( B \)
作为 \( A \)-模平坦(resp. 忠实平坦).
\begin{proposition}
  假设 \( B \) 作为 \( A \)-代数是一个平坦模, 那么函子 \( F = \bullet \otimes_A
  B \) 是正合函子.
\end{proposition}
\begin{proof}
  给定正合列 \( A' \xrightarrow{f} A \xrightarrow{g} A'' \),
  可以得到下面斜对角正合的交换图
% https://q.uiver.app/#q=WzAsMTMsWzIsMiwiQSciXSxbMywzLCJcXG9wZXJhdG9ybmFtZXtJbX1mIl0sWzEsMSwiXFxvcGVyYXRvcm5hbWV7S2VyfWYiXSxbMCwwLCIwIl0sWzQsNCwiMCJdLFsyLDQsIjAiXSxbNCwyLCJBIl0sWzYsMiwiQScnIl0sWzUsMSwiXFxvcGVyYXRvcm5hbWV7SW19IGciXSxbNiwwLCIwIl0sWzcsMywiXFxvcGVyYXRvcm5hbWV7Q29rZXJ9ZyJdLFs4LDQsIjAiXSxbNCwwLCIwIl0sWzMsMl0sWzIsMF0sWzAsMV0sWzUsMV0sWzEsNF0sWzEsNl0sWzAsNiwiZiJdLFs2LDcsImciXSxbNiw4XSxbOCw5XSxbOCw3XSxbNywxMF0sWzEwLDExXSxbMTIsOF1d
\[\begin{tikzcd}[sep=tiny]
	0 &&&& 0 && 0 \\
	& {\operatorname{Ker}f} &&&& {\operatorname{Im} g} \\
	&& {A'} && A && {A''} \\
	&&& {\operatorname{Im}f} &&&& {\operatorname{Coker}g} \\
	&& 0 && 0 &&&& 0
	\arrow[from=1-1, to=2-2]
	\arrow[from=1-5, to=2-6]
	\arrow[from=2-2, to=3-3]
	\arrow[from=2-6, to=1-7]
	\arrow[from=2-6, to=3-7]
	\arrow["f", from=3-3, to=3-5]
	\arrow[from=3-3, to=4-4]
	\arrow[from=3-5, to=2-6]
	\arrow["g", from=3-5, to=3-7]
	\arrow[from=3-7, to=4-8]
	\arrow[from=4-4, to=3-5]
	\arrow[from=4-4, to=5-5]
	\arrow[from=4-8, to=5-9]
	\arrow[from=5-3, to=4-4]
\end{tikzcd}\]
  作用 \( F \), 得到
% https://q.uiver.app/#q=WzAsMTMsWzIsMiwiRihBJykiXSxbMywzLCJGKFxcb3BlcmF0b3JuYW1le0ltfWYpIl0sWzEsMSwiRihcXG9wZXJhdG9ybmFtZXtLZXJ9ZikiXSxbMCwwLCIwIl0sWzQsNCwiMCJdLFsyLDQsIjAiXSxbNCwyLCJGKEEpIl0sWzYsMiwiRihBJycpIl0sWzUsMSwiRihcXG9wZXJhdG9ybmFtZXtJbX0gZykiXSxbNiwwLCIwIl0sWzcsMywiRihcXG9wZXJhdG9ybmFtZXtDb2tlcn1nKSJdLFs4LDQsIjAiXSxbNCwwLCIwIl0sWzMsMl0sWzIsMF0sWzAsMV0sWzUsMV0sWzEsNF0sWzEsNl0sWzAsNiwiRihmKSJdLFs2LDcsIkYoZykiXSxbNiw4XSxbOCw5XSxbOCw3XSxbNywxMF0sWzEwLDExXSxbMTIsOF1d
\[\begin{tikzcd}[sep=tiny]
	0 &&&& 0 && 0 \\
	& {F(\operatorname{Ker}f)} &&&& {F(\operatorname{Im} g)} \\
	&& {F(A')} && {F(A)} && {F(A'')} \\
	&&& {F(\operatorname{Im}f)} &&&& {F(\operatorname{Coker}g)} \\
	&& 0 && 0 &&&& 0
	\arrow[from=1-1, to=2-2]
	\arrow[from=1-5, to=2-6]
	\arrow[from=2-2, to=3-3]
	\arrow[from=2-6, to=1-7]
	\arrow[from=2-6, to=3-7]
	\arrow["{F(f)}", from=3-3, to=3-5]
	\arrow[from=3-3, to=4-4]
	\arrow[from=3-5, to=2-6]
	\arrow["{F(g)}", from=3-5, to=3-7]
	\arrow[from=3-7, to=4-8]
	\arrow[from=4-4, to=3-5]
	\arrow[from=4-4, to=5-5]
	\arrow[from=4-8, to=5-9]
	\arrow[from=5-3, to=4-4]
\end{tikzcd}\]
其中斜对角链也都是正合的, 由交换性
\begin{align*}
  \operatorname{Im}F(f) &= \operatorname{Im} (F(A') \to F(\operatorname{Im}f) \to
  F(A))\\ &= \operatorname{Im} (F(\operatorname{Im}f) \to F(A)) \\ &=
  \operatorname{Ker} (F(A) \to F(\operatorname{Im}g)) \\ &=
  \operatorname{Ker}(F(A) \to F(\operatorname{Im}g) \to F(A'')) \\ &=
  \operatorname{Ker}F(g)
\end{align*}
第二个等号是因为 \( F(A') \to F(\operatorname{Im}f) \) 是满射, 第四个等号是因为
\( F(\operatorname{Im}g) \to F(A'') \) 是单射.
\end{proof}

\begin{proposition}
  假设 \( P \) 为忠实平坦 \( A \)-模. 那么 \( A \)-模链复形
  \[
    (N): \quad N' \xrightarrow{\alpha} N \xrightarrow{\beta} N''
  \]
  正合当且仅当
  \[
    P \otimes_A (N): \quad P \otimes_A N' \xrightarrow{1 \otimes \alpha} P
    \otimes_A N \xrightarrow{1 \otimes \beta} P \otimes_A N''
  \]
  正合.

  反过来, 如果
  \[
    (N) \text{正合} \iff P \otimes_A (N) \text{正合}
  \]
  那么 \( P \) 是一个忠实平坦 \( A \)-模.
\end{proposition}
\begin{proof}
  记 \( C = \operatorname{Ker} \beta / \operatorname{Im} \alpha \), 考虑正合列
  \[
    N' \xrightarrow{\alpha} \operatorname{Ker} (\beta) \to C \to 0
  \]
  如果 \( P \) 平坦, 可以验证 \( P \otimes_A \operatorname{Ker}\beta \simeq
  \operatorname{Ker}(1 \otimes \beta) \), 那么复形链
  \[
    P \otimes_A N' \xrightarrow{1 \otimes \alpha} \operatorname{Ker} (1 \otimes
    \beta) \to P \otimes_A C \to 0
  \]
  正合. 那么
  \[
    (N) \text{正合} \iff C = 0 \iff P \otimes_A C = 0 \iff P \otimes_A (N)
    \text{正合}.
  \]
\end{proof}

\begin{corollary}
  假设 \( A \to B \) 忠实平坦. 如果 \( M \otimes_A B \) 作为 \( B \)-模平坦
  (resp. 忠实平坦), 那么 \( A \)-模 \( M \) 平坦 (resp. 忠实平坦).
\end{corollary}
\begin{proof}
  考虑复形链 \( (N): N' \to N \to N'' \). 如果 \( M \otimes_A B \) 平坦, 那么
  \[
    (N) \text{正合} \iff B \otimes_A (N) \text{正合} \implies ((N) \otimes_A B)
    \otimes_B (M \otimes_A B) \text{正合},
  \]
  其中 \( ((N) \otimes_A B) \otimes_{B} (M \otimes_A B) \simeq ((N) \otimes_A
  M)\otimes_A B  \), 其正合当且仅当 \( (N) \otimes_A M \) 正合. 因此 \( M \)
  平坦. 如果 \( B \otimes_A M \) 忠实平坦, 那么 \( (N) \) 正合当且仅当 \( B
  \otimes_A M \) 正合, 从而 \( M \) 忠实平坦.
\end{proof}

\begin{corollary}
  假设 \( A \to B \) 忠实平坦, \( M \) 为 \( A \)-模. 如果 \( B \)-模 \( B
  \otimes_A M \) 有限生成, 那么 \( M \) 有限生成.
\end{corollary}
\begin{proof}
  假设 \( 1 \otimes m_1, \ldots, 1 \otimes m_r \) 生成 \( B \otimes_A M \), 记
  \( N \) 为由 \( m_i \) 生成的 \( M \)-子模. 那么正合列 \( N \to M \to M / N
  \to 0 \) 在张量 \( B \) 后保持正合. 而 \( B \otimes M / N = 0 \), 结合 \( B \)
  忠实平坦, 得到 \( M / N = 0 \).
\end{proof}

\begin{proposition}
  假设 \( i: A \to B \) 为忠实平坦同态. 对每个 \( A \)-模 \( M \), 列
  \[
    0 \to M \xrightarrow{d_0} B \otimes_A M \xrightarrow{d_1} B \otimes_A B
    \otimes_A M
  \]
  其中
  \[
    \begin{cases}
    d_0(m) = 1 \otimes m\\
    d_1(b \otimes m) = 1 \otimes b \otimes m - b \otimes 1 \otimes m
    \end{cases}
  \]
  正合.
\end{proposition}
\begin{proof}
  假设存在 \( i \) 的截面 \( f: B \to A \) 使得 \( f \circ i =
  \operatorname{id}_{A} \), 定义
  \begin{align*}
    &k_0: B \otimes_A M \to M,\quad k_0(b \otimes m) = f(b) m\\ &k_1: B
    \otimes_A B \otimes_A M \to B \otimes_A M,\quad k_1(b \otimes b' \otimes m)
    = f(b)b' \otimes m.
  \end{align*}
  那么 \( k_0 d_0 = \operatorname{id}_M \), 因此 \( d_0 \) 单. 此外,
  \[
    k_1 \circ d_1 + d_0 \circ k_0 = \operatorname{id}_{B \otimes_A M}
  \]
  这说明, 如果 \( d_1(x) = 0 \), 那么 \( x = d_0(k_0 (x)) \).

  一般地, 因为 \( A \to B \) 忠实平坦, 只需要证明欲证列张量 \( B \) 后是正合的,
  也就是
  \[
    0 \to B \otimes_A M \xrightarrow{1 \otimes d_0} B \otimes_A B \otimes_A M
    \xrightarrow{1 \otimes d_1} B \otimes_A B \otimes_A B \otimes_A M
  \]
  但 \( 1 \otimes d_0 \) 截面是容易构造的:
  \[
    f: B \otimes_A B \otimes_A M \to B \otimes_A M, b \otimes_A b' \otimes_A m
    \mapsto (bb') \otimes_A m.
  \]
\end{proof}

\begin{corollary}
  如果 \( A \to B \) 忠实平坦, 那么这个态射是单射, 且其像集为下面映射一致的元素.
  \[
    \begin{cases}
    b \mapsto 1 \otimes b\\
    b \mapsto b \otimes 1
    \end{cases}:\quad B \to B \otimes_A B.
  \]
\end{corollary}
\begin{proof}
  取 \( M = A \).
\end{proof}

\begin{proposition}
  \label{proposition-faithfully-flat-TFAE-conditions}
  假设 \( \varphi: A \to B \) 平坦, 那么以下条件等价:
  \begin{enumerate}
    \item \( \varphi \) 忠实平坦.
    \item 对每个 \( A \) 的极大理想 \( \mathfrak{m} \), 理想 \(
      \varphi(\mathfrak{m})B \neq B \).
    \item 每个 \( A \) 的极大理想 \( \mathfrak{m} \) 都是某个 \( B \)
      的极大理想的逆像 \( \varphi^{-1}(\mathfrak{n}) \).
  \end{enumerate}
\end{proposition}
\begin{proof}
  (1) \( \implies \) (2) 假设 \( \mathfrak{m} \) 为 \( A \) 的一个极大理想,
  那么由忠实平坦性,
  \[
    0 \neq B \otimes_A A/ \mathfrak{m} \simeq B/\varphi(\mathfrak{m})B.
  \]

  (2) \( \implies \) (3) \( \mathfrak{m}^e \neq B \) 说明 \( \mathfrak{m}^e \)
  包含在一个极大理想 \( \mathfrak{n} \) 中, 于是由 \( \mathfrak{m} \) 的极大性
  \( \mathfrak{n}^c = \mathfrak{m} \).

  (3) \( \implies \) (1) 假设 \( M \) 是一个非零 \( A \)-模, 下面证明 \( B
  \otimes_A M \neq 0 \).
  设 \( x \in M \) 为非零元, 那么 \( Ax \) 为 \( M \) 的一个子模, 且有 \(  Ax
  \simeq A / \operatorname{ann} (x) \).
  再设 \( \mathfrak{m} \) 是包含 \( \operatorname{ann} (x) \) 的一个极大理想,
  那么由(3) 存在 \( B \) 的极大理想 \( \mathfrak{n} \) 使得 \(
  \varphi^{-1}(\mathfrak{n}) = \mathfrak{m} \).
  由于 \( \varphi \) 平坦, \( B \otimes_A Ax \simeq B/\varphi(\mathfrak{m})B \)
  可视为 \( B \otimes_A M \) 子模, 而 \( \varphi(\mathfrak{a})B \subset
  \varphi(\mathfrak{m})B \subset \mathfrak{n} \) 也就是 \( B \otimes_A Ax \neq 0
  \), 故 \( B \otimes_A M \neq 0 \).
\end{proof}

\begin{proposition}
  \label{proposition-faithfully-flat-homomorphism-induce-surjective-spectrum-morphism}
  假设 \( \varphi: A \to B \) 是一个忠实平坦同态, 那么 \( {}^a \varphi:
  \operatorname{Spec} B \to \operatorname{Spec} A \) 是满射.
\end{proposition}
\begin{proof}
  对任意 \( \mathfrak{p} \in \operatorname{Spec} A \), \( B \otimes_A
  \kappa(\mathfrak{p}) \neq 0 \).
\end{proof}

\begin{proposition}
  假设 \( A \to B \) 是一个平坦同态, \( \mathfrak{p}', \mathfrak{p} \) 为 \( A
  \) 中的素理想, \( \mathfrak{q} \) 为 \( B \) 中的素理想, 使得 \( \mathfrak{p}'
  \subset \mathfrak{p}, \mathfrak{q}^c = \mathfrak{p} \), 那么存在 \( B \)
  中的素理想 \( \mathfrak{q}' \) 使得 \( \mathfrak{q}' \subset \mathfrak{q},
  \mathfrak{q}'^c = \mathfrak{p}' \).
\end{proposition}
\begin{proof}
  由 \( A \to B \) 平坦性, 可以知道 \( A_{\mathfrak{p}} \to B_{\mathfrak{q}} \)
  的平坦性.
  \( A_{\mathfrak{p}} \) 极大理想只有 \( \mathfrak{p} A_{\mathfrak{p}} \), 而 \(
  (\mathfrak{q} B_{\mathfrak{q}})^c = \mathfrak{p} A_{\mathfrak{p}} \),
  由\cref{proposition-faithfully-flat-TFAE-conditions} 知道 \( A_{\mathfrak{p}}
  \to B_{\mathfrak{q}} \) 忠实平坦.
  由\cref{proposition-faithfully-flat-homomorphism-induce-surjective-spectrum-morphism},
  存在 \( (\mathfrak{q}'B_{\mathfrak{q}})^c = \mathfrak{p}' A_{\mathfrak{p}} \),
  于是 \( \mathfrak{q}'^c = \mathfrak{p}' \).
\end{proof}

\section{有向极限与逆极限}

\subsection{基本定义以及构造}

\paragraph{有向极限} 给定一个预序集范畴 \( (I, \leq) \), 给定范畴 \( \mathcal{M}
\) 其对象集是模 \( (M_i)_{i \in I} \), 态射集为 \( \left\lbrace \alpha^i_j: M_i
\to M_j: i \leq j \right\rbrace \), 我们称模 \( \varinjlim M_k \) 为逗号范畴 \(
(\alpha / \Delta) := (j_\alpha / \Delta) \)
\[
  \mathbf{1} \xrightarrow{\alpha} \mathcal{M}^I = \operatorname{Fct}(I,
  \mathcal{M}) \xleftarrow{\Delta} \mathcal{M}
\]
的始对象为 \emph{有向极限}, 其中 \( \mathbf{1} \)
为恰有一个对象和一个态射的范畴, \( \alpha \) 为函子 \( 1 \mapsto \left( (i
\mapsto j) \mapsto (M_i \to M_j) \right) \) , \( \Delta \)
为\href{https://en.wikipedia.org/wiki/Diagonal_functor}{对角函子}. 换句话说, \(
\varinjlim M_k \) 满足下面的交换图.
% https://q.uiver.app/#q=WzAsNCxbMCwwLCJNX2kiXSxbMiwwLCJNX2oiXSxbMSwxLCJcXHZhcmluamxpbSBNX2siXSxbMSwyLCJOIl0sWzAsMywiXFxiZXRhX2kiLDJdLFswLDEsIlxcYWxwaGFeaV9qIl0sWzAsMiwiXFxhbHBoYV9pIl0sWzEsMiwiXFxhbHBoYV9qIiwyXSxbMSwzLCJcXGJldGFfaiJdLFsyLDMsIlxcZXhpc3RzISIsMSx7InN0eWxlIjp7ImJvZHkiOnsibmFtZSI6ImRhc2hlZCJ9fX1dXQ==
\[\begin{tikzcd}
	{M_i} && {M_j} \\
	& {\varinjlim M_k} \\
	& N
	\arrow["{\alpha^i_j}", from=1-1, to=1-3]
	\arrow["{\alpha_i}", from=1-1, to=2-2]
	\arrow["{\beta_i}"', from=1-1, to=3-2]
	\arrow["{\alpha_j}"', from=1-3, to=2-2]
	\arrow["{\beta_j}", from=1-3, to=3-2]
	\arrow["{\exists!}"{description}, dashed, from=2-2, to=3-2]
\end{tikzcd}\]

\paragraph{有向极限的构造} 考察 \( M_i \) 的直积 \( \bigoplus_{i \in I} M_i \),
那么可以将 \( M_{i_0} \) 视为 \( \bigoplus_{i \in I} M_i \) 的子模, 后者对所有
\( i \neq i_0 \) 都满足 \( m_i = 0 \). 取 \( M \) 为 \( \bigoplus_{i \in I}
M_i \) 商去元素
\[
  m_i - \alpha^i_j(m_i),\quad m_i \in M_i,\quad i < j
\]
生成的子模即为所求.

\paragraph{逆极限}
类似地, 沿用上面的 \( I \), \( \mathcal{M} \) 对象集为 \( (M_i)_{i \in I} \),
态射集 \( \left\lbrace p^i_j: M_j \to M_i: i \leq j \right\rbrace \), 我们称模
\( \varprojlim M_k \) 为逗号范畴 \( (\Delta / \beta) := (\Delta / j_\beta) \)
\[
  \mathcal{M} \xrightarrow{\Delta} \mathcal{M}^{I^{\operatorname{op}}}
  \xleftarrow{j_\beta} \mathbf{1}
\]
的终对象为 \emph{逆极限}. 换句话说, \( \varprojlim M_k \) 满足下面交换图.
% https://q.uiver.app/#q=WzAsNCxbMCwwLCJNX2kiXSxbMiwwLCJNX2oiXSxbMSwxLCJcXHZhcnByb2psaW0gTV9rIl0sWzEsMiwiTiJdLFszLDIsIlxcZXhpc3RzISIsMSx7InN0eWxlIjp7ImJvZHkiOnsibmFtZSI6ImRhc2hlZCJ9fX1dLFszLDFdLFszLDBdLFsxLDAsInBeaV9qIiwyXSxbMiwwLCJwX2kiLDJdLFsyLDEsInBfaiJdXQ==
\[\begin{tikzcd}
	{M_i} && {M_j} \\
	& {\varprojlim M_k} \\
	& N
	\arrow["{p^i_j}"', from=1-3, to=1-1]
	\arrow["{p_i}"', from=2-2, to=1-1]
	\arrow["{p_j}", from=2-2, to=1-3]
	\arrow[from=3-2, to=1-1]
	\arrow[from=3-2, to=1-3]
	\arrow["{\exists!}"{description}, dashed, from=3-2, to=2-2]
\end{tikzcd}\]
特别地, 我们常取 \( I = \mathbb{N} \).

\paragraph{逆极限的构造} 给定义 \( A \)-模逆向系统 \( (M_n, \alpha_n) \),
我们定义 \( \varprojlim M_n \) 和 \( \varprojlim^1 M_n \) 为 \( A
\)-模同态的核与余核
\[
  \prod M_n \to \prod M_n,\quad (\ldots, m_n, \ldots) \mapsto (\ldots, m_n -
  \alpha_n(m_{n + 1}), \ldots)
\]

\subsection{基本性质}

\begin{proposition}
  对环 \( A \) 的每个乘性子集 \( S \), 有 \( S^{-1} A \simeq \varinjlim A_h \),
  其中 \( h \) 跑遍 \( S \), 偏序关系为整除关系.
\end{proposition}
\begin{proof}
  如果在 \( A \) 中有 \( h' = hq \), 那么 \( h \) 在 \( A_{h'} \) 中为单位,
  由局部化的泛性质存在唯一的同态 \( A_h \to A_{h'},\quad \frac{a}{h} \mapsto
  \frac{aq}{h'} \), 由此可以构造 \( \varinjlim A_h \). 又由有自然的 \( A_h \to
  S^{-1} A \) 以及有向极限的泛性质知道存在 \( \varinjlim A_h \to S^{-1}A \).
  又由有向极限的构造知道存在 \( S^{-1} A \to \varinjlim A_h \).
  对比这两个箭头知道, 它们互逆.
\end{proof}

\begin{proposition}
  假设 \( \left(M_i, \alpha^i_j\right), \left(N_i, \beta^i_j\right) \) 和 \(
  \left(P_i, \gamma^i_j\right) \) 为相对于有序集 \( I \) 的有向系统, 令
  \[
    \left(M_i, \alpha^i_j\right) \xrightarrow{(a_i)} \left(N_i, \beta^i_j\right)
    \xrightarrow{(b_i)} \left(P_i, \gamma^i_j\right)
  \]
  为有向系统列. 如果列
  \[
    M_i \xrightarrow{a_i} N_i \xrightarrow{b_i} P_i
  \]
  对所有 \( i \) 均正合, 那么
  \[
    \varinjlim M_i \xrightarrow{\varinjlim a_i} \varinjlim N_i
    \xrightarrow{\varinjlim b_i} \varinjlim P_i
  \]
  正合.
\end{proposition}
\begin{proof}
  %TODO: 完成此证明.
\end{proof}

\begin{proposition}
  假设对每个逆向系统 \( (M_n, \alpha_n) \) 和 \( A \)-模 \( N \)
  \[
    \operatorname{Hom}(\varprojlim M_n, N) \simeq \varprojlim
    \operatorname{Hom}(M_n, N)
  \]
\end{proposition}
\begin{proof}
  函子 \( \operatorname{Hom}(\bullet, N) \) 将逆向系统 \( (M_n, \alpha_n) \)
  映到逆向系统 \( \left(\operatorname{Hom}(M_n, N), (\alpha_n)_*\right) \). 映射
  \( \varprojlim M_n \to M_i \) 诱导了 \( \operatorname{Hom}(\varprojlim M_n, N)
  \to \operatorname{Hom}(M_i, N) \), 进而由逆极限的泛性质, 有态射
  \[
    \operatorname{Hom}(\varprojlim M_n, N) \to \varprojlim
    \operatorname{Hom}(M_n, N)
  \]
  反过来, 取 \( (f_i) \in \varprojlim \operatorname{Hom}(M_n, N) \), 取 \(
  (m_i) \in \varprojlim M_n \), \( (f_i) \) 诱导了 \( (m_i) \mapsto f_1(m_1) \).
  验证两者互逆即可.
\end{proof}

\begin{proposition}
  逆向系统列
  \[
    0 \to (M_n, \alpha_n) \to (N_n, \beta_n) \to (P_n, \gamma_n) \to 0
  \]
  诱导了正合列
  \[
    0 \to \varprojlim M_n \to \varprojlim N_n \to \varprojlim P_n \to
    \varprojlim\nolimits^1 M_n \to \varprojlim\nolimits^1 N_n \to
    \varprojlim\nolimits^1 P_n \to 0
  \]
\end{proposition}
\begin{proof}
  考察
  % https://q.uiver.app/#q=WzAsMTQsWzAsMiwiMCJdLFsxLDIsIlxccHJvZCBNX24iXSxbMSwxLCJcXHByb2QgTV9uIl0sWzIsMSwiXFxwcm9kIE5fbiJdLFsyLDIsIlxccHJvZCBOX24iXSxbMywxLCJcXHByb2QgUF9uIl0sWzMsMiwiXFxwcm9kIFBfbiJdLFs0LDEsIjAiXSxbMSwwLCJcXHZhcnByb2psaW0gTV9uIl0sWzEsMywiXFx2YXJwcm9qbGltXjEgTV9uIl0sWzIsMywiXFx2YXJwcm9qbGltXjEgTl9uIl0sWzMsMywiXFx2YXJwcm9qbGltXjEgTl9uIl0sWzIsMCwiXFx2YXJwcm9qbGltIE5fbiJdLFszLDAsIlxcdmFycHJvamxpbSBQX24iXSxbMCwxXSxbMiwxXSxbMiwzXSxbMSw0XSxbMyw0XSxbMyw1XSxbNCw2XSxbNSw2XSxbNSw3XSxbOCwyXSxbMTIsM10sWzEzLDVdLFs4LDEyLCIiLDEseyJzdHlsZSI6eyJib2R5Ijp7Im5hbWUiOiJkYXNoZWQifX19XSxbMTIsMTMsIiIsMSx7InN0eWxlIjp7ImJvZHkiOnsibmFtZSI6ImRhc2hlZCJ9fX1dLFs5LDEwLCIiLDEseyJzdHlsZSI6eyJib2R5Ijp7Im5hbWUiOiJkYXNoZWQifX19XSxbMTAsMTEsIiIsMSx7InN0eWxlIjp7ImJvZHkiOnsibmFtZSI6ImRhc2hlZCJ9fX1dLFsxMyw5LCIiLDEseyJzdHlsZSI6eyJib2R5Ijp7Im5hbWUiOiJkYXNoZWQifX19XSxbMSw5XSxbNCwxMF0sWzYsMTFdXQ==
  \[\begin{tikzcd}
    & {\varprojlim M_n} & {\varprojlim N_n} & {\varprojlim P_n} \\
    & {\prod M_n} & {\prod N_n} & {\prod P_n} & 0 \\
    0 & {\prod M_n} & {\prod N_n} & {\prod P_n} \\
    & {\varprojlim^1 M_n} & {\varprojlim^1 N_n} & {\varprojlim^1 N_n}
    \arrow[dashed, from=1-2, to=1-3]
    \arrow[from=1-2, to=2-2]
    \arrow[dashed, from=1-3, to=1-4]
    \arrow[from=1-3, to=2-3]
    \arrow[from=1-4, to=2-4]
    \arrow[dashed, from=1-4, to=4-2]
    \arrow[from=2-2, to=2-3]
    \arrow[from=2-2, to=3-2]
    \arrow[from=2-3, to=2-4]
    \arrow[from=2-3, to=3-3]
    \arrow[from=2-4, to=2-5]
    \arrow[from=2-4, to=3-4]
    \arrow[from=3-1, to=3-2]
    \arrow[from=3-2, to=3-3]
    \arrow[from=3-2, to=4-2]
    \arrow[from=3-3, to=3-4]
    \arrow[from=3-3, to=4-3]
    \arrow[from=3-4, to=4-4]
    \arrow[dashed, from=4-2, to=4-3]
    \arrow[dashed, from=4-3, to=4-4]
  \end{tikzcd}\]
  利用 \href{https://en.wikipedia.org/wiki/Snake_lemma}{蛇形引理} 即可.
\end{proof}

\begin{corollary}
  如果映射 \( \alpha_n: M_{n + 1} \to M_n \) 都是满射的, 那么 \( \varprojlim^1
  M_n = 0 \).
\end{corollary}
\begin{proof}
  %TODO: 证明.
\end{proof}

\paragraph{有向极限}
\begin{proposition}
  有向极限和张量积是交换的
  \[
    \varinjlim_{i \in I} M_i \otimes_A \varinjlim_{j \in J} N_j \simeq
    \varinjlim_{(i, j) \in I \times J} M_i \otimes_A N_j.
  \]
\end{proposition}
\begin{proof}
  %TODO: 完成此证明
  % 由张量积的泛性质, \( \left((m_i), (n_j)\right) \mapsto \left(m_i \otimes_A
  % n_j \right) \) 诱导了 \( (m_i) \otimes_A (n_j) \mapsto \left(m_i \otimes_A
  % n_j \right) \). 反过来, 在由张量积的泛性质 \( (m_i, n_j) \mapsto m_i
  % \otimes_A n_j \), 其中
\end{proof}

\subsection{代数张量积, 张量代数以及对称代数}
\paragraph{代数张量积} 假设 \( k \) 是一个环, \( A \) 和 \( B \) 为 \( k
\)-代数. 我们称 \( k \) 代数 \( C \) 和同态 \( i: A \to C \) 和 \( j: B \to C \)
称为 \emph{\( A \) 和 \( B \) 的张量积}, 如果其满足下述泛性质: 对每一对 \( k
\)-代数同态 \( f: A \to R \) 和 \( g: B \to R \), 存在唯一的同态 \( (f, g): C
\to R \) 使得 \( (f, g) \circ i = \alpha \) 和 \( (f, g) \circ j = \beta \).
% https://q.uiver.app/#q=WzAsNCxbMCwwLCJBIl0sWzEsMCwiQyJdLFsxLDEsIlIiXSxbMiwwLCJCIl0sWzMsMiwiZyJdLFswLDIsImYiLDJdLFswLDEsImkiXSxbMywxLCJqIiwyXSxbMSwyLCJcXGV4aXN0cyAhIChmLCBnKSIsMSx7InN0eWxlIjp7ImJvZHkiOnsibmFtZSI6ImRhc2hlZCJ9fX1dXQ==
\[\begin{tikzcd}
	A & C & B \\
	& R
	\arrow["i", from=1-1, to=1-2]
	\arrow["f"', from=1-1, to=2-2]
	\arrow["{\exists ! (f, g)}"{description}, dashed, from=1-2, to=2-2]
	\arrow["j"', from=1-3, to=1-2]
	\arrow["g", from=1-3, to=2-2]
\end{tikzcd}\]
其同构意义上是唯一的, 记作 \( A \otimes_k B \). 这个泛性质告诉我们
\[
  \operatorname{Hom}(A \otimes_k B, R) \simeq \operatorname{Hom}(A, R) \times
  \operatorname{Hom}(B, R).
\]

\paragraph{张量积的构造} 首先将 \( A \) 和 \( B \) 视为 \( k \)-模,
形成了模张量积 \( A \otimes_k B \), 那么存在乘法 \( A \otimes_k B \times A
\otimes_k B \to A \otimes_k B \) 使得
\[
  (a \otimes b)(a' \otimes b') = a a' \otimes aa' \otimes aa',\quad a, a \in
  A,\quad b, b' \in B.
\]
这使得 \( A \otimes_k B \) 成为一个环, 态射
\[
  c \mapsto c (1 \otimes 1) = c \otimes 1 = 1 \otimes c
\]
使得其变为一个 \( k \)-代数. 映射 \( i, j \) 构造如下.
\[
  i: A \to A \otimes_k B, a \mapsto a \otimes 1 \text{ 和 } j: B \to A \otimes_k
  B \to B, b \mapsto 1 \otimes b
\]

\paragraph{张量代数} 假设 \( M \) 是 \( A \)-模. 对每一个 \( A \neq 0 \), 置
\[
  T^r M = \underbrace{M \otimes_A \cdots \otimes_A M}_{r \text{个}}
\]
特别地, \( T^0 M = A \) 以及 \( T^1 M = M \). 定义
\[
  TM = \bigoplus_{r \geq 0} T^r M.
\]
这个模可以赋予一个非交换 \( A \)-代数, 称为 \emph{\( M \) 的张量代数},
其乘法如:
\begin{align*}
  T^r M \times T^s M &\to T^{r + s}M\\ (m_1 \otimes \cdots \otimes m_r, m_{r +
  1} \otimes \cdots \otimes m_{r + s}) &\mapsto m_1 \otimes \cdots \otimes m_{r
  + s}.
\end{align*}
对 \( (TM, M \to TM) \) 由下述泛性质: 由 \( M \) 到(未必交换的) \( A \)-代数 \(
R \) 的 \( A \)-线性映射能唯一分解为 \( A \)-代数同态 \( TM \to R \).
% https://q.uiver.app/#q=WzAsMyxbMCwwLCJNIl0sWzEsMCwiVE0iXSxbMSwxLCJSIl0sWzAsMiwiQVxcdGV4dHst57q/5oCnfSIsMl0sWzAsMV0sWzEsMiwiXFxleGlzdHMhQVxcdGV4dHst5Luj5pWwfSIsMCx7InN0eWxlIjp7ImJvZHkiOnsibmFtZSI6ImRhc2hlZCJ9fX1dXQ==
\[\begin{tikzcd}
	M & TM \\
	& R
	\arrow[from=1-1, to=1-2]
	\arrow["{A\text{-线性}}"', from=1-1, to=2-2]
	\arrow["{\exists!A\text{-代数}}", dashed, from=1-2, to=2-2]
\end{tikzcd}\]
如果 \( M \) 是一个基为 \( x_1, \ldots, x_n \) 自由 \( A \)-模, 那么 \( TM \)
是一个 \( A \) 的非交换记号 \( x_i \) 上的非交换多项式环.

\paragraph{对称代数} 一个 \( A \)-模 \( M \) 称为 \emph{对称代数 \(
\operatorname{Sym}(M) \)} 为 \( TM \) 商去由下面 \( T^2 M \) 元素生成的理想
\[
  m \otimes n - n \otimes m, \quad m, n \in M.
\]
这是分次代数 \( \operatorname{Sym}(M) = \bigoplus_{r \geq 0}
\operatorname{Sym}^r (M) \), 其中 \( \operatorname{Sym}^{r}(M) \) 为 \(
M^{\otimes r} \) 商去下面元素生成的理想
\[
  m_1 \otimes \cdots \otimes m_r - m_{\sigma(1)} \otimes \cdots \otimes
  m_{\sigma(r)},\quad m_i \in M,\quad \sigma \in \operatorname{Sym}(r)
\]
对 \( (\operatorname{Sym}(M), M \to \operatorname{Sym}(M)) \) 有下述泛性质:
对每个由 \( M \) 到一个交换的 \( A \)-代数 \( M \) 的 \( A \)-线性映射 \( M \to
R \) 能唯一分解为 \( A \)-代数同态 \( \operatorname{Sym}(M) \to R \).
% https://q.uiver.app/#q=WzAsMyxbMCwwLCJNIl0sWzEsMCwiXFxvcGVyYXRvcm5hbWV7U3ltfShNKSJdLFsxLDEsIlIiXSxbMCwyLCJBXFx0ZXh0ey3nur/mgKd9IiwyXSxbMCwxXSxbMSwyLCJcXGV4aXN0cyFBXFx0ZXh0ey3ku6PmlbB9IiwwLHsic3R5bGUiOnsiYm9keSI6eyJuYW1lIjoiZGFzaGVkIn19fV1d
\[\begin{tikzcd}
	M & {\operatorname{Sym}(M)} \\
	& R
	\arrow[from=1-1, to=1-2]
	\arrow["{A\text{-线性}}"', from=1-1, to=2-2]
	\arrow["{\exists!A\text{-代数}}", dashed, from=1-2, to=2-2]
\end{tikzcd}\]
如果 \( M \) 是一个基为 \( x_1, \ldots, x_n \) 自由 \( A \)-模, 那么 \(
\operatorname{Sym}M \) 是一个 \( A \) 的交换记号 \( x_i \) 上的交换多项式环.
