\chapter{生成代数}

\section{生成模}

\begin{proposition}
  \label{proposition-exact-squence-middle-submodules-equality}
  假设 \( A \) 是一个环且
  \[
    0 \to M' \xrightarrow{\alpha} M \xrightarrow{\beta} M'' \to 0
  \]
  是一个 \( A \)-模正合列, 那么
  \begin{enumerate}
    \item 如果 \( N, P \) 是 \( M \) 的子模, 使得
      \[
        N \subset P,\quad \alpha(M') \cap N = \alpha(M') \cap P,\quad \beta(N) =
        \beta(P)
      \]
      那么 \( N = P \).
    \item 如果 \( M' \) 和 \( M'' \) 都是有限生成的, 那么 \( M \)
      亦是有限生成的.
  \end{enumerate}
\end{proposition}
\begin{proof}
  (1) \( P \subset N \)
  设 \( p \in P \), 那么存在 \( n \in N \) 使得 \( \beta(n) = \beta(p) \), 所以
  \( p - n = \operatorname{ker} p = \operatorname{im} \alpha \).
  而 \( p - n \in P \), 又由条件 \( p - n \in N \). 因此 \( p = (p - n) + n
  \in N \).

  (2) 假设 \( S', S'' \) 分别为 \( M', M \) 的有限集, 使得 \( \left\langle {S'}
  \right\rangle = M',\; \left\langle \beta (S'') \right\rangle = M'' \).
  设 \( N = \alpha(S') \cup S'' \) 由 (1) 可以知道 \( N = M \).
  而 \( \alpha(S') \cup S'' \) 是一个有限集, 于是 \( M \) 有限生成.
\end{proof}

\begin{lemma}
  每个有限生成理想的生成元集包含一个有限的生成元集.
\end{lemma}
\begin{proof}
  令 \( S \) 为理想 \( \mathfrak{a} \) 的一个生成元集, \( \mathfrak{a} \)
  由有限集 \( \left\lbrace a_1, \ldots, a_n \right\rbrace \) 生成. 那么 \( a_i
  \) 由 \( S \) 的有限子集 \( S_i \) 生成.
  于是 \( \bigcup S_i \) 有限且生成 \( \mathfrak{a} \).
\end{proof}

\subsection{有限生成模与局部化}

\begin{proposition}
  假设 \( M \) 是一个有限生成 \( A \)-模.
  如果 \( S^{-1} M = 0 \), 那么存在 \( h \in S \) 使得 \( M_h = 0 \).
\end{proposition}

\section{Noetherian 环}

\begin{proposition}
  \label{proposition-Noetherian-ring-iff-condition}
  给定一个环 \( A \), 下面条件等价
  \begin{enumerate}
    \item \( A \) 的每个理想都是有限生成的.
    \item \( A \) 的每个理想上升链 \( \mathfrak{a}_1 \subseteq \mathfrak{a}_2 \subseteq
      \cdots \) 都将终止, 也就是存在 \( m \) 使得
      \[
        \mathfrak{a}_m = \mathfrak{a}_{m + 1} = \cdots.
      \]
    \item \( A \) 的任何非空的理想集都存在一个极大元.
  \end{enumerate}
  一个环 \( A \) 称为是\emph{noetherian的}, 如果 \( A \) 满足上面等价条件.
\end{proposition}
\begin{proof}
  (1) \( \implies \) (2)
  如果 \( \mathfrak{a}_1 \subseteq \mathfrak{a}_2 \subseteq \cdots \)
  是一个上升链, 那么 \( \mathfrak{a} = \bigcup \mathfrak{a}_i \) 是一个理想.
  由假设, \( \mathfrak{a} \) 有有限个生成元, 设其为 \( a_1, \ldots, a_n \).
  因此, 能找到一个 \( m \in \mathbb{N} \) 使得所有 \( a_i \in \mathfrak{a}_m \)
  成立, 因此
  \[
    \mathfrak{a}_m = \mathfrak{a}_{m + 1} = \cdots = \mathfrak{a}.
  \]

  (2) \( \implies \) (3) 如果 \( \Sigma \) 没有极大元,
  那么能找到一个无穷的上升链, 与假设矛盾.

  (3) \( \implies \) (1) 假设 \( \mathfrak{a} \) 是 \( A \) 的任意一个理想.
  设 \( \Sigma \) 为所有包含于 \( \mathfrak{a} \) 的有限生成理想组成的集合.
  由假设, \( \Sigma \) 有一个极大元 \( \mathfrak{m} \), 下面证明 \( \mathfrak{m}
  = \mathfrak{a} \), 于是 \( \mathfrak{a} \) 有限生成.
  反之不然, 假设 \( \mathfrak{a} \neq \mathfrak{m} \).
  那么存在 \( x \in \mathfrak{a} \setminus \mathfrak{m} \), 但 \( \mathfrak{m}
  \subsetneq \left( x, \mathfrak{m} \right) \) 与极大性矛盾.
\end{proof}

\begin{remark}
  \label{remark-noetherian-ring}
  \begin{enumerate}
    \item 主理想整环和域是 noetherian 的.
    \item noetherian 环的商自然是 noetherian 环.
  \end{enumerate}
\end{remark}

\begin{proposition}
  \label{proposition-Noetherian-module-iff-condition}
  假设 \( A \) 是一个环. 那么下面条件对 \( A \)-模 \( M \) 是等价的
  \begin{enumerate}
    \item 每个 \( M \) 的子模是有限生成的.
    \item 每个 \( M \) 的子模上升链是终止的.
    \item 每个 \( M \) 的非空子模集族有极大元.
  \end{enumerate}
  一个 \( A \)-模 \( M \) 是\emph{noetherian}的, 如果其满足上面等价条件.
\end{proposition}

\subsection{Noetherian 环的基本性质}

\begin{proposition}
  \label{proposition-exact-sequence-noetherian-module}
  假设 \( A \) 是一个环且
  \[
    0 \to M' \xrightarrow{\alpha} M \xrightarrow{\beta} M'' \to 0
  \]
  是一个 \( A \)-模正合列, 那么 \( M \) 是 noetherian 的当且仅当 \( M' \) 和 \(
  M'' \) 都是 noetherian 的.
  特别地, \( M_1 \oplus M_2 \) noetherian 当且仅当 \( M_1 \) 和 \( M_2 \) 都
  noetherian.
\end{proposition}
\begin{proof}
  只证明\( \impliedby \).
  设 \( \mathfrak{a}_1 \subset \mathfrak{a}_2 \subset \cdots \) 为 \( M \)
  的一个子模升链.
  由假设 \( \alpha(M') \cap \mathfrak{a}_1 \subset \alpha(M') \cap
  \mathfrak{a}_2 \subset \cdots , \beta(\mathfrak{a}_1) \subset
  \beta(\mathfrak{a}_2) \subset \cdots \) 终止, 于是可以设 \( m, n \in
  \mathbb{N} \) 使得 \( \alpha(M') \cap \mathfrak{a}_m = \alpha(M') \cap
  \mathfrak{a}_{m + 1} = \cdots, \beta(\mathfrak{a}_n) = \beta(\mathfrak{a}_{n +
  1}) = \cdots \).
  取 \( k = \max(m, n) \),
  那么由\cref{proposition-exact-squence-middle-submodules-equality} \(
  \mathfrak{a}_{k} = \mathfrak{a}_{k + 1} = \cdots \).
\end{proof}

\begin{proposition}
  \label{proposition-finite-generated-module-over-noetherian-ring}
  每个 noetherian 环 \( A \) 上的有限生成模 \( M \) 是 noetherian 的.
\end{proposition}
\begin{proof}
  对 \( M \) 的极小生成元个数 \( n \) 进行归纳.
  如果 \( n = 1 \), 那么由\cref{remark-noetherian-ring} \( M \simeq
  A/\mathfrak{a} \) noetherian 其中 \( \mathfrak{a} \) 为 \( A \) 的理想.
  如果 \( n > 1 \), \( M = \left\langle a_1,
  \ldots, a_n \right\rangle \) 有一个子模 \( N = \left\langle a_1, \ldots, a_{n
  - 1} \right\rangle \).
  由归纳假设正合列
  \[
    0 \to N \to M \to M / N \to 0
  \]
  中 \( N, M/N \) noetherian,
  应用\cref{proposition-exact-sequence-noetherian-module} \( M \) noetherian.
\end{proof}

\paragraph{Hilbert 基定理}

\begin{theorem}[Hilbert基]
  \label{theorem-Hilbert-basis}
  每个 noetherian 环上的有限生成代数 noetherian.
\end{theorem}

\subsection{Noetherian 模的伴随素理想}

假设 \( M \) 是一个 noetherian 环 \( A \) 上的有限生成模. 给定 \( x \in M \),
我们称 \( A \) 的理想
\[
  \operatorname{ann}(x) = \left\lbrace a \in A: ax = 0 \right\rbrace
\]
为 \( x \) 的\emph{零化子}.

\begin{lemma}
  假设 \( M \) 是一个 noetherian 环 \( A \) 上的有限生成模. \( \left\lbrace
  \operatorname{ann}(x) : x \in M \right\rbrace \) 的极大元是素理想.
\end{lemma}
\begin{proof}
  \( A \) noetherian, 故理想集族有极大元. 如果 \( ab \in
  \mathfrak{a} \), 那么 \( abx = 0 \), 只能 \( ax = 0 \) 或 \( bx = 0 \), 否则
  有 \( \operatorname{ann}(x) \subsetneq \operatorname{ann}(ax) \),
  这与极大性矛盾.
\end{proof}

\begin{proposition}
  Noetherian 环 \( A \) 上的有限生成模 \( M \) 有一个有限子模链
  \[
    M \supseteq M_r \supseteq \cdots \supseteq M_1 \supseteq 0
  \]
  使得每个商 \( M_i / M_{i - 1} \) 同构于 \( A / \mathfrak{p}_i \), 其中 \(
  \mathfrak{p}_i \) 为一些素理想.
\end{proposition}
\begin{proof}
  对每个 \( x \in M \), \( M \) 的子模 \( Ax \) 同构于 \( A /
  \operatorname{ann}(x) \). 如果 \( M \) 非零, 那么存在 \( x_1 \) 使得 \(
  \operatorname{ann}(x) \) 为素理想. 令 \( Ax = M_1 \), 这时 \( M_1 = A /
  \operatorname{ann}(x_1) \).

  如果 \( M \neq M_1 \), 我们可以对 \( M / M_1 \) 作相同的讨论, 知道存在 \( x_2
  + M_1 \) 使得 \( \operatorname{ann}(x_2 + M_1) \) 是素的, 且有子模 \( M_2 \)
  使得 \( M_2 / M_1 = A / \operatorname{ann}(x_2 + M_1) \). 如果 \( M_2 / M_1
  \neq M / M_1 \), 那么考察 \( \frac{M / M_1}{M_2 / M_1} \simeq M / M_2, \ldots \)

  我们能够得到一个上升子模链 \( 0 \subseteq M_1 \subseteq M_2 \subseteq \cdots
  \), 使得 \( M_{i -  1} \subsetneq M_{i} \), 且 \( M_{i} / M_{i - 1} \simeq A /
  \mathfrak{p}_{i} \), 其中 \( \mathfrak{p}_i \) 是一些素理想.
  \cref{proposition-finite-generated-module-over-noetherian-ring} 告诉我们, \( M
  \) 是 noetherian 的, 因此这个链有限. 上面的构造在 \( M_n \neq M \)
  时总能继续进行, 因此此链只能终止在 \( M \).
\end{proof}

\begin{corollary}
  如果环 \( A \) notherian, 每个有限生成 \( A \)-代数有限表现.
\end{corollary}
\begin{proof}
  每个有限生成 \( A \)-代数 \( B \) 形如 \( A[X_1, \ldots, X_n] / \mathfrak{a}
  \), \( \mathfrak{a} \) 作为 \( A[X_1, \ldots, X_n] \) 的理想有限生成.
\end{proof}

\subsection{Noetherian 环与根}

\begin{proposition}
  在 noetherian 环中, 每个理想都包含其根的一个幂; 特别地, 幂零根的某个幂就是零.
\end{proposition}
\begin{proof}
  假设 \( a_1, \ldots, a_n \) 生成 \( \operatorname{rad}(\mathfrak{a}) \).
  对每个 \( a_i \) 都存在 \( r_i \) 使得 \( a_i^{r_i} \in \mathfrak{a} \).
  设 \( r = r_1 + r_2 + \cdots + r_n \), 那么对任意 \( a \in
  \operatorname{rad}(\mathfrak{a}), a^r \in \mathfrak{a} \).
\end{proof}

\begin{proposition}
  假设 \( (A, \mathfrak{m}) \) 是一个 noetherian 局部环.
  \( \mathfrak{m} \) 由 \( a_1, \ldots, a_n \) 生成当且仅当
  \( a_1 + \mathfrak{m}^2,\ldots, a_n + \mathfrak{m}^{2} \) 张成 \( A /
  \mathfrak{m} \)-线性空间 \( \mathfrak{m} / \mathfrak{m}^2 \).
  特别地, \( \mathfrak{m} \) 生成元集最小数等于 \( \mathfrak{m} / \mathfrak{m}^2
  \) 维数.
\end{proposition}
\begin{proof}
  将 \(  \mathfrak{m}, N = \left\langle a_1, \ldots, a_n \right\rangle \) 视为
  \( A \)-模, 于是 \( \mathfrak{m} = N + \mathfrak{m}^2 \).
  由\cref{theorem-Nakayama-lemma}, \( \mathfrak{m} = N \).
\end{proof}

\begin{theorem}[Krull 交定理]
  \label{theorem-Krull-intersection}
  假设 \( \mathfrak{a} \) 是 noetherian 环 \( A \) 的一个理想. 如果 \(
  \mathfrak{a} \) 包含于所有 \( A \) 的极大理想, 那么
  \[
    \bigcap_{n \geq 1} \mathfrak{a}^n = 0.
  \]
\end{theorem}

\section{noetherian 环与局部化}

\begin{proposition}
  如果 \( A \) 是 noetherian 的, \( S \) 为 \( A \) 的一个乘性子集, 那么 \(
  S^{-1} A \) 亦是 noetherian 的.
\end{proposition}
\begin{proof}
  \( \mathfrak{b}^c \) 有限生成, \( \mathfrak{b} = \mathfrak{b}^{ce} \)
  有限生成.
\end{proof}

\section{域上有限生成代数}

\subsection{Noether 正规定理}

\begin{lemma}
  假设 \( f \in k[X_1, \ldots, X_d, T] \). 对恰当的 \( m \in \mathbb{N} \),
  \[
    f\left(X_1 + T^m, X_2 + T^{m^2}, \ldots, X_d + T^{m^d}, T\right)
  \]
  形如 \( c_0 T^r + c_1 T^{r - 1} + \cdots + c_r \) 其中 \( c_0 \in k^{\times}
  \).
\end{lemma}

\begin{lemma}
  假设 \( A = k[x_1, \ldots, x_n] \) 是一个有限生成 \( k \)-代数, 令 \(
  \left\lbrace x_1, \ldots, x_d \right\rbrace \) 为 \( \left\lbrace x_1, \ldots,
  x_n \right\rbrace \) 的极大代数独立子集. 如果 \( n > d \), 那么存在一个 \( m
  \in \mathbb{N} \) 使得 \( A \) 在子代数 \( k\left[x_1 - x_n^m,\ldots, x_d -
  x_n^{m^d}, x_{d + 1}, \ldots, x_{n - 1}\right] \).
\end{lemma}

\begin{theorem}[Noether正规定理]
  每个域 \( k \) 上的有限生成代数包含一个多项式代数 \( R \), 使得 \( A \) 是一个
  \( R \)-代数. 换句话说, 存在 \( A \) 的元素 \( y_1, \ldots, y_r \) 在 \( k \)
  中代数独立, 使得 \( A \) 在 \( k[y_1, \ldots, y_r] \) 上有限.
\end{theorem}

\subsection{Nakayama 引理}

\begin{theorem}[Zariski 引理]
  \label{theorem-Zariski-lemma}
  假设 \( K/k \) 为域扩张.
  如果 \( K \) 是有限生成 \( k \)-代数, 那么 \( K \) 在 \( k \) 上代数.
  特别地, \( K \) 在 \( k \) 上有限.
\end{theorem}
\begin{proof}
  设 \( K = k[x_1, \ldots, x_r] \), 其中 \( x_1, \ldots, x_r \) 为 \( K \)
  的一组 \( k \)-最小生成元, 下面对 \( r \) 作归纳.
  \( r = 0 \) 的情况平凡.

  考虑 \( r \geq 1 \).
  如果 \( K \) 不在 \( k \) 上代数, 那么存在至少一个 \( x_i \) 不在 \( k \)
  上代数, 通过一个置换不妨设其为 \( x_1 \).
  于是 \( k[x_1] \) 是带一个未定元的 \( k \) 多项式环, 且 \( K \) 是一个有限生成
  \( k(x_1) \)-代数, 其一组生成元为 \( x_2, \ldots, x_r \).
  于是由归纳假设 \( x_2, \ldots, x_r \) 在 \( k(x_1) \) 上代数.
  因此存在 \( c \in k[x_1] \) 使得 \( cx_2, \ldots, cx_r \) 在 \( k[x_1] \)
  上整.

  对任意 \( f \in k(x_1) \subseteq K = k[x_1, \ldots, x_r] \), 存在足够大的 \( N
  \) 使得 \( c^N f \in k[x_1, cx_2, \ldots, cx_r] \).
  于是 \( c^N f \) 在 \( k[x_1] \) 上整.
  而 \( k[x_1] \) 作为域上多项式环是一个唯一因子分解整环, 是整闭的, 于是 \( c^N
  f \in k[x_1] \) 这与 \( k[x_1] \) 有无穷多个不可约元矛盾.
\end{proof}
%TODO: 分别考虑是不是代数闭域

\begin{corollary}
  假设 \( A \) 是一个有限生成 \( k \)-代数.
  \( A \) 的每个极大理想 \( \mathfrak{m} \) 都是 \( A \) 到某个 \( k
  \)-有限生成扩域的同态核.
\end{corollary}

\begin{corollary}
  假设 \( k \subseteq K \subseteq A \) 为 \( k \)-代数, \( K \) 是一个域, \( A
  \) 为一个 \( k \)-上有限生成代数.
  那么 \( K \) 在 \( k \) 上代数.
\end{corollary}
\begin{proof}
  假设 \( \mathfrak{m} \) 为 \( A \) 的一个极大理想, 那么 \( K \cap \mathfrak{m}
  = (0) \) 于是 \( k \subseteq K \subseteq A/\mathfrak{m} \).
  由\cref{theorem-Zariski-lemma} \( A/\mathfrak{m} \) 在 \( k \) 上代数, 所以 \(
  K \) 在 \( k \) 上代数.
\end{proof}

\subsection{零点定理}

\begin{theorem}[Nullstellensatz]
  \label{theorem-Nullstellensatz}
  假设 \( k \) 为一个域, \( \overline{k} \) 为其代数闭包.
  \( k[X_1, \ldots, X_n] \) 中的每个真理想都在 \( \overline{k}^n \)
  中有公共零点.
  换句话说, 存在 \( (a_1, \ldots, a_n) \in \overline{k}^n \) 使得 \( f(a_1,
  \ldots, a_n) = 0 \) 对所有 \( f \in \mathfrak{a} \) 成立.
\end{theorem}
\begin{proof}
  考虑找到一个 \( k \)-代数同态 \( \varphi: k[X_1, \ldots, X_n] \to \overline{k}
  \), 其核包含 \( \mathfrak{a} \).
  如果能找到这样的 \( \varphi \), 设 \( \alpha_i = \varphi(X_i) \).
  那么对任意 \( f \in \mathfrak{a} \), 则 \( 0 = \varphi f(X_1, \ldots, X_n) =
  f(\alpha_1, \ldots, \alpha_n) \).

  假设 \( \mathfrak{m} \) 是 \( k[X_1, \ldots, X_n] \) 的一个极大理想, 那么域 \( K
  = k[X_1, \ldots, X_n]/ \mathfrak{m} \) 是一个有限生成 \( k \)-代数.
  由\cref{theorem-Zariski-lemma}, \( K \) 在 \( k \) 上代数, 于是有嵌入 \( K
  \hookrightarrow \overline{k} \).
  考虑复合 \( k \to K \hookrightarrow \overline{k} \) 即可.
\end{proof}

\begin{corollary}
  如果 \( k \) 代数闭, 那么 \( k[X_1, \ldots, X_n] \) 的所有极大理想即
  \[
    \left\lbrace (X_1 - a_1,\ldots, X_n - a_n): (a_1, \ldots, a_n) \in k^n
    \right\rbrace.
  \]
\end{corollary}
\begin{proof}
  \( (X_1 - a_1,\ldots, X_n - a_n) \) 为 \( k[X_1, \ldots, X_n] \)
  极大理想是因为 \( k[X_1, \ldots, X_n]/(X_1 - a_1,\ldots, X_n - a_n) \simeq k
  \).

  假设 \( \mathfrak{m} \) 是 \( K[X_1, \ldots, X_n] \) 的一个极大理想.
  由\cref{theorem-Nullstellensatz}, \( \mathfrak{m} \) 有公共零点 \( (a_1,
  \ldots, a_n) \in k^n \).
  将 \( f \in K[X_1, \ldots, X_n] \) 写成 \( X_i - a_i \) 的多项式, 其常数项即
  \( f(a_1, \ldots, a_n) \).
  于是对 \( f \in \mathfrak{m} \) 有 \( f(a_1, \ldots, a_n) = 0 \), 换句话说 \(
  \mathfrak{m} \subseteq (X_1 - a_1, \ldots, X_n - a_n) \), 结合极大性,
  只能相等.
\end{proof}

\begin{theorem}[强 Nullstellensatz]
  \label{theorem-strong-Nullstellensatz}
  假设 \( \mathfrak{a} \) 为 \( k[X_1, \ldots, X_n] \) 的任意理想.
  假设 \( Z(\mathfrak{a}) \) 为 \( \mathfrak{a} \) 在 \( \overline{k}^n \)
  的所有公共零点集.
  如果一个多项式 \( h \in k[X_1, \ldots, X_n] \) 在 \( Z(\mathfrak{a}) \)
  中取值均为零, 那么 \( h \in \operatorname{rad}(\mathfrak{a}) \).
\end{theorem}
\begin{proof}
  只需证明 \( h \neq 0 \) 的情况.
  由\cref{theorem-Hilbert-basis}, \( k[X_1, \ldots, X_n] \) noetherian, 于是 \(
  \mathfrak{a} \) 有限生成, 设其由 \( g_1, \ldots, g_m \) 生成.
  考虑下面关于变元 \( X_1, \ldots, X_n, Y \) 的方程组
  \[
    \begin{split}
      g_i(X_1, \ldots, X_n) &= 0, i = 1, \ldots, m\\
      1 - Yh(X_1, \ldots, X_n) &= 0.
    \end{split}
  \]
  这样的方程组是没有解的: 如果 \( (a_1, \ldots, a_n, b) \) 满足前 \( m \) 个,
  那么 \( (a_1, \ldots, a_n) \in \mathfrak{a} \), 于是 \( h(a_1, \ldots, a_n) =
  0 \), 最后一个方程变成 \( 1 = 0 \), 这是不可能的.
  因此, 由\cref{theorem-Nullstellensatz}, \( (g_1, \ldots, g_m, 1 - Yh) = k[X_1,
  \ldots, X_n, Y] \).
  换句话说, 存在 \( f_i \in k[X_1, \ldots, X_n, Y] \) 使得
  \[
    1 = \sum_{i = 1}^mf_i g_i + f_{m + 1}(1 - Yh).
  \]
  考虑态射
  \[
    k[X_1, \ldots, X_n, Y] \to k(X_1, \ldots, X_n),\quad X_i \mapsto X_i, Y
    \mapsto h^{-1},
  \]
  其将 \( \sum_{i = 1}^mf_i g_i + f_{m + 1}(1 - Yh) \) 映到
  \[
    1 = \sum_i f_i(X_1, \ldots, X_n, h^{-1})g_i(X_1, \ldots, X_n).
  \]
  对足够大的 \( N \), 两边乘以 \( h^N \) 知道 \( h^N \in \mathfrak{a} \).
\end{proof}

\begin{proposition}
  \label{proposition-radical-as-maximal-intersection-in-finite-generated-algebra-over-field}
  假设 \( A \) 是一个有限生成 \( k \)-代数, \( \mathfrak{a} \) 为 \( A \)
  的理想.
  那么
  \[
    \operatorname{rad}(\mathfrak{a}) = \bigcap_{\mathfrak{m} \supseteq
    \mathfrak{a} \text{极大}} \mathfrak{m}.
  \]
  特别地, 如果 \( A \) 约化, 那么 \( \bigcap_{\mathfrak{m} \text{极大}}
  \mathfrak{m} = 0 \).
\end{proposition}
\begin{proof}
  只需对 \( A = k[X_1, \ldots, X_n] \) 的情况证明.
  由\cref{proposition-radical-as-prime-intersection}包含关系 \( \operatorname{rad}(\mathfrak{a}) \subseteq \bigcap_{\mathfrak{m}
  \supseteq \mathfrak{a}} \mathfrak{m} \) 是已知的.
  下面证明\( \operatorname{rad}(\mathfrak{a}) \subseteq \bigcap_{\mathfrak{m}
  \supseteq \mathfrak{a}} \mathfrak{m} \).

  \cref{theorem-Nullstellensatz} 保证了 \( Z(\mathfrak{a}) \neq \varnothing \),
  取任意 \( (a_1, \ldots, a_n) \in Z(\mathfrak{a}) \).
  假设 \( h \in \bigcap_{\mathfrak{m} \supseteq \mathfrak{a}} \mathfrak{m} \),
  考虑赋值映射
  \[
    \operatorname{ev}: k[X_1, \ldots, X_n] \to \overline{k},\quad f \mapsto f(a_1, \ldots, a_n),
  \]
  其中 \( \overline{k} \) 为 \( k \) 的代数闭包.
  \( \operatorname{Im} \operatorname{ev} \) 在 \( k \) 上代数, \(
  \operatorname{Im} \operatorname{ev} \) 是一个域.
  因此 \( \operatorname{ker} \operatorname{ev} \) 是一个包含 \( \mathfrak{a} \)
  的极大理想.
  由 \( h \) 定义, \( h \in \operatorname{ker} \operatorname{ev} \), 换句话说 \(
  h(a_1, \ldots, a_n) = 0 \).
  由\cref{theorem-strong-Nullstellensatz} \( h \in
  \operatorname{rad}(\mathfrak{a}) \).
\end{proof}
