\chapter{Number Theory}

\section{Ideal norms and the Dedekind-Kummer theorem}

\subsection{The module index}

\( A \): a Dedekind domain;
\( K = \operatorname{Frac} A \);
\( V \): a \( n \)-dimensional \( K \)-vector space;
\( M, N \): \( A \)-lattices in \( V \);
\( \mathfrak{p} \): a prime of \( A \).
Since \( A_{\mathfrak{p}} \) is a PID, we have \( M_{\mathfrak{p}} \) is free of finite rank since
\begin{enumerate}
  \item \( M_{\mathfrak{p}} \) is a product of a free modules and a torsion by the structure of modules over a PID.
  \item \( M_{\mathfrak{p}} \) is in a vector space, then it is torsion free.
    Moreover, its basis generates \( V \), and hence it is of rank \( n = \operatorname{dim}_K V \).
\end{enumerate}
Then \( M_{\mathfrak{p}} \simeq A_{\mathfrak{p}}^n \simeq N_{\mathfrak{p}} \).
Choose an \( A_{\mathfrak{p}} \)-module isomorphism \( \phi_{\mathfrak{p}}: M_{\mathfrak{p}} \xrightarrow{\sim} N_{\mathfrak{p}} \) and let \( \hat{\phi}_{\mathfrak{p}} \) denote the unique \( K \)-linear map \( V \to V \) extending \( \phi_{\mathfrak{p}} \).
The linear map is an isomorphism and therefore has nonzero determinant(in a same basis).

The \emph{module index} \( [M_\mathfrak{p}: N_{\mathfrak{p}}]_{A_\mathfrak{p}} \) is the principal fractional \( A_{\mathfrak{p}} \)-ideals generated by \( \det \hat{\phi}_{\mathfrak{p}} \):
\[
  [M_{\mathfrak{p}}: N_{\mathfrak{p}}]_{A_{\mathfrak{p}}} = \left(\det \hat{\phi}_{\mathfrak{p}}\right)
\]
The ideal does not depend on our choice of \( \phi_{\mathfrak{p}} \) because any other choice can be written as \( \phi_1 \phi_{\mathfrak{p}} \phi_2 \), where the determinants of isomorphisms \( \phi_1: M_{\mathfrak{p}} \xrightarrow{\sim} M_{\mathfrak{p}} \) and \( \phi_2: N_{\mathfrak{p}} \xrightarrow{\sim} N_{\mathfrak{p}} \) are just units.

The \emph{module index} \( [M : N]_A \) is the \( A \)-module
\[
  [M_{\mathfrak{p}}: N_{\mathfrak{p}}]_A = \bigcap_{\mathfrak{p}} [M_{\mathfrak{p}} : N_{\mathfrak{p}}]_{A_{\mathfrak{p}}},
\]
where \( \mathfrak{p} \) ranges over primes of \( A \) and the intersection takes place in \( K \).

\begin{proposition}
  The module index \( [M: N]_A \) is a nonzero fractional ideal of \( A \) and
  \[
    \left([M: N]_A\right)_{\mathfrak{p}} = [M_{\mathfrak{p}}: N_{\mathfrak{p}}]_{A_{\mathfrak{p}}}.
  \]
\end{proposition}

The previous proposition implies that the module index \( [M: N]_A \) is an element of the ideal group \( \mathcal{I}_A \).
If \( M, N, P \) are \( A \)-lattices in \( V \), then
\[
  [M: N]_A [N: P]_A = [M : P]_A,
\]
since
\begin{enumerate}
  \item for each prime \( \mathfrak{p} \) we can write any isomorphism \( M_{\mathfrak{p}} \xrightarrow{\sim} P_{\mathfrak{p}} \), as a composition of isomorphisms \( M_{\mathfrak{p}} \xrightarrow{\sim} N_{\mathfrak{p}} \xrightarrow{\sim} P_{\mathfrak{p}} \);
  \item the determinant map is multiplicative w.r.t. composition and multiplication of fractional ideals is compatible with localization.
\end{enumerate}
Taking \( P = M \) yields the identity
\[
  [M: N]_A [N: M]_A = [M: M]_A = A,
\]
thus \( [M : N]_A \) and \( [N : M]_A \) are inverses in the ideal group \( \mathcal{I}_A \).

\begin{theorem}
  Let \( N \subseteq M \), then the quotient module is a direct sum of cyclic \( A \)-modules:
  \[
    M / N \simeq A / I_1 \oplus \cdots \oplus A / I_n,
  \]
  where \( I_1, \cdots, I_n \) are nonzero ideals of \( A \).
  Then
  \[
    [M : N]_A = I_1 \cdots I_n.
  \]
\end{theorem}

\subsection{The ideal norm}
