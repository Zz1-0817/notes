\chapter{基础知识}

\section{代数扩张与超越扩张初步}

\section{分裂域}

假设 \( f \in F[X] \). 那么一个包含 \( F \) 的域 \( E \) 称为\emph{分裂} \( f
\) 如果 \( f \) 在 \( E[X] \) 中分裂, 也就是
\[
  f(X) = a \prod_{i = 1}^m(X - \alpha_i),\quad a \in F,\quad \alpha_i \in E.
\]
如果 \( E \) 分裂 \( f \) 且由 \( f \) 的根生成, 那么称 \( E \) 为 \( f \)
的\emph{分裂域} 或者\emph{根域}.

\begin{proposition}
  \label{proposition-polynomial-splitting-field-degree}
  每个多项式 \( f \in F[X] \) 都有一个分裂域 \( E_f \), 并且
  \[
    [E_f, F] \leq (\deg f) !
  \]
\end{proposition}
\begin{proof}
  假设 \( f \) 有根 \( \alpha_1, \ldots, \alpha_n \), 对 \( n \) 做归纳假设.
  \( n = 1 \) 时, \( \alpha_1 \in F \) 结论显然.
  对 \( n > 1 \), 设 \( \alpha_1 \) 在 \( F \) 的极小多项式为 \( f_1 \), 于是 \(
  f_1 \mid f \).
  由归纳假设 \( \widetilde{f} = f / (X - \alpha_1) \) 在 \( F[\alpha_1] \)
  的分裂域 \( E_{\widetilde{f}} \) 存在且满足 \( [E_{\widetilde{f}}:F[\alpha_1]]
  \leq (\deg \widetilde{f})! \).
  而 \( E_{\widetilde{f}} = F[\alpha_1][\alpha_2, \ldots, \alpha_n] =
  F[\alpha_1, \alpha_2, \ldots, \alpha_n] \), 因此 \( E_{\widetilde{f}} \) 为 \(
  f \) 在 \( F \) 中的分裂域, 且
  \[
    [E_{\widetilde{f}}: F] = [E_{\widetilde{f}}: F[\alpha_1]][F[\alpha_1]:F]
    \leq (\deg \widetilde{f})!\deg f_1 \leq (\deg f)!.
  \]
\end{proof}

\subsection{单扩张的域同态}

\begin{proposition}
  \label{proposition-simple-extension-and-field-homomorphism}
  假设 \( F(\alpha) \) 是域 \( F \) 的单扩张, \( \Omega \) 是 \( F \)
  的第二个扩张.
  \begin{enumerate}
    \item 假设 \( \alpha \) 在 \( F \) 上超越. 那么对每个 \( F \)-同态 \(
      \varphi: F(\alpha) \to \Omega, \varphi(\alpha) \) 在 \( F \) 上超越,
      且映射 \( \varphi \mapsto \varphi(\alpha) \) 给出了一一对应
      \[
        \left\lbrace F\text{-同态} F(\alpha) \to \Omega \right\rbrace
        \leftrightarrow \left\lbrace \Omega \text{在} F \text{上的超越元}
        \right\rbrace.
      \]
    \item 假设 \( \alpha \) 在 \( F \) 上代数, 令 \( f(X) \) 为其极小多项式.
      对每个 \( F \)-同态 \( \varphi: F[\alpha] \to \Omega \), \(
      \varphi(\alpha) \) 是 \( f(X) \) 在 \( \Omega \) 中的根, 并且映射 \(
      \varphi \mapsto \varphi(\alpha) \) 定义了一一映射
      \[
        \left\lbrace F \text{-同态} \varphi: F[\alpha] \to \Omega \right\rbrace
        \leftrightarrow \left\lbrace f \text{在} \Omega \text{中的根}
        \right\rbrace.
      \]
      特别地, 这样的映射数即 \( f \) 在 \( \Omega \) 的不同根数.
  \end{enumerate}
\end{proposition}
\begin{proof}
  \( (1) \)
  应用反证法, 如果 \( \varphi(\alpha) \) 在 \( F \) 上代数, 那么存在多项式 \(
  f(X) \in F[X] \), 使得 \( f(\varphi(\alpha)) = 0 \).
  于是 \( \varphi(f(\alpha)) = 0 \), 这只能 \( f(\alpha) = 0 \), 与 \( \alpha \)
  超越矛盾.
  取定一个超越元 \( f(\alpha) \), 双射的另一边由 \( \alpha \mapsto f(\alpha) \)
  给出.
  \( (2) \)
  与 \( (1) \) 是完全类似的.
\end{proof}

我们能将上面结果推广如下, 证明是完全一致的.
\begin{proposition}
  \label{proposition-extended-simple-extension-and-field-homomorphism}
  假设 \( F(\alpha) \) 是 \( F \) 的单扩张, \( \varphi_0: F \to \Omega \)
  是一个由 \( F \) 到第二个域 \( \Omega \) 的同态.
  \begin{enumerate}
    \item 如果 \( \alpha \) 在 \( F \) 上超越, 那么映射 \( \varphi \mapsto
      \varphi(\alpha) \) 定义了一个一一映射
      \[
        \left\lbrace \varphi_0 \text{的扩张} \varphi: F(\alpha) \to \Omega
        \right\rbrace \leftrightarrow \left\lbrace \Omega \text{在} \varphi_0(F)
        \text{的超越元} \right\rbrace
      \]
    \item 如果 \( \alpha \) 在 \( F \) 上代数且其极小多项式为 \( f(X) \),
      那么映射 \( \varphi \mapsto \varphi(\alpha) \) 定义了一个一一映射
      \[
        \left\lbrace \varphi_0 \text{的扩张} \varphi: F[\alpha] \to \Omega
        \right\rbrace \leftrightarrow \left\lbrace \varphi_0 f \text{在} \Omega
        \text{的根} \right\rbrace.
      \]
      特别地, 这样的映射数即 \( \varphi_0f \) 在 \( \Omega \) 的不同根数.
  \end{enumerate}
\end{proposition}

\begin{proposition}
  \label{proposition-single-polynomial-extension-number}
  假设 \( \sigma: F_1 \to F_2 \) 为域同构, \( f \in F[X] \), \( E \) 为 \( F_2
  \) 的一个由 \( \sigma f \) 的根生成的扩张, \( \Omega \) 为一个分裂 \( \sigma f
  \) 的 \( F_2 \) 扩域.
  那么存在一个 \( \sigma \) 扩张 \( \varphi: E \to \Omega \),
  且这样的同态个数至多有 \( [E: F_2] \);
  个数恰为 \( [E:F_2] \) 的一个充分条件是 \( \sigma f \) 在 \( \Omega \)
  的根不同.
\end{proposition}
\begin{proof}
  通过给 \( f \) 乘一个常数, 可以假设 \( f \) 首一系数, 从而 \( \sigma f \)
  首一系数.
  由 \( \sigma f \) 在 \( \Omega \) 上分裂, 可以假设 \( \sigma f = \prod_i(X -
  \beta_i) \in \Omega[X] \).
  于是, \( E \) 由一些 \( \beta_i \) 生成, 不妨设 \( E = F_2[\beta_1, \ldots,
  \beta_r] \).

  假设 \( \beta_1 \) 在 \( F_2 \) 的极小多项式为 \( g_1 \), 那么 \( g_1 \mid
  \sigma f \).
  由\cref{proposition-extended-simple-extension-and-field-homomorphism}, \(
  \sigma \) 的扩张 \( F_2[\beta_1] \to \Omega \) 个数至多为 \( \deg g_1 =
  [F_2[\beta_1]: F_2] \), 相等当且仅当 \( g_1 \) 的根两两不同.
  如果 \( \sigma f \) 的根两两不同, 那么 \( g_1 \) 的根亦两两不同.
  重复此过程, 得到 \( F_2[\beta_1, \ldots, \beta_i] \to \Omega \) 的扩张 \(
  F_2[\beta_1, \ldots, \beta_{i + 1}] \to \Omega \) 个数至多为 \( [F_2[\beta_1,
  \ldots, \beta_{i + 1}]: F_2[\beta_1, \ldots, \beta_{i}]] \), 并且相等如果 \(
  \sigma f \) 根两两不同.

  通过限制到 \( F_2[\beta_1, \ldots, \beta_{i}] \), 每个 \( \sigma \) 扩张 \( E
  \to \Omega \) 都可以视为逐步扩张
  \[
    F_2[\beta_1] \to \Omega, \ldots, F_2[\beta_1,\ldots ,\beta_r] = E \to
    \Omega,
  \]
  于是其个数至多为 \( [F_2[\beta_1]:F_2] [F_2[\beta_1, \beta_2]:F_2[\beta_1]]
  \cdots [E: F_2[\beta_1,\ldots,\beta_{r - 1}]] = [E:F_2] \), 且取等如果 \(
  \sigma f \) 根两两不同.
\end{proof}

\begin{corollary}
  \label{corollary-splitting-fields-isomorphic}
  假设 \( \sigma: F_1 \to F_2 \) 是域同构.
  如果 \( E_1 \) 和 \( E_2 \) 分别是 \( f \) 和 \( \sigma f \) 的分裂域, 那么 \(
  \sigma \) 的扩张\( E_1 \to E_2 \) 是同构.
  特别地, \( f \) 的不同分裂域是 \( F \)-同构的.
\end{corollary}

\begin{corollary}
  \label{corollary-finite-extension-homomorphisms-numbers}
  假设 \( \sigma: F_1 \to F_2 \) 为域同构, \( E \) 和 \( L \) 为 \( F_2 \)
  的域扩张, 其中 \( E \) 在 \( F_2 \) 上有限.
  那么 \( \sigma \) 的扩张 \( E \to L \) 个数至多有 \( [E: F_2] \) 个.
\end{corollary}
\begin{proof}
  设 \( E = F_2[\beta_1, \ldots, \beta_r] \), \( \beta_i \) 在 \( F_2 \)
  上的极小多项式为 \( f_i \).
  记 \( f \) 为不同 \( f_i \) 的乘积, \( f \) 在 \( L \) 上的分裂域为 \( \Omega
  \).
  由\cref{proposition-single-polynomial-extension-number}, \( \sigma \) 扩张
  \( E \to \Omega \) 个数至多有 \( [E : F_2] \) 个.
  而 \( \sigma \) 扩张 \( E \to L \) 复合包含映射 \( L \to \Omega \) 也是扩张 \(
  E \to \Omega \), 故扩张 \( E \to L \) 个数至多有 \( [E: F_2] \) 个.
\end{proof}

\section{代数闭域初步}

\paragraph{代数闭域} \( f \in F[X] \) 称为在 \( F[X] \) 中\emph{分裂}, 如果 \( f \) 是 \( F[X] \)
中次数为 \( 1 \) 的多项式的积.

\begin{proposition}
  给定一个域 \( \Omega \), 那么以下条件等价:
  \begin{enumerate}
    \item 每个 \( \Omega[X] \) 的非常多项式都在 \( \Omega[X] \) 中分裂.
    \item 每个 \( \Omega[X] \) 的非常多项式都在 \( \Omega \) 中至少有一根.
    \item 每个 \( \Omega[X] \) 的不可约多项式次数都为 \( 1 \).
    \item \( \Omega \) 的有限扩张都是 \( \Omega \).
  \end{enumerate}
  如果 \( \Omega \) 满足上面任一条件, 那么称 \( \Omega \) 是\emph{代数闭的}.
\end{proposition}

\paragraph{代数闭包} 域 \( \Omega \) 称为是 \( F \) 的\emph{代数闭包}, 如果 \(
\Omega \) 代数闭并且在 \( F \) 上代数.
\begin{proposition}
  \label{proposition-algebraic-closure-as-a-splitting-field}
  如果 \( \Omega \) 在 \( F \) 上代数, 每个 \( f \in F[X] \) 都在 \( \Omega[X]
  \) 上分裂, 那么 \( \Omega \) 代数闭.
\end{proposition}
\begin{proof}
  假设 \( \alpha \) 为 \( f = a_0X^n + a_1 X^{n - 1} + \cdots + a_n \in
  \Omega[X] \) 的根.
  因为 \( F[a_0, \ldots, a_n, \alpha] = F[a_0,\ldots, a_n][\alpha] \), \( \alpha
  \) 在 \( F \) 上有限, 因此 \( \alpha \in \Omega \).
\end{proof}

\begin{proposition}
  假设 \( F \) 是一个域, \( \Omega \) 是一个包含 \( F \) 的整环, 那么
  \[
    \overline{F} := \left\lbrace \alpha \in \Omega: \alpha \text{在} F
    \text{上代数} \right\rbrace
  \]
  为一个域, 称为\emph{\( F \) 在 \( \Omega \) 上的代数闭包}.
\end{proposition}

\begin{corollary}
  假设 \( \Omega \) 是一个代数闭包.
  对每个 \( \Omega \) 的子域 \( F \), \( F \) 在 \( \Omega \) 的代数闭包 \( E
  \), 是 \( F \) 的代数闭包.
\end{corollary}
\begin{proof}
  每个 \( F[X] \) 的多项式在 \( \Omega[X] \) 中分裂, 自然在 \( E[X] \) 分裂.
  由\cref{proposition-algebraic-closure-as-a-splitting-field}立刻得到结论.
\end{proof}

\paragraph{代数闭包同构}

\begin{theorem}
  \label{theorem-algebraic-closure-isomorphic}
  假设 \( \sigma: F_1 \simeq F_2 \) 为域同构.
  假设 \( \Omega \) 是域 \( F_2 \) 的一个代数闭包, \( E \) 是 \( F_1 \)
  的代数扩张. 那么存在 \( \sigma_1 \) 的扩张 \( E \to \Omega \), 并且如果 \( E
  \) 是 \( F_1 \) 的代数闭包, 那么每个这样的同态都是同构.
\end{theorem}
\begin{proof}
  \( \mathcal{S} \) 为所有二元组 \( (N, \tau) \) 组成的集合, 其中 \( N \) 为 \(
  E/F_1 \) 的子扩张, \( \tau: N \to \Omega \) 为 \( \sigma_1 \) 的扩张.
  定义 \( \mathcal{S} \) 上的一个偏序关系 \( (N_1, \tau_1) \leq (N_2, \tau_2) \)
  当且仅当 \( N_1 \subseteq N_2, \left. \tau_2 \right\vert_{N_1} = \tau_1 \).
  由此定义, 每个 \( \mathcal{S} \) 的上链都有上界, 于是由 Zorn 引理, \(
  \mathcal{S} \) 有一个极大元 \( (E_0, \tau_0) \), 下面证明 \( E_0 = E \).

  假设不然, 有 \( E_0 \subsetneq E \).
  那么存在 \( \alpha \in E \setminus E_0 \).
  由\cref{proposition-simple-extension-and-field-homomorphism}知道存在 \( \tau_0
  \) 的扩张 \( E_0[\alpha] \to \Omega \), 这与极大性矛盾.

  假设 \( E \) 是 \( F_1 \) 的代数闭包, 对任意 \( \alpha \in \Omega \), 设 \(
  \alpha \) 在 \( F_2 \) 上的极小多项式为 \( f \).
  结合 \( \sigma_1 \) 单, \( \sigma_1 \) 将 \( \sigma^{-1}f \) 的所有根映到 \( f
  \) 的所有根, 因此\( \alpha \in \operatorname{Im} E \).
\end{proof}



\section{分裂域与正规扩张}

\begin{proposition}
  \label{proposition-isomorphism-extend-to-splitting-fields}
  假设 \( \sigma: K \to L \) 是一个域同构, \( S \subseteq K[X] \)
  是一个多项式集, \( S' = \left\lbrace \sigma f: f \in S \right\rbrace \subseteq
  L[X] \).
  如果 \( F \) 是 \( S \) 在 \( K \) 的分裂域, \( M \) 是 \( S' \) 在 \( L \)
  的分裂域, 那么 \( \sigma \) 能扩张为同构 \( F \simeq M \).
\end{proposition}
\begin{proof}
  \( S \) 只有一个元素的情况由\cref{corollary-splitting-fields-isomorphic}保证.

  假设 \( S \) 任意, 令 \( \mathcal{S} \) 为所有三元组 \( (E, N, \tau) \)
  组成的集合, 其中 \( E \) 是 \( F/S \) 的中间域, \( N \) 是 \( M/L \) 的中间域,
  \( \tau: E \simeq N \) 为域同构.
  定义 \( \mathcal{S} \) 的一个偏序关系 \( (E_1, N_1, \tau_1) \leq (E_2, N_2,
  \tau_2) \) 当且仅当 \( E_1 \subseteq E_2, N_1 \subseteq N_2 \) 且 \( \left.
  \tau_2 \right\vert_{E_1} = \tau_1 \).
  由此定义 \( \mathcal{S} \) 的任意上升链都有上界,
  进而通过\href{https://en.wikipedia.org/wiki/Zorn's_lemma}{Zorn 引理}可以得到
  \( \mathcal{S} \) 的一个极大元 \( (F_0, M_0, \tau_0) \).
  下面说明 \( F_0 = F, M_0 = M \) 以完成证明.

  假设不然, \( F_0 \subsetneq F \), 那么存在 \( f \in S \) 在 \( F_0 \)
  上不分裂. 又由\cref{corollary-splitting-fields-isomorphic}, 可以找到 \( F/K \)
  包含 \( F_0 \) 分裂 \( f \) 的子域 \( F_1 \) 通过 \( \tau_0 \) 的一个扩张同构于
  \( M_0 \) 的一个包含于 \( M \) 的扩域.
  这与 \( (F_0, M_0, \tau_0) \) 极大性矛盾.
  类似地, \( M_0 = M \).
\end{proof}

\begin{theorem}
  \label{theorem-normal-extension-TFAE-conditions}
  假设 \( F \) 是 \( K \) 的代数扩张, 那么下面条件等价:
  \begin{enumerate}
    \item \( F \) 在 \( K \) 上正规.
    \item \( F \) 是 \( K \) 上某个多项式集合 \( S \subseteq K[X] \) 的分裂域.
    \item 如果 \( \overline{K} \) 是 \( K \) 的包含 \( F \) 的代数闭包,
      那么对任意 \( K \)-单同态 \( \sigma: F \to \overline{K} \), 有 \(
      \operatorname{Im} \sigma = F \).
  \end{enumerate}
\end{theorem}
\begin{proof}
  \( (1) \implies (2) \)
  因为每个线性空间都由一组基(譬如:
  \href{https://en.wikipedia.org/wiki/Basis_(linear_algebra)#Proof_that_every_vector_space_has_a_basis}{wikipeida}),
  可以设 \( \left\lbrace u_i: i \in I \right\rbrace \) 为 \( F \) 的一组 \( K
  \)-基, 其中 \( I \) 是某个指标集.
  设 \( u_i, i \in I \) 在 \( K[X] \) 的极小多项式为 \( f_i \) 那么 \( F \) 是
  \( \left\lbrace f_i: i \in I \right\rbrace \) 的分裂域.

  \( (2) \implies (3) \)
  假设 \( F \) 是 \( K \) 关于 \( \left\lbrace f_i: i \in I\right\rbrace
  \subseteq F[X] \) 的分裂域.
  假设 \( \alpha_i \) 为 \( f_i \) 的一个根, 那么 \( \sigma(\alpha_i) \) 也是 \(
  f_i \) 的一个根.
  由于 \( \sigma\) 单, 其将 \( f_i \) 的根映到 \( f_i \) 的所有根, .
  而 \( F \) 由 \( \left\lbrace f_i: i \in I \right\rbrace \) 的根生成, 于是 \(
  \sigma(F) = F \).

  \( (3) \implies (1) \)
  假设 \( \alpha \in F \), 那么对 \( \alpha \) 在 \( K[X] \) 的任一根 \( \alpha'
  \), 有域同构\( \tau: K[\alpha] \simeq K[\alpha'] \).
  由\cref{theorem-algebraic-closure-isomorphic} 有 \( \tau \) 的扩张 \( \tau':
  \Omega \to \Omega \).
  结合假设并考虑限制 \( \left. \tau' \right\vert_{F} \), 知道 \( \alpha' \in F
    \).
\end{proof}

\section{可分扩张}

\subsection{重根}

\begin{proposition}
  假设 \( f \) 和 \( g \) 为 \( F[X] \) 上的多项式, \( \Omega \) 为 \( F \)
  的扩张.
  那么 \( \gcd_F(f, g) = \gcd_{\Omega}(f, g) \).
  特别地, \( F[X] \) 中不同的不可约多项式在 \( \Omega[X] \) 中亦没有公共根.
\end{proposition}

\begin{lemma}
  \( f \) 有重根当且仅当 \( f' \) 有重根.
\end{lemma}

\begin{proposition}
  \label{proposition-irreducible-polynomial-with-multiple-roots-TFAE-condition}
  假设 \( f \in F[X] \) 是一个不可约多项式, 那么以下等价:
  \begin{enumerate}
    \item \( f \) 有重根.
    \item \( \gcd(f, f') \neq 1 \).
    \item 存在 \( p \neq 0 \) 使得 \( \operatorname{char} F = p \) 并且 \( f \)
      为 \( X^p \) 的多项式.
    \item \( f \) 的所有根都是重根.
  \end{enumerate}
\end{proposition}

\begin{proposition}
  假设 \( f \in F[X] \) 是一个非零多项式, 那么以下等价:
  \begin{enumerate}
    \item \( \gcd(f, f') = 1 \).
    \item \( f \) 只有单根.
  \end{enumerate}
  我们称这样的 \( f \) 为\emph{可分多项式}.
\end{proposition}

\subsection{完美域}

一个域称为是\emph{完美的}, 如果其特征为零或其特征为 \( p \neq 0 \), 且每个 \( F
\) 的元素都是 \( F \) 中某个元素的 \( p \) 次幂.

\begin{proposition}
  一个域 \( F \) 是完美的当且仅当 \( F[X] \) 的每个不可约多项式都可分.
\end{proposition}
\begin{proof}
  只需证明 \( \operatorname{char} F = p \neq 0 \) 的情况.

  \( \implies \) 假设不然, 存在不可约多项式 \( f \in F[X] \) 不可分, 那么 \( (f,
  f') \neq 1 \).
  由\cref{proposition-irreducible-polynomial-with-multiple-roots-TFAE-condition}
  \( f \) 具有形式 \( a_0(X^p)^r + a_{1}(X^p)^{r - 1} + \cdots + a_r \).
  \( F \) 是完美域, 可以设 \( a_i = b_i^p \), 于是
  \[
    f = (b_0 X^r + b_1 X^{r - 1} + \cdots + b_r)^p
  \]
  与 \( f \) 不可约矛盾.

  \( \impliedby \) 假设不然, \( a \in F \) 不为 \( F \) 中元素的 \( p \) 次幂.
  设 \( f =  X^p - a \) 的一个根为 \( \alpha \notin F \), 那么 \( X^p - a = (X -
  \alpha)^p \);
  换句话说, \( f \) 不可分.
  \( f \) 的因子具有形式 \( (X - \alpha)^i, 1 \leq i \leq p \), 考察 \( X \)
  项系数知道 \( i\alpha \in F \), 只能 \( i = p \), 因此 \( X^p - a \) 在 \( F
  \) 中不可约.
  \( f \) 不可分不可约与假设矛盾.
\end{proof}
