\section{李代数}

\section{模}

下面假设 \( \mathfrak{g} \) 是一个域 \( \mathbb{F} \) 上李代数.

\begin{definition}[对偶模]
  假设 \( V \) 为一个 \( \mathfrak{g} \)-模. 那么 \( V \) 的对偶线性空间 \( V^*
  \) 可以按如下方式视为 \( \mathfrak{g} \)模: 对 \( f \in V^*, v \in V, x \in
  \mathfrak{g} \), \( x.f(v) = -f(x.v) \). 我们称按这种方式定义的 \(
  \mathfrak{g} \)-表示 \( V^* \) 为 \( V \) 的\emph{对偶模}.
\end{definition}

\begin{proposition}
  假设 \( e_1, e_2, \cdots, e_n \) 为 \( \mathfrak{g} \)-模 \( V \) 的一组基, \(
  e_1^*, e_2^*, \cdots, e_n^* \) 为 \( V^* \) 对偶基, 即 \( e_i^*(e_j) =
  \delta_{ij} \). 如果 \( x \in \mathfrak{g} \), 在基 \( e_1, e_2, \cdots, e_n
  \) 下线性变换对应矩阵为 \( M \), 那么 \( x \) 在基 \( e_1^*, e_2^*, \cdots,
  e_n^* \) 下矩阵为 \( -M^T \).
\end{proposition}
\begin{proof}
  记 \( M = \left(m_{ij}\right) \), 那么
  \[
    (x.e^*_i)(e_j) = -e^*_i(x.e_j) =
    -e_i^*(\sum_k m_{kj}e_k) = -m_{ij} = -m_{ij} e_j^*(e_j).
  \]
  从而 \( x.e^*_i = -\sum_{j}m_{ij}e^*_j \). 另一方面, 设 \( M' =
  \left(m'_{ij}\right) \) 为 \( x \) 在基 \( e^*_1, e^*_2, \cdots, e^*_n \)
  下的矩阵, 那么 \( x.e^*_i = \sum_j m'_{ji} e^*_i \). 因此 \( m'_{ji} = -m_{ij}
  \).
\end{proof}

\begin{definition}[张量模]
  假设 \( V, W \) 为两个 \( \mathfrak{g} \)-模. 考虑 \( V \otimes W \) 是域 \(
  \mathbb{F} \) 上的张量积, 我们考虑 \( V \otimes W \) 上的 \( \mathfrak{g}
  \)-模结构: 对 \( x \in \mathfrak{g}, v \in V, W \), 定义 \( x.(v \otimes w) =
  x.v \otimes w + v \otimes x.w \).
\end{definition}
