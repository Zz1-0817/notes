\section{Groups Acting on Sets}

\subsection{Actions}

\begin{definition}
  Let \( X \) be a set and let \( G \) be a group.
  A \emph{left action} of \( G \) on \( X \) is a mapping \( (g, x) \mapsto gx: G \times X \to X \) such that
  \begin{enumerate}
    \item \( 1x = x \), for all \( x \in X \);
    \item \( (g_1 g_2) x = g_1 (g_2x) \), all \( g_1, g_2 \in G, x \in X \).
  \end{enumerate}
  A set together with a left action of \( G \) is called a (left) \emph{\( G \)-set}.
  An action is \emph{trivial} if \( gx = x \) for all \( g \in G \).
\end{definition}

The conditions imply that, for each \( g \in G \), left translation by \( g \),
\[
  g_L: X \to X,\quad x \mapsto gx,
\]
has \( (g^{-1})_L \) as an inverse, and therefore \( g_L \) is a bijection, i.e. \( g_L \in \operatorname{Sym}(X) \).
(2) now says that
\[
  g \mapsto g_L: G \to \operatorname{Sym}(X)
\]
is a homomorphism.

\begin{definition}
  The action is said to be \emph{faithful}(or \emph{effective}) if the homomorphism is injective, i.e., if
  \[
    gx = x \textmd{ for all } x \in X \implies g = 1.
  \]
\end{definition}

\begin{example}
  \begin{enumerate}
    \item Every subgroup of the symmetric group \( S_n \) acts faithfully on \( \left\lbrace 1, 2, \cdots, n \right\rbrace \).
    \item Every subgroup \( H \) of a group \( G \) acts faithfully on \( G \) by left translation,
      \[
        H \times G \to G, \quad (h, x) \mapsto hx.
      \]
    \item The \emph{group of rigid motions} of \( \mathbb{R}^n \) is the group ob bijections \( \mathbb{R}^n \to \mathbb{R}^n \) preserving lengths.
  \end{enumerate}
\end{example}

\begin{definition}
  A \emph{map of} \( G \)-sets is a map \( \varphi: X \to Y \) such that
  \[
    \varphi(gx) = g\varphi(x),\quad \textmd{ all } g \in X,\quad x \in X.
  \]
  An \emph{isomorphism} of \( G \)-sets is a bijective \( G \)-maps; its inverse is then also a \( G \)-map.
\end{definition}

\paragraph{Orbits}

\begin{definition}
  Let \( G \) act on \( X \).
  A subset \( S \subseteq X \) is said to be \emph{stable} under the action of \( G \) if
  \[
    g \in G, x \in S \implies gx \in S.
  \]
  The action of \( G \) on \( X \) then induces an action of \( G \) on \( S \).
\end{definition}

\begin{definition}
  Write \( x \sim_G y \) if \( y = gx \), for some \( g \in G \).
  This is an equivalence relation.
  The equivalence classes are called \( G \)-\emph{orbits}.
  Thus the \( G \)-orbits partition \( X \).
  Write \( G \backslash X \) for the set of orbits.
\end{definition}
By definition, the \( G \)-orbit containing \( x_0 \) is
\[
  G x_0 = \left\lbrace g x_0: g \in G \right\rbrace.
\]

\begin{example}
  \begin{enumerate}
    \item Suppose \( G \) acts on \( X \), and let \( \alpha \in G \) be an element of order \( n \).
      Then the orbits of \( \left\langle \alpha \right\rangle \) are the sets of the form
      \[
        \left\lbrace x_0, \alpha x_0, \cdots, \alpha^{n - 1}x_0 \right\rbrace.
      \]
      These elements need not be distinct!
  \end{enumerate}
\end{example}

Note that a subset of \( X \) is stable \( \iff \) it is a union of orbits.

\begin{definition}
  The action of \( G \) on \( X \) is said to be \emph{transitive}, and \( G \) is said to act \emph{trasitively} on \( X \),h, if there is only one orbit.
  The set \( X \) is then called a \emph{homogeneous} \( G \)-set.
\end{definition}
\begin{example}
  \begin{enumerate}
    \item \( S_n \) acts transitively on \( \left\lbrace 1, 2, \cdots, n \right\rbrace \).
    \item For any subgroup \( H \) of a group \( G \), \( G \) acts transitively on \( G / H \), but the action of \( G \) on itself by conjugation is never transitive if \( G \neq 1 \).
  \end{enumerate}
\end{example}

\begin{definition}
  The action of \( G \) on \( X \) is \emph{doubly transitive} if for any two pairs \( (x_1, x_2), (y_1, y_2) \) of elements of \( X \) with \( x_1 \neq x_2 \) and \( y_1 \neq y_2 \), there exists a single \( g \in G \) such that \( gx_1 = y_1 \) and \( gx_2 = y_2 \).
  Define \( k \)-\emph{fold transitivity} for \( k \geq 3 \) similarly.
\end{definition}

\paragraph{Stabilizers}

\begin{definition}
  Let \( G \) act on \( X \).
  The \emph{stabilizers} (or \emph{isotropy group}) of an element \( x \in X \) is
  \[
    \operatorname{Stab}(x) = \left\lbrace g \in G: gx = x \right\rbrace.
  \]
  It is a subgroup, but it need not be a normal subgroup.
  THe action is \emph{free} if \( \operatorname{Stab}(x) = \left\lbrace e \right\rbrace \) for all \( x \).
\end{definition}

\begin{lemma}
  For any \( g \in G \) and \( x \in X \),
  \[
    \operatorname{Stab}(gx) = g \cdot \operatorname{Stab}(x) \cdot g^{-1}.
  \]
\end{lemma}

\begin{example}
  \begin{enumerate}
    \item Let \( G \) act on itself by conjugation.
      Then
      \[
        \operatorname{Stab}(x) = \left\lbrace g \in G: gx = xg \right\rbrace.
      \]
      This group is called the \emph{centralizer} \( C_G(x) \) of \( x \) in \( G \).
      It consists of all elements of \( G \) that commute with, i.e., centralize, \( x \).
      The intersection
      \[
        \bigcap_{x \in G} C_G(x) = \left\lbrace g \in G: gx = xg \textmd{ for all } x \in G \right\rbrace
      \]
      is the centre of \( G \).
    \item Let \( G \) act on \( G / H \) by left multiplication.
      Then \( \operatorname{Stab}(H) = H \), and the stabilizer of \( gH \) is \( gHg^{-1} \).
  \end{enumerate}
\end{example}

\begin{definition}
  For a subset \( S \) of \( X \), we define the \emph{stabilizer} of \( S \) to be
  \[
    \operatorname{Stab}(S) = \left\lbrace g \in G: gS = S \right\rbrace.
  \]
\end{definition}
We also have
\[
    \operatorname{Stab}(gS) = g \cdot \operatorname{Stab}(S) \cdot g^{-1}.
\]

\begin{example}
  Let \( G \) act on \( G \) by conjugation, and let \( H \) be a subgroup of \( G \).
  The stabilizer of \( H \) is called the \emph{normalizer} \( N_G(H) \) of \( H \) in \( G \):
  \[
    N_G(H) = \left\lbrace g \in G: g H g^{-1} = H \right\rbrace.
  \]
  \( N_G(H) \) is the largest subgroup of \( G \) containing \( H \) as a normal subgroup.
\end{example}

\paragraph{Transtive actions}

\begin{proposition}
  If \( G \) acts transitively on \( X \), then for any \( x_0 \in X \), the map
  \[
    g \operatorname{Stab}(x_0) \mapsto gx_0: G / \operatorname{Stab}(x_0) \to X
  \]
  is an isomorphism of \( G \)-sets.
\end{proposition}

\begin{corollary}
  Let \( G \) act on \( X \), and let \( O = Gx_0 \) be the orbit containing \( x_0 \).
  Then the cardinality of \( O \) is
  \[
    \left\vert O \right\vert = (G: \operatorname{Stab}(x_0)).
  \]
\end{corollary}
\begin{proof}
  \( G / \operatorname{Stab}(x_0) \to Gx_0 \) is a bijection.
\end{proof}

\begin{lemma}
  For any subgroup \( H \) of a group \( G \), \( \bigcap_{g \in G}g H g^{-1} \) is the largest normal subgroup contained in \( H \).
\end{lemma}

\begin{proposition}
  Let \( x_0 \in X \).
  If \( G \) acts transitively on \( X \), then
  \[
    \ker (G \to \operatorname{Sym}(X))
  \]
  is the largest normal subgroup contained in \( \operatorname{Stab}(x_0) \).
\end{proposition}

\begin{proof}
  When
  \[
    \ker(G \to \operatorname{Sym}(X)) = \bigcap_{x \in X} \operatorname{Stab}(x) = \bigcap_{g \in G}\operatorname{Stab}(gx_0) = \bigcap g \cdot \operatorname{Stab}(x_0) \cdot g^{-1}.
  \]
\end{proof}

\paragraph{The class equation}

When \( X \) is finite, it is a disjoint union of a finite number of orbits
\[
  X = \bigcup_{i = 1}^m O_i
\]
Hence:
\begin{proposition}
  The number of elements in \( X \) is
  \[
    \left\vert X \right\vert = \sum_{i = 1}^m \left\vert O_i \right\vert = \sum_{i = 1}^m (G: \operatorname{Stab}(x_i)),\quad x_i \in O_i.
  \]
\end{proposition}

\begin{proposition}[Class Equiation]
  \[
    \left\vert G \right\vert = \sum (G: C_G(x))
  \]
  where \( x \) runs over a set of representatives for the conjugacy classes, or
  \[
    \left\vert G \right\vert = \left\vert Z(G) \right\vert + \sum(G : C_G(y))
  \]
  where \( y \) runs over set of representatives for the conjugacy classes containing more than one element.
\end{proposition}

\begin{theorem}[Cauchy]
  If the prime \( p \) divides \( \left\vert G \right\vert \), then \( G \) contains an element of order \( p \).
\end{theorem}
\begin{proof}
  We use induction on \( \left\vert G \right\vert \).
  If for some \( y \) not in the centre of \( G \) and \( p \) does not divide \( (G: C_G(y)) \).
  Then \( p \) divides the order of \( C_G(y) \) and we apply induction.
  Thus we may suppose that \( p \) divides all of the terms \( (G: C_G(y)) \) in the class equation, and also divides \( Z(G) \).
  But \( Z(G) \) is commutative, and follows from the structure theorem of such groups that \( Z(G) \) will contain an element of order \( p \).
\end{proof}

\begin{corollary}
  A finite group \( G \) is a \( p \)-group \( \iff \) every element has order a power of \( p \).
\end{corollary}
\begin{proof}
  \( \implies \): Lagrange theorem.
  The converse follows from Cauchy's theorem: if not, suppose another \( p \neq p' \mid \left\vert G \right\vert \), then there exists an element \( a \in G \) with order \( p' \) contained in \( G \), a contradiction.
\end{proof}

\begin{corollary}
  Every group of order \( 2p, \) where \( p \) is odd prime, is cyclic or dihedral.
\end{corollary}
\begin{proof}
  From Cauchy's theorem, we know that such a \( G \) contains elements \( s \) and \( r \) of orders \( 2 \) and \( p \) respectively.
  Then \( H \) with index \( 2 \) is normal.
  Obviously, \( s \notin H \), and so \( G = H \cup Hs \):
  \[
    G = \left\lbrace 1, r, \cdots, r^{p - 1}, s, rs, \cdots, r^{p - 1}s \right\rbrace.
  \]
  As \( H \) is normal, \( srs^{-1} = r^i \) for some \( i \).
  Because \( s^2 = 1, r= s^2 r s^{-2} = s(srs^{-1})s^{-1} = r^{i^2} \), and so \( i^2 \equiv 1 \mod{p} \).
  Because \( \mathbb{Z} / p \mathbb{Z} \) is a field, its only elements with square \( 1 \) are \( \pm 1 \), and so \( i \equiv 1 \) or \( -1 \mod{p} \).
  In the first case, the group is commutative; in the second \( s r s^{-1} = r^{-1} \) and we have the dihedral group.
\end{proof}

\paragraph{\( p \)-groups}

\begin{theorem}
  Every nontrivial finite \( p \)-group has nontrivial centre.
\end{theorem}

\begin{corollary}
  A group of order \( p^n \) has normal subgroups of order \( p^m \) for all \( m \leq n \).
\end{corollary}
\begin{proof}
  We use induction on \( n \).
  The centre of \( G \) contains an element \( g \) of order \( p \), and so \( N = \left\langle g \right\rangle \) is a normal subgroup of \( G \) of order \( p \).
  Now apply the induction hypothesis to \( G / N \) and use the correponce theorem.
\end{proof}

\begin{proposition}
  Every group of order \( p^2 \) is commutative, and hence is isomophic to \( C_p \times C_p \) or \( C_{p^2} \).
\end{proposition}
\begin{proof}
  It suffices to show that \( Z(G) = G \) and apply the theorem of the structure of finite generated commutative groups.
  See the following lemma.
\end{proof}

\begin{lemma}
  Suppose \( G \) contains a subgroup \( H \) in its centre(hence \( H \) is normal) such that \( G / H \) is cyclic.
  Then \( G \) is commutative.
\end{lemma}
\begin{proof}
  Let \( a \) be an element of \( G \) whose image in \( G / H \) generates \( it \).
  Then every element of \( G \) can be written \( g = a^i h \) with \( h \in H, i \in \mathbb{Z} \).
  Now
  \[
    a^i h \cdot a^{i'} h' = a^{i'} h' \cdot a^i h,
  \]
  by using the fact that \( H \subseteq Z(G) \).
\end{proof}

\begin{example}
  \( p^3 \).
\end{example}

\paragraph{Actions on the left cosets}
The action of \( G \) on the set of left cosets \( G / H \) of \( H \) in \( G \) is a very useful tool in the study of groups.
We illustrate this with some examples.

Let \( X = G / H \). Recall that, for any \( g \in G \),
\[
  \operatorname{Stab}(gH) = g H g^{-1}
\]
and the kernel of
\[
  G \to \operatorname{Sym}(X)
\]
is the largest normal subgroup \( \bigcap_{g \in G} gH g^{-1} \) of \( G \) contained in \( H \).

%TODO: complete this.

\subsection{Permutation Groups}

Consider \( \operatorname{Sym}(X) \), where \( X \) has \( n \) elements and consider a permutation
\[
  \sigma = \begin{pmatrix}
    1 & 2 & 3 &\cdots &n\\
    \sigma(1) & \sigma(2) & \sigma(3) &\cdots & \sigma(n)
  \end{pmatrix}
\]
The ordered pairs \( (i, j) \) with \( i < j \) and \( \sigma(i) > \sigma(j) \) are called the \emph{inversions} of \( \sigma \), and \( \sigma \) is said to be \emph{even} or \emph{odd} according as the number its inversions is even or odd.
The \emph{signature}, \( \operatorname{sign}(\sigma) \), of \( \sigma \) is \( + 1 \) or \( -1 \) according as \( \sigma \) is even or odd.

\begin{proposition}
  \( \operatorname{sign}(\sigma) \operatorname{sign}(\tau) = \operatorname{sign}(\sigma \tau) \).
\end{proposition}
\begin{proof}
  For a permutation \( \sigma \), consider the products
  \[
    V = \prod_{1 \leq i < j \leq n} (j - i) = (2 - 1)(3 - 1) \cdots (n - 1)(3 - 2) \cdots (n - 2) \cdots (n - (n - 1))
  \]
  and
  \[
    \sigma V = \prod_{1 \leq i < j \leq n}(\sigma(j) - \sigma(i)).
  \]
  Both products run over the \( 2 \)-element subsets \( \left\lbrace i, j \right\rbrace \) of \( \left\lbrace 1, 2, \cdots, n \right\rbrace \) and the terms correponding to a subset are the same except that each inversion introduces a negative sgin.
  Therefore,
  \[
    \sigma V = \operatorname{sign}(\sigma)(V).
  \]

  Now let \( P \) be the additive group of maps \( \mathbb{Z}^n \to \mathbb{Z} \).
  For \( f \in P \) and \( \sigma \in S_n \), let \( \sigma f \) denote the element of \( P \) defined by
  \[
    (\sigma f)(z_1, \cdots, z_n) = f(z_{\sigma(1)}, \cdots, z_{\sigma(n)}).
  \]
  For \( z \in \mathbb{Z}^n \) and \( \sigma \in S_n \), let \( z^\sigma \) denote the element of \( \mathbb{Z}^n \) such that \( (z^\sigma)_i = z_{\sigma(i)} \).
  Then \( (z^\sigma)^\tau = z^{\sigma \tau} \).
  By definition, we have
  \[
    \sigma(\tau f) = (\sigma \tau)f.
  \]
  Let \( p \) be the element of \( P \) defined by
  \[
    p(z_1, \cdots, z_n) = \prod_{1 \leq i < j \leq n}(z_j - z_i).
  \]
  Then
  \[
    \sigma p = \operatorname{sign}(\sigma) p.
  \]
\end{proof}

\begin{definition}
  From the preceding proposition, ``sign'' is a homomorphism \( S_n \to \left\lbrace \pm 1 \right\rbrace \).
  When \( n \geq 2 \), it is surjective, and so its kernel is a normal subgroup of \( S_n \) or order \( n! / 2 \), called the \emph{alternating group} \( A_n \).
\end{definition}

\begin{definition}
  A \emph{cycle} is a permutation of the following form
  \[
    i_1 \mapsto i_2 \mapsto i_3 \mapsto \cdots \mapsto i_r \mapsto i_1,\quad \textmd{remaining} i \textmd{'s fixed.}
  \]
  The \( i_j \) are required to be distinct.
  We denote this cycle by \( (i_1 i_2 \cdots i_r) \), and call \( r \) its \emph{length}.
  A cycle of length \( 2 \) is a \emph{transposition}.
  A cycle of length \( 1 \) is the identity map.
  The \emph{support of the cycle} \( (i_1 \cdots i_r ) \) is the set \( \left\lbrace i_1, \cdots, i_r \right\rbrace \), and cycles are said to be \emph{disjoint} if their supports are disjoint.
\end{definition}
\begin{remark}
  Disjoint cycles commute.
  And if
  \[
    \sigma = (i_1 \cdots i_r)(j_1 \cdots j_s) \cdots (l_1 \cdots l_u)
  \]
  then
  \[
    \sigma^m = (i_1 \cdots i_r)^m(j_1 \cdots j_s)^m \cdots (l_1 \cdots l_u)^m
  \]
  and it follows that \( \sigma \) has order \( \operatorname{lcm}(r,s,\cdots, u) \).
\end{remark}

\begin{proposition}
  Every permutation can be written as a product of disjoint cycles.
\end{proposition}
\begin{example}
  \[
    \begin{pmatrix}
    1 & 2 & 3 & 4 & 5 & 6 & 7 & 8\\
    5 & 7 & 4 & 2 & 1 & 3 & 6 & 8
    \end{pmatrix} = (15)(27634)(8).
  \]
\end{example}

\begin{corollary}
  Each permutation \( \sigma \) can be written as a product of transpositions;
  the number of transpositions in such a product is even or odd according as \( \sigma \) is even or odd.
\end{corollary}
\begin{proof}
  \( (i_1 i_2 \cdots i_r) = (i_1 i_2) \cdots (i_{r - 2}i_{r - 1})(i_{r - 1} i_r) \).
\end{proof}
\begin{remark}
  The signature of a cycle of length \( r \) is \( (-1)^{r - 1} \), that is, an \( r \)-cycle is even or odd according as \( r \) is odd or even.
\end{remark}

\begin{corollary}
  The alternating group \( A_n \) is generated by cycles of length three.
\end{corollary}
\begin{proof}
  \[
    (ij)(kl) = \begin{cases}
      (ij) (jl) = (ijl) & j = k,\\
      (ij)(jk)(jk)(kl) = (ijk)(jkl) & i,j,k,l \textmd{ distinct }\\
      1 & (ij) = (kl)
    \end{cases}
  \]
\end{proof}

\paragraph{Conjugacy class}

\begin{example}
  In \( S_n \), the conjugate of a cycle is given by
  \[
    g(i_1 \cdots i_k)g^{-1} = (g(i_1) \cdots g(i_k)).
  \]
\end{example}

We shall determine the conjugacy classes in \( S_n \).
By a \emph{partition} of \( n \), we mean a sequence of integers \( n_1, \cdots, n_k \) such that
\[
  1 \leq n_1 \leq n_2 \leq \cdots n_k \leq n \textmd{ and } n_1 + n_2 + \cdots + n_k = n.
\]
\begin{proposition}
  Two elements \( \sigma \) and \( \tau \) of \( S_n \) are conjugate \( \iff \) if they define the same partitions of \( n \).
\end{proposition}
\begin{proof}
  \( \impliedby \): Since \( \sigma \) and \( \tau \) define the same partitions of \( n \), their decompositions into products of disjoint cycles have the same type:
  \begin{align*}
    \sigma = (i_1 \cdots i_r)(j_1 \cdots j_s)\cdots(l_1 \cdots l_u),\\
    \tau = (i'_1 \cdots i'_r)(j'_1 \cdots j'_s)\cdots(l'_1 \cdots l'_u).
  \end{align*}
  If we define \( g \) to be
  \[
    \begin{pmatrix}
      i_1 &\cdots &i_r &j_1 &\cdots &j_s &\cdots &l_1 &\cdots &l_u\\
      i'_1 &\cdots &i'_r &j'_1 &\cdots &j'_s &\cdots &l'_1 &\cdots &l'_u
    \end{pmatrix}
  \]
\end{proof}
\begin{remark}
  For \( 1 < k \leq n \), there are \( \frac{n(n - 1) \cdots (n - k + 1)}{k} \) distinct \( k \)-cycles in \( S_n \).
  The \( 1 / k \) is needed so that we don't count
  \[
    (i_1 i_2 \cdots i_k) = (i_k i_1 \cdots i_{k - 1}) = \cdots
  \]
  \( k \) times.
  Similarly, it is possible to compute the number of elements in any conjugacy class in \( S_n \), but a little care is needed when the partition of \( n \) has several terms equal.
  For example, the number of permutation in \( S_4 \) of type \( (ab)(cd) \) is
  \[
    \frac{1}{2}\left(\frac{4 \cdot 3}{2} \cdot \frac{2 \cdot 1}{2}\right) = 3.
  \]
  The \( \frac{1}{2} \) is needed so that we don't count \( (ab)(cd) = (cd)(ab) \) twice.
  For \( S_4 \) we have the follow table:
  \begin{table}[H]
    \centering
    \begin{tabular}{cccc}
      partition & element & No. in Conj. Class & Parity\\
      \( 1 + 1 + 1 + 1 \) & 1 & 1 & even\\
      \( 1 + 1 + 2 \) & \( (ab) \) & 6 & odd\\
      \( 1 + 3 \) & \( (abc) \) & 8 & even\\
      \( 2 + 2 \) & \( (ab)(cd) \) & 3 & even\\
      \( 4 \) & \( (abcd) \) & 6 & odd
    \end{tabular}
  \end{table}
  Note that \( A_4 \) contains exactly \( 3 \) elements of order \( 2 \), namely those of type \( 2 + 2 \), and that with \( 1 \) they form a subgroup \( V \).
  This group is a union of conjugacy classes, and is therefore a normal subgroup of \( S_4 \).
\end{remark}

\paragraph{Alternating groups}

\begin{theorem}[Galois]
  The group \( A_n \) is simple if \( n \geq 5 \).
\end{theorem}

\begin{remark}
  For \( n = 2 \), \( A_n \) is trivial, and for \( n = 3 \), \( A_n \) is cyclic of order \( 3 \), and hence simple.
\end{remark}

\begin{lemma}
  Let \( N \) be a normal subgroup of \( A_n(n \geq 5) \);
  if \( N \) contains a cycle of length three, then it contains all cycles of length three, and so equal \( A_n \).
\end{lemma}
\begin{proof}
  Let \( \gamma \) be the cycle of length three in \( N \), and let \( \sigma \) be a second cycle of length three in \( A_n \).
  We know that \( \sigma = g \gamma g^{-1} \) for some \( g \in S_n \).
  \begin{itemize}
    \item If \( g \in A_n \), then this shows that \( \sigma \) is also in \( N \).
    \item If not, because \( n \geq 5 \), there exists a transposition \( t \in S_n \) disjoint from \( \sigma \).
      Then \( tg \in A_n \), and
      \[
        \sigma = t \sigma t^{-1} = tg \gamma g^{-1} t^{-1},
      \]
      and so again \( \sigma \in N \).
  \end{itemize}
\end{proof}

\begin{lemma}
  Every normal subgroup \( N \) of \( A_n \), \( n \geq 5 \), \( N \neq 1 \), contains a cycle of length \( 3 \).
\end{lemma}
\begin{proof}
  Let \( \sigma \in N \), \( \sigma \neq 1 \).
  If \( \sigma \) is not a \( 3 \)-cycle, then \( \sigma' \neq 1 \), which fixes more elements of \( \left\lbrace 1, 2, \cdots, n \right\rbrace \) than does \( \sigma \).
  If \( \sigma' \) is not, we can apply the same construction.
  After a finite number of steps, we arrive at a \( 3 \)-cycle.

  Suppose \( \sigma \) is not a \( 3 \)-cycle.
  When we express it as a product of disjoint cycles, either it contains a cycle of length \( \geq 3 \) or else it is a product of transpositions.
  \begin{enumerate}
    \item \( \sigma = (i_1 i_2 i_3 \cdots) \cdots \).
      \( \sigma \) moves two numbers, say \( i_4, i_5 \) other than \( i_1, i_2, i_3 \) since \( \sigma \neq (i_1 i_2 i_3), (i_1 \cdots i_4) \).
      Let \( \gamma = (i_3 i_4 i_5) \), then \( \sigma_1 := \gamma \sigma \gamma^{-1} = (i_1 i_2 i_4 \cdots) \cdots \in N \), and is distinct from \( \sigma \).
      Thus \( \sigma' := \sigma_1 \sigma^{-1} \neq 1 \), but \( \sigma' = \gamma \sigma \gamma^{-1} \sigma^{-1} \) fixes \( i_2 \) and all elements other than \( i_1, \cdots, i_5 \) fixed by \( \sigma \).
      Therefore, it fixes more elements than \( \sigma \).
    \item \( \sigma = (i_1 i_2)(i_3 i_4) \cdots \).
      form \( \gamma, \sigma_1, \sigma' \) as the first case with \( i_4 \) as in (2) and \( i_5 \) any element distinct from \( i_1, i_2, i_3, i_4 \).
      Then \( \sigma_1 = (i_1 i_2)(i_4 i_5) \cdots \) is distinct from \( \sigma \) because it acts differently on \( i_4 \).
      Thus \( \sigma' = \sigma_1 \sigma^{-1} \neq 1 \), but \( \sigma' \) fixes \( i_1 \) and \( i_2 \), and all elements \( \neq i_1, \cdots, i_5 \) not fixed by \( \sigma \).
      Therefore it fixes at least one more element than \( \sigma \).
  \end{enumerate}
\end{proof}

\begin{corollary}
  For \( n \geq 5 \), the only normal subgroups of \( S_n \) are \( 1 \), \( A_n \), and \( S_n \).
\end{corollary}
\begin{proof}
  If \( N \) is normal in \( S_n \), then either \( N \cap A_n = A_n \) or \( N \cap A_n = \left\lbrace 1 \right\rbrace \).
  In the second case, the map \( x \mapsto x A_n: N \to S_n / A_n \) is injective, but it can't have order \( 2 \) because no conjugacy class in \( S_n \) consists of a single element.
\end{proof}

\begin{remark}
  A group \( G \) is said to be \emph{solvable} if there exist subgroups
  \[
    G = G_0 \supseteq G_1 \supseteq \cdots \supseteq G_{i - 1} \supseteq G_i \supseteq \cdots \supseteq G_r = \left\lbrace 1 \right\rbrace
  \]
  such that each \( G_i \) is normal in \( G_{i - 1} \) and each quotient \( G_{i - 1} / G_i \) is commutative.
  Thus \( A_n \) is not solvable if \( n \geq 5 \).

  Let \( f(x) \in \mathbb{Q}[x] \) be of degree \( n \).
  In Galois theory, one attaches to \( f \) a subgroup \( G_f \) of the group of permutations of the roots of \( f \) and shows that the roots of \( f \) can be obtained from the coefficients of \( f \) by the algebraic operations of addition, subtraction, multiplication, division, and the extraction of \( m \)th roots \( \iff \) \( G_f \) is solvable.
\end{remark}

\subsection{The Todd-Coxeter algotithm}

Let \( G \) be a group described by a finite presentation, and let \( H \) be a subgroup described by a generated set.
Then the \emph{Todd-Coxeter algorithm } is a strategy for writing down the set of left cosets of \( H \) in \( G \) together with the action of \( G \) on the set.

Let \( G = \left\langle a, b, c \mid a^3, b^2, c^2, cba \right\rangle \) and let \( H \) be the subgroup generated by \( c \).
The operation of \( G \) on the set of cosets is described by the action of generators which must satisfy the following rules
\begin{enumerate}
  \item Each generator acts as a permutaion.
  \item The relations act trivially.
  \item The generators of \( H \) fix the coset \( 1H \).
  \item The operation on the cosets is transitive.
\end{enumerate}

\paragraph{Primitive actions}

\begin{definition}
  Let \( G \) be a group acting on a set \( X \), and let \( \pi \) be a partition of \( X \).
  We say that \( \pi \) is \emph{stabilized} by \( G \) if
  \[
    A \in \pi \implies gA \in \pi.
  \]
\end{definition}
\begin{remark}
  It suffices to check the condition for a set of generators for \( G \).
\end{remark}

\begin{example}
  \begin{enumerate}
    \item The subgroup \( G = \left\langle (1234) \right\rangle \) of \( S_4 \) stabilizes the partition \( \left\lbrace \left\lbrace 1, 3 \right\rbrace, \left\lbrace 2, 4 \right\rbrace \right\rbrace \) of \( \left\lbrace 1, 2, 3, 4 \right\rbrace \).
    \item Identify \( X = \left\lbrace 1, 2, 3, 4 \right\rbrace \) with the set of vertices of the square on which \( D_4 \) acts in the usual way, namely, with \( r = (1234) \), \( s = (24) \).
      Then \( D_4 \) stabilizes the partition \( \left\lbrace \left\lbrace 1, 3 \right\rbrace, \left\lbrace 2, 4 \right\rbrace \right\rbrace \)(opposite vertices stay opposite).
  \end{enumerate}
\end{example}

\begin{definition}
  The group \( G \) always stabilizes the trivial partitions of \( X \), namely, the set of all one-element subsets of \( X \), and \( \left\lbrace X \right\rbrace \).
  When it stabilizes only those partitions, we say that the action is \emph{primitive}; otherwise it is \emph{imprimitive}.
  A subgroup of \( \operatorname{Sym}(X) \) is said to be \emph{primitive} if it acts primitively on \( X \).
\end{definition}
Obviously, \( S_n \) itself is primitive.

\begin{example}
  A doubly transitive action is primitive: if it stabilized
  \[
    \left\lbrace \left\lbrace x, x' \right\rbrace, \left\lbrace y, \cdots, \right\rbrace, \cdots \right\rbrace,
  \]
  then there would be no element sending \( (x, x') \) to \( (x, y) \).
\end{example}

For the remainder of this section, \( G \) is a finite group acting transitively on a set \( X \) with at least two elements.

\begin{proposition}
  The group \( G \) acts imprimitively \( \iff \) there is a proper subset \( A \) of \( X \) with at least \( 2 \) elements such that
  \begin{equation}
    \text{for each } g \in G, \text{ either } gA = A \text{ or } gA \cap A = \emptyset. \tag{*} \label{eq: block}
  \end{equation}
\end{proposition}
\begin{proof}
  \( \impliedby \): From such an \( A \), we can form a partition \( \left\lbrace A, g_1 A, g_2 A, \cdots \right\rbrace \) of \( X \), which is stabilized by \( G \)(Recall that we assume \( G  \) acts transitively on \( X \)).
\end{proof}

\begin{definition}
  A subset \( A \) of \( X \) satisfying \eqref{eq: block} is called \emph{block}.
\end{definition}

\begin{proposition}
  Let \( A \) be a block in \( X \) with \( \left\vert A \right\vert \geq 2 \) and \( A \neq X \).
  For any \( x \in A \),
  \[
    \operatorname{Stab}(x) \subsetneq \operatorname{Stab}(A) \subsetneq G.
  \]
\end{proposition}
\begin{proof}
  \( \operatorname{Stab}(A) \supseteq \operatorname{Stab}(x) \) because
  \[
    gx = x \implies gA \cap A \neq \emptyset \implies gA = A.
  \]
  Let \( y \in A, y \neq x \).
  Because \( G \) acts transitively on \( X \), there is a \( g \in G \) such that \( gx = y \).
  Then \( g \in \operatorname{Stab}(A) \), but \( g \notin \operatorname{Stab}(x) \).
  Let \( y \notin A \).
  There is a \( g \in G \) such that \( gx = y \), and then \( g \notin \operatorname{Stab}(x) \).
\end{proof}

\begin{theorem}
  The group \( G \) acts primitively on \( X \iff \) for one \( x \in X, \operatorname{Stab}(x) \) is a maximal subgroup(hence any) of \( G \).
\end{theorem}
\begin{proof}
  \( \impliedby \), suppose that there exists an \( x \) in \( X \) and a subgroup \( H \) such that
  \[
    \operatorname{Stab}(x) \subsetneq H \subsetneq G.
  \]
    \( A = Hx \) is a block \( \neq X \) with at least two elements
    Because \( H \neq \operatorname{Stab}(x), Hx \neq \left\lbrace x \right\rbrace \), and so \( \left\lbrace x \right\rbrace \subsetneq A \subsetneq X \).
    If \( g \in H \), then \( gA = A \).
    If \( g \notin H \), then \( gA \) is disjoint from \( A \): for suppose \( ghx = h' x \) for some \( h' \in H \);
    then \( h'^{-1}gh \in \operatorname{Stab}(x) \subseteq H \), say \( h'^{-1}gh = h'' \), and \( g = h'h''h^{-1} \in H \).
\end{proof}

\subsection{Sylow Theorem}

In this subsection, all group are finite.
Let \( G \) be a group and let \( p \) be a prime dividing \( (G: 1) \).
A subgroup of \( G \) is called a \emph{sylow} \emph{\( p \)-subgroup of} \( G \) if its order is the highest power of \( p \) dividing \( (G: 1) \).

\paragraph{The Sylow Theorems}

In the proofs, we frequently use that if \( O \) is an orbit for a group \( H \) acting on a set \( X \), and \( x_0 \in O \), then the map \( H \to X, h \mapsto h x_0 \) induces a bijection
\[
  H / \operatorname{Stab}(x_0) \to O;
\]
Therefore
\[
  (H: \operatorname{Stab}(x_0)) = \left\vert O \right\vert.
\]
In particular, when \( H \) is a \( p \)-group, \( \left\vert O \right\vert \) is a power of \( p \), and so either \( O \) consists of a single element, or \( \left\vert O \right\vert \) is divisible by \( p \).

\begin{lemma}
  Let \( H \) be a \( p \)-group acting on a finite set \( X \), and let \( X^H \) be the set of points fixed by \( H \);
  then
  \[
    \left\vert X \right\vert \equiv \left\vert X^H \right\vert \pmod{p}
  \]
\end{lemma}
\begin{proof}
  Immediately from the orbit equation.
\end{proof}
\begin{remark}
  When the lemma is applied to a \( p \)-group \( H \) acting on itself by conjugation, we find that
  \[
    (Z(H) : 1) \equiv (H : 1) \mod{p}
  \]
  and so \( p \mid (Z(H): 1) \).
\end{remark}

\begin{theorem}[Sylow I]
  Let \( G \) be a finite group, and let \( p \) be prime, then \( G \) has a subgroup of order \( p^r \).
\end{theorem}
\begin{proof}
  It suffices to prove this with \( p^r \) the highest power of \( p \) dividing \( (G : 1) \), and so from now on we assume that \( (G: 1) = p^r m \) with \( p \nmid m \).
  Let
  \[
    X = \left\lbrace \text{subsets of } G \text{ with } p^r \text{ elements} \right\rbrace,
  \]
  with the action of \( G \) defined by
  \[
    G \times X \to X,\quad (g, A) \mapsto gA.
  \]
  Let \( A \in X \), and let
  \[
    H = \operatorname{Stab}(A) \coloneq \left\lbrace g \in G: gA = A \right\rbrace.
  \]
  For any \( a_0 \in A, h \mapsto h a_0: H \to A \) is injective, since \( A \subseteq G \).
  And so \( (H : 1) \leq \left\vert A \right\vert = p^r \).
  In the equation
  \[
    (G : 1) = (G : H)(H : 1)
  \]
  we know that \( (G : 1) = p^r m, (H : 1) \leq p^r \) and that \( (G : H) \) is the number of elements in the orbits of \( A \).
  Observe that
  \begin{quote}
    If we can find an \( A \) such that \( p \) doesn't divide the number of elements in its orbit, then we can conclude that \( H = \operatorname{Stab}A \) has order \( p^r \).
  \end{quote}
  The number of elements in \( X \) is
  \[
    \left\vert X \right\vert = \binom{p^rm}{p^r} = \frac{(p^rm)(p^r m - 1) \cdots (p^rm- i) \cdots (p^rm - p^r + 1)}{p^r(p^r - 1) \cdots (p^r - i)\cdots (p^r - p^r + 1)}.
  \]
  Note that, because \( i < p^r \), the power of \( p \) dividing \( p^r m - i \) is the power of \( p \) dividing \( i \).
  The same is true for \( p^r - i \).
  Therefore the correponding terms on top and bottom are divisible by the same powers of \( p \), and so \( p \) does not divide \( \left\vert X \right\vert \).
  Because the orbits form a partition of \( X \),
  \[
    \left\vert X \right\vert = \sum \left\vert O_i \right\vert, \quad O_i\text{ the distinct orbits }.
  \]
  and so at least one of the \( \left\vert O_i \right\vert \) is not divisible by \( p \).
\end{proof}

\begin{theorem}[Sylow II]
  Let \( G \) be a finite group, and let \( \left\vert G \right\vert = p^r m \) with \( m \) not divisible by \( p \).
  \begin{enumerate}
    \item Any two Sylow \( p \)-subgroups are conjugate.
    \item Let \( s_p \) be the number of Sylow \( p \)-subgroups in \( G \);
      then \( s_p \equiv 1 \mod{p} \) and \( s_p \mid m \).
    \item Every \( p \)-subgroup of \( G \) is contained in a Sylow \( p \)-subgroup.
  \end{enumerate}
\end{theorem}
Recall that:
\[
  N_G(H) \coloneq \left\lbrace g \in G: g H g^{-1} = H \right\rbrace,
\]
and that the number of conjugates of \( H \) in \( G \) is \( (G: N_G(H)) \).

\begin{lemma}
  Let \( P \) be a Sylow \( p \)-subgroup of \( G \), and let \( H \) be a \( p \)-subgroup.
  If \( H \) normalizes \( P \), i.e., if \( H \subseteq N_G(P) \), then \( H \subseteq P \).
  In particular, no Sylow \( p \)-subgroup of \( G \) other than \( P \) normalizes \( P \).
\end{lemma}
\begin{proof}
  Because \( H \) and \( P \) are subgroups of \( N_G(P) \) with \( P \) normal in \( N_G(P) \), \( HP \) is a subgroup, and \( H / H \cap P \simeq HP / P \)
  Therefore \( (HP : P) \) is a power of \( p \), but
  \[
    (HP : 1) = (HP : P)(P : 1),
  \]
  and \( (P : 1) \) is the largest power of \( p \) dividing \( (G : 1) \), hence also the largest power of \( p \) dividing \( (HP : 1) \).
  Hence \( (HP : P) = 1 \), and \( H \subseteq P \).
\end{proof}

\begin{proof}[\textsc{proof of Sylow II}]
  \begin{enumerate}
    \item Let \( X \) be the set of Sylow \( p \)-subgroups in \( G \), and let \( G \) ac on \( X \) by conjugation,
      \[
        (g, P) \mapsto g P g^{-1}: \quad G \times X \to X.
      \]
      Let \( O \) be one of the \( G \)-orbits: we have to show \( O \) is all of \( X \).

      Let \( P \in O \), and let \( P \) act on \( O \) through the action of \( G \).
      This single \( G \)-orbit may break up into serveral \( P \)-orbits, one of which will be \( \left\lbrace P \right\rbrace \).
      In fact this is the only one-point orbit because
      \[
        \left\lbrace Q \right\rbrace \text{ is a } P \text{-orbit} \iff P \text{ normalizes } Q.
      \]
      We know that happens only for \( Q = P \) by the preceding lemma.
      Hence the number of elements in every \( P \)-orbit other than \( \left\lbrace P \right\rbrace \) is divisible by \( p \), and we have that \( \left\vert O \right\vert \equiv 1 \mod{p} \).

      Suppose there exists a \( P \notin O \).
      We again let \( P \) act on \( O \), but this time the argument shows that there are no one-point orbit, and so the number of elements in every \( P \)-orbit is divisible by \( p \)(the orbit equation).
      This implies that \( \# O \) is divisible by \( p \), which is a contradiction.
    \item Let \( P \) be a Sylow \( p \)-subgroup of \( G \).
      We have shown that \( s_p \equiv 1 \pmod{p} \).
      Then
      \[
        s_p = (G : N_G(P)) = \frac{(G : 1)}{(N_G(P) : 1)} = \frac{(G : 1)}{(N_G(P): P) \cdot (P : 1)} = \frac{m}{(N_G(P): P)}.
      \]
    \item Let \( H \) be a \( p \)-subgroup of \( G \), and let \( H \) act on the set \( X \) of Sylow \( p \)-subgroups by conjugation.
      Because \( \left\vert X \right\vert = s_p \) is not divisible by \( p \), \( X^H \) must be nonemepty(the first lemma in this subsection).
      But then \( H \) normalizes \( P \) and the preceding lemma implies that \( H \subseteq P \).
  \end{enumerate}
\end{proof}

\begin{corollary}
  A Sylow \( p \)-subgroup is normal \( \iff \) it is only Sylow \( p \)-subgroup.
\end{corollary}

\begin{corollary}
  Suppose that a group \( G \) has only one Sylow \( p \)-subgroup for each prime \( p \) dividing its order.
  Then \( G \) is a direct product of its Sylow \( p \)-subgroups.
\end{corollary}
\begin{proof}
  Let \( P_1, \cdots, P_k \) be Sylow subgroups of \( G \), and let \( \left\vert P_i \right\vert = p^{r_i}_i \), where the \( p_i \) are distinct primes.
  We shall prove by induction on \( k \) that it has order \( p^{r_1}_1 \cdots p^{r_k}_k \).
  We may suppose that \( k \geq 2 \) and tht \((P_1 \cdots P_{k - 1})P_{k - 1}\) has order \( p^{r_1}_1\cdots p^{r_{k - 1}}_{k - 1} \).
  Then \( P_1 \cdots P_{k - 1} \cap P_k = 1 \)(consider the order of element) then \( (P_1 \cdots P_{k - 1})P_k \) is the direct product of \( P_1 \cdots P_{k - 1} \) and \( P_k \), and so has order \( p^{r_1}_1 \cdots p^{r_k}_k \).
\end{proof}
