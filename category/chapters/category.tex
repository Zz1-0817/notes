\chapter{范畴的基本架构}

% \section{构成要素}
%
\section{形式化定义}

\subsection{范畴}

一个\emph{范畴}\( \mathcal{C} \) 系指以下资料:
\begin{enumerate}
  \item 集合 \( \operatorname{Ob}(\mathcal{C}) \), 其元素称作 \( \mathcal{C}
    \) 的\emph{对象}.
  \item 集合 \( \operatorname{Mor}(\mathcal{C}) \), 其元素称作 \( \mathcal{C}
    \) 的\emph{态射(或箭头)}, 配上一对映射 \( \operatorname{Mor}(\mathcal{C})
    \mathop{\rightrightarrows}\limits_{t}^{s} \operatorname{Ob}(\mathcal{C})
    \), 其中 \( s \) 和 \( t \) 分别给出态射的\emph{来源} 和\emph{目标}. 对
    \( X, Y \in \operatorname{Ob}(\mathcal{C}) \), 一般习惯记 \(
    \operatorname{Hom}_{\mathcal{C}}(X, Y) := s^{-1}(X) \cap t^{-1}(Y) \),
    或简单记作 \( \operatorname{Hom}(X, Y) \), 称为 \( \operatorname{Hom}
    \)-集, 其元素称为从 \( X \) 到 \( Y \) 的态射. 一般也将 \( f \in
    \operatorname{Hom}_{\mathcal{C}}(X, Y) \) 写作 \( f: X \to Y \) 或 \( X
    \xrightarrow{f} Y \), 故态射有时也叫作\emph{箭头}.
  \item 对每个对象 \( X \), 给定元素 \( \operatorname{id}_X \in
    \operatorname{Hom}_{\mathcal{C}}(X, X) \), 称为 \( X \)
    到自身的\emph{恒等态射}.
  \item 对任意 \( X, Y, Z \in \operatorname{Ob}(\mathcal{C}) \),
    给定态射间的\emph{合成态射}
    \begin{align*}
      \circ: \operatorname{Hom}_{\mathcal{C}}(Y, Z) \times
      \operatorname{Hom}_{\mathcal{C}}(X, Y) &\to
      \operatorname{Hom}_{\mathcal{C}}(X, Z)\\
      (f, g) &\mapsto f \circ g,
    \end{align*}
    不引起混淆的情况下, 将 \( f \circ g \) 简记作 \( fg \).
    恒等态射和合成态射满足
    \begin{itemize}
      \item 结合律: 对于任意态射 \( h, g, f \in
        \operatorname{Mor}(\mathcal{C}) \), 若合成 \( f(gh) \) 和 \( (fg)h \)
        都有定义, 则
        \[
          f(gh) = (fg)h.
        \]
        故两边都可以写成 \( f \circ g \circ h \) 或 \( fgh \);
      \item 对于任意态射 \( f \in \operatorname{Hom}_{\mathcal{C}}(X, Y) \),
        有
        \[
          f \circ \operatorname{id}_X = f = \operatorname{id}_Y \circ f.
        \]
    \end{itemize}
\end{enumerate}
其中, \( \operatorname{id}_X \) 被其性质唯一确定.


\begin{example}
  对象与态射集皆空的范畴称为 \emph{空范畴}, 记作 \( \mathbf{0} \).
\end{example}

\paragraph{小范畴} 一个范畴 \( \mathcal{C} \) 称为是\emph{小范畴} 如果 \(
\operatorname{Ob}(\mathcal{C}) \) 是一个集合而非真类.

\paragraph{反范畴} 给定一个范畴 \( \mathcal{C} \) , 称范畴 \(
\mathcal{C}^{\operatorname{op}} \) 为 \( \mathcal{C} \) 的\emph{反范畴}:
给定任意 \( \mathcal{C} \), 其 \emph{反范畴} \( \mathcal{C}^{\operatorname{op}}
\) 定义为
\begin{enumerate}
  \item \( \operatorname{Ob}(\mathcal{C}^{\operatorname{ob}}) =
    \operatorname{Ob})(\mathcal{C}) \).
  \item 对任意对象 \( X, Y \), \(
    \operatorname{Hom}_{\mathcal{C}^{\operatorname{op}}}(X, Y) :=
    \operatorname{Hom}_{\mathcal{C}}(Y, X) \).
  \item 态射 \( f \in \operatorname{Hom}_{\mathcal{C}^{\operatorname{op}}}(Y,
    Z), g \in \operatorname{Hom}_{\mathcal{C}^{\operatorname{op}}}(X, Y) \) 在
    \( \mathcal{C}^{\operatorname{op}} \) 中的合成 \( f
    \circ^{\operatorname{op}} g \) 定义为 \( \mathcal{C} \) 中的 \( g \circ f
    \).
  \item 恒等映射定义同 \( \mathcal{C} \).
\end{enumerate}

\paragraph{子范畴, 全子范畴} 称 \( \mathcal{C}' \) 为 \( \mathcal{C} \)
的\emph{子范畴}, 如果
  \begin{enumerate}
    \item \( \operatorname{Ob}(\mathcal{C}') \subseteq
      \operatorname{Ob}(\mathcal{C}) \).
    \item \( \operatorname{Mor}(\mathcal{C}') \subseteq
      \operatorname{Mor}(\mathcal{C}) \), 并保持恒等映射.
    \item 来源映射以及目标映射是 \( \operatorname{Mor}(\mathcal{C'})
      \mathop{\rightrightarrows}\limits_{t}^{s} \operatorname{Ob}(\mathcal{C'})
      \) 是 \( \mathcal{C} \) 的限制.
    \item \( \mathcal{C}' \) 中态射的合成亦是 \( \mathcal{C} \) 的限制.
  \end{enumerate}
更进一步地, 如果 \( \operatorname{Hom}_{\mathcal{C}'}(X, Y) =
\operatorname{Hom}_{\mathcal{C}}(X, Y) \), 那么称 \( \mathcal{C}' \) 是 \(
\mathcal{C} \) 的\emph{全子范畴}.

\section{对象与态射}

假设下面对象和态射都在一个给定的范畴 \( \mathcal{C} \) 中讨论

\subsection{对象同构}

态射 \( f: X \to Y \) 称为是\emph{可逆的}, 如果存在 \( g: Y \to X \), 使得 \( fg
= \operatorname{id}_Y, gf = \operatorname{id}_X \), 则称 \( f \) 是 \emph{同构},
而 \( g \) 称为 \( f \) 的\emph{逆}.

\( f \) 的逆 \( g \) 存在性蕴含着唯一性: 假设 \( g' \) 也为 \( f \) 的逆, 那么
\[
  g' = g'(fg) = (g'f) = g.
\]

\subsection{态射}

假设有一个态射 \( f: X \to Y \).
\begin{enumerate}
  \item 称 \( f \) 为\emph{单态射}, 如果对任何对象 \( Z \) 和任一对态射 \( g, h: Z
    \to X \) 有 \( fg = fh \) 当且仅当 \( g = h \).
  \item 称 \( f \) 为\emph{满态射}, 如果对任何对象 \( Z \) 和任一对态射 \( g,
    h: Y \to Z \) 有 \( gf = hf \) 当且仅当 \( g = h \).
  \item 如果存在 \( g: Y \to X \) 使得 \( gf = \operatorname{id}_X \), 那么称
    \( f \) \emph{左可逆}, \( g \) 为 \( f \) 的\emph{左逆(收缩)}.
  \item 如果存在 \( g: Y \to X \) 使得 \( fg = \operatorname{id}_Y \), 那么称
    \( f \) \emph{右可逆}, \( g \) 为 \( f \) 的\emph{右逆(截面)}.
\end{enumerate}
由定义立刻可以知道, 态射 \( f \) 左可逆蕴含 \( f \) 单;
\( f \) 右可逆蕴含 \( f \) 满;
态射可逆当且仅当其左右均可逆.

对一般的范畴来说, 态射 \( f \) 单(满)不一定蕴含着 \( f \) 由左(右)逆.
然而, 对集合范畴来说,
这个命题为真\href{https://math.stackexchange.com/questions/1134772/does-an-injective-function-always-have-a-left-inverse}{stackexchange}

\subsection{始对象, 终对象与零对象}
假设给定了一个范畴 \( \mathcal{C} \).
\begin{enumerate}
  \item 如果 \( X \in \operatorname{Ob}(\mathcal{C}) \) 使得 \(
    \operatorname{Hom}_{\mathcal{C}}(X, Y) \) 有且仅有一个元素对所有 \( Y \in
    \operatorname{Ob}(\mathcal{C}) \) 成立, 那么我们称 \( X \) 是 \( \mathcal{C}
    \) 中的\emph{始对象}.
  \item 如果 \( X \in \operatorname{Ob}(\mathcal{C}) \) 使得 \(
    \operatorname{Hom}_{\mathcal{C}}(Y, X) \) 有且仅有一个元素对所有 \( Y \in
    \operatorname{Ob}(\mathcal{C}) \) 成立, 那么我们称 \( X \) 是 \( \mathcal{C}
    \) 中的\emph{终对象}.
  \item 如果 \( X \in \operatorname{Ob}(\mathcal{C}) \) 既是始对象又是终对象,
    那么我们称其为\emph{零对象}.
\end{enumerate}

\begin{proposition}
  如果 \( X, X' \) 为 \( \mathcal{C} \) 中的始对象, 那么存在唯一的同构 \( X
  \simeq X' \). 类似地, 如果 \( X, X' \) 为 \( \mathcal{C} \) 中的终对象,
  那么存在唯一的同构 \( X \simeq X' \).
\end{proposition}

假设 \( \mathcal{C} \) 有零对象, 记作 \( \mathbf{0} \). 对任意 \( X, Y \in
\operatorname{Ob}(\mathcal{C}) \), 我们定义 \emph{零态射} \( 0: X \to Y \) 为复合
\[
  X \to 0 \to Y.
\]

\begin{example}
  下面是一些典型的例子.
  \begin{enumerate}
    \item 集合范畴 \( \operatorname{Set} \): \( \varnothing \) 是始对象, \(
      \left\lbrace \operatorname{pt} \right\rbrace \) 是终对象.
    \item 带基点的集合范畴 \( \operatorname{Set}_{\bullet} \): \( (\left\lbrace
      \operatorname{pt} \right\rbrace, \operatorname{pt}) \) 是零对象.
    \item 群范畴 \( \operatorname{Grp} \): 平凡群 \( \left\lbrace 1
      \right\rbrace \) 是零对象, 取常值 \( 1 \) 的同态是零态射.
    \item 域 \( \Bbbk \) 上的向量空间范畴 \(
      \operatorname{Vect}(\Bbbk) \): 零空间是零对象, 零映射是零态射.
    \item 离散范畴既没有始对象, 也没有终对象.
  \end{enumerate}
\end{example}


\subsection{模型初步}

\paragraph{偏序范畴} 预序集可以视为两个对象间至多只有一个态射的范畴.

对于预序集 \( (P, \leq) \), 定义范畴使得其对象集为 \( P \), 而存在态射 \( p \to
p' \) 当且仅当 \( p \leq p' \), 并且这样的态射唯一.
作为特例, \( 0 \) 给出空范畴 \( \mathbf{0} \), 而 \( 1 \)
给出恰有一个对象和一个态射的范畴 \( \mathbf{1} \).

\paragraph{离散范畴} 给定集合 \( S \). \( S \) 对应的 \emph{离散范畴} \(
\operatorname{Disc}(S) \)定义如:
\( \operatorname{Ob}(\operatorname{Disc}(S)) = S \), 而态射由恒等映射 \(
\left\lbrace \operatorname{id}_{x}: x \in S \right\rbrace \)组成.

\section{函子范畴}

\subsection{函子}

假设 \( \mathcal{C}', \mathcal{C} \) 为范畴, 一个\emph{函子} \( F: \mathcal{C}'
\to \mathcal{C} \) 指的是下面资料
\begin{enumerate}
  \item 对象间的映射 \( F: \operatorname{Ob}(\mathcal{C}') \to
    \operatorname{Ob}(\mathcal{C}) \).
  \item 态射间的映射 \( F: \operatorname{Mor}(\mathcal{C}') \to
    \operatorname{Mor}(\mathcal{C}) \), 使得
    \begin{itemize}
      \item 对每个 \( X, Y \in \operatorname{Ob}(\mathcal{C}') \) 皆有映射 \(
        F: \operatorname{Hom}_{\mathcal{C}'}(X, Y) \to
        \operatorname{Hom}_{\mathcal{C}}(FX, FY) \).
      \item \( F(g \circ f) = F(g) \circ F(f), F(\operatorname{id}_X) =
        \operatorname{id}_{FX} \).
    \end{itemize}
\end{enumerate}
对于 \( F: \mathcal{C}_1 \to \mathcal{C}_2, G: \mathcal{C}_2 \to C_3 \),
\emph{合成函子} \( G \circ F \) 指的是分别取合成映射
\begin{align*}
  &\operatorname{Ob}(\mathcal{C}_1) \xrightarrow{F}
  \operatorname{Ob}(\mathcal{C}_2) \xrightarrow{G}
  \operatorname{Ob}(\mathcal{C}_3)\\
  &\operatorname{Mor}(\mathcal{C}_1) \xrightarrow{F}
  \operatorname{Mor}(\mathcal{C}_2) \xrightarrow{G}
  \operatorname{Mor}(\mathcal{C}_3)
\end{align*}

\begin{remark}
  \label{remark-opposite-category-functor}
  从 \( \mathcal{C}' \) 到 \( \mathcal{C} \) 和从 \(
  (\mathcal{C}')^{\operatorname{op}} \) 到 \( \mathcal{C}^{\operatorname{op}} \)
  的函子是一回事. 为了区别, 对于函子 \( F: \mathcal{C}' \to \mathcal{C} \)
  \emph{反范畴间的函子} 记作 \( F^{\operatorname{op}}:
  (\mathcal{C}')^{\operatorname{op}} \to \mathcal{C}^{\operatorname{op}} \).
\end{remark}

\paragraph{本质满, 忠实, 全}

对于函子 \( F: \mathcal{C}' \to \mathcal{C} \).
\begin{enumerate}
  \item 称 \( F \) 是\emph{本质满}的, 如果 \( \mathcal{C} \) 中任意对象都同构于某个
    \( FX \).
  \item 称 \( F \) 是\emph{忠实}的, 如果对所有的 \( X, Y \in
    \operatorname{Ob} (\mathcal{C}') \), 映射 \(
    \operatorname{Hom}_{\mathcal{C}'}(X, Y) \to
    \operatorname{Hom}_{\mathcal{C}}(FX, FY) \) 都是单射.
    特别地, 范畴 \( \mathcal{C} \) 称为是\emph{具体的}, 如果存在忠实函子 \( U:
    \mathcal{C} \to \operatorname{Set} \).
  \item 称 \( F \) 是\emph{全}的, 如果对所有的 \( X, Y \in
    \operatorname{Ob} (\mathcal{C}') \), 映射 \(
    \operatorname{Hom}_{\mathcal{C}'}(X, Y) \to
    \operatorname{Hom}_{\mathcal{C}}(FX, FY) \) 都是满射.
\end{enumerate}

\begin{example}
  一些经典的例子包括
  \begin{enumerate}
    \item 考虑群范畴 \( \operatorname{Grp} \). 对于任一个群 \( G \),
      总是可以忘掉 \( G \) 的群结构而视之为集合.
      群同态当然也可以视为集合间的映射. 此程序给出 \emph{忘却函子 \(
      \operatorname{Grp} \to \operatorname{Set} \) }.
      准此要领可对其他结构定义忘却函子, 例如 \( \operatorname{Top} \to
      \operatorname{Set}, \operatorname{Vect(\Bbbk)} \to \operatorname{Ab}
      \)(忘记纯量乘法)等等. 这类函子显然忠实而非全.
    \item 考虑域 \( \Bbbk \) 上的线性空间范畴 \( \operatorname{Vect}(\Bbbk) \).
      对于任意 \( \Bbbk \)-线性空间 \( V \), 定义其对偶空间
      \[
        V^{\vee} := \operatorname{Hom}_{\Bbbk}(V, \Bbbk) = \left\lbrace \Bbbk
        \text{-线性映射} V \to \Bbbk \right\rbrace.
      \]
      任一线性映射 \( f: V_1 \to V_2 \) 诱导对偶空间的反向映射
      \begin{align*}
        f^{\vee}: V_2^{\vee} &\to V_1^{\vee},\\
        (\lambda: V_2 \to \Bbbk) &\mapsto \lambda \circ f.
      \end{align*}
      容易知道 \( D: V \mapsto V^{\vee}, f \mapsto f^{\vee} \) 定义了函子 \( D:
      \operatorname{Vect}(\Bbbk)^{\operatorname{op}} \to
      \operatorname{Vect}(\Bbbk) \), 我们称其为 \emph{对偶函子}. 可以验证 \( D
      \) 是忠实的.
  \end{enumerate}
\end{example}

\subsection{函子的函子 -- 自然变换}

\paragraph{自然变换} 函子 \( F, G:\mathcal{C}' \to \mathcal{C} \)
之间的\emph{自然变换} \( \theta \) 是一族态射
\[
  \theta_X \in \operatorname{Hom}_{\mathcal{C}}(FX, GX),\quad X \in
  \operatorname{Ob}(\mathcal{C}')
\]
使得下图对所有 \( \mathcal{C}' \) 中的态射 \( f: X \to Y \) 交换
% https://q.uiver.app/#q=WzAsNCxbMCwwLCJGWCJdLFsxLDAsIkdYIl0sWzAsMSwiRlkiXSxbMSwxLCJHWSJdLFswLDIsIkZmIiwyXSxbMCwxLCJcXHRoZXRhX1giXSxbMSwzLCJHZiJdLFsyLDMsIlxcdGhldGFfWSIsMl1d
\[\begin{tikzcd}
  FX & GX \\
  FY & GY
  \arrow["{\theta_X}", from=1-1, to=1-2]
  \arrow["Ff"', from=1-1, to=2-1]
  \arrow["Gf", from=1-2, to=2-2]
  \arrow["{\theta_Y}"', from=2-1, to=2-2]
\end{tikzcd}\]
我们也将自然变换 \( \theta: F \to G \) 称为从\emph{函子 \( F \) 到 \( G \) 的态射}.

\begin{definition}
自然变换的合成包括横合成和纵合成两种.

考虑 \( \mathcal{C}' \) 到 \( \mathcal{C} \) 的三个函子间的态射 \(
\theta: F \to G, \psi: G \to H \). \emph{纵合成} \( \psi \circ \theta \) 的定义为
\( \left\lbrace \psi_X \circ \theta_X: X \in \operatorname{Ob}(\mathcal{C}')
\right\rbrace \)
% https://q.uiver.app/#q=WzAsNSxbMCwwLCJcXG1hdGhjYWx7Q30nIl0sWzIsMCwiXFxtYXRoY2Fse0N9Il0sWzMsMCwiXFxsZWFkc3RvIl0sWzQsMCwiXFxtYXRoY2Fse0N9JyJdLFs2LDAsIlxcbWF0aGNhbHtDfSJdLFswLDEsIkciLDFdLFswLDEsIkYiLDAseyJvZmZzZXQiOi0zLCJjdXJ2ZSI6LTN9XSxbMCwxLCJIIiwyLHsib2Zmc2V0IjoyLCJjdXJ2ZSI6M31dLFszLDQsIkYiLDAseyJjdXJ2ZSI6LTN9XSxbMyw0LCJIIiwyLHsiY3VydmUiOjN9XSxbNiw1LCJcXHRoZXRhIiwwLHsic2hvcnRlbiI6eyJzb3VyY2UiOjMwLCJ0YXJnZXQiOjMwfX1dLFs1LDcsIkgiLDAseyJzaG9ydGVuIjp7InNvdXJjZSI6MzAsInRhcmdldCI6MzB9fV0sWzgsOSwiXFxwc2kgXFxjaXJjIFxcdGhldGEiLDEseyJzaG9ydGVuIjp7InNvdXJjZSI6MjAsInRhcmdldCI6MjB9fV1d
\[\begin{tikzcd}
  {\mathcal{C}'} && {\mathcal{C}} & \leadsto & {\mathcal{C}'} && {\mathcal{C}}
  \arrow[""{name=0, anchor=center, inner sep=0}, "G"{description}, from=1-1, to=1-3]
  \arrow[""{name=1, anchor=center, inner sep=0}, "F", shift left=3, curve={height=-18pt}, from=1-1, to=1-3]
  \arrow[""{name=2, anchor=center, inner sep=0}, "H"', shift right=2, curve={height=18pt}, from=1-1, to=1-3]
  \arrow[""{name=3, anchor=center, inner sep=0}, "F", curve={height=-18pt}, from=1-5, to=1-7]
  \arrow[""{name=4, anchor=center, inner sep=0}, "H"', curve={height=18pt}, from=1-5, to=1-7]
  \arrow["\theta", shorten <=5pt, shorten >=5pt, Rightarrow, from=1, to=0]
  \arrow["H", shorten <=4pt, shorten >=4pt, Rightarrow, from=0, to=2]
  \arrow["{\psi \circ \theta}"{description}, shorten <=5pt, shorten >=5pt, Rightarrow, from=3, to=4]
\end{tikzcd}\]
考虑函子 \( 
  % https://q.uiver.app/#q=WzAsMyxbMCwwLCJcXG1hdGhjYWx7Q30nJyJdLFsxLDAsIlxcbWF0aGNhbHtDfSciXSxbMiwwLCJcXG1hdGhjYWx7Q30iXSxbMCwxLCJGXzEiLDAseyJvZmZzZXQiOi0xfV0sWzAsMSwiRl8yIiwyLHsib2Zmc2V0IjoxfV0sWzEsMiwiR18yIiwyLHsib2Zmc2V0IjoxfV0sWzEsMiwiR18xIiwwLHsib2Zmc2V0IjotMX1dXQ==
  \begin{tikzcd}
    {\mathcal{C}''} & {\mathcal{C}'} & {\mathcal{C}}
    \arrow["{F_1}", shift left, from=1-1, to=1-2]
    \arrow["{F_2}"', shift right, from=1-1, to=1-2]
    \arrow["{G_2}"', shift right, from=1-2, to=1-3]
    \arrow["{G_1}", shift left, from=1-2, to=1-3]
  \end{tikzcd}
  \) 及态射 \( \theta: F_1 \to F_2, \psi: G_1 \to G_2 \). 将\emph{横合成}
  定义为 \(
  \psi \circ \theta: G_1 \circ F_1 \to G_2 \circ F_2 \). 具体地, \( \theta:
  F_1 \to F_2 \) 以及 \( G_1 \) 给出了下面交换图表
% https://q.uiver.app/#q=WzAsOSxbMCwwLCJGXzFYIl0sWzAsMiwiRl8xWSJdLFsyLDAsIkZfMlgiXSxbMiwyLCJGXzJZIl0sWzQsMCwiR18xIEZfMSBYIl0sWzYsMCwiR18xRl8yWCJdLFs0LDIsIkdfMUZfMVkiXSxbNiwyLCJHXzFGXzJZIl0sWzMsMSwiXFxsZWFkc3RvIl0sWzAsMSwiRl8xIGYiLDJdLFswLDIsIlxcdGhldGFfWCJdLFsyLDMsIkZfMiBmIl0sWzEsMywiXFx0aGV0YV9ZIiwyXSxbNCw2LCJHXzFGXzFmIiwyXSxbNCw1LCJHXzFcXHRoZXRhX1giXSxbNSw3LCJHXzFGXzJmIl0sWzYsNywiR18xXFx0aGV0YV9ZIiwyXV0=
\[\begin{tikzcd}
{F_1X} && {F_2X} && {G_1 F_1 X} && {G_1F_2X} \\
&&& \leadsto \\
{F_1Y} && {F_2Y} && {G_1F_1Y} && {G_1F_2Y}
\arrow["{\theta_X}", from=1-1, to=1-3]
\arrow["{F_1 f}"', from=1-1, to=3-1]
\arrow["{F_2 f}", from=1-3, to=3-3]
\arrow["{G_1\theta_X}", from=1-5, to=1-7]
\arrow["{G_1F_1f}"', from=1-5, to=3-5]
\arrow["{G_1F_2f}", from=1-7, to=3-7]
\arrow["{\theta_Y}"', from=3-1, to=3-3]
\arrow["{G_1\theta_Y}"', from=3-5, to=3-7]
\end{tikzcd}\]
上图右边交换图在自然变换 \( \psi \) 下有交换图
% https://q.uiver.app/#q=WzAsOCxbMCwyLCJHXzEgRl8xIFgiXSxbMiwyLCJHXzEgRl8yIFgiXSxbNCwyLCJHXzEgRl8yIFkiXSxbNiwyLCJHXzIgRl8yIFkiXSxbMCw0LCJHXzEgRl8xIFkiXSxbMiw0LCJHXzEgRl8yIFkiXSxbMiwwLCJHXzJGXzIgWCJdLFswLDAsIkdfMiBGXzEgWCJdLFswLDQsIkdfMSBGXzEgZiIsMl0sWzEsNSwiR18xRl8yIGYiXSxbNCw1LCJHXzEgXFx0aGV0YV9ZIiwyXSxbMCwxLCJHXzFcXHRoZXRhX1giLDFdLFswLDcsIlxccHNpX3tGXzEgWH0iXSxbMSw2LCJcXHBzaV97Rl8yIFh9IiwyXSxbNyw2LCJHXzIgXFx0aGV0YV9YIl0sWzEsMiwiR18xIEZfMiBmIiwxXSxbNiwzLCJHXzIgRl8yIGYiXSxbMiwzLCJcXHBzaV97Rl8yIFl9IiwxXSxbNSwzLCJcXHBzaV97Rl8yWX0iLDJdXQ==
\[\begin{tikzcd}
{G_2 F_1 X} && {G_2F_2 X} \\
\\
{G_1 F_1 X} && {G_1 F_2 X} && {G_1 F_2 Y} && {G_2 F_2 Y} \\
\\
{G_1 F_1 Y} && {G_1 F_2 Y}
\arrow["{G_2 \theta_X}", from=1-1, to=1-3]
\arrow["{G_2 F_2 f}", from=1-3, to=3-7]
\arrow["{\psi_{F_1 X}}", from=3-1, to=1-1]
\arrow["{G_1\theta_X}"{description}, from=3-1, to=3-3]
\arrow["{G_1 F_1 f}"', from=3-1, to=5-1]
\arrow["{\psi_{F_2 X}}"', from=3-3, to=1-3]
\arrow["{G_1 F_2 f}"{description}, from=3-3, to=3-5]
\arrow["{G_1F_2 f}", from=3-3, to=5-3]
\arrow["{\psi_{F_2 Y}}"{description}, from=3-5, to=3-7]
\arrow["{G_1 \theta_Y}"', from=5-1, to=5-3]
\arrow["{\psi_{F_2Y}}"', from=5-3, to=3-7]
\end{tikzcd}\]
因此横合成为
\begin{equation}
(\psi \circ \theta)_{X} = G_2(\theta_X) \circ \psi_{F_1 X} =
\psi_{F_2 X} \circ G_1(\theta_X)  \text{ 以及 } (\psi \circ \theta)_Y =
\psi_{F_2 Y} \circ G_1(\theta_Y).
\label{equation-natural-transformation-horizontal-composite}
\end{equation}
我们可以简要地写作
% https://q.uiver.app/#q=WzAsNixbMCwwLCJcXG1hdGhjYWx7Q30nJyJdLFsxLDAsIlxcbWF0aGNhbHtDfSciXSxbMiwwLCJcXG1hdGhjYWx7Q30iXSxbMywwLCJcXGxlYWRzdG8iXSxbNCwwLCJcXG1hdGhjYWx7Q30nJyJdLFs2LDAsIlxcbWF0aGNhbHtDfSJdLFswLDEsIkZfMiIsMix7ImN1cnZlIjoyfV0sWzAsMSwiRl8xIiwwLHsiY3VydmUiOi0yfV0sWzEsMiwiR18yIiwyLHsiY3VydmUiOjJ9XSxbMSwyLCJHXzEiLDAseyJjdXJ2ZSI6LTJ9XSxbNCw1LCJHXzIgRl8yIiwyLHsiY3VydmUiOjN9XSxbNCw1LCJHXzFGXzEiLDAseyJjdXJ2ZSI6LTN9XSxbNyw2LCJcXHRoZXRhIiwxLHsic2hvcnRlbiI6eyJzb3VyY2UiOjIwLCJ0YXJnZXQiOjIwfX1dLFs5LDgsIlxccHNpIiwxLHsic2hvcnRlbiI6eyJzb3VyY2UiOjIwLCJ0YXJnZXQiOjIwfX1dLFsxMSwxMCwiXFxwc2kgXFxjaXJjIFxcdGhldGEiLDEseyJzaG9ydGVuIjp7InNvdXJjZSI6MjAsInRhcmdldCI6MjB9fV1d
\[\begin{tikzcd}
	{\mathcal{C}''} & {\mathcal{C}'} & {\mathcal{C}} & \leadsto & {\mathcal{C}''} && {\mathcal{C}}
	\arrow[""{name=0, anchor=center, inner sep=0}, "{F_2}"', curve={height=12pt}, from=1-1, to=1-2]
	\arrow[""{name=1, anchor=center, inner sep=0}, "{F_1}", curve={height=-12pt}, from=1-1, to=1-2]
	\arrow[""{name=2, anchor=center, inner sep=0}, "{G_2}"', curve={height=12pt}, from=1-2, to=1-3]
	\arrow[""{name=3, anchor=center, inner sep=0}, "{G_1}", curve={height=-12pt}, from=1-2, to=1-3]
	\arrow[""{name=4, anchor=center, inner sep=0}, "{G_2 F_2}"', curve={height=18pt}, from=1-5, to=1-7]
	\arrow[""{name=5, anchor=center, inner sep=0}, "{G_1F_1}", curve={height=-18pt}, from=1-5, to=1-7]
	\arrow["\theta"{description}, shorten <=3pt, shorten >=3pt, Rightarrow, from=1, to=0]
	\arrow["\psi"{description}, shorten <=3pt, shorten >=3pt, Rightarrow, from=3, to=2]
	\arrow["{\psi \circ \theta}"{description}, shorten <=5pt, shorten >=5pt, Rightarrow, from=5, to=4]
\end{tikzcd}\]
\end{definition}

\begin{proposition}
  纵横合成都是函子间的态射, 满足结合律 \( (\phi \circ \psi) \circ \theta = \phi
  \circ (\psi \circ \theta) \). 并且纵横合成有下述关系: 对于图
  % https://q.uiver.app/#q=WzAsMyxbMCwwLCJcXG1hdGhjYWx7Q31fMSJdLFsyLDAsIlxcbWF0aGNhbHtDfV8yIl0sWzQsMCwiXFxtYXRoY2Fse0N9XzMiXSxbMCwxLCIiLDAseyJjdXJ2ZSI6LTR9XSxbMSwyLCIiLDAseyJjdXJ2ZSI6LTR9XSxbMSwyLCIiLDAseyJjdXJ2ZSI6NH1dLFswLDEsIiIsMCx7ImN1cnZlIjo0fV0sWzAsMV0sWzEsMl0sWzMsNywiXFx0aGV0YSIsMCx7InNob3J0ZW4iOnsic291cmNlIjoyMCwidGFyZ2V0IjoyMH19XSxbNyw2LCJcXHBzaSIsMCx7InNob3J0ZW4iOnsic291cmNlIjoyMCwidGFyZ2V0IjoyMH19XSxbNCw4LCJcXHRoZXRhJyIsMCx7InNob3J0ZW4iOnsic291cmNlIjoyMCwidGFyZ2V0IjoyMH19XSxbOCw1LCJcXHBzaSciLDAseyJzaG9ydGVuIjp7InNvdXJjZSI6MjAsInRhcmdldCI6MjB9fV1d
  \[\begin{tikzcd}
    {\mathcal{C}_1} && {\mathcal{C}_2} && {\mathcal{C}_3}
    \arrow[""{name=0, anchor=center, inner sep=0}, curve={height=-24pt}, from=1-1, to=1-3]
    \arrow[""{name=1, anchor=center, inner sep=0}, curve={height=24pt}, from=1-1, to=1-3]
    \arrow[""{name=2, anchor=center, inner sep=0}, from=1-1, to=1-3]
    \arrow[""{name=3, anchor=center, inner sep=0}, curve={height=-24pt}, from=1-3, to=1-5]
    \arrow[""{name=4, anchor=center, inner sep=0}, curve={height=24pt}, from=1-3, to=1-5]
    \arrow[""{name=5, anchor=center, inner sep=0}, from=1-3, to=1-5]
    \arrow["\theta", shorten <=3pt, shorten >=3pt, Rightarrow, from=0, to=2]
    \arrow["\psi", shorten <=3pt, shorten >=3pt, Rightarrow, from=2, to=1]
    \arrow["{\theta'}", shorten <=3pt, shorten >=3pt, Rightarrow, from=3, to=5]
    \arrow["{\psi'}", shorten <=3pt, shorten >=3pt, Rightarrow, from=5, to=4]
  \end{tikzcd}\]
  下面互换律成立
  \[
    \left( \psi' \mathop{\circ}\limits_{\text{纵}} \theta' \right)
    \mathop{\circ}\limits_{\text{横}} \left( \psi
    \mathop{\circ}\limits_{\text{纵}} \theta \right) = \left( \psi'
  \mathop{\circ}\limits_{\text{横}} \psi \right)
  \mathop{\circ}\limits_{\text{纵}} \left( \theta'
  \mathop{\circ}\limits_{\text{横}} \theta \right)
  \]
\end{proposition}
\begin{proof}
  对于第一个论断, 仔细考虑之, 发现是只能讨论横合成间的合成律或纵合成间的合成律,
  因为如果 \( \phi \circ \psi \) 是横合成, \( \psi \circ \theta \) 是纵合成,
  % https://q.uiver.app/#q=WzAsMyxbMCwwLCJcXG1hdGhjYWx7Q30nJyJdLFsyLDAsIlxcbWF0aGNhbHtDfSciXSxbNCwwLCJcXG1hdGhjYWx7Q30iXSxbMCwxLCJIIiwyLHsiY3VydmUiOjN9XSxbMCwxLCJGIiwwLHsiY3VydmUiOi0zfV0sWzEsMiwiSyIsMix7ImN1cnZlIjozfV0sWzEsMiwiSiIsMCx7ImN1cnZlIjotM31dLFswLDEsIkciLDFdLFs2LDUsIlxccGhpIiwwLHsic2hvcnRlbiI6eyJzb3VyY2UiOjMwLCJ0YXJnZXQiOjMwfX1dLFs0LDcsIlxcdGhldGEiLDAseyJzaG9ydGVuIjp7InNvdXJjZSI6MzAsInRhcmdldCI6MzB9fV0sWzcsMywiXFxwc2kiLDAseyJzaG9ydGVuIjp7InNvdXJjZSI6MzAsInRhcmdldCI6MzB9fV1d
  \[\begin{tikzcd}
    {\mathcal{C}''} && {\mathcal{C}'} && {\mathcal{C}}
    \arrow[""{name=0, anchor=center, inner sep=0}, "H"', curve={height=18pt}, from=1-1, to=1-3]
    \arrow[""{name=1, anchor=center, inner sep=0}, "F", curve={height=-18pt}, from=1-1, to=1-3]
    \arrow[""{name=2, anchor=center, inner sep=0}, "G"{description}, from=1-1, to=1-3]
    \arrow[""{name=3, anchor=center, inner sep=0}, "K"', curve={height=18pt}, from=1-3, to=1-5]
    \arrow[""{name=4, anchor=center, inner sep=0}, "J", curve={height=-18pt}, from=1-3, to=1-5]
    \arrow["\theta", shorten <=4pt, shorten >=4pt, Rightarrow, from=1, to=2]
    \arrow["\psi", shorten <=4pt, shorten >=4pt, Rightarrow, from=2, to=0]
    \arrow["\phi", shorten <=7pt, shorten >=7pt, Rightarrow, from=4, to=3]
  \end{tikzcd}\]
  那么, 由上图所示, \( \left( \phi \mathop{\circ}\limits_{\text{横}} (\psi
  \mathop{\circ}\limits_{\text{纵}} \theta) \right) \) 有意义, 而 \( \left(
(\phi \mathop{\circ}\limits_{\text{横}} \psi) \mathop{\circ}\limits_{\text{纵}}
\theta \right) \) 没有定义
  \begin{itemize}
    \item 如果 \( \phi \circ \psi \) 和 \( \psi \circ \theta \) 都是纵合成
    % https://q.uiver.app/#q=WzAsMTAsWzAsMSwiXFxtYXRoY2Fse0N9JyJdLFsyLDEsIlxcbWF0aGNhbHtDfSJdLFszLDAsIkZYIl0sWzMsMiwiRlkiXSxbNSwwLCJHWCJdLFs1LDIsIkdZIl0sWzcsMCwiSFgiXSxbNywyLCJIWSJdLFs5LDAsIkpYIl0sWzksMiwiSlkiXSxbMCwxLCJGIiwwLHsiY3VydmUiOi01fV0sWzAsMSwiSiIsMix7ImN1cnZlIjo1fV0sWzAsMSwiRyIsMSx7ImN1cnZlIjotMn1dLFswLDEsIkgiLDEseyJjdXJ2ZSI6Mn1dLFsyLDMsIkZmIiwyXSxbMiw0LCJcXHBoaV9YIl0sWzMsNSwiXFxwaGlfWSJdLFs0LDUsIkdmIiwyXSxbNCw2LCJcXHBzaV9YIl0sWzUsNywiXFxwc2lfWSJdLFs2LDcsIkhmIiwyXSxbNiw4LCJcXHRoZXRhX1giXSxbNyw5LCJcXHRoZXRhX1kiXSxbOCw5LCJKZiIsMl0sWzEwLDEyLCJcXHBoaSIsMCx7InNob3J0ZW4iOnsic291cmNlIjozMCwidGFyZ2V0IjozMH19XSxbMTIsMTMsIlxccHNpIiwwLHsic2hvcnRlbiI6eyJzb3VyY2UiOjMwLCJ0YXJnZXQiOjMwfX1dLFsxMywxMSwiXFx0aGV0YSIsMCx7InNob3J0ZW4iOnsic291cmNlIjozMCwidGFyZ2V0IjozMH19XV0=
      \[\begin{tikzcd}
        &&& FX && GX && HX && JX \\
        {\mathcal{C}'} && {\mathcal{C}} \\
        &&& FY && GY && HY && JY
        \arrow["{\phi_X}", from=1-4, to=1-6]
        \arrow["Ff"', from=1-4, to=3-4]
        \arrow["{\psi_X}", from=1-6, to=1-8]
        \arrow["Gf"', from=1-6, to=3-6]
        \arrow["{\theta_X}", from=1-8, to=1-10]
        \arrow["Hf"', from=1-8, to=3-8]
        \arrow["Jf"', from=1-10, to=3-10]
        \arrow[""{name=0, anchor=center, inner sep=0}, "F", curve={height=-30pt}, from=2-1, to=2-3]
        \arrow[""{name=1, anchor=center, inner sep=0}, "J"', curve={height=30pt}, from=2-1, to=2-3]
        \arrow[""{name=2, anchor=center, inner sep=0}, "G"{description}, curve={height=-12pt}, from=2-1, to=2-3]
        \arrow[""{name=3, anchor=center, inner sep=0}, "H"{description}, curve={height=12pt}, from=2-1, to=2-3]
        \arrow["{\phi_Y}", from=3-4, to=3-6]
        \arrow["{\psi_Y}", from=3-6, to=3-8]
        \arrow["{\theta_Y}", from=3-8, to=3-10]
        \arrow["\phi", shorten <=4pt, shorten >=4pt, Rightarrow, from=0, to=2]
        \arrow["\psi", shorten <=5pt, shorten >=5pt, Rightarrow, from=2, to=3]
        \arrow["\theta", shorten <=4pt, shorten >=4pt, Rightarrow, from=3, to=1]
      \end{tikzcd}\]
      \( \phi_X, \psi_X, \theta_X \) 和 \( \phi_Y, \psi_Y, \theta_Y \) 都是 \(
      \mathcal{C} \) 中的态射, 自然满足结合律.
    \item 如果 \( \phi \circ \psi \) 和 \( \psi \circ \theta \) 都是横合成
    % https://q.uiver.app/#q=WzAsNCxbMCwwLCJcXG1hdGhjYWx7Q31fNCJdLFsyLDAsIlxcbWF0aGNhbHtDfV8zIl0sWzQsMCwiXFxtYXRoY2Fse0N9XzIiXSxbNiwwLCJcXG1hdGhjYWx7Q31fMSJdLFswLDEsIkZfMiIsMix7ImN1cnZlIjozfV0sWzEsMiwiR18yIiwyLHsiY3VydmUiOjN9XSxbMSwyLCJHXzEiLDAseyJjdXJ2ZSI6LTN9XSxbMCwxLCJGXzEiLDAseyJjdXJ2ZSI6LTN9XSxbMiwzLCJIXzIiLDIseyJjdXJ2ZSI6M31dLFsyLDMsIkhfMSIsMCx7ImN1cnZlIjotM31dLFs3LDQsIlxcdGhldGEiLDAseyJzaG9ydGVuIjp7InNvdXJjZSI6MjAsInRhcmdldCI6MjB9fV0sWzksOCwiXFxwaGkiLDAseyJzaG9ydGVuIjp7InNvdXJjZSI6MjAsInRhcmdldCI6MjB9fV0sWzYsNSwiXFxwc2kiLDAseyJzaG9ydGVuIjp7InNvdXJjZSI6MjAsInRhcmdldCI6MjB9fV1d
    \[\begin{tikzcd}
      {\mathcal{C}_4} && {\mathcal{C}_3} && {\mathcal{C}_2} && {\mathcal{C}_1}
      \arrow[""{name=0, anchor=center, inner sep=0}, "{F_2}"', curve={height=18pt}, from=1-1, to=1-3]
      \arrow[""{name=1, anchor=center, inner sep=0}, "{F_1}", curve={height=-18pt}, from=1-1, to=1-3]
      \arrow[""{name=2, anchor=center, inner sep=0}, "{G_2}"', curve={height=18pt}, from=1-3, to=1-5]
      \arrow[""{name=3, anchor=center, inner sep=0}, "{G_1}", curve={height=-18pt}, from=1-3, to=1-5]
      \arrow[""{name=4, anchor=center, inner sep=0}, "{H_2}"', curve={height=18pt}, from=1-5, to=1-7]
      \arrow[""{name=5, anchor=center, inner sep=0}, "{H_1}", curve={height=-18pt}, from=1-5, to=1-7]
      \arrow["\theta", shorten <=5pt, shorten >=5pt, Rightarrow, from=1, to=0]
      \arrow["\psi", shorten <=5pt, shorten >=5pt, Rightarrow, from=3, to=2]
      \arrow["\phi", shorten <=5pt, shorten >=5pt, Rightarrow, from=5, to=4]
    \end{tikzcd}\]
    由 \eqref{equation-natural-transformation-horizontal-composite}, 知道
    \begin{align*}
      \left(\phi \mathop{\circ}\limits_{\text{横}} (\psi
      \mathop{\circ}\limits_{\text{横}} \theta)\right)_X &= \phi_{G_2F_2X} \circ
      H_1(\psi \mathop{\circ}\limits_{\text{横}} \theta)\\ &= \phi_{G_2 F_2 X}
      \circ H_1(\psi_{F_2X} \circ G_1(\theta_X)) \\ &= \phi_{G_2 F_2 X}
      \circ H_1(\psi_{F_2X}) \circ H_1 G_1(\theta_X)
    \end{align*}
    以及
    \begin{align*}
      \left((\phi \mathop{\circ}\limits_{\text{横}} \psi)
      \mathop{\circ}\limits_{\text{横}} \theta\right)_X &= (\phi
      \mathop{\circ}\limits_{\text{横}} \psi)_{F_2 X} H_1 G_1(\theta_X) \\ &=
      \phi_{G_2 F_2 X} \circ H_1(\psi_{F_2 X}) \circ H_1 G_1(\theta_X).
    \end{align*}
  \end{itemize}
\end{proof}


