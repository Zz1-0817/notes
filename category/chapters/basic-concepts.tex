\chapter{基本概念}

\section{范畴与态射}

\subsection{范畴}

\begin{definition}
  一个范畴 \( \mathcal{C} \) 系指以下资料:
  \begin{enumerate}
    \item 集合 \( \operatorname{Ob}(\mathcal{C}) \), 其元素称作 \( \mathcal{C}
      \) 的 \emph{对象}.
    \item 集合 \( \operatorname{Mor}(\mathcal{C}) \), 其元素称作 \( \mathcal{C}
      \) 的 \emph{态射}, 配上一对映射 \( \operatorname{Mor}(\mathcal{C})
      \mathop{\rightrightarrows}\limits_{t}^{s} \operatorname{Ob}(\mathcal{C})
      \), 其中 \( s \) 和 \( t \) 分别给出态射的 \emph{来源} 和 \emph{目标}. 对
      \( X, Y \in \operatorname{Ob}(\mathcal{C}) \), 一般习惯记 \(
      \operatorname{Hom}_{\mathcal{C}}(X, Y) := s^{-1}(X) \cap t^{-1}(Y) \),
      或简单记作 \( \operatorname{Hom}(X, Y) \), 称为 \( \operatorname{Hom}
      \)-集, 其元素称为从 \( X \) 到 \( Y \) 的态射. 一般也将 \( f \in
      \operatorname{Hom}_{\mathcal{C}}(X, Y) \) 写作 \( f: X \to Y \) 或 \( X
      \xrightarrow{f} Y \), 故态射有时也叫作 \emph{箭头}.
    \item 对每个对象 \( X \), 给定元素 \( \operatorname{id}_X \in
      \operatorname{Hom}_{\mathcal{C}}(X, X) \), 称为 \( X \) 到自身的
      \emph{恒等映射}.
    \item 对任意 \( X, Y, Z \in \operatorname{Ob}(\mathcal{C}) \), 给定态射间的
      \emph{合成态射}
      \begin{align*}
        \circ: \operatorname{Hom}_{\mathcal{C}}(Y, Z) \times
        \operatorname{Hom}_{\mathbb{C}}(X, Y) &\to
        \operatorname{Hom}_{\mathcal{C}}(X, Z)\\
        (f, g) &\mapsto f \circ g,
      \end{align*}
      不会混淆时, 将 \( f \circ g \) 简记作 \( fg \). 它满足
      \begin{itemize}
        \item 结合律: 对于任意态射 \( h, g, f \in
          \operatorname{Mor}(\mathcal{C}) \), 若合成 \( f(gh) \) 和 \( (fg)h \)
          都有定义, 则
          \[
            f(gh) = (fg)h.
          \]
          故两边都可以写成 \( f \circ g \circ h \) 或 \( fgh \);
        \item 对于任意态射 \( f \in \operatorname{Hom}_{\mathcal{C}}(X, Y) \),
          有
          \[
            f \circ \operatorname{id}_X = f = \operatorname{id}_Y \circ f.
          \]
      \end{itemize}
  \end{enumerate}
\end{definition}

\begin{remark}
  \( \operatorname{id}_X \) 被其性质唯一确定.
\end{remark}

\begin{definition}
  给定任意 \( \mathcal{C} \), 其 \emph{反范畴} \(
  \mathcal{C}^{\operatorname{op}} \) 定义为
  \begin{itemize}
    \item \( \operatorname{Ob}(\mathcal{C}^{\operatorname{ob}}) =
      \operatorname{Ob})(\mathcal{C}) \).
    \item 对任意对象 \( X, Y \), \(
      \operatorname{Hom}_{\mathcal{C}^{\operatorname{op}}}(X, Y) :=
      \operatorname{Hom}_{\mathcal{C}}(Y, X) \).
    \item 态射 \( f \in \operatorname{Hom}_{\mathcal{C}^{\operatorname{op}}}(Y,
      Z), g \in \operatorname{Hom}_{\mathcal{C}^{\operatorname{op}}}(X, Y) \) 在
      \( \mathcal{C}^{\operatorname{op}} \) 中的合成 \( f
      \circ^{\operatorname{op}} g \) 定义为 \( \mathcal{C} \) 中的 \( g \circ f
      \).
    \item 恒等映射定义同 \( \mathcal{C} \).
  \end{itemize}
\end{definition}

\begin{definition}
  对于态射 \( f: X \to Y \), 若存在 \( g: Y \to X \), 使得 \( fg =
  \operatorname{id}_Y, gf = \operatorname{id}_X \), 则称 \( f \) 是 \emph{同构},
  而 \( g \) 称为 \( f \) 的\emph{逆}.
\end{definition}

\begin{example}
  对象与态射集皆空的范畴称为 \emph{空范畴}, 记作 \( \mathbf{0} \).
\end{example}

\begin{definition}
  一个范畴 \( \mathcal{C} \) 称为是 \emph{小范畴} 如果 \(
  \operatorname{Ob}(\mathcal{C}) \) 是一个集合而非类.
\end{definition}

\begin{definition}
  称 \( \mathcal{C}' \) 为 \( \mathcal{C} \) 的 \emph{子范畴}, 如果
  \begin{enumerate}
    \item \( \operatorname{Ob}(\mathcal{C}') \subseteq
      \operatorname{Ob}(\mathcal{C}) \).
    \item \( \operatorname{Mor}(\mathcal{C}') \subseteq
      \operatorname{Mor}(\mathcal{C}) \), 并保持恒等映射.
    \item 来源映射以及目标映射是 \( \operatorname{Mor}(\mathcal{C'})
      \mathop{\rightrightarrows}\limits_{t}^{s} \operatorname{Ob}(\mathcal{C'})
      \) 是 \( \mathcal{C} \) 的限制.
    \item \( \mathcal{C}' \) 中态射的合成亦是 \( \mathcal{C} \) 的限制.
  \end{enumerate}
  如果 \( \operatorname{Hom}_{\mathcal{C}'}(X, Y) =
  \operatorname{Hom}_{\mathcal{C}}(X, Y) \), 那么称 \( \mathcal{C}' \) 是 \(
  \mathcal{C} \) 的全子范畴.
\end{definition}

\subsection{态射}

\begin{definition}
  假设 \( X, Y \) 为范畴 \( \mathcal{C} \) 中的对象, \( f: X \to Y \) 为态射.
  \begin{enumerate}
    \item 称 \( f \) 为\emph{单态射}, 如果对任何对象 \( Z \) 和任一对态射 \( g, h: Z
      \to X \) 有 \( fg = fh \) 当且仅当 \( g = h \).
    \item 称 \( f \) 为\emph{满态射}, 如果对任何对象 \( Z \) 和任一对态射 \( g,
      h: Y \to Z \) 有 \( gf = hf \) 当且仅当 \( g = h \).
    \item 如果存在 \( g: Y \to X \) 使得 \( gf = \operatorname{id}_X \), 那么称
      \( f \) \emph{左可逆}, \( g \) 为 \( f \) 的 \emph{左逆}.
    \item 如果存在 \( g: Y \to X \) 使得 \( fg = \operatorname{id}_Y \), 那么称
      \( f \) \emph{右可逆}, \( g \) 为 \( f \) 的 \emph{右逆}.
  \end{enumerate}
\end{definition}

\begin{proposition}
  左可逆蕴含单; 右可逆蕴含满; 态射可逆当且仅当其左右均可逆.
\end{proposition}

\section{函子与自然变换}

\subsection{函子}

\begin{definition}
  假设 \( \mathcal{C}', \mathcal{C} \) 为范畴, 一个\emph{函子} \( F:
  \mathcal{C}' \to \mathcal{C} \) 指的是下面资料
  \begin{enumerate}
    \item 对象间的映射 \( F: \operatorname{Ob}(\mathcal{C}') \to
      \operatorname{Ob}(\mathcal{C}) \).
    \item 态射间的映射 \( F: \operatorname{Mor}(\mathcal{C}') \to
      \operatorname{Mor}(\mathcal{C}) \), 使得
      \begin{itemize}
        \item 对每个 \( X, Y \in \operatorname{Ob}(\mathcal{C}') \) 皆有映射 \(
          F: \operatorname{Hom}_{\mathcal{C}'}(X, Y) \to
          \operatorname{Hom}_{\mathcal{C}}(FX, FY) \).
        \item \( F(g \circ f) = F(g) \circ F(f), F(\operatorname{id}_X) =
          \operatorname{id}_{FX} \).
      \end{itemize}
  \end{enumerate}
  对于 \( F: \mathcal{C}_1 \to \mathcal{C}_2, G: \mathcal{C}_2 \to C_3 \),
  合成函子的定义即分别取合成映射
  \begin{align*}
    &\operatorname{Ob}(\mathcal{C}_1) \xrightarrow{F}
    \operatorname{Ob}(\mathcal{C}_2) \xrightarrow{G}
    \operatorname{Ob}(\mathcal{C}_3)\\
    &\operatorname{Mor}(\mathcal{C}_1) \xrightarrow{F}
    \operatorname{Mor}(\mathcal{C}_2) \xrightarrow{G}
    \operatorname{Mor}(\mathcal{C}_3)
  \end{align*}
\end{definition}

\begin{definition}
  对于函子 \( F: \mathcal{C}' \to \mathcal{C} \).
  \begin{enumerate}
    \item 称 \( F \) 是 \emph{本质满} 的, 如果 \( \mathcal{C} \) 中任意对象都同构于某个
      \( FX \).
    \item 称 \( F \) 是 \emph{忠实} 的, 如果对所有的 \( X, Y \in
      \operatorname{Ob} (\mathcal{C}') \), 映射 \(
      \operatorname{Hom}_{\mathcal{C}'}(X, Y) \to
      \operatorname{Hom}_{\mathcal{C}}(FX, FY) \) 都是单射.
    \item 称 \( F \) 是 \emph{全} 的, 如果对所有的 \( X, Y \in
      \operatorname{Ob} (\mathcal{C}') \), 映射 \(
      \operatorname{Hom}_{\mathcal{C}'}(X, Y) \to
      \operatorname{Hom}_{\mathcal{C}}(FX, FY) \) 都是满射.
  \end{enumerate}
\end{definition}

\subsection{自然变换}

\begin{definition}
  函子 \( F, G:\mathcal{C}' \to \mathcal{C} \) 之间的自然变换 \( \theta \)
  是一族态射
  \[
    \theta_X \in \operatorname{Hom}_{\mathcal{C}}(FX, GX),\quad X \in
    \operatorname{Ob}(\mathcal{C}')
  \]
  使得下图对所有 \( \mathcal{C}' \) 中的态射 \( f: X \to Y \) 交换
  % https://q.uiver.app/#q=WzAsNCxbMCwwLCJGWCJdLFsxLDAsIkdYIl0sWzAsMSwiRlkiXSxbMSwxLCJHWSJdLFswLDIsIkZmIiwyXSxbMCwxLCJcXHRoZXRhX1giXSxbMSwzLCJHZiJdLFsyLDMsIlxcdGhldGFfWSIsMl1d
  \[\begin{tikzcd}
    FX & GX \\
    FY & GY
    \arrow["{\theta_X}", from=1-1, to=1-2]
    \arrow["Ff"', from=1-1, to=2-1]
    \arrow["Gf", from=1-2, to=2-2]
    \arrow["{\theta_Y}"', from=2-1, to=2-2]
  \end{tikzcd}\]
  我们也将自然变换 \( \theta: F \to G \) 称为从函子 \( F \) 到 \( G \) 的态射.
\end{definition}

\begin{definition}
自然变换的合成包括横合成和纵合成两种.
\begin{itemize}
  \item 考虑 \( \mathcal{C}' \) 到 \( \mathcal{C} \) 的三个函子间的态射 \(
    \theta: F \to G, \psi: G \to H \). \emph{纵合成} \( \psi \circ \theta \) 的定义为
    \( \left\lbrace \psi_X \circ \theta_X: X \in \operatorname{Ob}(\mathcal{C}')
    \right\rbrace \)
    % https://q.uiver.app/#q=WzAsNSxbMCwwLCJcXG1hdGhjYWx7Q30nIl0sWzIsMCwiXFxtYXRoY2Fse0N9Il0sWzMsMCwiXFxsZWFkc3RvIl0sWzQsMCwiXFxtYXRoY2Fse0N9JyJdLFs2LDAsIlxcbWF0aGNhbHtDfSJdLFswLDEsIkciLDFdLFswLDEsIkYiLDAseyJvZmZzZXQiOi0zLCJjdXJ2ZSI6LTN9XSxbMCwxLCJIIiwyLHsib2Zmc2V0IjoyLCJjdXJ2ZSI6M31dLFszLDQsIkYiLDAseyJjdXJ2ZSI6LTN9XSxbMyw0LCJIIiwyLHsiY3VydmUiOjN9XSxbNiw1LCJcXHRoZXRhIiwwLHsic2hvcnRlbiI6eyJzb3VyY2UiOjMwLCJ0YXJnZXQiOjMwfX1dLFs1LDcsIkgiLDAseyJzaG9ydGVuIjp7InNvdXJjZSI6MzAsInRhcmdldCI6MzB9fV0sWzgsOSwiXFxwc2kgXFxjaXJjIFxcdGhldGEiLDEseyJzaG9ydGVuIjp7InNvdXJjZSI6MjAsInRhcmdldCI6MjB9fV1d
    \[\begin{tikzcd}
      {\mathcal{C}'} && {\mathcal{C}} & \leadsto & {\mathcal{C}'} && {\mathcal{C}}
      \arrow[""{name=0, anchor=center, inner sep=0}, "G"{description}, from=1-1, to=1-3]
      \arrow[""{name=1, anchor=center, inner sep=0}, "F", shift left=3, curve={height=-18pt}, from=1-1, to=1-3]
      \arrow[""{name=2, anchor=center, inner sep=0}, "H"', shift right=2, curve={height=18pt}, from=1-1, to=1-3]
      \arrow[""{name=3, anchor=center, inner sep=0}, "F", curve={height=-18pt}, from=1-5, to=1-7]
      \arrow[""{name=4, anchor=center, inner sep=0}, "H"', curve={height=18pt}, from=1-5, to=1-7]
      \arrow["\theta", shorten <=5pt, shorten >=5pt, Rightarrow, from=1, to=0]
      \arrow["H", shorten <=4pt, shorten >=4pt, Rightarrow, from=0, to=2]
      \arrow["{\psi \circ \theta}"{description}, shorten <=5pt, shorten >=5pt, Rightarrow, from=3, to=4]
    \end{tikzcd}\]
  \item 考虑函子 \( 
    % https://q.uiver.app/#q=WzAsMyxbMCwwLCJcXG1hdGhjYWx7Q30nJyJdLFsxLDAsIlxcbWF0aGNhbHtDfSciXSxbMiwwLCJcXG1hdGhjYWx7Q30iXSxbMCwxLCJGXzEiLDAseyJvZmZzZXQiOi0xfV0sWzAsMSwiRl8yIiwyLHsib2Zmc2V0IjoxfV0sWzEsMiwiR18yIiwyLHsib2Zmc2V0IjoxfV0sWzEsMiwiR18xIiwwLHsib2Zmc2V0IjotMX1dXQ==
    \begin{tikzcd}
      {\mathcal{C}''} & {\mathcal{C}'} & {\mathcal{C}}
      \arrow["{F_1}", shift left, from=1-1, to=1-2]
      \arrow["{F_2}"', shift right, from=1-1, to=1-2]
      \arrow["{G_2}"', shift right, from=1-2, to=1-3]
      \arrow["{G_1}", shift left, from=1-2, to=1-3]
    \end{tikzcd}
    \) 及态射 \( \theta: F_1 \to F_2, \psi: G_1 \to G_2 \). 将\emph{横合成}
    定义为 \(
    \psi \circ \theta: G_1 \circ F_1 \to G_2 \circ F_2 \). 具体地, \( \theta:
    F_1 \to F_2 \) 以及 \( G_1 \) 给出了下面交换图表
% https://q.uiver.app/#q=WzAsOSxbMCwwLCJGXzFYIl0sWzAsMiwiRl8xWSJdLFsyLDAsIkZfMlgiXSxbMiwyLCJGXzJZIl0sWzQsMCwiR18xIEZfMSBYIl0sWzYsMCwiR18xRl8yWCJdLFs0LDIsIkdfMUZfMVkiXSxbNiwyLCJHXzFGXzJZIl0sWzMsMSwiXFxsZWFkc3RvIl0sWzAsMSwiRl8xIGYiLDJdLFswLDIsIlxcdGhldGFfWCJdLFsyLDMsIkZfMiBmIl0sWzEsMywiXFx0aGV0YV9ZIiwyXSxbNCw2LCJHXzFGXzFmIiwyXSxbNCw1LCJHXzFcXHRoZXRhX1giXSxbNSw3LCJHXzFGXzJmIl0sWzYsNywiR18xXFx0aGV0YV9ZIiwyXV0=
\[\begin{tikzcd}
	{F_1X} && {F_2X} && {G_1 F_1 X} && {G_1F_2X} \\
	&&& \leadsto \\
	{F_1Y} && {F_2Y} && {G_1F_1Y} && {G_1F_2Y}
	\arrow["{\theta_X}", from=1-1, to=1-3]
	\arrow["{F_1 f}"', from=1-1, to=3-1]
	\arrow["{F_2 f}", from=1-3, to=3-3]
	\arrow["{G_1\theta_X}", from=1-5, to=1-7]
	\arrow["{G_1F_1f}"', from=1-5, to=3-5]
	\arrow["{G_1F_2f}", from=1-7, to=3-7]
	\arrow["{\theta_Y}"', from=3-1, to=3-3]
	\arrow["{G_1\theta_Y}"', from=3-5, to=3-7]
\end{tikzcd}\]
上图右边交换图在自然变换 \( \psi \) 下有交换图
% https://q.uiver.app/#q=WzAsOCxbMCwyLCJHXzEgRl8xIFgiXSxbMiwyLCJHXzEgRl8yIFgiXSxbNCwyLCJHXzEgRl8yIFkiXSxbNiwyLCJHXzIgRl8yIFkiXSxbMCw0LCJHXzEgRl8xIFkiXSxbMiw0LCJHXzEgRl8yIFkiXSxbMiwwLCJHXzJGXzIgWCJdLFswLDAsIkdfMiBGXzEgWCJdLFswLDQsIkdfMSBGXzEgZiIsMl0sWzEsNSwiR18xRl8yIGYiXSxbNCw1LCJHXzEgXFx0aGV0YV9ZIiwyXSxbMCwxLCJHXzFcXHRoZXRhX1giLDFdLFswLDcsIlxccHNpX3tGXzEgWH0iXSxbMSw2LCJcXHBzaV97Rl8yIFh9IiwyXSxbNyw2LCJHXzIgXFx0aGV0YV9YIl0sWzEsMiwiR18xIEZfMiBmIiwxXSxbNiwzLCJHXzIgRl8yIGYiXSxbMiwzLCJcXHBzaV97Rl8yIFl9IiwxXSxbNSwzLCJcXHBzaV97Rl8yWX0iLDJdXQ==
\[\begin{tikzcd}
	{G_2 F_1 X} && {G_2F_2 X} \\
	\\
	{G_1 F_1 X} && {G_1 F_2 X} && {G_1 F_2 Y} && {G_2 F_2 Y} \\
	\\
	{G_1 F_1 Y} && {G_1 F_2 Y}
	\arrow["{G_2 \theta_X}", from=1-1, to=1-3]
	\arrow["{G_2 F_2 f}", from=1-3, to=3-7]
	\arrow["{\psi_{F_1 X}}", from=3-1, to=1-1]
	\arrow["{G_1\theta_X}"{description}, from=3-1, to=3-3]
	\arrow["{G_1 F_1 f}"', from=3-1, to=5-1]
	\arrow["{\psi_{F_2 X}}"', from=3-3, to=1-3]
	\arrow["{G_1 F_2 f}"{description}, from=3-3, to=3-5]
	\arrow["{G_1F_2 f}", from=3-3, to=5-3]
	\arrow["{\psi_{F_2 Y}}"{description}, from=3-5, to=3-7]
	\arrow["{G_1 \theta_Y}"', from=5-1, to=5-3]
	\arrow["{\psi_{F_2Y}}"', from=5-3, to=3-7]
\end{tikzcd}\]
因此横合成为 \( (\psi \circ \theta)_{X} = G_2(\theta_X) \circ \psi_{F_1 X} =
\psi_{F_2 X} \circ G_1(\theta_X)  \) 以及 \( (\psi \circ \theta)_Y =
\psi_{F_2 Y} \circ G_1(\theta_Y) \). 我们可以简要地写作
% https://q.uiver.app/#q=WzAsNixbMCwwLCJcXG1hdGhjYWx7Q30nJyJdLFsxLDAsIlxcbWF0aGNhbHtDfSciXSxbMiwwLCJcXG1hdGhjYWx7Q30iXSxbMywwLCJcXGxlYWRzdG8iXSxbNCwwLCJcXG1hdGhjYWx7Q30nJyJdLFs2LDAsIlxcbWF0aGNhbHtDfSJdLFswLDEsIkZfMiIsMix7ImN1cnZlIjoyfV0sWzAsMSwiRl8xIiwwLHsiY3VydmUiOi0yfV0sWzEsMiwiR18yIiwyLHsiY3VydmUiOjJ9XSxbMSwyLCJHXzEiLDAseyJjdXJ2ZSI6LTJ9XSxbNCw1LCJHXzIgRl8yIiwyLHsiY3VydmUiOjN9XSxbNCw1LCJHXzFGXzEiLDAseyJjdXJ2ZSI6LTN9XSxbNyw2LCJcXHRoZXRhIiwxLHsic2hvcnRlbiI6eyJzb3VyY2UiOjIwLCJ0YXJnZXQiOjIwfX1dLFs5LDgsIlxccHNpIiwxLHsic2hvcnRlbiI6eyJzb3VyY2UiOjIwLCJ0YXJnZXQiOjIwfX1dLFsxMSwxMCwiXFxwc2kgXFxjaXJjIFxcdGhldGEiLDEseyJzaG9ydGVuIjp7InNvdXJjZSI6MjAsInRhcmdldCI6MjB9fV1d
\[\begin{tikzcd}
	{\mathcal{C}''} & {\mathcal{C}'} & {\mathcal{C}} & \leadsto & {\mathcal{C}''} && {\mathcal{C}}
	\arrow[""{name=0, anchor=center, inner sep=0}, "{F_2}"', curve={height=12pt}, from=1-1, to=1-2]
	\arrow[""{name=1, anchor=center, inner sep=0}, "{F_1}", curve={height=-12pt}, from=1-1, to=1-2]
	\arrow[""{name=2, anchor=center, inner sep=0}, "{G_2}"', curve={height=12pt}, from=1-2, to=1-3]
	\arrow[""{name=3, anchor=center, inner sep=0}, "{G_1}", curve={height=-12pt}, from=1-2, to=1-3]
	\arrow[""{name=4, anchor=center, inner sep=0}, "{G_2 F_2}"', curve={height=18pt}, from=1-5, to=1-7]
	\arrow[""{name=5, anchor=center, inner sep=0}, "{G_1F_1}", curve={height=-18pt}, from=1-5, to=1-7]
	\arrow["\theta"{description}, shorten <=3pt, shorten >=3pt, Rightarrow, from=1, to=0]
	\arrow["\psi"{description}, shorten <=3pt, shorten >=3pt, Rightarrow, from=3, to=2]
	\arrow["{\psi \circ \theta}"{description}, shorten <=5pt, shorten >=5pt, Rightarrow, from=5, to=4]
\end{tikzcd}\]
\end{itemize}
\end{definition}

\begin{proposition}
  纵横合成都是函子间的态射, 满足结合律 \( (\phi \circ \psi) \circ \theta = \phi
  \circ (\psi \circ \theta) \). 并且纵横合成有下述关系: 对于图
  % https://q.uiver.app/#q=WzAsMyxbMCwwLCJcXG1hdGhjYWx7Q31fMSJdLFsyLDAsIlxcbWF0aGNhbHtDfV8yIl0sWzQsMCwiXFxtYXRoY2Fse0N9XzMiXSxbMCwxLCIiLDAseyJjdXJ2ZSI6LTR9XSxbMSwyLCIiLDAseyJjdXJ2ZSI6LTR9XSxbMSwyLCIiLDAseyJjdXJ2ZSI6NH1dLFswLDEsIiIsMCx7ImN1cnZlIjo0fV0sWzAsMV0sWzEsMl0sWzMsNywiXFx0aGV0YSIsMCx7InNob3J0ZW4iOnsic291cmNlIjoyMCwidGFyZ2V0IjoyMH19XSxbNyw2LCJcXHBzaSIsMCx7InNob3J0ZW4iOnsic291cmNlIjoyMCwidGFyZ2V0IjoyMH19XSxbNCw4LCJcXHRoZXRhJyIsMCx7InNob3J0ZW4iOnsic291cmNlIjoyMCwidGFyZ2V0IjoyMH19XSxbOCw1LCJcXHBzaSciLDAseyJzaG9ydGVuIjp7InNvdXJjZSI6MjAsInRhcmdldCI6MjB9fV1d
  \[\begin{tikzcd}
    {\mathcal{C}_1} && {\mathcal{C}_2} && {\mathcal{C}_3}
    \arrow[""{name=0, anchor=center, inner sep=0}, curve={height=-24pt}, from=1-1, to=1-3]
    \arrow[""{name=1, anchor=center, inner sep=0}, curve={height=24pt}, from=1-1, to=1-3]
    \arrow[""{name=2, anchor=center, inner sep=0}, from=1-1, to=1-3]
    \arrow[""{name=3, anchor=center, inner sep=0}, curve={height=-24pt}, from=1-3, to=1-5]
    \arrow[""{name=4, anchor=center, inner sep=0}, curve={height=24pt}, from=1-3, to=1-5]
    \arrow[""{name=5, anchor=center, inner sep=0}, from=1-3, to=1-5]
    \arrow["\theta", shorten <=3pt, shorten >=3pt, Rightarrow, from=0, to=2]
    \arrow["\psi", shorten <=3pt, shorten >=3pt, Rightarrow, from=2, to=1]
    \arrow["{\theta'}", shorten <=3pt, shorten >=3pt, Rightarrow, from=3, to=5]
    \arrow["{\psi'}", shorten <=3pt, shorten >=3pt, Rightarrow, from=5, to=4]
  \end{tikzcd}\]
  下面互换律成立
  \[
    \left( \psi' \mathop{\circ}\limits_{\text{纵}} \theta' \right)
    \mathop{\circ}\limits_{\text{横}} \left( \psi
    \mathop{\circ}\limits_{\text{纵}} \theta \right) = \left( \psi'
  \mathop{\circ}\limits_{\text{横}} \psi \right)
  \mathop{\circ}\limits_{\text{纵}} \left( \theta'
  \mathop{\circ}\limits_{\text{横}} \theta \right)
  \]
\end{proposition}

\section{泛性质}
\subsection{始对象, 终对象与零对象}
假设给定了一个范畴 \( \mathcal{C} \).
\begin{enumerate}
  \item 如果 \( X \in \operatorname{Ob}(\mathcal{C}) \) 使得 \(
    \operatorname{Hom}_{\mathcal{C}}(X, Y) \) 有且仅有一个元素对所有 \( Y \in
    \operatorname{Ob}(\mathcal{C}) \) 成立, 那么我们称 \( X \) 是 \( \mathcal{C}
    \) 中的 \emph{始对象}.
  \item 如果 \( X \in \operatorname{Ob}(\mathcal{C}) \) 使得 \(
    \operatorname{Hom}_{\mathcal{C}}(Y, X) \) 有且仅有一个元素对所有 \( Y \in
    \operatorname{Ob}(\mathcal{C}) \) 成立, 那么我们称 \( X \) 是 \( \mathcal{C}
    \) 中的 \emph{终对象}.
  \item 如果 \( X \in \operatorname{Ob}(\mathcal{C}) \) 既是始对象又是终对象,
    那么我们称其为 \emph{零对象}.
\end{enumerate}

\begin{proposition}
  如果 \( X, X' \) 为 \( \mathcal{C} \) 中的始对象, 那么存在唯一的同构 \( X
  \simeq X' \). 类似地, 如果 \( X, X' \) 为 \( \mathcal{C} \) 中的终对象,
  那么存在唯一的同构 \( X \simeq X' \).
\end{proposition}

假设 \( \mathcal{C} \) 有零对象, 记作 \( \mathbf{0} \). 对任意 \( X, Y \in
\operatorname{Ob}(\mathcal{C}) \), 我们定义零态射 \( 0: X \to Y \) 为复合
\[
  X \to 0 \to Y.
\]

\subsection{核与余核}

假设 \( \mathcal{C} \) 是一个带零对象 \( \mathbf{0} \) 的范畴,
约定下面讨论对象都是 \( \mathcal{C} \) 的对象, 态射都是 \( \mathcal{C} \)
的态射. 给定一个映射 \( f: B \to C \).

我们称 \( f \) 的 \emph{核} 为态射 \( i: A \to B \) 使得 \( fi = 0 \)
且满足泛性质: 每个使得 \( fe = 0 \) 的态射 \( e: A' \to B \) 都能分解为 \( e = ie'
\), 其中 \( e': A' \to A \) 唯一.

我们称 \( f: B \to C \) 的 \emph{余核} 为态射 \( p: C \to D \) 使得 \( pf = 0 \)
且满足泛性质: 每个使得 \( gf = 0 \) 的态射 \( g: C \to D' \) 都能分解为 \( g =
g'p \), 其中 \( g': D \to D' \) 唯一.

% https://q.uiver.app/#q=WzAsOCxbMCwxLCJBIl0sWzEsMSwiQiJdLFsyLDEsIkMiXSxbMSwwLCJBJyJdLFs0LDEsIkIiXSxbNSwxLCJDIl0sWzYsMSwiRCJdLFs1LDAsIkQnIl0sWzEsMiwiZiJdLFswLDEsImkiXSxbMywxLCJlIl0sWzMsMCwiXFxleGlzdHMgISBlJyIsMix7InN0eWxlIjp7ImJvZHkiOnsibmFtZSI6ImRhc2hlZCJ9fX1dLFs0LDUsImYiXSxbNSw2LCJwIl0sWzUsNywiZyJdLFs2LDcsIlxcZXhpc3RzICEgZyciLDIseyJzdHlsZSI6eyJib2R5Ijp7Im5hbWUiOiJkYXNoZWQifX19XV0=
\[\begin{tikzcd}
	& {A'} &&&& {D'} \\
	A & B & C && B & C & D
	\arrow["{\exists ! e'}"', dashed, from=1-2, to=2-1]
	\arrow["e", from=1-2, to=2-2]
	\arrow["i", from=2-1, to=2-2]
	\arrow["f", from=2-2, to=2-3]
	\arrow["f", from=2-5, to=2-6]
	\arrow["g", from=2-6, to=1-6]
	\arrow["p", from=2-6, to=2-7]
	\arrow["{\exists ! g'}"', dashed, from=2-7, to=1-6]
\end{tikzcd}\]

\begin{proposition}
  上面记号中, \( i \) 是单射, \( p \) 是满射.
\end{proposition}

\subsection{乘积与余积}

假设给定一个范畴 \( \mathcal{C} \) 以及 \( \mathcal{C} \) 的一个对象集 \(
\left\lbrace C_i: i \in I \right\rbrace \subseteq \operatorname{Ob}(\mathcal{C})
\).

我们称 \( \left\lbrace C_i: i \in I \right\rbrace \) 的\emph{乘积}(不一定存在)
指的是 \( \mathcal{C} \) 中的一个对象 \( \prod_{i \in I}C_i \) 加上一族态射 \( \pi_j: \prod C_i \to C_j, j
\in I \), 使得对每个 \( A \in \mathcal{C} \), 以及态射 \( \alpha_i: A \to C_i, i
\in I \), 都存在唯一的态射 \( \alpha: A \to \prod C_i \) 使得 \( \pi_i \alpha =
\alpha_i \).

对偶地, 我们称 \( \left\lbrace C_i: i \in I \right\rbrace \)
的\emph{余积}(不一定存在) 指的是 \( \mathcal{C} \) 中的一个对象 \( \coprod_{i
\in I}C_i \) 加上一族态射 \( \iota_j: C_j \to \coprod C_i, j \in I \),
使得对每个 \( A \in \mathcal{C} \), 以及态射 \( \alpha_i: C_i \to A, i \in I \),
都存在唯一的态射 \( \alpha: \coprod C_i \to A \) 使得 \( \alpha \iota_i =
\alpha_i \).

% https://q.uiver.app/#q=WzAsNixbMSwwLCJcXHByb2QgQ19pIl0sWzEsMSwiQ19qIl0sWzAsMSwiQSJdLFszLDEsIkEiXSxbNCwxLCJDX2oiXSxbNCwwLCJcXGNvcHJvZCBDX2kiXSxbMiwxLCJcXGFscGhhX2oiLDJdLFsyLDAsIlxcZXhpc3RzICEgXFxhbHBoYSIsMCx7InN0eWxlIjp7ImJvZHkiOnsibmFtZSI6ImRhc2hlZCJ9fX1dLFswLDEsIlxccGlfaiJdLFs0LDMsIlxcYWxwaGFfaiJdLFs0LDUsIlxcaW90YV9qIiwyXSxbNSwzLCJcXGV4aXN0cyAhIFxcYWxwaGEiLDIseyJzdHlsZSI6eyJib2R5Ijp7Im5hbWUiOiJkYXNoZWQifX19XV0=
\[\begin{tikzcd}
	& {\prod C_i} &&& {\coprod C_i} \\
	A & {C_j} && A & {C_j}
	\arrow["{\pi_j}", from=1-2, to=2-2]
	\arrow["{\exists ! \alpha}"', dashed, from=1-5, to=2-4]
	\arrow["{\exists ! \alpha}", dashed, from=2-1, to=1-2]
	\arrow["{\alpha_j}"', from=2-1, to=2-2]
	\arrow["{\iota_j}"', from=2-5, to=1-5]
	\arrow["{\alpha_j}", from=2-5, to=2-4]
\end{tikzcd}\]


\section{Ab-范畴}
给定一个范畴 \( \mathcal{A} \).
\begin{itemize}
  \item 如果每个 \( \operatorname{Hom}_{\mathcal{A}}(A, B) \)
    都给定了加法交换群结构并且符合关于此加法是分配的, 那么范畴 \( \mathcal{A} \)
    称为 \emph{\( \operatorname{Ab} \)-范畴}. 特别地, 给定 \( \mathcal{A} \)
    中的图
    \[
      A \xrightarrow{f} B \mathop{\rightrightarrows}\limits_{g}^{g'} C
      \xrightarrow{h} D,
    \]
    有 \( h(g + g')f = hgf + hg'f \).
  \item 给定一个 \( \operatorname{Ab} \)-范畴间的函子 \( F: \mathcal{B} \to
    \mathcal{A} \). 如果 \( \operatorname{Hom}_{\mathcal{B}}(B', B) \to
    \operatorname{Hom}_{\mathcal{A}}(FB', FB) \) 是一个群同态, 那么我们称 \( F
    \) 是一个\emph{加性函子}.
  \item 如果 \( \operatorname{Ab} \)-范畴 \( \mathcal{A} \)
    有一个零对象并且对每个 \( A, B \in \operatorname{\mathcal{A}} \) 存在 \( A
    \times B \), 那么我们称 \( \mathcal{A} \) 是 \emph{加性范畴}.
  \item 一个 \emph{abelian 范畴} 是一个加性范畴 \( \mathcal{A} \) 使得
    \begin{itemize}
      \item 每个 \( \mathcal{A} \) 中的态射有一个核和余核.
      \item 每个 \( \mathcal{A} \) 中的单射是它的核的余核.
      \item 每个 \( \mathcal{A} \) 中的满射是它的余核的核.
    \end{itemize}
\end{itemize}

给设给定了 abelian 范畴中的一个态射序列 \( A \xrightarrow{f} B \xrightarrow{g} C
\). 如果 \( \operatorname{ker} g = \operatorname{im} f \), 那么该序列称为是
\emph{正合} 的.
