\chapter{基本概念}

\section{范畴与态射}

\subsection{范畴}

\begin{definition}
  一个范畴 \( \mathcal{C} \) 系指以下资料:
  \begin{enumerate}
    \item 集合 \( \operatorname{Ob}(\mathcal{C}) \), 其元素称作 \( \mathcal{C}
      \) 的 \emph{对象}.
    \item 集合 \( \operatorname{Mor}(\mathcal{C}) \), 其元素称作 \( \mathcal{C}
      \) 的 \emph{态射}, 配上一对映射 \( \operatorname{Mor}(\mathcal{C})
      \mathop{\rightrightarrows}\limits_{t}^{s} \operatorname{Ob}(\mathcal{C})
      \), 其中 \( s \) 和 \( t \) 分别给出态射的 \emph{来源} 和 \emph{目标}. 对
      \( X, Y \in \operatorname{Ob}(\mathcal{C}) \), 一般习惯记 \(
      \operatorname{Hom}_{\mathcal{C}}(X, Y) := s^{-1}(X) \cap t^{-1}(Y) \),
      或简单记作 \( \operatorname{Hom}(X, Y) \), 称为 \( \operatorname{Hom}
      \)-集, 其元素称为从 \( X \) 到 \( Y \) 的态射. 一般也将 \( f \in
      \operatorname{Hom}_{\mathcal{C}}(X, Y) \) 写作 \( f: X \to Y \) 或 \( X
      \xrightarrow{f} Y \), 故态射有时也叫作 \emph{箭头}.
    \item 对每个对象 \( X \), 给定元素 \( \operatorname{id}_X \in
      \operatorname{Hom}_{\mathcal{C}}(X, X) \), 称为 \( X \) 到自身的
      \emph{恒等映射}.
    \item 对任意 \( X, Y, Z \in \operatorname{Ob}(\mathcal{C}) \), 给定态射间的
      \emph{合成态射}
      \begin{align*}
        \circ: \operatorname{Hom}_{\mathcal{C}}(Y, Z) \times
        \operatorname{Hom}_{\mathbb{C}}(X, Y) &\to
        \operatorname{Hom}_{\mathcal{C}}(X, Z)\\
        (f, g) &\mapsto f \circ g,
      \end{align*}
      不会混淆时, 将 \( f \circ g \) 简记作 \( fg \). 它满足
      \begin{itemize}
        \item 结合律: 对于任意态射 \( h, g, f \in
          \operatorname{Mor}(\mathcal{C}) \), 若合成 \( f(gh) \) 和 \( (fg)h \)
          都有定义, 则
          \[
            f(gh) = (fg)h.
          \]
          故两边都可以写成 \( f \circ g \circ h \) 或 \( fgh \);
        \item 对于任意态射 \( f \in \operatorname{Hom}_{\mathcal{C}}(X, Y) \),
          有
          \[
            f \circ \operatorname{id}_X = f = \operatorname{id}_Y \circ f.
          \]
      \end{itemize}
  \end{enumerate}
\end{definition}

\begin{remark}
  \( \operatorname{id}_X \) 被其性质唯一确定.
\end{remark}

\begin{example}
  下面是一些基本的例子
  \begin{enumerate}
    \item 预序集等同于仍两个对象间至多只有一个态射的范畴, 对于预序集 \( (P,
      \leq) \), 定义范畴使得其对象集为 \( P \), 而存在态射 \( p \to p' \)
      当且仅当 \( p \leq p' \), 此时这样的态射唯一.

      作为特例, \( 0 \) 给出空范畴 \( \mathbf{0} \), 而 \( 1 \)
      给出恰有一个对象和一个态射的范畴 \( \mathbf{1} \).
    \item 给定集合 \( S \), 定义相应的 \emph{离散范畴 \( \operatorname{Disc}(S)
      \) }: 其对象是 \( S \), 而态射只有恒等映射 \( \left\lbrace
      \operatorname{id}_{x}: x \in S \right\rbrace \).
  \end{enumerate}
\end{example}

\begin{definition}
  给定任意 \( \mathcal{C} \), 其 \emph{反范畴} \(
  \mathcal{C}^{\operatorname{op}} \) 定义为
  \begin{itemize}
    \item \( \operatorname{Ob}(\mathcal{C}^{\operatorname{ob}}) =
      \operatorname{Ob})(\mathcal{C}) \).
    \item 对任意对象 \( X, Y \), \(
      \operatorname{Hom}_{\mathcal{C}^{\operatorname{op}}}(X, Y) :=
      \operatorname{Hom}_{\mathcal{C}}(Y, X) \).
    \item 态射 \( f \in \operatorname{Hom}_{\mathcal{C}^{\operatorname{op}}}(Y,
      Z), g \in \operatorname{Hom}_{\mathcal{C}^{\operatorname{op}}}(X, Y) \) 在
      \( \mathcal{C}^{\operatorname{op}} \) 中的合成 \( f
      \circ^{\operatorname{op}} g \) 定义为 \( \mathcal{C} \) 中的 \( g \circ f
      \).
    \item 恒等映射定义同 \( \mathcal{C} \).
  \end{itemize}
\end{definition}

\begin{definition}
  对于态射 \( f: X \to Y \), 若存在 \( g: Y \to X \), 使得 \( fg =
  \operatorname{id}_Y, gf = \operatorname{id}_X \), 则称 \( f \) 是 \emph{同构},
  而 \( g \) 称为 \( f \) 的\emph{逆}.
\end{definition}

\begin{example}
  对象与态射集皆空的范畴称为 \emph{空范畴}, 记作 \( \mathbf{0} \).
\end{example}

\begin{definition}
  一个范畴 \( \mathcal{C} \) 称为是 \emph{小范畴} 如果 \(
  \operatorname{Ob}(\mathcal{C}) \) 是一个集合而非类.
\end{definition}

\begin{definition}
  称 \( \mathcal{C}' \) 为 \( \mathcal{C} \) 的 \emph{子范畴}, 如果
  \begin{enumerate}
    \item \( \operatorname{Ob}(\mathcal{C}') \subseteq
      \operatorname{Ob}(\mathcal{C}) \).
    \item \( \operatorname{Mor}(\mathcal{C}') \subseteq
      \operatorname{Mor}(\mathcal{C}) \), 并保持恒等映射.
    \item 来源映射以及目标映射是 \( \operatorname{Mor}(\mathcal{C'})
      \mathop{\rightrightarrows}\limits_{t}^{s} \operatorname{Ob}(\mathcal{C'})
      \) 是 \( \mathcal{C} \) 的限制.
    \item \( \mathcal{C}' \) 中态射的合成亦是 \( \mathcal{C} \) 的限制.
  \end{enumerate}
  更进一步地, 如果 \( \operatorname{Hom}_{\mathcal{C}'}(X, Y) =
  \operatorname{Hom}_{\mathcal{C}}(X, Y) \), 那么称 \( \mathcal{C}' \) 是 \(
  \mathcal{C} \) 的全子范畴.
\end{definition}

\subsection{态射}

\begin{definition}
  假设 \( X, Y \) 为范畴 \( \mathcal{C} \) 中的对象, \( f: X \to Y \) 为态射.
  \begin{enumerate}
    \item 称 \( f \) 为\emph{单态射}, 如果对任何对象 \( Z \) 和任一对态射 \( g, h: Z
      \to X \) 有 \( fg = fh \) 当且仅当 \( g = h \).
    \item 称 \( f \) 为\emph{满态射}, 如果对任何对象 \( Z \) 和任一对态射 \( g,
      h: Y \to Z \) 有 \( gf = hf \) 当且仅当 \( g = h \).
    \item 如果存在 \( g: Y \to X \) 使得 \( gf = \operatorname{id}_X \), 那么称
      \( f \) \emph{左可逆}, \( g \) 为 \( f \) 的 \emph{左逆}.
    \item 如果存在 \( g: Y \to X \) 使得 \( fg = \operatorname{id}_Y \), 那么称
      \( f \) \emph{右可逆}, \( g \) 为 \( f \) 的 \emph{右逆}.
  \end{enumerate}
\end{definition}

\begin{proposition}
  左可逆蕴含单; 右可逆蕴含满; 态射可逆当且仅当其左右均可逆.
\end{proposition}

\section{函子与自然变换}

\subsection{函子}

\begin{definition}
  假设 \( \mathcal{C}', \mathcal{C} \) 为范畴, 一个\emph{函子} \( F:
  \mathcal{C}' \to \mathcal{C} \) 指的是下面资料
  \begin{enumerate}
    \item 对象间的映射 \( F: \operatorname{Ob}(\mathcal{C}') \to
      \operatorname{Ob}(\mathcal{C}) \).
    \item 态射间的映射 \( F: \operatorname{Mor}(\mathcal{C}') \to
      \operatorname{Mor}(\mathcal{C}) \), 使得
      \begin{itemize}
        \item 对每个 \( X, Y \in \operatorname{Ob}(\mathcal{C}') \) 皆有映射 \(
          F: \operatorname{Hom}_{\mathcal{C}'}(X, Y) \to
          \operatorname{Hom}_{\mathcal{C}}(FX, FY) \).
        \item \( F(g \circ f) = F(g) \circ F(f), F(\operatorname{id}_X) =
          \operatorname{id}_{FX} \).
      \end{itemize}
  \end{enumerate}
  对于 \( F: \mathcal{C}_1 \to \mathcal{C}_2, G: \mathcal{C}_2 \to C_3 \),
  合成函子的定义即分别取合成映射
  \begin{align*}
    &\operatorname{Ob}(\mathcal{C}_1) \xrightarrow{F}
    \operatorname{Ob}(\mathcal{C}_2) \xrightarrow{G}
    \operatorname{Ob}(\mathcal{C}_3)\\
    &\operatorname{Mor}(\mathcal{C}_1) \xrightarrow{F}
    \operatorname{Mor}(\mathcal{C}_2) \xrightarrow{G}
    \operatorname{Mor}(\mathcal{C}_3)
  \end{align*}
\end{definition}

\begin{remark}
  \label{remark-opposite-category-functor}
  从 \( \mathcal{C}' \) 到 \( \mathcal{C} \) 和从 \(
  (\mathcal{C}')^{\operatorname{op}} \) 到 \( \mathcal{C}^{\operatorname{op}} \)
  的函子是一回事. 为了区别, 对于函子 \( F: \mathcal{C}' \to \mathcal{C} \)
  \emph{反范畴间的函子} 记作 \( F^{\operatorname{op}}:
  (\mathcal{C}')^{\operatorname{op}} \to \mathcal{C}^{\operatorname{op}} \).
\end{remark}

\begin{definition}
  对于函子 \( F: \mathcal{C}' \to \mathcal{C} \).
  \begin{enumerate}
    \item 称 \( F \) 是 \emph{本质满} 的, 如果 \( \mathcal{C} \) 中任意对象都同构于某个
      \( FX \).
    \item 称 \( F \) 是 \emph{忠实} 的, 如果对所有的 \( X, Y \in
      \operatorname{Ob} (\mathcal{C}') \), 映射 \(
      \operatorname{Hom}_{\mathcal{C}'}(X, Y) \to
      \operatorname{Hom}_{\mathcal{C}}(FX, FY) \) 都是单射.
    \item 称 \( F \) 是 \emph{全} 的, 如果对所有的 \( X, Y \in
      \operatorname{Ob} (\mathcal{C}') \), 映射 \(
      \operatorname{Hom}_{\mathcal{C}'}(X, Y) \to
      \operatorname{Hom}_{\mathcal{C}}(FX, FY) \) 都是满射.
  \end{enumerate}
\end{definition}

\begin{example}
  一些经典的例子包括
  \begin{enumerate}
    \item 考虑群范畴 \( \operatorname{Grp} \). 对于任一个群 \( G \),
      总是可以忘掉 \( G \) 的群结构而视之为集合.
      群同态当然也可以视为集合间的映射. 此程序给出 \emph{忘却函子 \(
      \operatorname{Grp} \to \operatorname{Set} \) }.
      准此要领可对其他结构定义忘却函子, 例如 \( \operatorname{Top} \to
      \operatorname{Set}, \operatorname{Vect(\Bbbk)} \to \operatorname{Ab}
      \)(忘记纯量乘法)等等. 这类函子显然忠实而非全.
    \item 考虑域 \( \Bbbk \) 上的线性空间范畴 \( \operatorname{Vect}(\Bbbk) \).
      对于任意 \( \Bbbk \)-线性空间 \( V \), 定义其对偶空间
      \[
        V^{\vee} := \operatorname{Hom}_{\Bbbk}(V, \Bbbk) = \left\lbrace \Bbbk
        \text{-线性映射} V \to \Bbbk \right\rbrace.
      \]
      任一线性映射 \( f: V_1 \to V_2 \) 诱导对偶空间的反向映射
      \begin{align*}
        f^{\vee}: V_2^{\vee} &\to V_1^{\vee},\\
        (\lambda: V_2 \to \Bbbk) &\mapsto \lambda \circ f.
      \end{align*}
      容易知道 \( D: V \mapsto V^{\vee}, f \mapsto f^{\vee} \) 定义了函子 \( D:
      \operatorname{Vect}(\Bbbk)^{\operatorname{op}} \to
      \operatorname{Vect}(\Bbbk) \), 我们称其为 \emph{对偶函子}. 可以验证 \( D
      \) 是忠实的.
  \end{enumerate}
\end{example}

\subsection{自然变换}

\begin{definition}
  函子 \( F, G:\mathcal{C}' \to \mathcal{C} \) 之间的自然变换 \( \theta \)
  是一族态射
  \[
    \theta_X \in \operatorname{Hom}_{\mathcal{C}}(FX, GX),\quad X \in
    \operatorname{Ob}(\mathcal{C}')
  \]
  使得下图对所有 \( \mathcal{C}' \) 中的态射 \( f: X \to Y \) 交换
  % https://q.uiver.app/#q=WzAsNCxbMCwwLCJGWCJdLFsxLDAsIkdYIl0sWzAsMSwiRlkiXSxbMSwxLCJHWSJdLFswLDIsIkZmIiwyXSxbMCwxLCJcXHRoZXRhX1giXSxbMSwzLCJHZiJdLFsyLDMsIlxcdGhldGFfWSIsMl1d
  \[\begin{tikzcd}
    FX & GX \\
    FY & GY
    \arrow["{\theta_X}", from=1-1, to=1-2]
    \arrow["Ff"', from=1-1, to=2-1]
    \arrow["Gf", from=1-2, to=2-2]
    \arrow["{\theta_Y}"', from=2-1, to=2-2]
  \end{tikzcd}\]
  我们也将自然变换 \( \theta: F \to G \) 称为从函子 \( F \) 到 \( G \) 的态射.
\end{definition}

\begin{definition}
自然变换的合成包括横合成和纵合成两种.

考虑 \( \mathcal{C}' \) 到 \( \mathcal{C} \) 的三个函子间的态射 \(
\theta: F \to G, \psi: G \to H \). \emph{纵合成} \( \psi \circ \theta \) 的定义为
\( \left\lbrace \psi_X \circ \theta_X: X \in \operatorname{Ob}(\mathcal{C}')
\right\rbrace \)
% https://q.uiver.app/#q=WzAsNSxbMCwwLCJcXG1hdGhjYWx7Q30nIl0sWzIsMCwiXFxtYXRoY2Fse0N9Il0sWzMsMCwiXFxsZWFkc3RvIl0sWzQsMCwiXFxtYXRoY2Fse0N9JyJdLFs2LDAsIlxcbWF0aGNhbHtDfSJdLFswLDEsIkciLDFdLFswLDEsIkYiLDAseyJvZmZzZXQiOi0zLCJjdXJ2ZSI6LTN9XSxbMCwxLCJIIiwyLHsib2Zmc2V0IjoyLCJjdXJ2ZSI6M31dLFszLDQsIkYiLDAseyJjdXJ2ZSI6LTN9XSxbMyw0LCJIIiwyLHsiY3VydmUiOjN9XSxbNiw1LCJcXHRoZXRhIiwwLHsic2hvcnRlbiI6eyJzb3VyY2UiOjMwLCJ0YXJnZXQiOjMwfX1dLFs1LDcsIkgiLDAseyJzaG9ydGVuIjp7InNvdXJjZSI6MzAsInRhcmdldCI6MzB9fV0sWzgsOSwiXFxwc2kgXFxjaXJjIFxcdGhldGEiLDEseyJzaG9ydGVuIjp7InNvdXJjZSI6MjAsInRhcmdldCI6MjB9fV1d
\[\begin{tikzcd}
  {\mathcal{C}'} && {\mathcal{C}} & \leadsto & {\mathcal{C}'} && {\mathcal{C}}
  \arrow[""{name=0, anchor=center, inner sep=0}, "G"{description}, from=1-1, to=1-3]
  \arrow[""{name=1, anchor=center, inner sep=0}, "F", shift left=3, curve={height=-18pt}, from=1-1, to=1-3]
  \arrow[""{name=2, anchor=center, inner sep=0}, "H"', shift right=2, curve={height=18pt}, from=1-1, to=1-3]
  \arrow[""{name=3, anchor=center, inner sep=0}, "F", curve={height=-18pt}, from=1-5, to=1-7]
  \arrow[""{name=4, anchor=center, inner sep=0}, "H"', curve={height=18pt}, from=1-5, to=1-7]
  \arrow["\theta", shorten <=5pt, shorten >=5pt, Rightarrow, from=1, to=0]
  \arrow["H", shorten <=4pt, shorten >=4pt, Rightarrow, from=0, to=2]
  \arrow["{\psi \circ \theta}"{description}, shorten <=5pt, shorten >=5pt, Rightarrow, from=3, to=4]
\end{tikzcd}\]
考虑函子 \( 
  % https://q.uiver.app/#q=WzAsMyxbMCwwLCJcXG1hdGhjYWx7Q30nJyJdLFsxLDAsIlxcbWF0aGNhbHtDfSciXSxbMiwwLCJcXG1hdGhjYWx7Q30iXSxbMCwxLCJGXzEiLDAseyJvZmZzZXQiOi0xfV0sWzAsMSwiRl8yIiwyLHsib2Zmc2V0IjoxfV0sWzEsMiwiR18yIiwyLHsib2Zmc2V0IjoxfV0sWzEsMiwiR18xIiwwLHsib2Zmc2V0IjotMX1dXQ==
  \begin{tikzcd}
    {\mathcal{C}''} & {\mathcal{C}'} & {\mathcal{C}}
    \arrow["{F_1}", shift left, from=1-1, to=1-2]
    \arrow["{F_2}"', shift right, from=1-1, to=1-2]
    \arrow["{G_2}"', shift right, from=1-2, to=1-3]
    \arrow["{G_1}", shift left, from=1-2, to=1-3]
  \end{tikzcd}
  \) 及态射 \( \theta: F_1 \to F_2, \psi: G_1 \to G_2 \). 将\emph{横合成}
  定义为 \(
  \psi \circ \theta: G_1 \circ F_1 \to G_2 \circ F_2 \). 具体地, \( \theta:
  F_1 \to F_2 \) 以及 \( G_1 \) 给出了下面交换图表
% https://q.uiver.app/#q=WzAsOSxbMCwwLCJGXzFYIl0sWzAsMiwiRl8xWSJdLFsyLDAsIkZfMlgiXSxbMiwyLCJGXzJZIl0sWzQsMCwiR18xIEZfMSBYIl0sWzYsMCwiR18xRl8yWCJdLFs0LDIsIkdfMUZfMVkiXSxbNiwyLCJHXzFGXzJZIl0sWzMsMSwiXFxsZWFkc3RvIl0sWzAsMSwiRl8xIGYiLDJdLFswLDIsIlxcdGhldGFfWCJdLFsyLDMsIkZfMiBmIl0sWzEsMywiXFx0aGV0YV9ZIiwyXSxbNCw2LCJHXzFGXzFmIiwyXSxbNCw1LCJHXzFcXHRoZXRhX1giXSxbNSw3LCJHXzFGXzJmIl0sWzYsNywiR18xXFx0aGV0YV9ZIiwyXV0=
\[\begin{tikzcd}
{F_1X} && {F_2X} && {G_1 F_1 X} && {G_1F_2X} \\
&&& \leadsto \\
{F_1Y} && {F_2Y} && {G_1F_1Y} && {G_1F_2Y}
\arrow["{\theta_X}", from=1-1, to=1-3]
\arrow["{F_1 f}"', from=1-1, to=3-1]
\arrow["{F_2 f}", from=1-3, to=3-3]
\arrow["{G_1\theta_X}", from=1-5, to=1-7]
\arrow["{G_1F_1f}"', from=1-5, to=3-5]
\arrow["{G_1F_2f}", from=1-7, to=3-7]
\arrow["{\theta_Y}"', from=3-1, to=3-3]
\arrow["{G_1\theta_Y}"', from=3-5, to=3-7]
\end{tikzcd}\]
上图右边交换图在自然变换 \( \psi \) 下有交换图
% https://q.uiver.app/#q=WzAsOCxbMCwyLCJHXzEgRl8xIFgiXSxbMiwyLCJHXzEgRl8yIFgiXSxbNCwyLCJHXzEgRl8yIFkiXSxbNiwyLCJHXzIgRl8yIFkiXSxbMCw0LCJHXzEgRl8xIFkiXSxbMiw0LCJHXzEgRl8yIFkiXSxbMiwwLCJHXzJGXzIgWCJdLFswLDAsIkdfMiBGXzEgWCJdLFswLDQsIkdfMSBGXzEgZiIsMl0sWzEsNSwiR18xRl8yIGYiXSxbNCw1LCJHXzEgXFx0aGV0YV9ZIiwyXSxbMCwxLCJHXzFcXHRoZXRhX1giLDFdLFswLDcsIlxccHNpX3tGXzEgWH0iXSxbMSw2LCJcXHBzaV97Rl8yIFh9IiwyXSxbNyw2LCJHXzIgXFx0aGV0YV9YIl0sWzEsMiwiR18xIEZfMiBmIiwxXSxbNiwzLCJHXzIgRl8yIGYiXSxbMiwzLCJcXHBzaV97Rl8yIFl9IiwxXSxbNSwzLCJcXHBzaV97Rl8yWX0iLDJdXQ==
\[\begin{tikzcd}
{G_2 F_1 X} && {G_2F_2 X} \\
\\
{G_1 F_1 X} && {G_1 F_2 X} && {G_1 F_2 Y} && {G_2 F_2 Y} \\
\\
{G_1 F_1 Y} && {G_1 F_2 Y}
\arrow["{G_2 \theta_X}", from=1-1, to=1-3]
\arrow["{G_2 F_2 f}", from=1-3, to=3-7]
\arrow["{\psi_{F_1 X}}", from=3-1, to=1-1]
\arrow["{G_1\theta_X}"{description}, from=3-1, to=3-3]
\arrow["{G_1 F_1 f}"', from=3-1, to=5-1]
\arrow["{\psi_{F_2 X}}"', from=3-3, to=1-3]
\arrow["{G_1 F_2 f}"{description}, from=3-3, to=3-5]
\arrow["{G_1F_2 f}", from=3-3, to=5-3]
\arrow["{\psi_{F_2 Y}}"{description}, from=3-5, to=3-7]
\arrow["{G_1 \theta_Y}"', from=5-1, to=5-3]
\arrow["{\psi_{F_2Y}}"', from=5-3, to=3-7]
\end{tikzcd}\]
因此横合成为
\begin{equation}
(\psi \circ \theta)_{X} = G_2(\theta_X) \circ \psi_{F_1 X} =
\psi_{F_2 X} \circ G_1(\theta_X)  \text{ 以及 } (\psi \circ \theta)_Y =
\psi_{F_2 Y} \circ G_1(\theta_Y).
\label{equation-natural-transformation-horizontal-composite}
\end{equation}
我们可以简要地写作
% https://q.uiver.app/#q=WzAsNixbMCwwLCJcXG1hdGhjYWx7Q30nJyJdLFsxLDAsIlxcbWF0aGNhbHtDfSciXSxbMiwwLCJcXG1hdGhjYWx7Q30iXSxbMywwLCJcXGxlYWRzdG8iXSxbNCwwLCJcXG1hdGhjYWx7Q30nJyJdLFs2LDAsIlxcbWF0aGNhbHtDfSJdLFswLDEsIkZfMiIsMix7ImN1cnZlIjoyfV0sWzAsMSwiRl8xIiwwLHsiY3VydmUiOi0yfV0sWzEsMiwiR18yIiwyLHsiY3VydmUiOjJ9XSxbMSwyLCJHXzEiLDAseyJjdXJ2ZSI6LTJ9XSxbNCw1LCJHXzIgRl8yIiwyLHsiY3VydmUiOjN9XSxbNCw1LCJHXzFGXzEiLDAseyJjdXJ2ZSI6LTN9XSxbNyw2LCJcXHRoZXRhIiwxLHsic2hvcnRlbiI6eyJzb3VyY2UiOjIwLCJ0YXJnZXQiOjIwfX1dLFs5LDgsIlxccHNpIiwxLHsic2hvcnRlbiI6eyJzb3VyY2UiOjIwLCJ0YXJnZXQiOjIwfX1dLFsxMSwxMCwiXFxwc2kgXFxjaXJjIFxcdGhldGEiLDEseyJzaG9ydGVuIjp7InNvdXJjZSI6MjAsInRhcmdldCI6MjB9fV1d
\[\begin{tikzcd}
	{\mathcal{C}''} & {\mathcal{C}'} & {\mathcal{C}} & \leadsto & {\mathcal{C}''} && {\mathcal{C}}
	\arrow[""{name=0, anchor=center, inner sep=0}, "{F_2}"', curve={height=12pt}, from=1-1, to=1-2]
	\arrow[""{name=1, anchor=center, inner sep=0}, "{F_1}", curve={height=-12pt}, from=1-1, to=1-2]
	\arrow[""{name=2, anchor=center, inner sep=0}, "{G_2}"', curve={height=12pt}, from=1-2, to=1-3]
	\arrow[""{name=3, anchor=center, inner sep=0}, "{G_1}", curve={height=-12pt}, from=1-2, to=1-3]
	\arrow[""{name=4, anchor=center, inner sep=0}, "{G_2 F_2}"', curve={height=18pt}, from=1-5, to=1-7]
	\arrow[""{name=5, anchor=center, inner sep=0}, "{G_1F_1}", curve={height=-18pt}, from=1-5, to=1-7]
	\arrow["\theta"{description}, shorten <=3pt, shorten >=3pt, Rightarrow, from=1, to=0]
	\arrow["\psi"{description}, shorten <=3pt, shorten >=3pt, Rightarrow, from=3, to=2]
	\arrow["{\psi \circ \theta}"{description}, shorten <=5pt, shorten >=5pt, Rightarrow, from=5, to=4]
\end{tikzcd}\]
\end{definition}

\begin{proposition}
  纵横合成都是函子间的态射, 满足结合律 \( (\phi \circ \psi) \circ \theta = \phi
  \circ (\psi \circ \theta) \). 并且纵横合成有下述关系: 对于图
  % https://q.uiver.app/#q=WzAsMyxbMCwwLCJcXG1hdGhjYWx7Q31fMSJdLFsyLDAsIlxcbWF0aGNhbHtDfV8yIl0sWzQsMCwiXFxtYXRoY2Fse0N9XzMiXSxbMCwxLCIiLDAseyJjdXJ2ZSI6LTR9XSxbMSwyLCIiLDAseyJjdXJ2ZSI6LTR9XSxbMSwyLCIiLDAseyJjdXJ2ZSI6NH1dLFswLDEsIiIsMCx7ImN1cnZlIjo0fV0sWzAsMV0sWzEsMl0sWzMsNywiXFx0aGV0YSIsMCx7InNob3J0ZW4iOnsic291cmNlIjoyMCwidGFyZ2V0IjoyMH19XSxbNyw2LCJcXHBzaSIsMCx7InNob3J0ZW4iOnsic291cmNlIjoyMCwidGFyZ2V0IjoyMH19XSxbNCw4LCJcXHRoZXRhJyIsMCx7InNob3J0ZW4iOnsic291cmNlIjoyMCwidGFyZ2V0IjoyMH19XSxbOCw1LCJcXHBzaSciLDAseyJzaG9ydGVuIjp7InNvdXJjZSI6MjAsInRhcmdldCI6MjB9fV1d
  \[\begin{tikzcd}
    {\mathcal{C}_1} && {\mathcal{C}_2} && {\mathcal{C}_3}
    \arrow[""{name=0, anchor=center, inner sep=0}, curve={height=-24pt}, from=1-1, to=1-3]
    \arrow[""{name=1, anchor=center, inner sep=0}, curve={height=24pt}, from=1-1, to=1-3]
    \arrow[""{name=2, anchor=center, inner sep=0}, from=1-1, to=1-3]
    \arrow[""{name=3, anchor=center, inner sep=0}, curve={height=-24pt}, from=1-3, to=1-5]
    \arrow[""{name=4, anchor=center, inner sep=0}, curve={height=24pt}, from=1-3, to=1-5]
    \arrow[""{name=5, anchor=center, inner sep=0}, from=1-3, to=1-5]
    \arrow["\theta", shorten <=3pt, shorten >=3pt, Rightarrow, from=0, to=2]
    \arrow["\psi", shorten <=3pt, shorten >=3pt, Rightarrow, from=2, to=1]
    \arrow["{\theta'}", shorten <=3pt, shorten >=3pt, Rightarrow, from=3, to=5]
    \arrow["{\psi'}", shorten <=3pt, shorten >=3pt, Rightarrow, from=5, to=4]
  \end{tikzcd}\]
  下面互换律成立
  \[
    \left( \psi' \mathop{\circ}\limits_{\text{纵}} \theta' \right)
    \mathop{\circ}\limits_{\text{横}} \left( \psi
    \mathop{\circ}\limits_{\text{纵}} \theta \right) = \left( \psi'
  \mathop{\circ}\limits_{\text{横}} \psi \right)
  \mathop{\circ}\limits_{\text{纵}} \left( \theta'
  \mathop{\circ}\limits_{\text{横}} \theta \right)
  \]
\end{proposition}
\begin{proof}
  对于第一个论断, 仔细考虑之, 发现是只能讨论横合成间的合成律或纵合成间的合成律,
  因为如果 \( \phi \circ \psi \) 是横合成, \( \psi \circ \theta \) 是纵合成,
  % https://q.uiver.app/#q=WzAsMyxbMCwwLCJcXG1hdGhjYWx7Q30nJyJdLFsyLDAsIlxcbWF0aGNhbHtDfSciXSxbNCwwLCJcXG1hdGhjYWx7Q30iXSxbMCwxLCJIIiwyLHsiY3VydmUiOjN9XSxbMCwxLCJGIiwwLHsiY3VydmUiOi0zfV0sWzEsMiwiSyIsMix7ImN1cnZlIjozfV0sWzEsMiwiSiIsMCx7ImN1cnZlIjotM31dLFswLDEsIkciLDFdLFs2LDUsIlxccGhpIiwwLHsic2hvcnRlbiI6eyJzb3VyY2UiOjMwLCJ0YXJnZXQiOjMwfX1dLFs0LDcsIlxcdGhldGEiLDAseyJzaG9ydGVuIjp7InNvdXJjZSI6MzAsInRhcmdldCI6MzB9fV0sWzcsMywiXFxwc2kiLDAseyJzaG9ydGVuIjp7InNvdXJjZSI6MzAsInRhcmdldCI6MzB9fV1d
  \[\begin{tikzcd}
    {\mathcal{C}''} && {\mathcal{C}'} && {\mathcal{C}}
    \arrow[""{name=0, anchor=center, inner sep=0}, "H"', curve={height=18pt}, from=1-1, to=1-3]
    \arrow[""{name=1, anchor=center, inner sep=0}, "F", curve={height=-18pt}, from=1-1, to=1-3]
    \arrow[""{name=2, anchor=center, inner sep=0}, "G"{description}, from=1-1, to=1-3]
    \arrow[""{name=3, anchor=center, inner sep=0}, "K"', curve={height=18pt}, from=1-3, to=1-5]
    \arrow[""{name=4, anchor=center, inner sep=0}, "J", curve={height=-18pt}, from=1-3, to=1-5]
    \arrow["\theta", shorten <=4pt, shorten >=4pt, Rightarrow, from=1, to=2]
    \arrow["\psi", shorten <=4pt, shorten >=4pt, Rightarrow, from=2, to=0]
    \arrow["\phi", shorten <=7pt, shorten >=7pt, Rightarrow, from=4, to=3]
  \end{tikzcd}\]
  那么, 由上图所示, \( \left( \phi \mathop{\circ}\limits_{\text{横}} (\psi
  \mathop{\circ}\limits_{\text{纵}} \theta) \right) \) 有意义, 而 \( \left(
(\phi \mathop{\circ}\limits_{\text{横}} \psi) \mathop{\circ}\limits_{\text{纵}}
\theta \right) \) 没有定义
  \begin{itemize}
    \item 如果 \( \phi \circ \psi \) 和 \( \psi \circ \theta \) 都是纵合成
    % https://q.uiver.app/#q=WzAsMTAsWzAsMSwiXFxtYXRoY2Fse0N9JyJdLFsyLDEsIlxcbWF0aGNhbHtDfSJdLFszLDAsIkZYIl0sWzMsMiwiRlkiXSxbNSwwLCJHWCJdLFs1LDIsIkdZIl0sWzcsMCwiSFgiXSxbNywyLCJIWSJdLFs5LDAsIkpYIl0sWzksMiwiSlkiXSxbMCwxLCJGIiwwLHsiY3VydmUiOi01fV0sWzAsMSwiSiIsMix7ImN1cnZlIjo1fV0sWzAsMSwiRyIsMSx7ImN1cnZlIjotMn1dLFswLDEsIkgiLDEseyJjdXJ2ZSI6Mn1dLFsyLDMsIkZmIiwyXSxbMiw0LCJcXHBoaV9YIl0sWzMsNSwiXFxwaGlfWSJdLFs0LDUsIkdmIiwyXSxbNCw2LCJcXHBzaV9YIl0sWzUsNywiXFxwc2lfWSJdLFs2LDcsIkhmIiwyXSxbNiw4LCJcXHRoZXRhX1giXSxbNyw5LCJcXHRoZXRhX1kiXSxbOCw5LCJKZiIsMl0sWzEwLDEyLCJcXHBoaSIsMCx7InNob3J0ZW4iOnsic291cmNlIjozMCwidGFyZ2V0IjozMH19XSxbMTIsMTMsIlxccHNpIiwwLHsic2hvcnRlbiI6eyJzb3VyY2UiOjMwLCJ0YXJnZXQiOjMwfX1dLFsxMywxMSwiXFx0aGV0YSIsMCx7InNob3J0ZW4iOnsic291cmNlIjozMCwidGFyZ2V0IjozMH19XV0=
      \[\begin{tikzcd}
        &&& FX && GX && HX && JX \\
        {\mathcal{C}'} && {\mathcal{C}} \\
        &&& FY && GY && HY && JY
        \arrow["{\phi_X}", from=1-4, to=1-6]
        \arrow["Ff"', from=1-4, to=3-4]
        \arrow["{\psi_X}", from=1-6, to=1-8]
        \arrow["Gf"', from=1-6, to=3-6]
        \arrow["{\theta_X}", from=1-8, to=1-10]
        \arrow["Hf"', from=1-8, to=3-8]
        \arrow["Jf"', from=1-10, to=3-10]
        \arrow[""{name=0, anchor=center, inner sep=0}, "F", curve={height=-30pt}, from=2-1, to=2-3]
        \arrow[""{name=1, anchor=center, inner sep=0}, "J"', curve={height=30pt}, from=2-1, to=2-3]
        \arrow[""{name=2, anchor=center, inner sep=0}, "G"{description}, curve={height=-12pt}, from=2-1, to=2-3]
        \arrow[""{name=3, anchor=center, inner sep=0}, "H"{description}, curve={height=12pt}, from=2-1, to=2-3]
        \arrow["{\phi_Y}", from=3-4, to=3-6]
        \arrow["{\psi_Y}", from=3-6, to=3-8]
        \arrow["{\theta_Y}", from=3-8, to=3-10]
        \arrow["\phi", shorten <=4pt, shorten >=4pt, Rightarrow, from=0, to=2]
        \arrow["\psi", shorten <=5pt, shorten >=5pt, Rightarrow, from=2, to=3]
        \arrow["\theta", shorten <=4pt, shorten >=4pt, Rightarrow, from=3, to=1]
      \end{tikzcd}\]
      \( \phi_X, \psi_X, \theta_X \) 和 \( \phi_Y, \psi_Y, \theta_Y \) 都是 \(
      \mathcal{C} \) 中的态射, 自然满足结合律.
    \item 如果 \( \phi \circ \psi \) 和 \( \psi \circ \theta \) 都是横合成
    % https://q.uiver.app/#q=WzAsNCxbMCwwLCJcXG1hdGhjYWx7Q31fNCJdLFsyLDAsIlxcbWF0aGNhbHtDfV8zIl0sWzQsMCwiXFxtYXRoY2Fse0N9XzIiXSxbNiwwLCJcXG1hdGhjYWx7Q31fMSJdLFswLDEsIkZfMiIsMix7ImN1cnZlIjozfV0sWzEsMiwiR18yIiwyLHsiY3VydmUiOjN9XSxbMSwyLCJHXzEiLDAseyJjdXJ2ZSI6LTN9XSxbMCwxLCJGXzEiLDAseyJjdXJ2ZSI6LTN9XSxbMiwzLCJIXzIiLDIseyJjdXJ2ZSI6M31dLFsyLDMsIkhfMSIsMCx7ImN1cnZlIjotM31dLFs3LDQsIlxcdGhldGEiLDAseyJzaG9ydGVuIjp7InNvdXJjZSI6MjAsInRhcmdldCI6MjB9fV0sWzksOCwiXFxwaGkiLDAseyJzaG9ydGVuIjp7InNvdXJjZSI6MjAsInRhcmdldCI6MjB9fV0sWzYsNSwiXFxwc2kiLDAseyJzaG9ydGVuIjp7InNvdXJjZSI6MjAsInRhcmdldCI6MjB9fV1d
    \[\begin{tikzcd}
      {\mathcal{C}_4} && {\mathcal{C}_3} && {\mathcal{C}_2} && {\mathcal{C}_1}
      \arrow[""{name=0, anchor=center, inner sep=0}, "{F_2}"', curve={height=18pt}, from=1-1, to=1-3]
      \arrow[""{name=1, anchor=center, inner sep=0}, "{F_1}", curve={height=-18pt}, from=1-1, to=1-3]
      \arrow[""{name=2, anchor=center, inner sep=0}, "{G_2}"', curve={height=18pt}, from=1-3, to=1-5]
      \arrow[""{name=3, anchor=center, inner sep=0}, "{G_1}", curve={height=-18pt}, from=1-3, to=1-5]
      \arrow[""{name=4, anchor=center, inner sep=0}, "{H_2}"', curve={height=18pt}, from=1-5, to=1-7]
      \arrow[""{name=5, anchor=center, inner sep=0}, "{H_1}", curve={height=-18pt}, from=1-5, to=1-7]
      \arrow["\theta", shorten <=5pt, shorten >=5pt, Rightarrow, from=1, to=0]
      \arrow["\psi", shorten <=5pt, shorten >=5pt, Rightarrow, from=3, to=2]
      \arrow["\phi", shorten <=5pt, shorten >=5pt, Rightarrow, from=5, to=4]
    \end{tikzcd}\]
    由 \eqref{equation-natural-transformation-horizontal-composite}, 知道
    \begin{align*}
      \left(\phi \mathop{\circ}\limits_{\text{横}} (\psi
      \mathop{\circ}\limits_{\text{横}} \theta)\right)_X &= \phi_{G_2F_2X} \circ
      H_1(\psi \mathop{\circ}\limits_{\text{横}} \theta)\\ &= \phi_{G_2 F_2 X}
      \circ H_1(\psi_{F_2X} \circ G_1(\theta_X)) \\ &= \phi_{G_2 F_2 X}
      \circ H_1(\psi_{F_2X}) \circ H_1 G_1(\theta_X)
    \end{align*}
    以及
    \begin{align*}
      \left((\phi \mathop{\circ}\limits_{\text{横}} \psi)
      \mathop{\circ}\limits_{\text{横}} \theta\right)_X &= (\phi
      \mathop{\circ}\limits_{\text{横}} \psi)_{F_2 X} H_1 G_1(\theta_X) \\ &=
      \phi_{G_2 F_2 X} \circ H_1(\psi_{F_2 X}) \circ H_1 G_1(\theta_X).
    \end{align*}
  \end{itemize}
\end{proof}

\subsection{函子间的同构, 范畴间的等价与同构}

\begin{definition}
  任何函子 \( F \) 到其自身有恒等态射 \( \operatorname{id}_F: F \to F \).
  给定函子的态射 \( \theta: F_1 \to F_2 \), 如果态射 \( \psi: F_2 \to F_1 \)
  满足 \( \psi \circ \theta = \operatorname{id}_{F_1}, \theta \circ \psi =
  \operatorname{id}_{F_2} \), 则称 \( \psi \) 是 \( \theta \) 的\emph{逆}.
  可逆的函子间的态射, 称为函子间的 \emph{同构}, 写作 \( \theta: F_1
  \xrightarrow{\sim} F_2 \).
\end{definition}

\( \theta \) 的逆如果存在, 则是唯一的: 将其中一个记作 \( \theta^{-1} \), 因为 \(
(\theta^{-1})_X := (\theta_X)^{-1}: F_2 X \xrightarrow{\sim} F_1 X \),
而可以直接验证知道 \( (\theta_{X})^{-1} \) 是唯一的. 由此推理还可看出, \( \theta
\) 可逆当且仅当每个 \( \theta_X \) 均可逆. 因此, 函子间同构 \( \theta: F_1
\xrightarrow{\sim} F_2 \) 的等价说法是 \( \theta_X: F_1X \xrightarrow{\sim} F_2
X \) 对变元 \( X \) 是 \emph{自然同构} 或 \emph{典范同构}. 另外, 由结合性,
我们可以知道同构的横纵合成都是同构.

\begin{definition}
  如果一对函子 \( % https://q.uiver.app/#q=WzAsMixbMCwwLCJcXG1hdGhjYWx7Q31fMSJdLFsxLDAsIlxcbWF0aGNhbHtDfV8yIl0sWzAsMSwiRiIsMCx7ImN1cnZlIjotMn1dLFsxLDAsIkciLDAseyJjdXJ2ZSI6LTJ9XV0=
\begin{tikzcd}
	{\mathcal{C}_1} & {\mathcal{C}_2}
	\arrow["F", curve={height=-12pt}, from=1-1, to=1-2]
	\arrow["G", curve={height=-12pt}, from=1-2, to=1-1]
\end{tikzcd} \) 满足以下性质: 存在函子之间的同构故 \( \theta: FG
\xrightarrow{\sim} \operatorname{id}_{\mathcal{C}_2}, \psi: GF
\xrightarrow{\sim} \operatorname{id}_{\mathcal{C}_1} \), 则称 \( G \) 是 \( F \)
的 \emph{拟逆函子}, 并称 \( F \) 是范畴 \( \mathcal{C}_1 \) 到 \( \mathcal{C}_2
\) 的\emph{等价}. 如果更进一步有 \( FG = \operatorname{id}_{\mathcal{C}_2}, GF =
\operatorname{id}_{\mathcal{C}_1} \), 则称 \( F \) 是范畴间的 \emph{同构}, 而 \(
G\) 是 \( F \) 的\emph{逆}.
\end{definition}

\begin{proposition}
  如果 \( G, G' \) 是函子 \( F: \mathcal{C}_1 \to \mathcal{C}_2 \) 的拟逆,
  那么存在函子的同构 \( G \simeq G' \).
\end{proposition}
\begin{proof}
  函子结合性告诉我们 \(  G = \operatorname{id}_{\mathcal{C}_1}G \simeq (G'
  F)G = G'(FG) \simeq G'\operatorname{id}_{\mathcal{C}_2} = G' \).
\end{proof}

\subsection{骨架范畴}

称一个全子范畴 \( \mathcal{C}' \subseteq \mathcal{C} \) 为 \( \mathcal{C} \)
的一副\emph{骨架}, 如果对 \( \mathcal{C} \) 的每个对象 \( X \) 都存在同构 \( X
\xrightarrow{\sim} Y \in \operatorname{Ob}(\mathcal{C}') \), 并且此 \( Y \in
\operatorname{Ob}(\mathcal{C}') \) 是唯一的. 自为骨架的范畴称为 \emph{骨架范畴}.

\begin{proposition}
  \label{propostion-category-skeleton}
  \begin{enumerate}
    \item 任意范畴 \( \mathcal{C} \) 总有一副骨架 \( \mathcal{C}' \),
  而且包含函子 \( \iota: \mathcal{C}' \hookrightarrow \mathcal{C} \) 是等价.
    \item 骨架范畴间的全忠实, 本质满函子都是同构.
\end{enumerate}
\end{proposition}
\begin{proof}
  (i) 利用 \href{https://en.wikipedia.org/wiki/Axiom_of_choice}{选择公理} 在 \(
  \operatorname{Ob}(\mathcal{C})\) 的每个同构类中选取代表元,
  由这些代表元构成的全子范畴记作 \( \mathcal{C}' \). 对每个 \( X \in
  \operatorname{Ob}(\mathcal{C}) \) 可以选定同构 \( \theta_X: X
  \xrightarrow{\sim} \kappa(X) \), 其中 \( \kappa(X) \in
  \operatorname{Ob}(\mathcal{C}') \). 不妨假设对于 \( X \in
  \operatorname{Ob}(\mathcal{C}') \) 有 \( \theta_X = \operatorname{id}_X \).
  置
  \[
    \kappa(f) := \theta_Y \circ f \circ \theta ^{-1}_X \in
    \operatorname{Hom}_{\mathcal{C}'}(\kappa(X), \kappa(Y)),\quad f \in
    \operatorname{Hom}_{\mathcal{C}}(X, Y),
  \]
  这是(唯一)一种方法将 \( \kappa: \operatorname{Ob}(\mathcal{C}) \to
  \operatorname{Ob}(\mathcal{C}') \) 延拓为函子并使得 \( \theta:
  \operatorname{id}_{\mathcal{C}} \xrightarrow{\sim} \iota \kappa \). 因此 \(
  \kappa \) 是 \( \iota \) 的拟逆函子. 另一方面, 我们有函子等式 \( \kappa \iota
  = \operatorname{id}_{\mathcal{C}'} \). 因此 \( \kappa \) 是 \( \iota \)
  的拟逆函子.

  (ii) 假设 \( F: \mathcal{C}_1 \to \mathcal{C}_2 \) 是骨架范畴间的全忠实,
  本质满函子. 对任意 \( \mathcal{C}_2 \) 中的对象, 存在 \( X \) 使得 \( Z \simeq
  FX\), 又 \( \mathcal{C}_2 \) 是一个骨架范畴, 因此 \( Z = FX \), 从而 \( F \)
  是对象集的一个满射. 如果 \( FX =
  FX' \); 这时, 考察 \( \operatorname{Hom}_{\mathcal{C}'}(FX, FX') \) 与 \(
  \operatorname{Hom}_{\mathcal{C}'}(FX', FX) \), 由函子的全假设, 存在 \( f \in
  \operatorname{Hom}_{\mathcal{C}}(X, X') \), 以及 \( g \in
  \operatorname{Hom}_{\mathcal{C}}(X', X) \), 使得 \( \operatorname{id}_{FX} =
  F(f) = F(g) \). 再由函子的忠实假设, 有 \( \operatorname{id}_{FX} = F(gf)
  \implies gf = \operatorname{id}_{X} \) 以及 \( \operatorname{id}_{FX'} = F(fg)
  \implies fg = \operatorname{id}_{X'}\), 换句话说 \( X \simeq X' \).
  再次利用骨架范畴的性质, \( X = X' \), 从而 \( F \) 是对象集的一个单射.
  加上全忠实性, 我们能构造 \( F \) 的逆.
\end{proof}

\begin{theorem}
  对于函子 \( F: \mathcal{C}_1 \to \mathcal{C}_2 \), 那么以下等价
  \begin{enumerate}
    \item \( F \) 是范畴等价.
    \item \( F \) 是全忠实, 本质满函子.
  \end{enumerate}
\end{theorem}
\begin{proof}
  如果 \( F \) 是一个范畴等价, 那么存在 \( F \) 的拟逆函子 \( G: \mathcal{C}_2
  \to \mathcal{C}_1 \) 和自然变换 \( \psi: GF \xrightarrow{\sim}
  \operatorname{id}_{\mathcal{C}_1}, \phi: FG \xrightarrow{\sim}
  \operatorname{id}_{\mathcal{C}_2} \).
  \begin{itemize}
    \item 本质满. 对 \( \mathcal{C}_2 \) 中的任何对象 \( Z \) 都有 \( \phi_Z:
      F(GZ) \xrightarrow{\sim} Z \), 故 \( F \) 本质满. 同理 \( G \) 本质满.
    \item 全忠实. \( \psi \) 的自然性告诉我们下图交换
      % https://q.uiver.app/#q=WzAsNCxbMCwwLCJHRlgiXSxbMCwxLCJHRlkiXSxbMSwwLCJYIl0sWzEsMSwiWSJdLFswLDEsIkdGKGYpIiwyXSxbMCwyLCJcXHBzaV9YIl0sWzIsMywiZiJdLFsxLDMsIlxccHNpX1kiLDJdXQ==
      \[\begin{tikzcd}
        GF(X) & X \\
        GF(Y) & Y
        \arrow["{\psi_X}", from=1-1, to=1-2]
        \arrow["{GF(f)}"', from=1-1, to=2-1]
        \arrow["f", from=1-2, to=2-2]
        \arrow["{\psi_Y}"', from=2-1, to=2-2]
      \end{tikzcd}\] 函子同构告诉我们 \( \psi_X^{-1} \) 存在, 从而 \( f =
      \psi_Y GF(f) \psi_{X}^{-1} \). 因此下图的合成是恒等映射
      % https://q.uiver.app/#q=WzAsOCxbMCwwLCJcXG9wZXJhdG9ybmFtZXtIb219KFgsIFkpIl0sWzEsMCwiXFxvcGVyYXRvcm5hbWV7SG9tfShGWCwgRlkpIl0sWzIsMCwiXFxvcGVyYXRvcm5hbWV7SG9tfShHRihYKSwgR0YoWSkpIl0sWzMsMCwiXFxvcGVyYXRvcm5hbWV7SG9tfShYLCBZKSJdLFswLDEsImYiXSxbMSwxLCJGZiJdLFsyLDEsIkdGKGYpIl0sWzMsMSwiXFxwc2lfWSBHRihmKVxccHNpX1heey0xfSJdLFswLDEsIkYiXSxbMSwyLCJHIl0sWzIsMywiXFxzaW0iXSxbNCw1LCIiLDAseyJzdHlsZSI6eyJ0YWlsIjp7Im5hbWUiOiJtYXBzIHRvIn19fV0sWzUsNiwiIiwwLHsic3R5bGUiOnsidGFpbCI6eyJuYW1lIjoibWFwcyB0byJ9fX1dLFs2LDcsIiIsMCx7InN0eWxlIjp7InRhaWwiOnsibmFtZSI6Im1hcHMgdG8ifX19XV0=
      \[\begin{tikzcd}
        {\operatorname{Hom}(X, Y)} & {\operatorname{Hom}(FX, FY)} & {\operatorname{Hom}(GF(X), GF(Y))} & {\operatorname{Hom}(X, Y)} \\
        f & Ff & {GF(f)} & {\psi_Y GF(f)\psi_X^{-1}}
        \arrow["F", from=1-1, to=1-2]
        \arrow["G", from=1-2, to=1-3]
        \arrow["\sim", from=1-3, to=1-4]
        \arrow[maps to, from=2-1, to=2-2]
        \arrow[maps to, from=2-2, to=2-3]
        \arrow[maps to, from=2-3, to=2-4]
      \end{tikzcd}\]
      因此, \( F \) 左可逆, 从而单; 而函子同构假设告诉我们上图最右边是一个同构,
      从而可以知道 \( G \) 是一个满同态. 同理, 交换 \( G, F \) 在上图的地位,
      可以知道 \( G \) 单, \( F \) 是一个满同态.
  \end{itemize}

  假设 \( F \) 是全忠实, 本质满函子. 取骨架 \( \iota_i: \mathcal{C}'_i \to
  \mathcal{C}_i \) 及其拟逆函子 \( \kappa_i \) (这里 \( i = 1, 2 \)). 因此函子
  \( F' := \kappa_2 \circ F \circ \iota_1: \mathcal{C}'_1 \to \mathcal{C}'_2 \)
  也是全忠实本质满函子, 因此由 \cref{propostion-category-skeleton} \( F' \)
  是范畴同构. 设 \( G := \iota_1 \circ F'^{-1} \circ \kappa_2 \), 则
  \begin{align*}
    GF = \iota_1 F'^{-1} \kappa_2 F \simeq \iota_1 F'^{-1} \kappa_2 F \iota_1
    \kappa_1 = \iota_1 \kappa_1 \simeq \operatorname{id}_{\mathcal{C}_1},\\
    FG = F \iota_1 F'^{-1} \kappa_2 \simeq \iota_2 \kappa_2 F \iota_1 F'^{-1}
    \kappa_2 = \iota_2 \kappa_2 \simeq \operatorname{id}_{\mathcal{C}_2}.
  \end{align*}
\end{proof}

\section{函子范畴}

\subsection{积范畴, 余积范畴以及 \texorpdfstring{\( \operatorname{Hom}
\)}{Hom}函子}

\begin{definition}
  假设 \( \left\lbrace \mathcal{C}_i: i \in I \right\rbrace \) 是一族范畴.
  \begin{itemize}
    \item \emph{积范畴} \( \prod_{i \in I} \mathcal{C}_i \) 定义为
      \begin{align*}
        \operatorname{Ob}\left( \prod_{i \in I} \mathcal{C}_i \right) &:=
        \prod_{i \in I} \operatorname{Ob}(\mathcal{C}_i),\\
        \operatorname{Hom}_{\prod_{i \in I}\mathcal{C}_i} \left( (X_i)_i,
      (Y_i)_i \right) &:= \prod_{i \in I} \operatorname{Hom}_{\mathcal{C}_i}
      (X_i, Y_i),
      \end{align*}
      其中我们以 \( (X_i)_i \) 表示 \( \prod_{i \in
      I}\operatorname{Ob}(\mathcal{C}_i) \) 的元素. 态射的合成是逐个分量定义的.
      我们有一族\emph{投影函子} \( \operatorname{pr}_j: \prod_{i \in
      I}\mathcal{C}_i \to \mathcal{C}_j \), 它将 \( (X_i)_i \) 映至 \( X_j \),
      在态射层面也是类似地投影到 \( j \) 分量.
    \item \emph{余积范畴 } \( \coprod_{i \in I}\mathcal{C}_i \) 定义为
      \begin{align*}
        \operatorname{Ob}\left( \coprod_{i \in I} \mathcal{C}_i \right) &:=
        \coprod_{i \in I}\operatorname{Ob}(\mathcal{C}_i),\\
        \operatorname{Hom}_{\prod_{i \in I} \mathcal{C}_i}(X_j, X_k) &:= \begin{cases}
          \operatorname{Hom}_{\mathcal{C}_i}(X_j, X_k), & j = k\\
          \varnothing, & j \neq k.
        \end{cases}
      \end{align*}
      其中对每个 \( j \in I, X_j \in \operatorname{Ob}(\mathcal{C}_j) \).
      态射的合成是在各个 \( \mathcal{C}_i \) 中个别定义的.
      % 我们有一族投影函子 \(
      % \iota_j: \mathcal{C}_j \to \coprod_{i \in I} \mathcal{C}_i \) 将 \(
      % \mathcal{C}_i \) 以自明的方式嵌入为全子范畴.
  \end{itemize}
  特别地, 如果 \( I \) 为有限集, 我们记 \( \mathcal{C}_1 \times \cdots \times
  \mathcal{C}_n \) 和 \( \mathcal{C}_1 \sqcup \cdots \sqcup \mathcal{C}_n \).
\end{definition}

\begin{definition}
  形如 \( F: \mathcal{C}_1 \times \mathcal{C}_2 \to \mathcal{C} \) 的函子称为
  \emph{二元函子}. \emph{多元函子} \( \mathcal{C}_1 \times \cdots \times
  \mathcal{C}_n \to \mathcal{C} \) 的定义类似.
\end{definition}

给定范畴 \( \mathcal{C} \), 则 \( (X, Y) \mapsto
\operatorname{Hom}_{\mathcal{C}}(X, Y) \) 定义了二元函子
\[
  \operatorname{Hom}: \mathcal{C}^{\operatorname{op}} \times
  \mathcal{C} \to \operatorname{Set}.
\]
\( \mathcal{C} \) 中的任一对态射 \( f: X' \to X, g: Y \to Y' \) 诱导了
\begin{align*}
  \operatorname{Hom}_{\mathcal{C}}(X, Y) &\to
  \operatorname{Hom}_{\mathcal{C}}(X', Y')\\
  \phi &\mapsto g \phi f.
\end{align*}
这就是函子对态射集的映射
\begin{align*}
  \operatorname{Hom}: \operatorname{Hom}_{\mathcal{C}^{op} \times
  \mathcal{C}}((X, Y), (X', Y')) &\to
  \operatorname{Hom}_{\operatorname{Set}}(\operatorname{Hom}_{\mathcal{C}}(X,
  Y), \operatorname{Hom}_{\mathcal{C}}(X', Y'))\\ (f, g) &\mapsto (\phi \mapsto g
  \phi f)
\end{align*}
有时也说这时 \( \phi \) 对 \( f \) 的\emph{拉回}, 对 \( g \) 的\emph{推出}.
习惯用符号 \( f^* \phi = \phi f \) 和 \( g_* \phi = g \phi \) 表示. 容易知道 \(
(f_1 f_2)^* = f_2^* f_1^* \) 和 \( (f_1 f_2)_* = (f_1)_* (f_2)_* \).
我们称这个二元函子为 \( \operatorname{Hom} 函子 \), 并在不混淆的情况下记为 \(
\operatorname{Hom}_{\mathcal{C}} \).
\subsection{函子范畴与范畴的中心}

\begin{definition}
  设 \( \mathcal{C}_1, \mathcal{C}_2 \) 为范畴, 定义 \emph{函子范畴 \(
  \operatorname{Fct}(\mathcal{C}_1, \mathcal{C}_2) \) }: 其对象是 \(
  \mathcal{C}_1 \) 到 \( \mathcal{C}_2 \) 的函子, 任两个对象 \( F, G \)
  之间的态射是自然变换 \( \theta: F \to G \); 态射 \( \theta: F \to G \) 与 \(
  \psi: G \to H \) 的合成是自然变换的纵合成 \( \psi \circ \theta: F \to H \).
\end{definition}

对函子 \( F, G: \mathcal{C}_1 \to \mathcal{C}_2 \), 于
\cref{remark-opposite-category-functor} 中, 我们讨论过, 它们在反范畴中有 \(
F^{\operatorname{op}}, G^{\operatorname{op}}: \mathcal{C}_1^{\operatorname{op}}
\to \mathcal{C}_2^{\operatorname{op}} \). 见下图
% https://q.uiver.app/#q=WzAsOSxbMCwwLCJGWCJdLFswLDIsIkZZIl0sWzIsMCwiR1giXSxbMiwyLCJHWSJdLFszLDEsIlxcbGVhZHN0byJdLFs0LDAsIkZee1xcb3BlcmF0b3JuYW1le29wfX1YIl0sWzQsMiwiRl57XFxvcGVyYXRvcm5hbWV7b3B9fVkiXSxbNiwwLCJHXntcXG9wZXJhdG9ybmFtZXtvcH19WCJdLFs2LDIsIkdee1xcb3BlcmF0b3JuYW1le29wfX1ZIl0sWzAsMiwiXFx2YXJwaGlfWCJdLFswLDEsIkZmIiwyXSxbMiwzLCJHZiJdLFsxLDMsIlxcdmFycGhpX1kiLDJdLFs2LDUsIkZee1xcb3BlcmF0b3JuYW1le29wfX1mIl0sWzgsNywiR157XFxvcGVyYXRvcm5hbWV7b3B9fWYiLDJdLFs4LDYsIlxcdmFycGhpXntcXG9wZXJhdG9ybmFtZXtvcH19X1kiXSxbNyw1LCJcXHZhcnBoaV57XFxvcGVyYXRvcm5hbWV7b3B9fV9YIiwyXV0=
\[\begin{tikzcd}
	FX && GX && {F^{\operatorname{op}}X} && {G^{\operatorname{op}}X} \\
	&&& \leadsto \\
	FY && GY && {F^{\operatorname{op}}Y} && {G^{\operatorname{op}}Y}
	\arrow["{\varphi_X}", from=1-1, to=1-3]
	\arrow["Ff"', from=1-1, to=3-1]
	\arrow["Gf", from=1-3, to=3-3]
	\arrow["{\varphi^{\operatorname{op}}_X}"', from=1-7, to=1-5]
	\arrow["{\varphi_Y}"', from=3-1, to=3-3]
	\arrow["{F^{\operatorname{op}}f}", from=3-5, to=1-5]
	\arrow["{G^{\operatorname{op}}f}"', from=3-7, to=1-7]
	\arrow["{\varphi^{\operatorname{op}}_Y}", from=3-7, to=3-5]
\end{tikzcd}\]
自然变换 \( \varphi: F \to G \) 在反范畴中被到转为 \(
\varphi^{\operatorname{op}}: G^{\operatorname{op}} \to F^{\operatorname{op}} \),
并且 \( (\varphi^{\operatorname{op}})^{\operatorname{op}} = \varphi \).
因此我们由下面命题

\begin{proposition}
  \label{proposition-opposite-functor-category}
  存在自然同构 \( \operatorname{Fct}(\mathcal{C}_1,
  \mathcal{C}_2)^{\operatorname{op}} \xrightarrow{\sim}
  \operatorname{Fct}(\mathcal{C}_1^{\operatorname{op}},
  \mathcal{C}_2^{\operatorname{op}}) \), 它将 \( \varphi \) 映至 \(
  \varphi^{\operatorname{op}} \).
\end{proposition}

\begin{definition}
  一个范畴 \( \mathcal{C} \) 的\emph{中心}为 \( Z(\mathcal{C}) :=
  \operatorname{End}(\operatorname{id}_{\mathcal{C}}) \), 也就是 \(
  \operatorname{Fct}(\mathcal{C}, \mathcal{C}) \) 中的 \(
  \operatorname{Hom}_{\operatorname{Fct}(\mathcal{C},
  \mathcal{C})}(\operatorname{id}_{\mathcal{C}},
  \operatorname{id}_{\mathcal{C}}) \).
\end{definition}

由自然变换(函子的态射)的定义知道, \( Z(\mathcal{C}) \) 中的元素是一族自同态 \(
\psi_X: X \to X \), 使得图表
% https://q.uiver.app/#q=WzAsNCxbMCwwLCJYIl0sWzAsMSwiWSJdLFsxLDAsIlgiXSxbMSwxLCJZIl0sWzAsMSwiZiIsMl0sWzAsMiwiXFxwc2lfWCJdLFsxLDMsIlxccHNpX1kiLDJdLFsyLDMsImYiXV0=
\[\begin{tikzcd}
	X & X \\
	Y & Y
	\arrow["{\psi_X}", from=1-1, to=1-2]
	\arrow["f"', from=1-1, to=2-1]
	\arrow["f", from=1-2, to=2-2]
	\arrow["{\psi_Y}"', from=2-1, to=2-2]
\end{tikzcd}\]
都每个 \( f: X \to Y \) 都交换.

\begin{proposition}
  中心 \( Z(\mathcal{C}) \) 对二元运算 \( \circ \) 总是交换的.
\end{proposition}
\begin{proof}
  对 \( \theta, \psi \in \operatorname{End}(\operatorname{id}_{\mathcal{C}}) \),
  取 \( X = Y \) 以及 \( f = \theta \) 立刻得到.
  % https://q.uiver.app/#q=WzAsNCxbMCwwLCJYIl0sWzAsMSwiWCJdLFsxLDAsIlgiXSxbMSwxLCJYIl0sWzAsMSwiXFx0aGV0YV9YIiwyXSxbMCwyLCJcXHBzaV9YIl0sWzEsMywiXFxwc2lfWCIsMl0sWzIsMywiXFx0aGV0YV9YIl1d
  \[\begin{tikzcd}
    X & X \\
    X & X
    \arrow["{\psi_X}", from=1-1, to=1-2]
    \arrow["{\theta_X}"', from=1-1, to=2-1]
    \arrow["{\theta_X}", from=1-2, to=2-2]
    \arrow["{\psi_X}"', from=2-1, to=2-2]
  \end{tikzcd}\]
\end{proof}

\begin{proposition}
  范畴等价 \( F: \mathcal{C}_1 \to \mathcal{C}_2 \) 诱导中心的同构 \(
  Z(\mathcal{C}_1) \simeq Z(\mathcal{C}_2) \).
\end{proposition}

\section{泛性质}
\subsection{始对象, 终对象与零对象}
假设给定了一个范畴 \( \mathcal{C} \).
\begin{enumerate}
  \item 如果 \( X \in \operatorname{Ob}(\mathcal{C}) \) 使得 \(
    \operatorname{Hom}_{\mathcal{C}}(X, Y) \) 有且仅有一个元素对所有 \( Y \in
    \operatorname{Ob}(\mathcal{C}) \) 成立, 那么我们称 \( X \) 是 \( \mathcal{C}
    \) 中的 \emph{始对象}.
  \item 如果 \( X \in \operatorname{Ob}(\mathcal{C}) \) 使得 \(
    \operatorname{Hom}_{\mathcal{C}}(Y, X) \) 有且仅有一个元素对所有 \( Y \in
    \operatorname{Ob}(\mathcal{C}) \) 成立, 那么我们称 \( X \) 是 \( \mathcal{C}
    \) 中的 \emph{终对象}.
  \item 如果 \( X \in \operatorname{Ob}(\mathcal{C}) \) 既是始对象又是终对象,
    那么我们称其为 \emph{零对象}.
\end{enumerate}

\begin{proposition}
  如果 \( X, X' \) 为 \( \mathcal{C} \) 中的始对象, 那么存在唯一的同构 \( X
  \simeq X' \). 类似地, 如果 \( X, X' \) 为 \( \mathcal{C} \) 中的终对象,
  那么存在唯一的同构 \( X \simeq X' \).
\end{proposition}

假设 \( \mathcal{C} \) 有零对象, 记作 \( \mathbf{0} \). 对任意 \( X, Y \in
\operatorname{Ob}(\mathcal{C}) \), 我们定义 \emph{零态射} \( 0: X \to Y \) 为复合
\[
  X \to 0 \to Y.
\]

\begin{example}
  下面是一些典型的例子.
  \begin{enumerate}
    \item 集合范畴 \( \operatorname{Set} \): \( \varnothing \) 是始对象, \(
      \left\lbrace \operatorname{pt} \right\rbrace \) 是终对象.
    \item 带基点的集合范畴 \( \operatorname{Set}_{\bullet} \): \( (\left\lbrace
      \operatorname{pt} \right\rbrace, \operatorname{pt}) \) 是零对象.
    \item 群范畴 \( \operatorname{Grp} \): 平凡群 \( \left\lbrace 1
      \right\rbrace \) 是零对象, 取常值 \( 1 \) 的同态是零态射.
    \item 域 \( \Bbbk \) 上的向量空间范畴 \(
      \operatorname{Vect}(\Bbbk) \): 零空间是零对象, 零映射是零态射.
    \item 离散范畴既没有始对象, 也没有终对象.
  \end{enumerate}
\end{example}

\subsection{逗号范畴}

\begin{example}
  \label{example-free-vector-space}
  选定域 \( \Bbbk \), 定义函子 \( V: \operatorname{Set} \to
  \operatorname{Vect}(\Bbbk) \) 如下
  \begin{itemize}
    \item 对于集合 \( X \), 命 \( V(X) := \bigoplus_{x \in X} \Bbbk x \) 为以 \(
      X \) 为基的 \( \Bbbk \)-向量空间.
    \item 任意映射 \( f: X \to Y \) 皆诱导出线性映射 \( V(f): V(X) \to V(Y) \),
      它由在基上的限制 \( f \) 刻画.
  \end{itemize}
  令 \( U: \operatorname{Vect}(\Bbbk) \to \operatorname{Set} \) 为忘却函子, 则
  \( x \mapsto V(X) \) 给出态射 \( \iota: X \to UV(X) \).

  定义范畴 \( (X / U) \) 使得其对象形如\( (W, i: X \to U(W)) \), 其中 \( W \in
  \operatorname{Ob}\left( \operatorname{Vect} \Bbbk \right) \), 而 \( X
  \xrightarrow{i} U(W) \) 是 \( \operatorname{Set} \) 中的态射.
  态射定为使下图交换的线性映射 \( h: W_1 \to W_2 \)
  % https://q.uiver.app/#q=WzAsMyxbMSwwLCJYIl0sWzAsMSwiVShXXzEpIl0sWzIsMSwiVShXXzIpIl0sWzAsMSwiaV8xIiwyXSxbMCwyLCJpXzIiXSxbMSwyLCJVKGgpIiwyXV0=
  \[\begin{tikzcd}
    & X \\
    {U(W_1)} && {U(W_2)}
    \arrow["{i_1}"', from=1-2, to=2-1]
    \arrow["{i_2}", from=1-2, to=2-3]
    \arrow["{U(h)}"', from=2-1, to=2-3]
  \end{tikzcd}\]
  我们断言 \( (V(X), \iota) \) 是 \( (X / U) \) 中的始对象. 这说的无非是对任意
  \( (W, i) \in \operatorname{Ob}(X / U) \), 存在唯一的 \( h: V(X) \to W \)
  使得图表
  % https://q.uiver.app/#q=WzAsMyxbMSwwLCJYIl0sWzAsMSwiVShWKFgpKSJdLFsyLDEsIlUoV18yKSJdLFswLDEsIlxcaW90YSIsMl0sWzAsMiwiaSJdLFsxLDIsIlUoaCkiLDJdXQ==
  \[\begin{tikzcd}
    & X \\
    {U(V(X))} && {U(W_2)}
    \arrow["\iota"', from=1-2, to=2-1]
    \arrow["i", from=1-2, to=2-3]
    \arrow["{U(h)}"', from=2-1, to=2-3]
  \end{tikzcd}\]
  交换.
\end{example}

\begin{definition}
  对于函子 \( \mathcal{A} \xrightarrow{S} \mathcal{C} \xleftarrow{T} \mathcal{B}
  \) 定义\emph{逗号范畴\( (S/T) \)} 如下
  \begin{itemize}
    \item 对象: 形如 \( (A, B, f) \), 其中 \( A \in
      \operatorname{Ob}(\mathcal{A}), B \in \operatorname{Ob}(\mathcal{B}), f:
      SA \to TB \).
    \item 态射: 从 \( (A, B, f) \) 到 \( (A', B', f') \) 的态射形如 \( (g, h)
      \), 其中 \( g: A \to A', h: B \to B' \) 分别是 \( \mathcal{A}, \mathcal{B}
      \) 中的态射, 使得下图交换
      % https://q.uiver.app/#q=WzAsNCxbMCwwLCJTQSJdLFsxLDAsIlNBJyJdLFswLDEsIlRCIl0sWzEsMSwiVEInIl0sWzIsMywiVGgiLDJdLFswLDIsImYiLDJdLFswLDEsIlNnIl0sWzEsMywiZiciXV0=
      \[\begin{tikzcd}
        SA & {SA'} \\
        TB & {TB'}
        \arrow["Sg", from=1-1, to=1-2]
        \arrow["f"', from=1-1, to=2-1]
        \arrow["{f'}", from=1-2, to=2-2]
        \arrow["Th"', from=2-1, to=2-2]
      \end{tikzcd}\]
      态射的合成是 \( (g_1, h_1) \circ (g_2, h_2) = (g_1 \circ g_2, h_1 \circ
      h_2) \), 而 \( (A, B, f) \) 到自身的恒等态射是 \( (\operatorname{id}_A,
      \operatorname{id}_B) \).
  \end{itemize}
\end{definition}

我们有明显的左, 右投影函子 \( P: (S / T) \to \mathcal{A} \) 和 \( Q: (S / T) \to
\mathcal{B} \).

\begin{example}
  \label{example-comma-category}
  设 \( \mathbf{1} \) 为先前定义的预序集范畴. 指定 \( \mathcal{C}
  \) 中的一个对象 \( X \) 相当于指定了一个函子 \( j_X: \mathbf{1} \to \mathcal{C}
  \).
  \begin{itemize}
    \item  考虑函子 \( T: \mathcal{C}'
      \to \mathcal{C} \) 以及 \( X \in \operatorname{Ob}(\mathcal{C}) \). 对应于
      \( \mathbf{1} \xrightarrow{j_X} \mathcal{C} \xleftarrow{T} \mathcal{C}' \)
      的逗号范畴 \( (X / T) := (j_X / T) \) 的对象形如 \( (W, X \xrightarrow{i}
      TW)_{W \in \operatorname{Ob}(\mathcal{C}')} \), 而 \( (W_1, i_1), (W_2,
      i_2) \) 间的态射是使得下图交换的 \( h: W_1 \to W_2 \)
      % https://q.uiver.app/#q=WzAsMyxbMSwwLCJYIl0sWzAsMSwiVFdfMSJdLFsyLDEsIlRXXzIiXSxbMCwxLCJpXzEiLDJdLFswLDIsImlfMiJdLFsxLDIsIlRoIiwyXV0=
      \[\begin{tikzcd}
        & X \\
        {TW_1} && {TW_2}
        \arrow["{i_1}"', from=1-2, to=2-1]
        \arrow["{i_2}", from=1-2, to=2-3]
        \arrow["Th"', from=2-1, to=2-3]
      \end{tikzcd}\]
      这正是 \cref{example-free-vector-space} 中使用的范畴.
    \item  考虑 \( \mathcal{C}'
      \xrightarrow{T} \mathcal{C} \xleftarrow{j_X} \mathbf{1} \). 逗号范畴 \( (T
      / X) \) 的对象形如 \( (W, TW \xrightarrow{p} X)_{W \in
      \operatorname{Ob}(\mathcal{C}')} \), 从 \( (W_1, p_1) \) 到 \( (W_2, p_2)
      \) 的态射是使得下图交换的 \( f: W_1 \to W_2 \)
      % https://q.uiver.app/#q=WzAsMyxbMCwwLCJUIFdfMSJdLFsyLDAsIlQgV18yIl0sWzEsMSwiWCJdLFswLDIsInBfMSIsMl0sWzEsMiwicF8yIl0sWzAsMSwiVGYiXV0=
      \[\begin{tikzcd}
        {T W_1} && {T W_2} \\
        & X
        \arrow["Tf", from=1-1, to=1-3]
        \arrow["{p_1}"', from=1-1, to=2-2]
        \arrow["{p_2}", from=1-3, to=2-2]
      \end{tikzcd}\]
    \item 逗号范畴 \( (\operatorname{id}_{\mathcal{C}},
      \operatorname{id}_{\mathcal{C}}) \) 的对象是 \( \mathcal{C} \)
      中的所有态射 \( f: X \to Y \), 两个对象 \( f: X \to Y \), \( f': X' \to Y'
      \) 之间的态射是交换图表
      % https://q.uiver.app/#q=WzAsNCxbMCwwLCJYIl0sWzAsMSwiWCciXSxbMSwxLCJZJyJdLFsxLDAsIlkiXSxbMCwzLCJmIl0sWzAsMV0sWzMsMl0sWzEsMiwiZiciLDJdXQ==
      \[\begin{tikzcd}
        X & Y \\
        {X'} & {Y'}
        \arrow["f", from=1-1, to=1-2]
        \arrow[from=1-1, to=2-1]
        \arrow[from=1-2, to=2-2]
        \arrow["{f'}"', from=2-1, to=2-2]
      \end{tikzcd}\]
      \( (\operatorname{id}_{\mathcal{C}} / \operatorname{id}_{\mathcal{C}}) \)
      也叫 \( \mathcal{C} \) 的箭头范畴.
  \end{itemize}
\end{example}


\subsection{核与余核}

假设 \( \mathcal{C} \) 是一个带零对象 \( \mathbf{0} \) 的范畴,
约定下面讨论对象都是 \( \mathcal{C} \) 的对象, 态射都是 \( \mathcal{C} \)
的态射. 给定一个映射 \( f: B \to C \).

我们称 \( f \) 的 \emph{核} 为态射 \( i: A \to B \) 使得 \( fi = 0 \)
且满足泛性质: 每个使得 \( fe = 0 \) 的态射 \( e: A' \to B \) 都能分解为 \( e = ie'
\), 其中 \( e': A' \to A \) 唯一.

我们称 \( f: B \to C \) 的 \emph{余核} 为态射 \( p: C \to D \) 使得 \( pf = 0 \)
且满足泛性质: 每个使得 \( gf = 0 \) 的态射 \( g: C \to D' \) 都能分解为 \( g =
g'p \), 其中 \( g': D \to D' \) 唯一.

% https://q.uiver.app/#q=WzAsOCxbMCwxLCJBIl0sWzEsMSwiQiJdLFsyLDEsIkMiXSxbMSwwLCJBJyJdLFs0LDEsIkIiXSxbNSwxLCJDIl0sWzYsMSwiRCJdLFs1LDAsIkQnIl0sWzEsMiwiZiJdLFswLDEsImkiXSxbMywxLCJlIl0sWzMsMCwiXFxleGlzdHMgISBlJyIsMix7InN0eWxlIjp7ImJvZHkiOnsibmFtZSI6ImRhc2hlZCJ9fX1dLFs0LDUsImYiXSxbNSw2LCJwIl0sWzUsNywiZyJdLFs2LDcsIlxcZXhpc3RzICEgZyciLDIseyJzdHlsZSI6eyJib2R5Ijp7Im5hbWUiOiJkYXNoZWQifX19XV0=
\[\begin{tikzcd}
	& {A'} &&&& {D'} \\
	A & B & C && B & C & D
	\arrow["{\exists ! e'}"', dashed, from=1-2, to=2-1]
	\arrow["e", from=1-2, to=2-2]
	\arrow["i", from=2-1, to=2-2]
	\arrow["f", from=2-2, to=2-3]
	\arrow["f", from=2-5, to=2-6]
	\arrow["g", from=2-6, to=1-6]
	\arrow["p", from=2-6, to=2-7]
	\arrow["{\exists ! g'}"', dashed, from=2-7, to=1-6]
\end{tikzcd}\]

\begin{proposition}
  上面记号中, \( i \) 是单射, \( p \) 是满射.
\end{proposition}

\subsection{乘积与余积}

假设给定一个范畴 \( \mathcal{C} \) 以及 \( \mathcal{C} \) 的一个对象集 \(
\left\lbrace C_i: i \in I \right\rbrace \subseteq \operatorname{Ob}(\mathcal{C})
\).

我们称 \( \left\lbrace C_i: i \in I \right\rbrace \) 的\emph{乘积}(不一定存在)
指的是 \( \mathcal{C} \) 中的一个对象 \( \prod_{i \in I}C_i \) 加上一族态射 \( \pi_j: \prod C_i \to C_j, j
\in I \), 使得对每个 \( A \in \mathcal{C} \), 以及态射 \( \alpha_i: A \to C_i, i
\in I \), 都存在唯一的态射 \( \alpha: A \to \prod C_i \) 使得 \( \pi_i \alpha =
\alpha_i \).

对偶地, 我们称 \( \left\lbrace C_i: i \in I \right\rbrace \)
的\emph{余积}(不一定存在) 指的是 \( \mathcal{C} \) 中的一个对象 \( \coprod_{i
\in I}C_i \) 加上一族态射 \( \iota_j: C_j \to \coprod C_i, j \in I \),
使得对每个 \( A \in \mathcal{C} \), 以及态射 \( \alpha_i: C_i \to A, i \in I \),
都存在唯一的态射 \( \alpha: \coprod C_i \to A \) 使得 \( \alpha \iota_i =
\alpha_i \).

% https://q.uiver.app/#q=WzAsNixbMSwwLCJcXHByb2QgQ19pIl0sWzEsMSwiQ19qIl0sWzAsMSwiQSJdLFszLDEsIkEiXSxbNCwxLCJDX2oiXSxbNCwwLCJcXGNvcHJvZCBDX2kiXSxbMiwxLCJcXGFscGhhX2oiLDJdLFsyLDAsIlxcZXhpc3RzICEgXFxhbHBoYSIsMCx7InN0eWxlIjp7ImJvZHkiOnsibmFtZSI6ImRhc2hlZCJ9fX1dLFswLDEsIlxccGlfaiJdLFs0LDMsIlxcYWxwaGFfaiJdLFs0LDUsIlxcaW90YV9qIiwyXSxbNSwzLCJcXGV4aXN0cyAhIFxcYWxwaGEiLDIseyJzdHlsZSI6eyJib2R5Ijp7Im5hbWUiOiJkYXNoZWQifX19XV0=
\[\begin{tikzcd}
	& {\prod C_i} &&& {\coprod C_i} \\
	A & {C_j} && A & {C_j}
	\arrow["{\pi_j}", from=1-2, to=2-2]
	\arrow["{\exists ! \alpha}"', dashed, from=1-5, to=2-4]
	\arrow["{\exists ! \alpha}", dashed, from=2-1, to=1-2]
	\arrow["{\alpha_j}"', from=2-1, to=2-2]
	\arrow["{\iota_j}"', from=2-5, to=1-5]
	\arrow["{\alpha_j}", from=2-5, to=2-4]
\end{tikzcd}\]

\section{可表函子}

\subsection{Yoneda 引理}

假设 \( \mathcal{C} \) 是一个范畴. 定义

\[
  \mathcal{C}^{\wedge} := \operatorname{Fct}(\mathcal{C}^{\operatorname{op}},
  \operatorname{Set}),\quad \mathcal{C}^{\vee} :=
  \operatorname{Fct}(\mathcal{C}^{\operatorname{op}},
  \operatorname{Set}^{\operatorname{op}}) = \operatorname{Fct}(\mathcal{C},
  \operatorname{Set})^{\operatorname{op}}.
\]
基于一些几何学的渊源, 也把 \( \mathcal{C}^{\wedge} \) 称为 \( \mathcal{C} \) 的
\emph{预层} 范畴. \cref{proposition-opposite-functor-category} 告诉我们
\[
  (\mathcal{C}^{\vee})^{\operatorname{op}} =
  (\mathcal{C}^{\operatorname{op}})^{\wedge}.
\]
定义函子
\begin{align*}
  h_{\mathcal{C}}: \mathcal{C} &\to \mathcal{C}^{\wedge}\\
  S &\mapsto \operatorname{Hom}_{\mathcal{C}}(\cdot, S).
\end{align*}
其中这里的 \( \operatorname{Hom}_{\mathcal{C}} \) 是我们先前定义的 \(
\operatorname{Hom} \) 函子. 我们有自然的求值函子 \( \operatorname{ev}^{\wedge}:
\mathcal{C}^{\operatorname{op}} \times \mathcal{C}^{\wedge} \to
\operatorname{Set} \), 其将 \( (S, A) \) 映至集合 \( A(S) \). 同理定义函子
\begin{align*}
  k_{\mathcal{C}}: \mathcal{C} &\to \mathcal{C}^{\vee}, &\operatorname{ev}^{\vee}:
  (\mathcal{C}^{\vee})^{\operatorname{op}} \times \mathcal{C} &\to
  \operatorname{Set}\\
  S &\mapsto \operatorname{Hom}_{\mathcal{C}}(S, \cdot) & (B, S) &\mapsto B(S).
\end{align*}
我们称 \( h_{\mathcal{C}}, k_{\mathcal{C}} \) 为 \emph{Yoneda 嵌入}.

\begin{theorem}[Yoneda引理]
  \label{theorem-Yoneda-lemma}
  对于 \( S \in \operatorname{Ob}(\mathcal{C}) \) 和 \( A \in
  \operatorname{Ob}(\mathcal{C}^{\wedge}) \), 映射
  \begin{align*}
    \operatorname{Hom}_{\mathcal{C}^{\wedge}}(h_{\mathcal{C}}(S), A) &\to A(S)\\
    \left[ \operatorname{Hom}_{\mathcal{C}}(\cdot, S) \xrightarrow{\phi}
  A(\cdot) \right] &\mapsto \phi_S (\operatorname{id}_S)
  \end{align*}
  是双射; 它给出函子的同构 \(
  \operatorname{Hom}_{\mathcal{C}^{\wedge}}(h_{\mathcal{C}}(\cdot), \cdot)
  \xrightarrow{\sim} \operatorname{ev}^{\wedge} \). 函子 \( h_{\mathcal{C}} \)
  是全忠实的.
  同理, 存在自然的函子同构 \( \operatorname{Hom}_{\mathcal{C}^{\vee}}(\cdot,
  k_{\mathcal{C}}(\cdot)) \xrightarrow{\sim} \operatorname{ev}^{\vee} \). 函子
  \( k_{\mathcal{C}} \) 是全忠实的.
\end{theorem}
\begin{proof}
  \( \phi \mapsto \phi_S(\operatorname{id}_S) \) 如图
% https://q.uiver.app/#q=WzAsNCxbMCwxLCJcXG1hdGhjYWx7Q31ee1xcb3BlcmF0b3JuYW1le29wfX0iXSxbMSwwLCJcXG9wZXJhdG9ybmFtZXtTZXR9Il0sWzIsMSwiXFxwaGlfUyhcXG9wZXJhdG9ybmFtZXtpZH1fUykgXFxpbiBBKFMpIl0sWzIsMCwiXFxvcGVyYXRvcm5hbWV7aWR9X1MgXFxpbiBoX3tcXG1hdGhjYWx7Q319KFMpKFMpID0gXFxvcGVyYXRvcm5hbWV7SG9tfV97XFxtYXRoY2Fse0N9fShTLCBTKSJdLFswLDEsImhfe1xcbWF0aGNhbHtDfX0oUykiLDAseyJjdXJ2ZSI6LTV9XSxbMCwxLCJBIiwyLHsiY3VydmUiOjV9XSxbMywyLCIiLDAseyJzdHlsZSI6eyJ0YWlsIjp7Im5hbWUiOiJtYXBzIHRvIn19fV0sWzQsNSwiXFxwaGkiLDEseyJzaG9ydGVuIjp7InNvdXJjZSI6MjAsInRhcmdldCI6MjB9fV0sWzcsMiwiIiwwLHsic2hvcnRlbiI6eyJzb3VyY2UiOjIwfSwibGV2ZWwiOjEsInN0eWxlIjp7InRhaWwiOnsibmFtZSI6Im1hcHMgdG8ifX19XV0=
\[\begin{tikzcd}
	& {\operatorname{Set}} & {\operatorname{id}_S \in h_{\mathcal{C}}(S)(S) = \operatorname{Hom}_{\mathcal{C}}(S, S)} \\
	{\mathcal{C}^{\operatorname{op}}} && {\phi_S(\operatorname{id}_S) \in A(S)}
	\arrow[maps to, from=1-3, to=2-3]
	\arrow[""{name=0, anchor=center, inner sep=0}, "{h_{\mathcal{C}}(S)}", curve={height=-30pt}, from=2-1, to=1-2]
	\arrow[""{name=1, anchor=center, inner sep=0}, "A"', curve={height=30pt}, from=2-1, to=1-2]
	\arrow[""{name=2, anchor=center, inner sep=0}, "\phi"{description}, shorten <=9pt, shorten >=9pt, Rightarrow, from=0, to=1]
	\arrow[shorten <=13pt, maps to, from=2, to=2-3]
\end{tikzcd}\]
对 \( \mathcal{C} \) 中任一态射 \( f: T \to S \), 我们由下面交换图
% https://q.uiver.app/#q=WzAsOCxbMCwwLCJcXG9wZXJhdG9ybmFtZXtIb219X3tcXG1hdGhjYWx7Q319KFMsIFMpIl0sWzAsMywiXFxvcGVyYXRvcm5hbWV7SG9tfV97XFxtYXRoY2Fse0N9fShULCBTKSJdLFszLDMsIkEoVCkiXSxbMywwLCJBKFMpIl0sWzEsMSwiXFxvcGVyYXRvcm5hbWV7aWR9X1MiXSxbMSwyLCJmIl0sWzIsMSwiXFxwaGlfUyhcXG9wZXJhdG9ybmFtZXtpZH1fUykiXSxbMiwyLCJcXHBoaV9UKGYpID0gQShmKShcXHBoaV9TKFxcb3BlcmF0b3JuYW1le2lkfV9TKSkiXSxbMCwxLCJmXioiLDJdLFszLDIsIkEoZikiXSxbMSwyLCJcXHBoaV9UIiwyXSxbMCwzLCJcXHBoaV9TIl0sWzQsNSwiIiwxLHsic3R5bGUiOnsidGFpbCI6eyJuYW1lIjoibWFwcyB0byJ9fX1dLFs2LDcsIiIsMSx7InN0eWxlIjp7InRhaWwiOnsibmFtZSI6Im1hcHMgdG8ifX19XSxbNCw2LCIiLDEseyJzdHlsZSI6eyJ0YWlsIjp7Im5hbWUiOiJtYXBzIHRvIn19fV0sWzUsNywiIiwxLHsic3R5bGUiOnsidGFpbCI6eyJuYW1lIjoibWFwcyB0byJ9fX1dXQ==
\[\begin{tikzcd}
	{\operatorname{Hom}_{\mathcal{C}}(S, S)} &&& {A(S)} \\
	& {\operatorname{id}_S} & {\phi_S(\operatorname{id}_S)} \\
	& f & {\phi_T(f) = A(f)(\phi_S(\operatorname{id}_S))} \\
	{\operatorname{Hom}_{\mathcal{C}}(T, S)} &&& {A(T)}
	\arrow["{\phi_S}", from=1-1, to=1-4]
	\arrow["{f^*}"', from=1-1, to=4-1]
	\arrow["{A(f)}", from=1-4, to=4-4]
	\arrow[maps to, from=2-2, to=2-3]
	\arrow[maps to, from=2-2, to=3-2]
	\arrow[maps to, from=2-3, to=3-3]
	\arrow[maps to, from=3-2, to=3-3]
	\arrow["{\phi_T}"', from=4-1, to=4-4]
\end{tikzcd}\]
这说明 \( \phi \) 完全由每个 \( \phi_S(\operatorname{id}_S) \) 决定,
因此映射的单满性显然.

第二个论断是第一个论断的推论, 见下图
% https://q.uiver.app/#q=WzAsNixbMCwxLCJcXG1hdGhjYWx7Q31ee1xcb3BlcmF0b3JuYW1le29wfX0gXFx0aW1lc1xcbWF0aGNhbHtDfV57XFx3ZWRnZX0iXSxbMiwxLCJcXG9wZXJhdG9ybmFtZXtTZXR9Il0sWzMsMiwiXFxvcGVyYXRvcm5hbWV7SG9tfV97XFxtYXRoY2Fse0N9fShoX3tcXG1hdGhjYWx7Q319KFQpLCBBKFQpKSJdLFs1LDIsIkEoVCkiXSxbMywwLCJcXG9wZXJhdG9ybmFtZXtIb219X3tcXG1hdGhjYWx7Q319KGhfe1xcbWF0aGNhbHtDfX0oUyksIEEoUykpIl0sWzUsMCwiQShTKSJdLFswLDEsIlxcb3BlcmF0b3JuYW1le0hvbX1fe1xcbWF0aGNhbHtDfX0oaF97XFxtYXRoY2Fse0N9fShcXGNkb3QpLCBcXGNkb3QpIiwwLHsiY3VydmUiOi0zfV0sWzAsMSwiXFxvcGVyYXRvcm5hbWV7ZXZ9XntcXHdlZGdlfSIsMix7ImN1cnZlIjozfV0sWzIsMywiIiwwLHsic3R5bGUiOnsidGFpbCI6eyJuYW1lIjoiYXJyb3doZWFkIn19fV0sWzQsMl0sWzUsM10sWzQsNSwiIiwyLHsic3R5bGUiOnsidGFpbCI6eyJuYW1lIjoiYXJyb3doZWFkIn19fV0sWzYsNywiIiwwLHsic2hvcnRlbiI6eyJzb3VyY2UiOjIwLCJ0YXJnZXQiOjIwfSwic3R5bGUiOnsidGFpbCI6eyJuYW1lIjoiYXJyb3doZWFkIn19fV1d
\[\begin{tikzcd}
	&&& {\operatorname{Hom}_{\mathcal{C}}(h_{\mathcal{C}}(S), A(S))} && {A(S)} \\
	{\mathcal{C}^{\operatorname{op}} \times\mathcal{C}^{\wedge}} && {\operatorname{Set}} \\
	&&& {\operatorname{Hom}_{\mathcal{C}}(h_{\mathcal{C}}(T), A(T))} && {A(T)}
	\arrow[tail reversed, from=1-4, to=1-6]
	\arrow[from=1-4, to=3-4]
	\arrow[from=1-6, to=3-6]
	\arrow[""{name=0, anchor=center, inner sep=0}, "{\operatorname{Hom}_{\mathcal{C}}(h_{\mathcal{C}}(\cdot), \cdot)}", curve={height=-18pt}, from=2-1, to=2-3]
	\arrow[""{name=1, anchor=center, inner sep=0}, "{\operatorname{ev}^{\wedge}}"', curve={height=18pt}, from=2-1, to=2-3]
	\arrow[tail reversed, from=3-4, to=3-6]
	\arrow[shorten <=5pt, shorten >=5pt, Rightarrow, 2tail reversed, from=0, to=1]
\end{tikzcd}\]

关于全忠实性, 取 \( A = h_{\mathcal{C}}(T) \), 给定 \( f: T \to S \). 由前面所述
\( h_{\mathcal{C}}(f) \) 诱导的态射(即自然变换) \(
\operatorname{Hom}_{\mathcal{C}}(\cdot, S) \to
\operatorname{Hom}_{\mathcal{C}}(\cdot, T) \) 由 \( \operatorname{id}_S \mapsto
\phi_S(\operatorname{id}_S)\) 决定, 而后者正是 \( f \in
\operatorname{Hom}_{\mathcal{C}}(T, S) \).
% https://q.uiver.app/#q=WzAsNyxbMCwwLCJTIl0sWzEsMCwiVCJdLFsyLDAsIlxcbWF0aG9we1xcbGVhZHN0b31cXGxpbWl0c157aF97XFxtYXRoY2Fse0N9fX0iXSxbMywwLCJoX3tcXG1hdGhjYWx7Q319KFMpPVxcb3BlcmF0b3JuYW1le0hvbX1fe1xcbWF0aGNhbHtDfX0oXFxjZG90LCBTKSJdLFs0LDAsImhfe1xcbWF0aGNhbHtDfX0oVCk9XFxvcGVyYXRvcm5hbWV7SG9tfV97XFxtYXRoY2Fse0N9fShcXGNkb3QsIFQpIl0sWzMsMSwiXFxvcGVyYXRvcm5hbWV7SG9tfV97XFxtYXRoY2Fse0N9fShTLCBTKVxcbmlcXG9wZXJhdG9ybmFtZXtpZH1fUyJdLFs0LDEsImYgPSBcXHBoaV9TKFxcb3BlcmF0b3JuYW1le2lkfV9TKVxcaW4gXFxvcGVyYXRvcm5hbWV7SG9tfV97XFxtYXRoY2Fse0N9fShTLCBUKSJdLFsxLDAsImYiLDJdLFszLDRdLFs1LDYsIiIsMix7InN0eWxlIjp7InRhaWwiOnsibmFtZSI6Im1hcHMgdG8ifX19XSxbOSw4LCIiLDIseyJzaG9ydGVuIjp7InNvdXJjZSI6MjAsInRhcmdldCI6MjB9fV1d
\[\begin{tikzcd}
	S & T & {\mathop{\leadsto}\limits^{h_{\mathcal{C}}}} & {h_{\mathcal{C}}(S)=\operatorname{Hom}_{\mathcal{C}}(\cdot, S)} & {h_{\mathcal{C}}(T)=\operatorname{Hom}_{\mathcal{C}}(\cdot, T)} \\
	&&& {\operatorname{Hom}_{\mathcal{C}}(S, S)\ni\operatorname{id}_S} & {f = \phi_S(\operatorname{id}_S)\in \operatorname{Hom}_{\mathcal{C}}(S, T)}
	\arrow["f"', from=1-2, to=1-1]
	\arrow[""{name=0, anchor=center, inner sep=0}, from=1-4, to=1-5]
	\arrow[""{name=1, anchor=center, inner sep=0}, maps to, from=2-4, to=2-5]
	\arrow[shorten <=4pt, shorten >=4pt, Rightarrow, from=1, to=0]
\end{tikzcd}\]
\end{proof}

我们将经常忽略符号 \( h_{\mathcal{C}} \) 或 \( k_{\mathcal{C}} \), 将 \(
\mathcal{C} \) 看作 \( \mathcal{C}^{\wedge} \) 或 \( \mathcal{C}^{\vee} \)
的全子范畴. %TODO: 为什么能看成全子范畴

\begin{definition}
  \begin{enumerate}
    \item 称函子 \( A: \mathcal{C}^{\operatorname{op}} \to \operatorname{Set} \)
      是\emph{可表函子}, 如果存在 \( X \in \operatorname{Ob}(\mathcal{C}) \)
      以及同构 \( \phi: h_{\mathcal{C}}(X) \xrightarrow{\sim} A \), 并称 \( (X,
      \phi) \) 是其代表元.
    \item 类似地, 称函子 \( B: \mathcal{C} \to \operatorname{Set}
      \)是\emph{可表函子}, 如果存在 \( X \in \operatorname{Ob}(\mathcal{C}) \)
      以及同构 \( \phi: k_{\mathcal{C}}(X) \xrightarrow{\sim} B \), 并称 \( (X,
      \phi) \) 是其代表元.
  \end{enumerate}
\end{definition}

\begin{lemma}
  如果函子 \( A: \mathcal{C}^{\operatorname{op}} \to \operatorname{Set} \) 可表,
  则其代表元 \( (X, \phi: h_{\mathcal{C}}(X) \xrightarrow{\sim} A) \)
  在至多差一个唯一同构的意义下是唯一的. 对函子 \( B: \mathcal{C} \to
  \operatorname{Set} \) 也有类似的性质.
\end{lemma}
\begin{proof}
  由 \cref{theorem-Yoneda-lemma}, 对于任意 \( Y \in
  \operatorname{Ob}(\mathcal{C}) \) 与 \( \psi: h_{\mathcal{C}}(Y) \to A \),
  存在唯一的 \( f \in \operatorname{Hom}_{\mathcal{C}}(Y, X) \) 使得下图交换
% https://q.uiver.app/#q=WzAsMyxbMCwwLCJoX3tcXG1hdGhjYWx7Q319KFkpIl0sWzIsMCwiaF97XFxtYXRoY2Fse0N9fShYKSJdLFsxLDEsIkEiXSxbMCwyLCJcXHBzaSIsMl0sWzEsMiwiXFxwaGksIFxcc2ltZXEiXSxbMCwxLCJoX3tcXG1hdGhjYWx7Q319KGYpIl1d
\[\begin{tikzcd}
	{h_{\mathcal{C}}(Y)} && {h_{\mathcal{C}}(X)} \\
	& A
	\arrow["{h_{\mathcal{C}}(f)}", from=1-1, to=1-3]
	\arrow["\psi"', from=1-1, to=2-2]
	\arrow["{\phi, \simeq}", from=1-3, to=2-2]
\end{tikzcd}\]
(见 \cref{example-comma-category})考察 \( \mathcal{C}
\xrightarrow{h_{\mathcal{C}}} \mathcal{C}^{\wedge} \xleftarrow{j_A} \mathbf{1}
\) 对应的逗号范畴 \( (h_{\mathcal{C}} / A) \). \( (X, \phi) \) 是其终对象,
故唯一.
\end{proof}

\section{伴随函子}

\subsection{伴随对}

\begin{definition}
  \emph{伴随对} 意指下面资料 \( (F, G, \varphi) \), 其中 \( % https://q.uiver.app/#q=WzAsMixbMCwwLCJcXG1hdGhjYWx7Q31fMSJdLFsxLDAsIlxcbWF0aGNhbHtDfV8yIl0sWzEsMCwiRyIsMCx7ImN1cnZlIjotMn1dLFswLDEsIkYiLDAseyJjdXJ2ZSI6LTJ9XV0=
\begin{tikzcd}
	{\mathcal{C}_1} & {\mathcal{C}_2}
	\arrow["F", curve={height=-12pt}, from=1-1, to=1-2]
	\arrow["G", curve={height=-12pt}, from=1-2, to=1-1]
\end{tikzcd} \) 是一对函子, 而 \( \varphi \) 是函子的同构
\[
  \varphi: \operatorname{Hom}_{\mathcal{C}_2}(F(\cdot), \cdot)
  \xrightarrow{\sim} \operatorname{Hom}_{\mathcal{C}_1}(\cdot, G(\cdot)).
\]
一般称 \( G \) 是 \( F \) 的右伴随, \( F \) 是左伴随.
\end{definition}

\begin{example}
  考察对偶函子 \( D: \operatorname{Vect}(\Bbbk)^{\operatorname{op}} \to
  \operatorname{Vect}(\Bbbk), DV := V^{\vee} \), 存在自然的同构
  \begin{align*}
    \varphi_{V, W}: \operatorname{Hom}_{\Bbbk}(V, W^{\vee}) &\to
    \operatorname{Hom}_{\Bbbk}(W, V^{\vee})\\ f &\mapsto (w \mapsto (v \mapsto
    f(v)(w))
  \end{align*}
  其中 \( V, W \) 是 \( \Bbbk \)-向量空间. 这是因为, 两边都自然地同构于双线性型 \(
  B: V \times W \to \Bbbk \) 构成的空间. 我们可以将其重写为
  \[
    \varphi_{VW}: \operatorname{Hom}_{\operatorname{Vect}(\Bbbk)}(V, DW)
    \xrightarrow{\sim}
    \operatorname{Hom}_{\operatorname{Vect}(\Bbbk)^{\operatorname{op}}}
    (D^{\operatorname{op}}V, W)
  \]
  并且对变元 \( V, W \) 满足函子性, 因此得到了伴随对 \( (D^{\operatorname{op}},
  D, \varphi^{-1}) \).
\end{example}

\subsection{单位与余单位}

设 \( (F, G, \varphi) \) 为伴随对, 定义态射
\[
  \eta = (\eta_X)_{X \in \operatorname{Ob}(\mathcal{C}_1)}:
  \operatorname{id}_{\mathcal{C}_1} \to GF
\]
如
\begin{align*}
  \varphi: \operatorname{Hom}_{\mathcal{C}_2}(FX, FX) &\to
  \operatorname{Hom}_{\mathcal{C}_1}(X, GFX)\\
  \operatorname{id}_{FX} &\mapsto \eta_X.
\end{align*}
同理, 定义态射 \( \varepsilon: (\varepsilon_X)_X: FG \to
\operatorname{id}_{\mathcal{C}_2} \) 如
\begin{align*}
  \varphi^{-1}: \operatorname{Hom}_{\mathcal{C}_1}(GX, GX) &\to
  \operatorname{Hom}_{\mathcal{C}_2}(FGX, X) \\ \operatorname{id}_{GX}
  &\mapsto \varepsilon_X
\end{align*}
我们下面说明 \( \eta \) 确实构成了自然变换. 首先, 给定 \( \mathcal{C}_1 \)
中的态射 \( h: X; \to X \), 由 \( \varphi \) 的自然性, 我们有下面交换图表
% https://q.uiver.app/#q=WzAsMTQsWzAsMCwiXFxvcGVyYXRvcm5hbWV7SG9tfShGWCwgRlgpIl0sWzAsMywiXFxvcGVyYXRvcm5hbWV7SG9tfShGWCcsIEZYKSJdLFswLDYsIlxcb3BlcmF0b3JuYW1le0hvbX0oRlgnLCBGWCcpIl0sWzMsNiwiXFxvcGVyYXRvcm5hbWV7SG9tfShYJywgR0ZYJykiXSxbMywzLCJcXG9wZXJhdG9ybmFtZXtIb219KFgnLCBHRlgpIl0sWzMsMCwiXFxvcGVyYXRvcm5hbWV7SG9tfShYLCBHRlgpIl0sWzEsMSwiXFxvcGVyYXRvcm5hbWV7aWR9X3tGWH0iXSxbMiwxLCJcXGV0YV9YIl0sWzIsMiwiXFx2YXJwaGkoRmgpIl0sWzEsMiwiRmgiXSxbMiw0LCJcXHZhcnBoaShGaCkiXSxbMSw0LCJGaCJdLFsxLDUsIlxcb3BlcmF0b3JuYW1le2lkfV97RlgnfSJdLFsyLDUsIlxcZXRhX3tYJ30iXSxbMCwxLCIoRmgpXioiLDJdLFsyLDEsIihGaClfKiJdLFsyLDMsIlxcdmFycGhpIiwyXSxbMSw0LCJcXHZhcnBoaSIsMV0sWzAsNSwiXFx2YXJwaGkiXSxbNSw0LCJoXioiXSxbMyw0LCIoR0ZoKV8qIiwyXSxbNiw3LCIiLDIseyJzdHlsZSI6eyJ0YWlsIjp7Im5hbWUiOiJtYXBzIHRvIn19fV0sWzYsOSwiIiwwLHsic3R5bGUiOnsidGFpbCI6eyJuYW1lIjoibWFwcyB0byJ9fX1dLFs5LDgsIiIsMCx7InN0eWxlIjp7InRhaWwiOnsibmFtZSI6Im1hcHMgdG8ifX19XSxbNyw4LCIiLDIseyJzdHlsZSI6eyJ0YWlsIjp7Im5hbWUiOiJtYXBzIHRvIn19fV0sWzExLDEwLCIiLDIseyJzdHlsZSI6eyJ0YWlsIjp7Im5hbWUiOiJtYXBzIHRvIn19fV0sWzEyLDExLCIiLDIseyJzdHlsZSI6eyJ0YWlsIjp7Im5hbWUiOiJtYXBzIHRvIn19fV0sWzEzLDEwLCIiLDAseyJzdHlsZSI6eyJ0YWlsIjp7Im5hbWUiOiJtYXBzIHRvIn19fV0sWzEyLDEzLCIiLDAseyJzdHlsZSI6eyJ0YWlsIjp7Im5hbWUiOiJtYXBzIHRvIn19fV1d
\[\begin{tikzcd}
	{\operatorname{Hom}(FX, FX)} &&& {\operatorname{Hom}(X, GFX)} \\
	& {\operatorname{id}_{FX}} & {\eta_X} \\
	& Fh & {\varphi(Fh)} \\
	{\operatorname{Hom}(FX', FX)} &&& {\operatorname{Hom}(X', GFX)} \\
	& Fh & {\varphi(Fh)} \\
	& {\operatorname{id}_{FX'}} & {\eta_{X'}} \\
	{\operatorname{Hom}(FX', FX')} &&& {\operatorname{Hom}(X', GFX')}
	\arrow["\varphi", from=1-1, to=1-4]
	\arrow["{(Fh)^*}"', from=1-1, to=4-1]
	\arrow["{h^*}", from=1-4, to=4-4]
	\arrow[maps to, from=2-2, to=2-3]
	\arrow[maps to, from=2-2, to=3-2]
	\arrow[maps to, from=2-3, to=3-3]
	\arrow[maps to, from=3-2, to=3-3]
	\arrow["\varphi"{description}, from=4-1, to=4-4]
	\arrow[maps to, from=5-2, to=5-3]
	\arrow[maps to, from=6-2, to=5-2]
	\arrow[maps to, from=6-2, to=6-3]
	\arrow[maps to, from=6-3, to=5-3]
	\arrow["{(Fh)_*}", from=7-1, to=4-1]
	\arrow["\varphi"', from=7-1, to=7-4]
	\arrow["{(GFh)_*}"', from=7-4, to=4-4]
\end{tikzcd}\]
因此, \( \varphi(Fh) = (GFh)_*(\eta_{X'}) = h_*(\eta_X) \), 换句话说, 下面图交换
% https://q.uiver.app/#q=WzAsNCxbMCwwLCJYJyJdLFswLDEsIlgiXSxbMSwwLCJHRlgnIl0sWzEsMSwiR0ZYIl0sWzAsMSwiaCIsMl0sWzIsMywiR0ZoIl0sWzAsMiwiXFxldGFfe1gnfSJdLFsxLDMsIlxcZXRhX1giLDJdXQ==
\[\begin{tikzcd}
	{X'} & {GFX'} \\
	X & GFX
	\arrow["{\eta_{X'}}", from=1-1, to=1-2]
	\arrow["h"', from=1-1, to=2-1]
	\arrow["GFh", from=1-2, to=2-2]
	\arrow["{\eta_X}"', from=2-1, to=2-2]
\end{tikzcd}\]
因此 \( \eta \) 是一个自然变换; 同理可以知道 \( \varepsilon \) 亦是自然变换.
我们称 \( \eta \) 是伴随对 \( (F, G, \varphi) \) 的 \emph{单位}, 而 \(
\varepsilon \) 是其 \emph{余单位}.

此外, 单位和余单位由下式确定了 \( \varphi \)
\begin{equation}
\begin{aligned}
  \varphi(f) &= Gf \circ \eta_X: X \to GY,\quad \forall f: FX \to Y,\\
  \varphi^{-1}(g) &= \varepsilon_Y \circ Fg: FX \to Y,\quad \forall g: X \to GY.
\end{aligned}
\label{equation-determine-adjoint-pair-isomorphism}
\end{equation}
% https://q.uiver.app/#q=WzAsOCxbMCwwLCJcXG9wZXJhdG9ybmFtZXtIb219KEZYLCBGWCkiXSxbMywwLCJcXG9wZXJhdG9ybmFtZXtIb219KFgsIEdGWCkiXSxbMCwzLCJcXG9wZXJhdG9ybmFtZXtIb219KEZYLCBZKSJdLFszLDMsIlxcb3BlcmF0b3JuYW1le0hvbX0oWCwgR1kpIl0sWzEsMSwiXFxvcGVyYXRvcm5hbWV7aWR9X3tGWH0iXSxbMSwyLCJmIl0sWzIsMiwiXFx2YXJwaGkoZikgPSBHZiBcXGNpcmMgXFxldGFfWCJdLFsyLDEsIlxcZXRhX1giXSxbMCwyLCJmXyoiLDJdLFswLDEsIlxcdmFycGhpIl0sWzEsMywiKEdmKV8qIl0sWzIsMywiXFx2YXJwaGkiLDJdLFs0LDUsIiIsMCx7InN0eWxlIjp7InRhaWwiOnsibmFtZSI6Im1hcHMgdG8ifX19XSxbNCw3LCIiLDIseyJzdHlsZSI6eyJ0YWlsIjp7Im5hbWUiOiJtYXBzIHRvIn19fV0sWzcsNiwiIiwyLHsic3R5bGUiOnsidGFpbCI6eyJuYW1lIjoibWFwcyB0byJ9fX1dLFs1LDYsIiIsMCx7InN0eWxlIjp7InRhaWwiOnsibmFtZSI6Im1hcHMgdG8ifX19XV0=
\[\begin{tikzcd}
	{\operatorname{Hom}(FX, FX)} &&& {\operatorname{Hom}(X, GFX)} \\
	& {\operatorname{id}_{FX}} & {\eta_X} \\
	& f & {\varphi(f) = Gf \circ \eta_X} \\
	{\operatorname{Hom}(FX, Y)} &&& {\operatorname{Hom}(X, GY)}
	\arrow["\varphi", from=1-1, to=1-4]
	\arrow["{f_*}"', from=1-1, to=4-1]
	\arrow["{(Gf)_*}", from=1-4, to=4-4]
	\arrow[maps to, from=2-2, to=2-3]
	\arrow[maps to, from=2-2, to=3-2]
	\arrow[maps to, from=2-3, to=3-3]
	\arrow[maps to, from=3-2, to=3-3]
	\arrow["\varphi"', from=4-1, to=4-4]
\end{tikzcd}\]
反转箭头就得到了 \( \varphi^{-1}(g) \) 的情形.

\begin{lemma}
  对于伴随对 \( (F, G, \varphi) \) ,相应的单位 \( \eta:
  \operatorname{id} \to GF \) 和余单位 \( \varepsilon: FG \to \operatorname{id}
  \), 我们有自然变换间的等式.
  \begin{equation}
    \begin{aligned}
      \left( G \xrightarrow{\eta G} (GF)G = G(FG) \xrightarrow{G \varepsilon} G
      \right) = \operatorname{id}_{G},\\
      \left( F \xrightarrow{F\eta} F(GF) = (FG)F \xrightarrow{\varepsilon F} G
      \right) = \operatorname{id}_{F}.
    \end{aligned}
    \label{equation-identity-and-coidentity}
  \end{equation}
\end{lemma}
\begin{proof}
  如图, 这些都是自然变换的纵合成
% https://q.uiver.app/#q=WzAsNCxbMCwwLCJcXG1hdGhjYWx7Q31fMiJdLFszLDAsIlxcbWF0aGNhbHtDfV8xIl0sWzQsMCwiXFxtYXRoY2Fse0N9XzEiXSxbNywwLCJcXG1hdGhjYWx7Q31fMiJdLFswLDEsIkciLDAseyJjdXJ2ZSI6LTR9XSxbMCwxLCJHIiwyLHsiY3VydmUiOjR9XSxbMCwxLCIoR0YpRyA9IEcoRkcpIiwxXSxbMiwzLCJGIiwwLHsiY3VydmUiOi00fV0sWzIsMywiRiIsMix7ImN1cnZlIjo0fV0sWzIsMywiRihHRik9KEZHKUYiLDFdLFs0LDYsIlxcZXRhIEciLDAseyJzaG9ydGVuIjp7InNvdXJjZSI6MjAsInRhcmdldCI6MjB9fV0sWzYsNSwiRyBcXHZhcmVwc2lsb24iLDAseyJzaG9ydGVuIjp7InNvdXJjZSI6MjAsInRhcmdldCI6MjB9fV0sWzcsOSwiRiBcXGV0YSIsMCx7InNob3J0ZW4iOnsic291cmNlIjoyMCwidGFyZ2V0IjoyMH19XSxbOSw4LCJcXHZhcmVwc2lsb24gRiIsMCx7InNob3J0ZW4iOnsic291cmNlIjoyMCwidGFyZ2V0IjoyMH19XV0=
\[\begin{tikzcd}
	{\mathcal{C}_2} &&& {\mathcal{C}_1} & {\mathcal{C}_1} &&& {\mathcal{C}_2}
	\arrow[""{name=0, anchor=center, inner sep=0}, "G", curve={height=-24pt}, from=1-1, to=1-4]
	\arrow[""{name=1, anchor=center, inner sep=0}, "G"', curve={height=24pt}, from=1-1, to=1-4]
	\arrow[""{name=2, anchor=center, inner sep=0}, "{(GF)G = G(FG)}"{description}, from=1-1, to=1-4]
	\arrow[""{name=3, anchor=center, inner sep=0}, "F", curve={height=-24pt}, from=1-5, to=1-8]
	\arrow[""{name=4, anchor=center, inner sep=0}, "F"', curve={height=24pt}, from=1-5, to=1-8]
	\arrow[""{name=5, anchor=center, inner sep=0}, "{F(GF)=(FG)F}"{description}, from=1-5, to=1-8]
	\arrow["{\eta G}", shorten <=3pt, shorten >=3pt, Rightarrow, from=0, to=2]
	\arrow["{G \varepsilon}", shorten <=3pt, shorten >=3pt, Rightarrow, from=2, to=1]
	\arrow["{F \eta}", shorten <=3pt, shorten >=3pt, Rightarrow, from=3, to=5]
	\arrow["{\varepsilon F}", shorten <=3pt, shorten >=3pt, Rightarrow, from=5, to=4]
\end{tikzcd}\]
对任意 \( Y \in \operatorname{Ob}(\mathcal{C}_2) \), 由
\eqref{equation-determine-adjoint-pair-isomorphism} 有
\[
  \operatorname{id}_{GY} = \varphi(\varphi^{-1}(\operatorname{id}_{GY})) =
  \varphi(\varepsilon_Y) = G(\varepsilon_Y) \circ \eta_{GY}: GY \to GY.
\]
\end{proof}

\begin{proposition}
  对于给定的函子 \( % https://q.uiver.app/#q=WzAsMixbMCwwLCJcXG1hdGhjYWx7Q31fMSJdLFsxLDAsIlxcbWF0aGNhbHtDfV8yIl0sWzAsMSwiRiIsMCx7ImN1cnZlIjotMn1dLFsxLDAsIkciLDAseyJjdXJ2ZSI6LTJ9XV0=
\begin{tikzcd}
	{\mathcal{C}_1} & {\mathcal{C}_2}
	\arrow["F", curve={height=-12pt}, from=1-1, to=1-2]
	\arrow["G", curve={height=-12pt}, from=1-2, to=1-1]
\end{tikzcd} \), 以下的映射互为逆
\begin{align*}
  \left\lbrace \varphi: (F, G, \varphi) \text{是伴随对} \right\rbrace
  &\leftrightarrow \left\lbrace (\eta, \varepsilon): \text{满足
  \eqref{equation-identity-and-coidentity}} \right\rbrace\\
  \varphi &\mapsto \left(\eta_X := \varphi(\operatorname{id}_{FX}), \varepsilon_Y :=
  \varphi^{-1}(\operatorname{id}_{GY})\right),\\ \varphi(f) := Gf \circ \eta_X
  &\mapsfrom (\eta, \varepsilon).
\end{align*}
因此伴随对亦可用资料 \( (F, G, \eta, \varepsilon) \) 描述, 这样不必涉及 \(
\operatorname{Hom} \) 集.
\end{proposition}


\section{Ab-范畴}
给定一个范畴 \( \mathcal{A} \).
\begin{itemize}
  \item 如果每个 \( \operatorname{Hom}_{\mathcal{A}}(A, B) \)
    都给定了加法交换群结构并且符合关于此加法是分配的, 那么范畴 \( \mathcal{A} \)
    称为 \emph{\( \operatorname{Ab} \)-范畴}. 特别地, 给定 \( \mathcal{A} \)
    中的图
    \[
      A \xrightarrow{f} B \mathop{\rightrightarrows}\limits_{g}^{g'} C
      \xrightarrow{h} D,
    \]
    有 \( h(g + g')f = hgf + hg'f \).
  \item 给定一个 \( \operatorname{Ab} \)-范畴间的函子 \( F: \mathcal{B} \to
    \mathcal{A} \). 如果 \( \operatorname{Hom}_{\mathcal{B}}(B', B) \to
    \operatorname{Hom}_{\mathcal{A}}(FB', FB) \) 是一个群同态, 那么我们称 \( F
    \) 是一个\emph{加性函子}.
  \item 如果 \( \operatorname{Ab} \)-范畴 \( \mathcal{A} \)
    有一个零对象并且对每个 \( A, B \in \operatorname{\mathcal{A}} \) 存在 \( A
    \times B \), 那么我们称 \( \mathcal{A} \) 是 \emph{加性范畴}.
  \item 一个 \emph{abelian 范畴} 是一个加性范畴 \( \mathcal{A} \) 使得
    \begin{itemize}
      \item 每个 \( \mathcal{A} \) 中的态射有一个核和余核.
      \item 每个 \( \mathcal{A} \) 中的单射是它的核的余核.
      \item 每个 \( \mathcal{A} \) 中的满射是它的余核的核.
    \end{itemize}
\end{itemize}

给设给定了 abelian 范畴中的一个态射序列 \( A \xrightarrow{f} B \xrightarrow{g} C
\). 如果 \( \operatorname{ker} g = \operatorname{im} f \), 那么该序列称为是
\emph{正合} 的.
