\section{Countability and Separation Axioms}

\subsection{The Countability Axioms}

\begin{definition}
  A space \( X \) is said to have a \emph{countable basis at} \( x \) if there is a countable collection \( \mathcal{B} \) of neighborhoods of \( x \) such that each neighborhood of \( x \) contains at least one of the elements of \( \mathcal{B} \).
  A space that has a countable basis at each of its points is said to satisfy the \emph{first countability axim}, or to be \emph{first-countable}.
\end{definition}

\begin{theorem}
  Let \( X \) be a topological space.
  \begin{enumerate}
    \item Let \( A \) be a subset of \( X \).
      If there is a sequence of points of \( A \) converging to \( x \), then \( x \in \overline{A} \);
      the converse holds if \( X \) is first-countable.
    \item Let \( f: X \to Y \).
      If \( f \) is continuous, then for every convergent sequence \( x_n \to x \) in \( X \), the sequence \( f(x_n) \) converges to \( f(x) \).
      The converse holds if \( X \) is first-countable.
  \end{enumerate}
\end{theorem}
\begin{sketchproof}
  (2) Let \( A \) be a closed set in \( Y \), and let \( B = f^{-1}(A) \).
  We want to show that \( B \) is closed.
  Let \( x \in \overline{B} \), then there exists a sequence \( x_n \) in \( A \) converging to \( x \).
  Then \( f(x_n) \) converges to \( f(x) \) by condition, hence \( f(x) \in f(B) = ff^{-1}(A) \subseteq A \), which implies that \( x \in B \).
\end{sketchproof}

\begin{definition}
  If a space \( X \) has a countable basis for its topology, then \( X \) is said to satisfy the \emph{second countability axiom}, or to be \emph{second-countable}.
\end{definition}

\begin{theorem}
  A subspace of a first-countable space is first-countable, and a countable product of first-countable spaces is first-countable.
  A subspace of a second-countable space is second-countable, and a countable product of second-countable spaces is second-countable.
\end{theorem}

\begin{definition}
  A subset \( A \) of space \( X \) is said to be \emph{dense} in \( X \) if \( \overline{A} = X \).
\end{definition}

\begin{theorem}
  Suppose that \( X \) has a countable basis.
  Then
  \begin{enumerate}
    \item Every open covering of \( X \) contains a countable subcollection covering \( X \).
    \item There exists a countable subset of \( X \) that is dense in \( X \).
  \end{enumerate}
\end{theorem}

\begin{definition}
  A space for which every open covering contains a countable subcovering is called a \emph{Lindel\"{o}f space}.
  A space having a countable dense subset is often said to be \emph{separable}.
\end{definition}

\subsection{The Separation Axioms}

\paragraph{Concepts and basis properties}

\begin{definition}
  Let \( X \) be a topological space.
  We say that
  \begin{enumerate}
    \item \( X \) is \( T_0 \), or \emph{Kolmogorov}, if any two distinct points \( a, b \) in \( X \) are topological distinguishable, i.e. there exists a open set contains \( a \) but not \( b \) (or contains \( b \) but not \( a \)).
    \item \( X \) is \( T_1 \), or \emph{Fr\'{e}chet}, if there exists open sets \( U, V \subset X \) such that $x_1 \in U \backslash V \textmd{ and } x_2 \in V \backslash U$.
    \item \( X \) is \( T_2 \), or \emph{Hausdorff}, if for any \( x_1 \neq x_2 \in X \), there exists open set \( U_1 \ni x_1 \) and \( U_2 \ni x_2 \) such that \( U_1 \cap U_2 = \varnothing \).
    \item \( X \) is \emph{regular}, if for each pair consisting of a point \( x \) and a closed set \( B \) disjoint from \( x \), there exist disjoint open sets containing \( x \) and \( B \), respectively.
    \item \( X \) is \emph{normal}, if for each pair \( A, B \) of disjoint closed sets of \( X \), there exist disjoint open sets containing \( A \) and \( B \), respectively.
  \end{enumerate}
\end{definition}

\begin{lemma}
  Let \( X \) be a topological space.
  \begin{enumerate}
    \item \( X \) is regular \( \iff \) given a point \( x \) of \( X \) and a neighborhood \( U \) of \( x \), there is a neighborhood \( V \) of \( x \) such that \( \overline{V} \subseteq U \).
    \item \( X \) is normal \( \iff \) given a closed set \( A \) and an open set \( U \) containing \( A \), there is an open set \( V \) containing \( A \) such that \( \overline{V} \subseteq U \).
  \end{enumerate}
\end{lemma}
\begin{sketchproof}
  (1) \( \implies \) Suppose that \( X \) is regular, and suppose that the point \( x \) and the neighborhood \( U \) of \( x \) are given.
  Let \( B = X - U \), then \( B \) is a closed set.
  By hypothesis, there exist disjoint open sets \( V \) and \( W \) containing \( x \) and \( B \), respectively.
  The set \( \overline{V} \) is disjoint from \( B \), since \( y \in B \), the set \( W \) is a neighborhood of \( y \) disjoint from \( V \).
  Therefore, \( \overline{V} \subseteq U \), as desired.
\end{sketchproof}

\begin{theorem}
  \begin{enumerate}
    \item A subspace of a Hausdorff space is Hausdorff;
      a product of Hausdorff spaces is Hausdorff.
    \item A subspace of a regular space is regular;
      a product of regular spaces is regular.
  \end{enumerate}
\end{theorem}

\paragraph{Normal spaces}

\begin{theorem}
  Every regular space with a countable basis is normal.
\end{theorem}
\begin{proof}
  Let \( A \) and \( B \) be disjoint closed subsets of \( X \).
  Each point \( x \) of \( A \) has neighborhood \( U \) not intersecting \( B \).
  Using regularity, choose a neighborhood \( V \) of \( x \) whose closure lies in \( U \);
  finally, choose an element of \( \mathcal{B} \) containing \( x \) and contained in \( V \).
  By choosing such a basis element for each \( x \) in \( A \), we construct a countable covering of \( A \) by open sets whose closures do not intersect \( B \).
  Since this covering of \( A \) is countable, we can index it with the positive integers; let us denote it by \( \left\lbrace U_n \right\rbrace \).

  Similarly, choose a countable collection \( \left\lbrace V_n \right\rbrace \) of open sets covering \( B \), such that each set \( \overline{V}_n \) is disjoint from \( A \).
  The sets \( U = \bigcup U_n \) and \( V = \bigcup V_n \) are open sets containing \( A \) and \( B \), respectively, but they need not be disjoint.
  \underline{We perform the following simple trick to construct two open sets that are disjoint.}
  Given \( n \), define
  \[
    U'_n = U_n - \bigcup_{i = 1}^n \overline{V}_i \text{ and } V'_n = V_n - \bigcup_{i = 1}^n \overline{U}_i.
  \]
  Note that each set \( U'_n \) is open, being the difference of an open set \( U_n \) and a closed set \( \bigcup_{i=1}^n \overline{V}_i \).
  Similarly, each set \( V'_n \) is open.
  The collection \( \left\lbrace U'_n \right\rbrace \) covers \( A \), because each \( x \) in \( A \) belongs to \( U_n \) for some \( n \), and \( x \) belongs to none of the sets \( \overline{V}_i \).
  Similarly, the collection \( \left\lbrace V'_n \right\rbrace \) covers \( B \).
  Finally, the open sets
  \[
    U' = \bigcup_{n \in \mathbb{Z}_+}U'_n \text{ and } V' = \bigcup_{n \in \mathbb{Z}_+} V'_n
  \]
  are disjoint.
  For if \( x \in U' \cap V' \), then \( x \in U'_j \cap V'_k \) for some \( j \) and \( k \).
  Suppose that \( j \leq k \).
  It follows from the definition of \( U'_j \) that \( x \in U_j \);
  and since \( j \leq k \) it follows from the definition of \( V'_k \) that \( x \notin \overline{U}_j \).
  A similar contradiction arises if \( j \geq k \).
\end{proof}

\begin{theorem}
  Every metrizable space is normal.
\end{theorem}

\begin{theorem}
  Every compact Hausdorff space is normal.
\end{theorem}

\paragraph{Well-ordered set}

\begin{theorem}
  Every well-ordered set \( X \) is normal in the order topology.
\end{theorem}
\begin{proof}
  TODO %TODO
\end{proof}

\subsection{Normal Spaces}

\paragraph{Urysohn Lemma}

\begin{theorem}[Urysohn Lemma]
  Let \( X \) be a normal space;
  let \( A \) and \( B \) be disjoint closed subsets of \( X \).
  Let \( [a, b] \) be a closed inteval in the real line.
  Then there exists a continuous map
  \[
    f: X \to [a, b]
  \]
  such that \( f(x) = a \) for every \( x \) in \( A \), and \( f(x) = b \) for every \( x \) in \( B \).
\end{theorem}
\begin{proof}
  \begin{enumerate}
    \item Let \( P \) be the set all rational numbers in the inteval \( [0, 1] \).
      We shall define, for each \( p \) in \( P \), an open set \( U_p \) of \( X \), in such a way that whenever \( p < q \), we have
      \[
        \overline{U}_p \subseteq U_q.
      \]
      Thus, the sets \( U_p \) will simply ordered by inclusion in the same way their subscripts are ordered by the usual ordered in the real line.

      Now define the sets \( U_p \) as follows: First, define \( U_1 = X - B \).
      Second because \( A \) is a closed set contained in the open set \( U_1 \), we may by normality of \( X \) choose an open set \( U_0 \) such that
      \[
        A \subseteq U_0 \text{ and } \overline{U}_0 \subseteq \overline{U}_1.
      \]
      In general, let \( P_n \) denote the set consisting of the first \( n \) rational numbers in the sequence.
      Suppose that \( U_p \) is defined for all rational numbers \( p \) belonging to the set \( P_n \), satisfying the condition
      \begin{equation}
        p < q \implies \overline{U}_p \subseteq U_q \tag{*}\label{eq: Urysohn lemma construction}
      \end{equation}
      Let \( r \) denote the next rational number in the sequence; we wish to define \( U_r \).

      Consider the set \( P_{n+1} = P_n \cup \left\lbrace r \right\rbrace \).
      It is a finite subset of the interval \( [0, 1] \), and as such, it has a simple ordering derived from the usual order relation \( < \) on the real line.
      In a finite simply ordered set, every element has an immediate predecessor and an immediate successor.
      The number \( 0 \) is the smallest element, and \( 1 \) is the largest element, of the simply ordered set \( P_{n + 1} \), and \( r \) is neither \( 0 \) nor \( 1 \).
      So \( r \) has an immediate predecessor \( p \) in \( P_{n + 1} \) and an immediate successor \( q \) in \( P_{n + 1} \).
      The sets \( U_p \) and \( U_q \) are already defined, and \( \overline{U}_p \subseteq U_q \) by the induction hypothesis.
      Using normality of \( X \), we can find an open set \( U_r \) of \( X \) such that
      \[
        \overline{U}_P \subseteq U_r \text{ and } \overline{U}_r \subseteq U_q.
      \]
      We assert that \eqref{eq: Urysohn lemma construction} now holds for every pair of elements of \( P_{n + 1} \).
      If both elements lie in \( P_n \), \eqref{eq: Urysohn lemma construction} holds by the induction hypothesis.
      If one of them is \( r \) and the other is a point \( s \) of \( P_n \), then either \( s \leq p \), in which case
      \[
        \overline{U}_s \subseteq \overline{U}_p \subseteq U_r,
      \]
      or \( s \geq q \), in which case
      \[
        \overline{U}_r \subseteq U_q \subseteq U_s.
      \]
      Thus, for every pair of elements of \( P_{n + 1} \) relation \eqref{eq: Urysohn lemma construction} holds.
    \item Now we have defined \( U_p \) for all rational numbers \( p \) in the interval \( [0, 1] \).
      We extend this definition to all rational numbers \( p \) in \( \mathbb{R} \) by defining
      \begin{align*}
        &U_p = \emptyset & \text{ if } p < 0,\\
        &U_p = X & \text{ if } p > 1
      .\end{align*}
      It is still true that for any pair for rational numbers \( p \) and \( q \),
      \[
        p < q \implies \overline{U}_p \subseteq U_q.
      \]
    \item Given a point \( x \) of \( X \), let us define \( \mathbb{Q}(x) \) to be the set of those rational numbers \( p \) such that the corresponding open sets \( U_p \) contain \( x \):
      \[
        \mathbb{Q}(x) = \left\lbrace p: x \in U_p \right\rbrace.
      \]
      This set contains no number less than \( 0 \), since no \( x \) is in \( U_p \) for \( p < 0 \).
      And it contains every number greater than \( 1 \), since every \( x \) is in \( U_p \) for \( p > 1 \).
      Therefore, \( \mathbb{Q}(x) \) is bounded below, and its greatest lower bound is a point of the interval \( [0, 1] \).
      Define
      \[
        f(x) = \inf \mathbb{Q}(x) = \inf \left\lbrace p: x \in U_p \right\rbrace.
      \]
    \item We show that \( f \) is the desired function.
      If \( x \in A \), then \( x \in U_p \) for every \( p \geq 0 \), so that \( \mathbb{Q}(x) \) equals the set of all nonnegative rationals, and \( f(x) = \inf \mathbb{Q}(x) = 0 \).
      Similarly, if \( x \in B \), then \( x \in U_p \) for no \( p \leq 1 \), so that \( \mathbb{Q}(x) \) consists of all rational numbers greater than \( 1 \), and \( f(x) = 1 \).
      
      To show that \( f \) is continuous, we first observe the following elementary facts:
      \begin{enumerate}
        \item \( x \in \overline{U}_r \implies f(x) \leq r \): if \( x \in \overline{U}_r \), then \( x \in U_s \) for every \( s > r \).
        \item \( x \notin U_r \implies f(x) \geq r \): if \( x \notin U_r \), then \( x \) is not in \( U_s \) for any \( s < r \).
      \end{enumerate}
      Now we prove the continuity of \( f \).
      Given a point \( x_0 \) of \( X \) and an open interval \( (c, d) \) in \( \mathbb{R} \) containing the point \( f(x_0) \), we wish to find a neighborhood \( U \) of \( x_0 \) such that \( f(U) \subseteq (c, d) \).
      Choose rational numbers \( p \) and \( q \) such that
      \[
        c < p < f(x_0) < q < d
      \]
      We assert that the open set
      \[
        U = U_q - \overline{U}_p
      \]
      is the desired neighborhood of \( x_0 \).
      \begin{itemize}
        \item We note that \( x_0 \in U \).
          For the fact that \( f(x_0) < q \) implies \( x_0 \in U_q \), while the fact that \( f(x_0) > p \) implies that \( x_0 \notin \overline{U}_p \).
        \item We show that \( f(U) \subseteq (c, d) \).
          Let \( x \in U \).
          Then \( x \in U_q \subseteq \overline{U}_q \), so that \( f(x) \leq q \).
          And \( x \notin \overline{U}_p \), so that \( x \notin U_p \) and \( f(x) \geq p \).
          Thus, \( f(x) \in [p, q] \subseteq (c, d) \), as desired.
      \end{itemize}
  \end{enumerate}
\end{proof}

\begin{definition}
  If \( A \) and \( B \) are two subsets of the topological space \( X \), and if there is a continuous function \( f: X \to [0, 1] \) such that \( f(A) = \left\lbrace 0 \right\rbrace \) and \( f(B) = \left\lbrace 1 \right\rbrace \), we say that \( A \) and \( B \) \emph{can be separated by a continuous function}.
\end{definition}

\begin{definition}
  A space \( X \) is \emph{completely regular} if for each point \( x_0 \) and each closed set \( A \) not containing \( x_0 \), there is a continuous function \( f: X \to [0, 1] \) such that \( f(x_0) = 1 \) and \( f(A) = \left\lbrace 0 \right\rbrace \).
\end{definition}
\begin{remark}
  \begin{enumerate}
    \item A normal space is completely regular by the Urysohn lemma.
    \item A completely regular space is regular.
  \end{enumerate}
\end{remark}

\begin{theorem}
  A subspace of a completely regular space is completely regular.
  A product of completely regular spaces is completely regular.
\end{theorem}
\begin{proof}
  TODO %TODO
\end{proof}

\paragraph{Urysohn metrization theorem}

\begin{theorem}[Urysohn metrization theorem]
  Every regular space \( X \) with a countable basis is metrizable.
\end{theorem}
\begin{proof}
  We shall prove that \( X \) metrizable by imbedding \( X \) in a metrizable space \( Y \); that is, by showing \( X \) homeomorphic with a subspace of \( Y \).
  \begin{enumerate}
    \item We can prove the following by Urysohn Lemma.
      \begin{claim}
        There exists a countable collection of continuous functions \( f_n: X \to [0, 1] \) having the properties that given any point \( x_0 \) of \( X \) and any neighborhood \( U \) of \( x_0 \), there exists an index \( n \) such that \( f_n \) is positive at \( x_0 \) and vanishes outside \( U \).
      \end{claim}
      \begin{claimproof}
        Let \( \left\lbrace B_n \right\rbrace \) be a countable basis for \( X \).
        For each pair \( n, m \) of indices for which \( \overline{B}_n \subseteq B_m \), apply the Urysohn lemma to choose a continuous function \( g_{n, m}: X \to [0, 1] \) such that \( g_{n, m} (\overline{B}_n) = \left\lbrace 1 \right\rbrace \) and \( g_{n, m}(X - B_m) = \left\lbrace 0 \right\rbrace \).
        Then the collection \( \left\lbrace g_{n, m} \right\rbrace \) satisfies our requirement: given \( x_0 \) and given a neighborhood \( U \) of \( x_0 \), one can choose a basis element \( B_m \) containing \( x_0 \) that is contained in \( U \).
        Using regularity, one can then choose \( B_n \) so that \( x_0 \in B_n \) and \( \overline{B}_n \subseteq B_m \).
        Then \( n, m \) is a pair of indices for which the function \( g_{n, m} \) is defined; and it is positive at \( x_0 \) and vanishes outside \( U \).
        Because the collection \( \left\lbrace g_{n, m} \right\rbrace \) is indexed with a subset of \( \mathbb{Z}_+ \times \mathbb{Z}_+ \), it is countable; therefore, it can be reindexed with the positive integers, giving us the desired collection \( \left\lbrace f_n \right\rbrace \).
      \end{claimproof}
    \item (First version) Given the function \( f_n \) as above, take \( \mathbb{R}^\omega \) in the product topology and define a map \( F: X \to \mathbb{R}^\omega \) by the rule
      \[
        F(x) = (f_1(x), f_2(x), \cdots,).
      \]
      We assert that \( F \) is an embedding.
      \begin{enumerate}
        \item First, \( F \) is continuous because \( \mathbb{R}^\omega \) has the product topology and each \( f_n \) is continuous.
        \item Second, \( F \) is injective by the construction, hence \( F \) is a bijection of \( X \) to its image \( Z \coloneq F(X) \).
        \item Finally, We want to show that \( F \) is a homeomorphism.
          \( F \) is continuous by the universal property of the product topology.
          We need only show that for each open set \( U \) in \( X \), the set \( F(U) \) is open in \( Z \).

          Let \( z_0 \) be a point of \( F(U) \).
          We shall find an open set \( W \) of \( Z \) such that
          \[
            z_0 \in W \subseteq F(U).
          \]
          Let \( x_0 \) be the point of \( U \) such that \( F(x_0) = z_0 \).
          Chooose an index \( N \) for which \( f_N(x_0) > 0 \) and \( f_N(X - U) = \left\lbrace 0 \right\rbrace \).
          Take the open ray \( (0, +\infty) \) in \( \mathbb{R} \), and let \( V \) be the open set
          \[
            V = \pi^{-1}_N((0, + \infty))
          \]
          of \( \mathbb{R}^\omega \).
          Let \( W = V \cap Z \); then \( W \) is open in \( Z \), by definition of the subspace topology.
          Then
          \begin{claim}
            \( z_0 \in W \subseteq F(U) \).
            And thus \( F \) is an embedding of \( X \) in \( \mathbb{R}^\omega \).
          \end{claim}
          \begin{claimproof}
            First \( z_0 \in W \) because
            \[
              \pi_N(z_0) = \pi_N(F(x_0)) = f_N(x_0) > 0.
            \]
            Second, \( W \subseteq F(U) \).
            For if \( x \in W \), then \( z = F(x) \) for some \( x \in X \), and \( \pi_N(z) \in (0, + \infty) \).
            Since \( \pi_N(z) = \pi_N(F(z)) = f_N(x) \), and \( f_N \) vanishes outside \( U \), the point \( x \) must be in \( U \).
            Then \( z = F(x) \) is in \( F(U) \), as desired.
          \end{claimproof}
      \end{enumerate}
  \end{enumerate}
\end{proof}

In Step (2) of the preceding proof, we actually proved something stronger than the result stated there.
For later use, we state it here.

\begin{theorem}[Embedding theorem]
  Suppose that \( \left\lbrace f_\alpha \right\rbrace_{\alpha \in J} \) is an indexed family of continuous functions \( f_\alpha: X \to \mathbb{R} \) satisfying the requirement that for each point \( x_0 \) of \( X \) and each neighborhood \( U \) of \( x_0 \), there is an index \( \alpha \) such that \( f_\alpha \) is positive at \( x_0 \) and vanishes outside \( U \).
  Then the function \( F: X \to \mathbb{R}^J \) defined by
  \[
    F(x) = (f_\alpha(x))_{\alpha \in J}
  \]
  is an embedding of \( X \) in \( \mathbb{R}^J \).
  If \( f_\alpha \) maps \( X \) into \( [0, 1] \) for each \( \alpha \), then \( F \) embeds \( X \) in \( [0, 1]^J \).
\end{theorem}

\begin{theorem}
  A space \( X \) is completely regular \( \iff \) it is homeomorphic to a subspace of \( [0, 1]^J \) for some \( J \).
\end{theorem}

\paragraph{The Tietza Extension Theorem}

\begin{theorem}[Tietze extension theorem]
  Let \( A \) be a closed subspace of \( X \).
  Then \( X \) is a normal space
  \begin{enumerate}
    \item \( \iff \) Any continuous map \( f \) of \( A \) into the closed interval \( [a, b] \) of \( \mathbb{R} \) may be extended to a continuous map \( \tilde{f} \) of all of \( X \) into \( [a, b] \).
    \item \( \implies \) Any continuous map \( f \) of \( A \) into \( \mathbb{R} \) may be extended to a continuous map \( \tilde{f} \) of all of \( X \) into \( \mathbb{R} \).
  \end{enumerate}
\end{theorem}
\begin{proof}
  \begin{enumerate}
    \item For convenience, we let \( a = -1, b = 1 \).
      \( \impliedby \): Let \( A, B \) be disjoint closed sets of \( X \), then \( A \cup B \) is closed in \( X \) and
      \[
        f: A \cup B \to [-1, 1],\quad x \mapsto \left\lbrace \begin{array}{ll}
          -1, & x \in A\\
          1, & x \in B
        \end{array} \right.
      \]
      is a continuous function on \( A \cup B \).
      \( f \) can be extended to \( \widetilde{f}: X \to [-1, 1] \), and consider \( \widetilde{f}^{-1}(-\infty, 0) \) and \( \widetilde{f}^{-1}((0, +\infty)) \).

      \( \implies \): Let
      \[
        A_1 := \left\lbrace x \in A \mid f(x) \geq 1 / 3 \right\rbrace \textmd{ and } B_1 := \left\lbrace x \in A \mid f(x) \leq - 1 / 3 \right\rbrace,
      \]
      then \( A_1 \) and \( B_1 \) are disjoint closed set in \( X \).
      By Urysohn Lemma, there exists a continuous map \( g: X \to [-1/3, 1/3] \) such that
      \[
        g(A_1) = 1 / 3 \textmd{ and } g(B_1) = -1 / 3.
      \]
      And \( \left\vert f(x) - \left. g \right\vert_A(x) \right\vert \leq 2 / 3 \) for all \( x \in A \).

        Denote \( f_1 = f \), we find a continuous map \( g_1: X \to [-1/3, 1/3] \) such that
        \[
          \left\vert f_1(x) - \left. g \right\vert_A(x) \right\vert \leq 2 / 3 \textmd{ for all }  x \in A .
        \]
        let \( f_2 = f_1 - \left. g_1 \right\vert_A \), repeat the procedure above, we can find a continuous map \( g_2: X \to [-\frac{1}{3} \cdot \frac{1}{3}, \frac{1}{3} \cdot \frac{1}{3}] \), such that
        \[
           \left\vert f_2(x) - \left. g \right\vert_A(x) \right\vert \leq (2 / 3)^2  \textmd{ for all }  x \in A .
        \]
        Continue repeating, we can find a series of continuous maps
        \[
          g_n: X \to \left[-\frac{1}{3}(\frac{1}{3})^{n - 1}, \frac{1}{3}(\frac{1}{3})^{n - 1}\right]
        \]
        such that
        \[
          \left\vert f_n(x) - \left. g_n \right\vert_A(x) \right\vert \leq (2 / 3)^n, \textmd{ for all } x \in A
        \]
        where \( f_n = f_{n - 1} - \left. g_{n - 1} \right\vert_A \).
        Let \( \widetilde{f}(x) := \sum_{n = 1}^\infty g_n(x) \).
        By construction,\( \sum_{n = 1}^\infty g_n \) is defined everywhere.
        To show that \( \tilde{f} \) is continuous, we show that \( s_n \coloneq \sum_{i = 1}^n g_n \) converges to \( g \) uniformly.
        \begin{align*}
        \left\vert s_k(x) - s_n(x) \right\vert &= \left\vert \sum_{i = n + 1}^k g_i(x) \right\vert\\
                                               &\leq \frac{1}{3} \sum_{i = n + 1}^k \left(\frac{2}{3}\right)^{i - 1}\\
                                               &< \frac{1}{3} \sum_{i = n + 1}^\infty \left(\frac{2}{3}\right)^{i - 1} = \left(\frac{2}{3}\right)^n
        .\end{align*}
    Let \( k \to \infty \), we see that
    \[
      \left\vert g(x) - s_n(x) \right\vert \leq \left(\frac{2}{3}\right)^n.
    \]
    \item Consider \( f_1 := \arctan \circ f: A \to (-\pi/2, \pi/2) \).
      By (1), \( f_1 \) can be extended to a continuous function
      \[
        \widetilde{f}_1: X \to [-\pi / 2, \pi / 2]
      \]
      Let \( B = \widetilde{f}_1^{-1}(-\pi / 2) \cup \widetilde{f}_1^{-1}(\pi / 2) \), then \( B \) is closed in \( X \) and \( B \cap X = \emptyset \).
      By Urysohn Lemma, there exists a continuous function \( g: X \to [0, 1] \) such that
      \[
        g(A) = 1 \textmd{ and } g(B) = 0.
      \]
      We define \( h(x) := \widetilde{f}_1(x) g(x) \), then \( h \) is the continuous function map \( X \) to \( (-\pi / 2, \pi / 2) \).
  \end{enumerate}
\end{proof}

\paragraph{Partition of unity}

\begin{definition}
  If \( \phi: X \to \mathbb{R} \), then the \emph{support} of \( \phi \) is defined to be the closure of the set \( \phi^{-1}(\mathbb{R} - \left\lbrace 0 \right\rbrace) \), and we use \( \operatorname{Supp} \) to denote the support of \( \phi \).
\end{definition}

\begin{definition}
  Let \( \left\lbrace U_1, \cdots, U_n \right\rbrace \) be a finite indexed open covering of the space \( X \).
  An indexed family of continuous functions
  \[
    \phi_i: X \to [0, 1] \quad \text{for } i = 1,\cdots, n,
  \]
  is said to be a \emph{partition of unity}(or shortly P.O.U.) dominated by \( \left\lbrace U_i \right\rbrace \) if
  \begin{enumerate}
    \item \( \operatorname{Supp} \phi_i \subseteq U_i \) for each \( i \).
    \item \( \sum_{i = 1}^n \phi_i(x) = q \) for each \( x \).
  \end{enumerate}
  Moreover, if \( \left\lbrace \operatorname{Supp} \phi_i \right\rbrace \) is locally finite, we say such P.O.U. is locally finite.
\end{definition}

TODO%TODO
