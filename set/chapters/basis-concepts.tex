\chapter{基础知识}

\section{集合论公理}

\subsection{公式}

集合论中的公式由原子公式
\[
  x \in y,\quad x = y
\]
组成.

\subsection{Russell 悖论}

\paragraph{概括公理模式(Axiom Schema of Comprehension)} 如果 \( P \) 是一个性质,
那么存在一个集合 \( Y = \left\lbrace x: P(x) \right\rbrace \).

\subsection{Zermelo-Fraenkel 公理}

\paragraph{外延公理(Axiom of Extensionality)} 假设 \( X, Y \) 为两个集合.
如果 \( X \) 和 \( Y \) 有相同的元素, 那么 \( X = Y \).
\[
  \forall u (u \in X \leftrightarrow u \in Y) \rightarrow X = Y.
\]

\paragraph{配对公理(Axiom of Pairing)} 对任意集合 \( a \) 和 \( b \),
存在一个集合 \( \left\lbrace a, b \right\rbrace \) 包含且只包含 \( a \) 和 \( b
\).
\[
  \forall a \forall b \exists c \forall x (x \in c \leftrightarrow x = a \vee
  x = b)
\]

\paragraph{分离公理模式(Axiom Schema of Separation)} 假设 \( P \) 为关于集合 \(
X \) 的一个性质, 并以 \( P(x) \) 表示集合 \( x \in X \) 满足性质.
那么集合 \( \left\lbrace x \in X: P(x) \right\rbrace \) 存在.

\paragraph{并集公理(Axiom of Union)} 对任意集合 \( X \), 存在并集 \( \bigcup X
:= \left\lbrace x: x \in X \right\rbrace \).

\paragraph{幂集公理(Axiom Power Set)} 对任意集合 \( X \), 其子集构成一个集合 \(
P(X) := \left\lbrace u: u \subset X\right\rbrace \).

\paragraph{无穷公理(Axiom of Infinity)} 存在无穷集.

\paragraph{替换公理模式(Axiom Schema of Replacement)}. 设 \( F \) 为一个以集合
\( X \) 为定义域的函数, 那么存在集合 \( F(X) := \left\lbrace F(x) : x \in X
\right\rbrace \).

\paragraph{正则公理(Axiom of Regularity)} 每个一个非空集合都有一个关于属于 \(
\in \) 的极小元.

\paragraph{选择公理(Axiom of Choice)} 每一族非空集合都有一个选择函数.

\section{序}

集合 \( P \) 上的一个二元关系 \( < \) 称为 \( P \) 上的\emph{偏序}, 如果
\begin{enumerate}
  \item 对任意 \( p \in P \), \( p \not< p \).
  \item 如果 \( p < q, q < r \), 那么 \( p < r \).
\end{enumerate}
此时 \( (P, <) \) 称为一个\emph{偏序集}.
\( P \) 上的偏序 \( < \) 称为\emph{线性序}, 如果对所有\( p, q \in P \),
下面之一成立
\[
  p < q,\quad p = q,\quad q < p.
\]
如果 \( < \) 是一个偏序(线性序), 那么关系 \( \leq \)
也被称为\emph{偏序}(\emph{线性序})(而 \( < \) 有时被称为\emph{严格}序), 其中符号
\( \leq \) 意指 \( p \leq q \) 如果 \( p < q \) 或称 \( q = p \) 成立.

假设 \( (P, <) \) 是一个偏序集, \( X \) 是 \( P \) 的一个非空子集, \( a \in P
\), 那么
\begin{enumerate}
  \item \( a \) 称为 \( X \) 的一个\emph{极大元}, 如果 \( a \in X \) 且 
    \( (\forall x \in X) a \not < x \).
    \( a \) 称为 \( X \) 的一个\emph{极小元}, 如果 \( a \in X \) 且
    \( (\forall x \in X) x \not < a \).
  \item \( a \) 称为 \( X \) 的一个\emph{最大元}, 如果 \( a \in X \) 且
    \( (\forall x \in X) x \leq a \).
    \( a \) 称为 \( X \) 的一个\emph{最小元}, 如果 \( a \in X \) 且
    \( (\forall x \in X) a \leq x \).
  \item \( a \) 称为 \( X \) 的一个\emph{上界}, 如果
    \( (\forall x \in X) x \leq a\).
    \( a \) 称为 \( X \) 的一个\emph{下界}, 如果
    \( (\forall x \in X) a \leq x\).
  \item \( a \) 称为 \( X \) 的一个\emph{上确界}, 如果 \( a \) 是 \( X \)
    的一个最小上界.
    \( a \) 称为 \( X \) 的一个\emph{下确界}, 如果 \( a \) 是 \( X \)
    的一个最大下界.
\end{enumerate}

假设 \( (P, <), (Q, <) \) 是偏序集, \( f: P \to Q \) 是一个函数.
称 \( f \) 是\emph{保序的}如果当 \( x < y \) 时有 \( f(x) < f(y) \).
如果 \( (P, <), (Q, <) \) 都是线性序的, 那么包含函数也被称为\emph{上升}函数.

一个 \( P \) 到 \( Q \) 的一一对应的满函数 \( f \) 称为是 \( P \) 和 \( Q \)的\emph{同构},
如果 \( f \) 和 \( f^{-1} \) 都是保序映射, 并且, 此时称 \( (P, <) \) 同构于 \(
(Q, <)\).
\( (P, <) \) 到自身的同构称为 \( (P, <) \) 的\emph{自同构}.

\subsection{良序}

集合 \( P \) 上的线性序 \( < \) 称为是\emph{良序}, 如果 \( P \)
的每个非空子集都有一个极小元.

\begin{lemma}
  \label{lemma-well-order-set-increasing-funcion-to-itself}
  如果 \( (W, <) \) 是一个良序集, \( f: W \to W \) 是一个上升函数, 那么对每个
  \( x \in W \) 有 \( f(x) \geq x \).
\end{lemma}
\begin{proof}
  由分离公理模式, 可以考虑集合 \( X = \left\lbrace x \in W: f(x) < x
  \right\rbrace \), 只需证明 \( X = \varnothing \) 即可.
  如果 \( X \neq \varnothing \), 怎么由良序集定义 \( X \)
  有一个极小元 \( z \).
  那么由 \( f(x) < x \) 有 \( f(f(x)) < f(x) \), 这与 \( x \) 的极小性矛盾.
\end{proof}

\begin{corollary}
  良序集的自同构只有恒等映射.
\end{corollary}

\begin{corollary}
  如果两个良序集 \( W_1, W_2 \) 是同构的, 那么 \( W_1 \) 到 \( W_2 \)
  的同构是唯一的.
\end{corollary}

\paragraph{前段} 如果 \( W \) 是一个良序集, \( u \in W \), 那么 \( \left\lbrace
x \in W: x < u \right\rbrace \) 称为由 \( u \) 给出的 \( W \) 的\emph{前段}.

\begin{lemma}
  \label{lemma-well-order-set-not-isomorphic-to-its-initial-segment}
  没有良序集同构于其自身的前段.
\end{lemma}
\begin{proof}
  假设不然, 存在 \( (P, <) \) 是通过 \( f \) 同构于其前段 \( \left\lbrace x: x
  \in u \right\rbrace \) 的良序集, 那么 \( f(u) < u \),
  这与\cref{lemma-well-order-set-increasing-funcion-to-itself}矛盾.
\end{proof}

\begin{theorem}
  假设 \( W_1 \) 和 \( W_2 \) 为两个良序集, 那么以下条件中有且只有满足
  \begin{enumerate}
    \item \( W_1 \) 同构于 \( W_2 \).
    \item \( W_1 \) 同构于 \( W_2 \) 的一个前段.
    \item \( W_2 \) 同构于 \( W_1 \) 的一个前段.
  \end{enumerate}
\end{theorem}
\begin{proof}
  如果 \( u \in W_i \), 以 \( W_i(u) \) 记 \( W_i \) 中 \( u \) 给出的前段.
  令函数 \( f \) 为
  \[
    f = \left\lbrace (x, y) \in W_1 \times W_2: W_1(x) \text{同构于} W_2(y)
    \right\rbrace.
  \]

  \( f \) 是一个一一对应, 即如果 \( W_1(x) \) 同构与 \( W_2(y_1), W_2(y_2) \),
  那么 \( y_1 = y_2 \).
  如果 \( y_1 < y_2 \), 不仿设 \( y_1 < y_2 \), 那么 \( W_2(y_1) \) 是 \( W_2(y_2)
  \) 的一个前段, 而 \( W_2(y_2) \) 同构于 \( W_1(x) \), 因此同构于 \( W_1(y_1)
  \), 这与\cref{lemma-well-order-set-not-isomorphic-to-its-initial-segment}矛盾.

  \( f \) 是一个保序映射: 如果 \( h \) 是 \( W_1(x) \) 和 \( W_2(y) \)
  间的同构, 且 \( x' < x \), 那么 \( W_1(x') \) 和 \( W_2(h(x
  ')) \) 是同构的, 于是 \( f(x') = h(x') \).
  换句话说, \( x' < x \implies f(x') = h(x') < y = f(x) \).

  如果 \( \operatorname{dom} f = W_1 \) 且 \( \operatorname{ran} f = W_2 \),
  那么 (1) 成立.

  如果 \( \operatorname{ran} f \neq W_2 \), 由 \( W_2 \) 是良序集, 可以取 \(
  W_2 \setminus \operatorname{ran} f \) 的极小元 \( y_0 \).
  实际上, \( \operatorname{ran} f = W_2(y_0) \).
  \( \operatorname{ran} f \supset W_2(y_0) \) 由极小性保证;
  \( \operatorname{ran} f \subset W_2(y_0) \) 是因为, 对任意 \( y \in
  \operatorname{ran} f \), 如果存在 \( y \in \operatorname{ran} f \) 使得 \( y
  \notin W_2(y_0) \), 即 \( y \geq y_0 \), 那么总能得到 \( y_0 \in
  \operatorname{ran} f \), 这与假设矛盾.
  这时, 还有 \( \operatorname{dom}f = W_1 \).
  否则, 再由 \( W_1 \) 良序, 取 \( W_1 \setminus \operatorname{dom} f \) 极小元
  \( x_0 \).
  类似地, 可以知道 \( \operatorname{dom} f = W_1(x_0) \).
  于是 \( W_1(x_0) \) 同构于 \( W_2(y_0) \), 换句话说 \( (x_0, y_0) \in f \),
  这与 \( x_0, y_0 \) 取法矛盾.
  总结之, \( \operatorname{ran}f = W_2(y_0), \operatorname{dom} f = W_1 \),
  这正是 (2).

  类似于 (2), 如果 \( \operatorname{ran} f \neq W_1 \), 那么可以知道这时满足
  (3).
\end{proof}

\subsection{序数}

一个集合 \( T \) 称为是\emph{传递的}, 如果每个 \( T \) 的元素都是 \( T \)
的子集.
\begin{remark}
  \( T \) 传递 \( \iff \bigcup T \subset T \iff T \subset \mathcal{P}(T) \).
\end{remark}
一个集合称为一个\emph{序数}, 如果其传递且关于 \( \in \) 良序.
我们常用小写希腊字母 \( \alpha, \beta, \gamma, \ldots \) 来记序数,
并将所有序数组成的类记作 \( \operatorname{Ord} \).

\begin{lemma}
  \label{lemma-ordinal-number}
  \begin{enumerate}
    \item \( 0 = \varnothing \) 是一个序数.
    \item 如果 \( \alpha \) 是一个序数, \( \beta \in \alpha \), 那么 \( \beta \)
      是一个序数.
    \item 如果 \( \alpha \neq \beta \) 为序数, \( \alpha \subset \beta \), 那么
      \( \alpha \in \beta \).
    \item 如果 \( \alpha, \beta \) 为序数, 那么要么 \( \alpha \subset \beta \)
      或 \( \beta \subset \alpha \).
  \end{enumerate}
\end{lemma}
\begin{proof}
  (1) 和 (2) 的证明是直接依定义的.

  (3) 如果 \( \alpha \subsetneq \beta \), 由 \( \beta \) 良序, 可以取 \( \beta -
  \alpha \) 中的一个极小元素 \( \gamma \).
  于是 \( \alpha \) 即 \( \gamma \) 给出的前段, 而由传递性 \( \gamma \)
  的前段就是 \( \gamma \), 于是 \( \alpha = \gamma \in \beta \).

  (4) 可以知道 \( \gamma =  \alpha \cap \beta \) 也是一个序数, 只需证明 \( \gamma =
  \alpha \) 或 \( \gamma = \beta \) 之一成立.
  假设不然, \( \gamma \in \alpha \) 和 \( \gamma \in \beta \) 同时成立, 则 \(
  \gamma \in \gamma \) 这与 \( \in \) 是严格序矛盾.
\end{proof}

\begin{proposition}
  \begin{enumerate}
    \item \( < \) 是 \( \operatorname{Ord} \) 中的线性序.
    \item 对每个 \( \alpha \in \operatorname{Ord} \), \( \alpha = \left\lbrace
      \beta: \beta < \alpha \right\rbrace \).
    \item 如果 \( C \) 是一个序数的非空类, 那么 \( \bigcap C \) 也是一个序数,
      且 \( \bigcap C \in C \), \( \bigcap C = \inf C \).
    \item 如果 \( X \) 是序数组成的一个非空集合, 那么 \( \bigcup X \)
      也是一个序数, 且 \( \bigcup = \sup X \).
    \item 对每个 \( \alpha \in \operatorname{Ord} \), \( \alpha \cup
      \left\lbrace \alpha \right\rbrace \) 是一个序数, 且 \( \alpha \cup
      \left\lbrace \alpha \right\rbrace = \inf \left\lbrace \beta \in
        \operatorname{Ord}: \beta >
      \alpha \right\rbrace \)(一般来说, 这个不一定是个集合).
      于是, 可以定义 \( \alpha \) 的\emph{后继} \( \alpha + 1 = \alpha \cup
      \left\lbrace \alpha \right\rbrace \).
  \end{enumerate}
\end{proposition}

\begin{theorem}
  每一个良序集都同构于一个唯一的序数.
\end{theorem}
\begin{proof}
  唯一性由\cref{lemma-ordinal-number}和\cref{lemma-well-order-set-not-isomorphic-to-its-initial-segment}保证.
  给定一个良序集 \( W \) 定义 \( \Omega \) 上的函数 \( F(x) = \alpha_x \),
  如果序数 \( \alpha_x \) 同构于 \( x \) 在 \( W \) 给出的前段.
  对每个 \( x \) 这样的 \( \alpha_x \) 都是存在的, 否则考虑最小不存在这样 \(
  \alpha_x \) 的 \( x \)

  此时
  由替换公理知道, \( F(W) \) 是一个集合.
\end{proof}

\begin{theorem}[超限归纳法]
  假设 \( B \) 是良序集 \( (A, \leq) \) 的一个子集, 如果条件: 每个 \( a \in A \),
  \( A \) 的前段 \( \left\lbrace c \in A: c < a \right\rbrace \) 包含在 \(
  B \) 中蕴含 \( a \in B \), 那么 \( B = A \).
\end{theorem}
\begin{proof}
  下面证明 \( A \setminus B = \varnothing \).
  如果 \( A \setminus B \neq \varnothing \), 那么由良序集定义, \( A \setminus B
  \) 存在一个极小元 \( a \).
  由极小性有 \( \left\lbrace c \in A: c < a \right\rbrace \subseteq B \), 结合
  \( B \) 性质知道 \( a \in B \), 这与 \( a \) 的定义矛盾.
\end{proof}

\begin{theorem}[超限归纳法]
  \label{theorem-transfinite-induction}
  假设 \( C \) 是为一类序数, 且
  \begin{enumerate}
    \item \( 0 \in C \);
    \item 如果 \( \alpha \in C \), 那么 \( \alpha + 1 \in C \);
    \item 如果 \( \alpha \) 是非零极限序数, 对所有 \( \beta < \alpha \) 有 \(
      \beta \in C  \), 那么 \( \alpha \in C \);
  \end{enumerate}
  那么 \( C = \operatorname{On} \).
\end{theorem}

\begin{theorem}[良序定理, Zermelo]
  每一个集合都能被赋予一个良序结构.
\end{theorem}
\begin{proof}
  假设 \( A \) 是一个集合, 由幂集公理 \( \mathcal{P}(A) \) 是一个集合.
  由分离公理模式 \( \mathcal{A} = \left\lbrace x \in \mathcal{P}(A): x \notin
  \varnothing \right\rbrace \) 是一个集合.
  由选择公理, 存在 \( \mathcal{A} \) 的一个选择函数 \( f \).


  % 下面将应用超限归纳法\eqref{theorem-transfinite-induction}构造一个 \( A \)
  % 到某个基数的双射.
  %
  % 首先将 \( c(A) \mapsto 0 \), 并逐步扩张到双射 \( c(A \setminus X)  \),
  % 其中 \( c(X) \mapsto \alpha \) 已经定义.
  % 假设 \( s_{\beta} \) 对所有 \( \beta < \alpha \) 已定义
\end{proof}
