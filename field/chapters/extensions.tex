\chapter{域扩张}

\section{同态与扩张}

\begin{proposition}
  假设 \( F(\alpha) \) 是域 \( F \) 的单扩张, \( \Omega \) 是 \( F \)
  的第二个扩张.
  \begin{enumerate}
    \item 假设 \( \alpha \) 在 \( F \) 上超越. 那么对每个 \( F \)-同态 \(
      \varphi: F(\alpha) \to \Omega, \varphi(\alpha) \) 在 \( F \) 上超越,
      且映射 \( \varphi \mapsto \varphi(\alpha) \) 给出了一一对应
      \[
        \left\lbrace F\text{-同态} F(\alpha) \to \Omega \right\rbrace
        \leftrightarrow \left\lbrace \Omega \text{在} F \text{上的超越元}
        \right\rbrace.
      \]
    \item 假设 \( \alpha \) 在 \( F \) 上代数, 令 \( f(X) \) 为其极小多项式.
      对每个 \( F \)-同态 \( \varphi: F[\alpha] \to \Omega \), \(
      \varphi(\alpha) \) 是 \( f(X) \) 在 \( \Omega \) 中的根, 并且映射 \(
      \varphi \mapsto \varphi(\alpha) \) 定义了一一映射
      \[
        \left\lbrace F \text{-同态} \varphi: F[\alpha] \to \Omega \right\rbrace
        \leftrightarrow \left\lbrace f \text{在} \Omega \text{中的根}
        \right\rbrace.
      \]
      特别地, 这样的映射数即 \( f \) 在 \( \Omega \) 的不同根数.
  \end{enumerate}
\end{proposition}
\begin{proof}
  (i) \( \varphi \) 是一个非平凡域态射意味着这是个单射. 如果 \( \varphi(\alpha)
  \) 在 \( F \) 上代数, 那么存在多项式 \( f(X) \in F[X] \), 使得 \(
  f(\varphi(X)) = 0 \). 又由同态的性质 \( \varphi(f(\alpha)) = 0 \), 这只能 \(
  f(\alpha) = 0 \), 矛盾. 至于一一映射另一边的构造是显然的.
  (ii) 与 (i) 是完全类似的.
\end{proof}

我们能将上面结果推广如下, 证明是完全一致的.
\begin{proposition}
  假设 \( F(\alpha) \) 是 \( F \) 的单扩张, \( \varphi_0: F \to \Omega \)
  是一个由 \( F \) 到第二个域 \( \Omega \) 的同态.
  \begin{enumerate}
    \item 如果 \( \alpha \) 在 \( F \) 上超越, 那么映射 \( \varphi \mapsto
      \varphi(\alpha) \) 定义了一个一一映射
      \[
        \left\lbrace \varphi_0 \text{的扩张} \varphi: F(\alpha) \to \Omega
        \right\rbrace \leftrightarrow \left\lbrace \Omega \text{在} \varphi_0(F)
        \text{的超越元} \right\rbrace
      \]
    \item 如果 \( \alpha \) 在 \( F \) 上代数且其极小多项式为 \( f(X) \),
      那么映射 \( \varphi \mapsto \varphi(\alpha) \) 定义了一个一一映射
      \[
        \left\lbrace \varphi_0 \text{的扩张} \varphi: F[\alpha] \to \Omega
        \right\rbrace \leftrightarrow \left\lbrace \varphi_0 f \text{在} \Omega
        \text{的根} \right\rbrace.
      \]
      特别地, 这样的映射数即 \( \varphi_0f \) 在 \( \Omega \) 的不同根数.
  \end{enumerate}
\end{proposition}

\section{分裂域}

假设 \( f \in F[X] \). 那么一个包含 \( F \) 的域 \( E \) 称为 \emph{分裂} \( f
\) 如果 \( f \) 在 \( E[X] \) 中分裂, 也就是
\[
  f(X) = a \prod_{i = 1}^m(X - \alpha_i),\quad a \in F,\quad \alpha_i \in E.
\]
如果 \( E \) 分裂 \( f \) 且由 \( f \) 的根生成, 那么称 \( E \) 为 \( f \) 的
\emph{分裂域} 或者 \emph{根域}.

\begin{proposition}
  每个多项式 \( f \in F[X] \) 都有一个分裂域 \( E_f \), 并且
  \[
    [E_f, F] \leq (\deg f) !
  \]
\end{proposition}
\begin{proof}
  如果 \( f \) 的根都在 \( F \) 中, 则问题平凡.
  我们按归纳法证明该命题.
  设 \( f \) 的一个根为 \( \alpha_1 \notin F \). 考虑 \( F_1 = F[\alpha_1] \),
  由归纳假设, \( f_1 = {f} / {(X - \alpha_1)} \) 对 \( F_1 \) 有分裂域
  \[
    E_{f_1} = F_1[\alpha_2, \ldots, \alpha_{\deg f_1}] = F[\alpha_1][\alpha_2,
    \ldots, \alpha_{\deg f_1}] = F[\alpha_1][\alpha_2, \ldots, \alpha_{\deg
    f_1}].
  \]
\end{proof}

\begin{proposition}
  假设 \( f \in F[X] \). 假设 \( E \) 为由 \( f \) 的一些根生成的 \( F \) 的扩张, \(
  \Omega\) 是一个分裂 \( f \) 的 \( F \) 扩张. 那么存在 \( F \)-同态 \( \varphi:
  E \to \Omega \), 并且这样的同态个数至多有 \( [E : F] \) 个. 等于 \( [E : F] \)
  的一个充分条件是 \( f \) 在 \( \Omega \) 的根不同.
\end{proposition}
\begin{proof}
  不妨假设 \( f \) 是首一系数的. 由假设 \( f = \prod (X - \beta_i) \in \Omega[X]
  \).
\end{proof}
