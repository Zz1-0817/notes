\chapter{域扩张性质}

\section{可分性}

\subsection{纯不可分扩张}

假设 \( F/K \) 是一个域扩张.
一个元素 \( u \in F \) 称为在 \( K \) 上\emph{纯不可分}, 如果 \( u \) 在 \( K[X]
\) 上的极小多项式具有形式 \( f = (X - u)^m \).
\( F \) 称为 \( K \) 的\emph{纯不可分扩张}, 如果\( F \) 的每个元素都在 \( K \)
上纯不可分.

\begin{proposition}
  \label{proposition-both-separable-purely-inseparable}
  假设 \( F/K \) 是一个域扩张.
  那么 \( u \in F \) 在 \( K \) 上同时可分及纯不可分当且仅当 \( u \in K \).
\end{proposition}

\paragraph{纯不可分扩张的等价条件}

\begin{lemma}
  \label{lemma-power-of-element-separable}
  假设 \( F/K \) 是一个域扩张, 其中 \( \operatorname{char} K = p \neq 0 \).
  如果 \( u \in F \) 在 \( K \) 上代数, 那么存在某个 \( n \in \mathbb{N} \) 使得
  \( u^{p^n} \) 在 \( K \) 上可分.
\end{lemma}
\begin{proof}
  假设 \( u \) 的极小多项式为 \( (X - u)^m \), 对 \( m \) 进行归纳.
  \( m = 1 \) 时结果显然.
  设 \( m > 1 \),
  由\cref{proposition-irreducible-polynomial-with-multiple-roots-TFAE-condition},
  \( (X - u)^m \) 可以视为 \( X^p \) 的多项式.
  于是可以设 \( m = p^r n \) 其中 \( \gcd(p, n) = 1, r > 1 \).
  因此 \( (X - u)^m = (X^{p^r} - u^{p^r})^n \), 换句话说 \( u^{p^r} \) 纯不可分,
  由归纳假设知道存在 \( n \in \mathbb{N} \) 使得 \( (u^{p^r})^{p^n} = u^{p^{r +
  n}} \) 在 \( K \) 上可分.
\end{proof}

\begin{theorem}
  \label{theorem-purely-inseparable-extension-TFAE-conditions}
  假设 \( F/K \) 为代数扩张, \( \operatorname{char} K \neq 0 \),
  那么以下条件等价:
  \begin{enumerate}
    \item \( F \) 在 \( K \) 上纯不可分.
    \item \( F \) 的任意元素 \( u \) 在 \( K \) 上极小多项式具有形式 \( x^{p^n}
      - a \in K[X] \).
    \item 如果 \( u \in F \), 那么存在 \( n \in \mathbb{N} \) 使得 \( u^{p^n}
      \in K \).
    \item \( F \) 在 \( K \) 上的所有可分元由 \( K \) 本身组成.
    \item \( F \) 由 \( K \) 上一个纯不可分元集合生成.
  \end{enumerate}
\end{theorem}
\begin{proof}
  \( (1) \implies (2) \)
  如果 \( u \in F \) 纯不可分,
  由\cref{proposition-irreducible-polynomial-with-multiple-roots-TFAE-condition},
  \( u \in F \) 在 \( K \) 上的极小多项式 \( (X - u)^m \) 可视为 \( X^p \)
  的多项式.
  于是可设 \( m = p^rn \) 其中 \( \gcd(p, n) = 1, r > 1 \).
  考察 \( (X - u)^{m} = (X^{p^r} - u^{p^r})^n \) 的 \( X^{p^r(n - 1)} \)
  项系数知道 \( u^{p^r} \in K \), 只能 \( n = 1 \).

  \( (2) \implies (3) \implies (1), (2) \implies (5), (3) \implies (4) \)
  是直接的.
  \( (4) \implies (3) \) 由\cref{lemma-power-of-element-separable}得到.

  \( (5) \implies (3) \)
  显然对任意非零 \( k \in K \), \( ku \) 的亦纯不可分.
  假设 \( u_1, u_2 \in F \) 在 \( K \) 上纯不可分, 那么由 \( (1) \implies (2) \)
  \( u_i \) 的极小多项式形如 \( x^{p^{n_i}} - a_i \in K[X] \).
  于是 \( (u_1 + u_2)^{p^{n_1 + n_2}} \in K \).
\end{proof}

\begin{corollary}
  假设 \( F \) 是 \( E \) 的纯可分扩张, \( E \) 是 \( K \) 的纯可分扩张, 那么 \(
  F \) 是 \( K \) 的纯可分扩张.
\end{corollary}

\begin{corollary}
  \label{corollary-purely-inseparable-over-middle-extension}
  假设 \( F/K \) 为一个域扩张.
  如果 \( u \in F \) 在 \( K \) 上纯不可分, 那么 \( u \) 在 \( F/K \)
  的任意中间域 \( M \) 上纯不可分.
  因此, 如果 \( F \) 在 \( K \) 上纯不可分, 那么 \( F \) 在 \( F/K \) 任意中间域
  \( M \) 上纯不可分.
\end{corollary}
\begin{proof}
  由\cref{theorem-purely-inseparable-extension-TFAE-conditions},
  如果 \( u \in F \) 那么存在 \( n \in \mathbb{N} \) 使得 \( u^{p^n} \in K
  \subseteq M \).
  再由\cref{theorem-purely-inseparable-extension-TFAE-conditions}, \( M[u] \) 是
  \( M \) 的纯不可分扩张, 于是 \( u \) 在 \( M \) 上纯不可分.
\end{proof}

\begin{corollary}
  \label{corollary-finite-purely-inseparable-degree}
  假设 \( F/K \) 为有限纯不可分扩张, 其中 \( \operatorname{char} K = p \neq 0
  \), 那么存在 \( n \geq 0 \) 使得 \( [F:K] = p^n \).
\end{corollary}
\begin{proof}
  假设 \( F = K[\alpha_1, \ldots, \alpha_r] \),
  那么由\cref{corollary-purely-inseparable-over-middle-extension}, \(
  F[\alpha_1,\ldots,\alpha_{i + 1}]/F[\alpha_1,\ldots,\alpha_i] \) 纯不可分.
  再由\cref{theorem-purely-inseparable-extension-TFAE-conditions}, \(
  F[\alpha_1,\ldots,\alpha_{i + 1}]/F[\alpha_1,\ldots,\alpha_i] \)
  的次数具有形式 \( p^{n_i} \).
  注意到 \( F \) 是 \( K \) 的逐步扩张即可.
\end{proof}

\paragraph{可分扩张}

\begin{proposition}
  \label{proposition-separate-elements-generate-separable-extension}
  假设 \( F/K \) 是一个域扩张, \( F \) 由 \( K \) 上一个可分元素集 \( S \) 生成,
  那么 \( F \) 在 \( K \) 上可分.
\end{proposition}
\begin{proof}
  对任意 \( u \in F \), 那么存在 \( v_1,\ldots,v_n \in S \) 使得 \( u \in
  K(v_1,\ldots,v_n) \).
  假设 \( v_i \) 在 \( F \) 上的极小多项式为 \( f_i \), 考虑不同 \( f_i \)
  的乘积 \( f \).
  由\cref{theorem-finite-Galois-TFAE-conditions}, \( f \) 在 \( K \) 的分裂域是
  \( K \) 的 Galois 扩张且包含 \( K(v_1, \ldots, v_n) \).
  于是 \( v \) 在 \( K \) 上可分.
\end{proof}

\subsection{可分子扩张与纯不可分子扩张}

\begin{theorem}
  \label{theorem-separable-and-purely-inseparable-subextension}
  假设 \( F/K \) 是一个代数扩张, \( S \) 是 \( F \) 中所有在 \( K \)
  可分元素构成的集合, \( P \) 是 \( F \) 中所有在 \( K \)
  纯不可分元素构成的集合.
  那么
  \begin{enumerate}
    \item \( S \) 是 \( K \) 的可分扩张域.
    \item \( F \) 在 \( S \) 上纯不可分.
    \item \( P \) 是 \( K \) 的纯不可分扩张域.
    \item \( P \cap S = K \).
    \item \( F \) 在 \( P \) 上可分 \( \iff F = SP \).
  \end{enumerate}
\end{theorem}
\begin{proof}
  \( (1) \)
  由\cref{proposition-separate-elements-generate-separable-extension}得到.
  \( (2) \)
  由\cref{lemma-power-of-element-separable}和\cref{theorem-purely-inseparable-extension-TFAE-conditions}得到.
  \( (3) \) 由\cref{theorem-purely-inseparable-extension-TFAE-conditions}得到.
  \( (4) \) 由\cref{proposition-both-separable-purely-inseparable}得到.

  \( (5) \) \( \implies \) \( F \) 在 \( P \) 上可分, 从而在中间域 \( SP \)
  上可分.
  而由 \( (2) \), \( F \) 在 \( S \) 上纯不可分, 所以在 \( SP \) 上纯不可分.
  因此由\cref{proposition-both-separable-purely-inseparable}知道 \( F = SP \).
  \( \impliedby \)
  由\cref{proposition-separate-elements-generate-separable-extension}直接得到.
\end{proof}

\begin{corollary}
  假设 \( F \) 是 \( E \) 的可分扩张, \( E \) 是 \( K \) 的可分扩张, 那么 \( F
  \) 是 \( K \) 的可分扩张.
\end{corollary}
\begin{proof}
  假设 \( S \) 为 \( F \) 为 \( K \) 中所有可分元构成的子域.
  显然 \( S \supseteq E \), 故 \( F \) 在 \( S \) 上可分.
  由\cref{theorem-separable-and-purely-inseparable-subextension}, \( F \) 在 \(
  S \) 上纯不可分.
  由\cref{proposition-both-separable-purely-inseparable}, \( F = S \).
\end{proof}

\begin{corollary}
  假设 \( F/K \) 是一个代数扩张, 且 \( \operatorname{char} K = p \neq 0 \). 如果
  \( F \) 在 \( K \) 上可分, 那么对每个 \( n \geq 1 \), \( F = KF^{p^n} \).
  反过来, 如果 \( [F: K] \) 有限且 \( F = KF^p \), 那么 \( F \) 在 \( K \)
  上可分.
  特别地, \( u \in F \) 在 \( K \) 上可分当且仅当 \( K(u^p) = K(u) \).
\end{corollary}
\begin{proof}
  如果 \( F \) 在 \( K \) 上可分, 那么 \( F \) 在 \( KF^{p^n} \)
  中可分且由\cref{theorem-purely-inseparable-extension-TFAE-conditions}在 \(
  KF^{p^n} \) 中纯不可分.
  于是由\cref{proposition-both-separable-purely-inseparable}, \( F = KF^{p^n} \)

  下面假设 \( F/K \) 有限, \( S \) 为 \( F \) 在 \( K \) 中可分元构成的域且 \(
  KF^p \), \( F = K(u_1, \ldots, u_m) \) 那么 \( F = S(u_1, \ldots, u_m) \).
  由\cref{theorem-separable-and-purely-inseparable-subextension} 每个 \( u_i \)
  都在 \( S \) 上纯不可分,
  于是由\cref{theorem-purely-inseparable-extension-TFAE-conditions}, 对所有 \( i
  \) 存在 \( n \in \mathbb{N} \) 使得 \( u_i^{p^n} \in S \), 故 \( F^{p^n}
  \subseteq S \).
  由 \cref{theorem-purely-inseparable-extension-TFAE-conditions}, \( S \) 在 \(
  F^{p^n} \) 上纯不可分, 故在 \( KF^{p^n} \) 上纯不可分.
  \( S \) 当然在 \( KF^{p^n} \) 上可分,
  因此由\cref{proposition-both-separable-purely-inseparable}, \( S = KF^{p^n} \).
  注意到 \( F^{p^t} = (K(u_1, \ldots, u_m))^{p^t} = K^{p^t}(u_1^{p^t}, \ldots,
  u_m^{p^t}) \), 因此
  \[
    KF^{p^t} = K(K^{p^t}(u_1^{p^t}, \ldots, u_m^{p^t})) = K(u_1^{p^t}, \ldots,
    u^{p^t}_m).
  \]
  假设 \( F = KF^p \) 即 \( K(u_1, \ldots, u_m) = K(u_1^p, \ldots, u_m^p) \),
  由此可以得到 \( K(u_1, \ldots, u_m) = K(u_1^{p^n}, \ldots, u^{p^n}_m) \),
  换句话说 \( F = S \).
\end{proof}

\subsection{可分次数}

假设 \( F/K \) 是一个代数扩张, \( S \) 由 \( F \) 中所有在 \( K \)
可分的元素组成.
维数 \( [S:K] \) 称为 \( F \) 在 \( K \) 上的\emph{可分次数}, 记作 \( [F:K]_s
\).
维数 \( [F:S] \) 称为 \( F \) 在 \( K \) 上的\emph{不可分次数}, 记作 \( [F:S]_i
\).

\begin{lemma}
  \label{lemma-composite-monomoprhism-to-normal-extension}
  假设 \( F/E, E/K \) 为域扩张, \( N \) 为 \( K \) 的一个包含 \( F \)
  的正规扩张.
  记 \( r \) 是不同 \( E \)-单同态 \( F \to N \) 的基数, \( t \) 是不同 \( K
  \)-单同态 \( E \to N \) 的基数, 那么 \( rt \) 是不同 \( K \)-单同态 \( F \to N
  \) 的基数.
\end{lemma}
\begin{proof}
  假设 \( r, t < \infty \), \( \tau_1, \ldots, \tau_r \) 为不同的 \( E \)-单同态
  \( F \to N \), \( \sigma_1, \ldots, \sigma_t \) 为不同 \( E \)-单同态 \( E \to
  N\).
  由\cref{proposition-isomorphism-extend-to-splitting-fields}以及\cref{theorem-normal-extension-TFAE-conditions},
  \( \sigma_i \) 能扩张为 \( N \) 的 \( K \)-自同构, 记其一个这样的扩张为 \(
  \widetilde{\sigma_i} \).
  于是复合 \( \widetilde{\sigma_i}\tau_j \) 是一个 \( K \)-单同态 \( F \to N \).
  如果 \( \widetilde{\sigma_i} \tau_j = \widetilde{\sigma_a} \tau_b \) 那么 \(
  \widetilde{\sigma_a}^{-1} \widetilde{\sigma_i} \tau_j = \tau_b \), 也就是 \(
  \left. \widetilde{\sigma_a}^{-1} \widetilde{\sigma_i} \right\vert_{E} =
    \operatorname{id}_E \), 因此 \( \sigma_i = \sigma_a \).
  从而 \( \tau_j = \tau_b \). 所以 \( \widetilde{\sigma_i}\tau_j \) 两两不同.

  假设 \( \sigma: F \to N \) 是某个 \( K \)-单同态, 存在某个 \( \sigma_i \) 使得
  \( \sigma_i = \left. \sigma \right\vert_{E} \).
  于是 \( \left. \widetilde{\sigma_i}^{-1} \sigma \right\vert_E =
    \operatorname{id}_E \) 存在 \( \tau_j = \sigma_i^{-1} \sigma \).
\end{proof}

\begin{proposition}
  \label{proposition-separable-degree}
  假设 \( F/K \) 是有限域扩张, \( N \) 是域 \( K \) 包含 \( F \) 的正规扩张.
  那么不同的 \( K \)-单同态 \( F \to N \) 个数恰为 \( [F:K]_s \).
\end{proposition}
\begin{proof}
  假设 \( S \) 为 \( F \) 在 \( K \) 上所有可分元组成的域.
  % 由\cref{proposition-isomorphism-extend-to-splitting-fields}以及\cref{theorem-normal-extension-TFAE-conditions},
  % 每个 \( K \)-单同态能扩张为 \( N \) 的 \( K \)-自同构, 从而能限制为 \( K
  % \)-单同态 \( F \to N \).
  % 由有限性及\cref{proposition-single-polynomial-extension-number}知道扩张的自同构个数即
  % \( [S:K] = [F:K]_s \).
  假设 \( \sigma, \tau \) 是两个 \( S \)-单同态 \( F \to N \), 那么对任意 \( u
  \in F \), 由\cref{lemma-power-of-element-separable} 存在 \( n \in \mathbb{N}
  \) 使得 \( u^{p^n} \in S \), 于是
  \[
    \sigma(u)^{p^n} = \sigma(u^{p^n}) = \tau(u^{p^n}) = \tau(u)^{p^n}.
  \]
  所以 \( \sigma(u) = \tau(u) \), 换句话说 \( \sigma = \tau \).

  由\cref{lemma-composite-monomoprhism-to-normal-extension}, 只需要考虑 \( F \)
  可分的情况.
  对 \( n = [F:K] \) 进行归纳, 取 \( u \in F \setminus K \), 设 \( [K(u):K] = r
  > 1 \).
  如果 \( r < n \),
  那么对\( F/K(u) \) 和 \( K(u)/K \)
  应用归纳假设及应用\cref{lemma-composite-monomoprhism-to-normal-extension}完成归纳.
  如果 \( r = n \),
  那么\cref{proposition-simple-extension-and-field-homomorphism}完成证明.
\end{proof}

\begin{corollary}
  如果 \( F/E, E/K \) 是域扩张, 那么
  \[
    [F:E]_s[E:K]_s = [F:K]_s \text{ 且 } [F:E]_i [E:K]_i = [F:K]_i.
  \]
\end{corollary}

\begin{corollary}
  \label{corollary-minimal-polynomial-of-algebraic-element}
  假设 \( f \in K[X] \) 不可约多项式, \( F \) 为 \( f \) 在 \( K \)
  上的分裂域, \( u_1 \) 为 \( f \) 在 \( F \) 的一个根, 那么
  \begin{enumerate}
    \item 每个 \( f \) 的根都有重数 \( [K(u_1): K]_i \), 因此在 \( F[X] \) 中
      \[
        f = ((X - u_1)\cdots (X - u_n))^{[K(u_1):K]_i},
      \]
      其中 \( u_1, \ldots, u_n \) 为 \( f \) 的不同根且 \( n \in [K(u_1):K]_s
      \).
    \item \( u_i^{[K(u_1):K]_i} \) 在 \( K \) 上可分.
  \end{enumerate}
\end{corollary}
\begin{proof}
  \( (1) \)
  存在 \( K \)-同构 \( \sigma: K[u_1] \simeq K[u_i],\quad u_1 \mapsto u_i \).
  由\cref{proposition-isomorphism-extend-to-splitting-fields}, \( \sigma \)
  可以扩张为同构 \( \widetilde{\sigma}: F \simeq F \).
  \( F[X] \) 是一个唯一因子分解整环, 而 \( \widetilde{\sigma}f = f \), 因此 \(
  u_i \) 的重数相等.
  由\cref{proposition-single-polynomial-extension-number}, 这样的扩张个数为 \(
  n \), 而由\cref{proposition-separable-degree} \( n = [F:K]_s \).
\end{proof}

\begin{corollary}
  假设 \( F/K \) 是一个代数扩张, \( S \) 为 \( F \) 在 \( K \)
  中所有可分元构成的子域, \( P \) 为 \( F \) 在 \( K \)
  中所有纯不可分元构成的子域.
  如果 \( F \) 在 \( K \) 上正规, 那么 \( S \) 在 \( K \) 上 Galois, \( F \) 在
  \( P \) 上 Galois, 且
  \[
    \operatorname{Gal}(S/K) \simeq \operatorname{Gal}(F/P) =
    \operatorname{Aut}(F/K).
  \]
\end{corollary}
\begin{proof}
  \( S \) 在 \( K \) 上 Galois.
  \( u \in S \) 在\( K[X] \) 上极小多项式 \( f \) 可分, 故 \( f \)
  的所有根可分且都落在 \( F \) 中, 自然落在 \( S \) 中. 因此 \( S \) 在 \( K \)
  上正规.
  而可分性由\cref{theorem-separable-and-purely-inseparable-subextension}保证

  \( F \) 在 \( P \) 上 Galois.
  假设 \( u \in F \) 在 \( K \) 的极小多项式为 \( f \),
  那么由\cref{corollary-minimal-polynomial-of-algebraic-element}知道 \( f \)
  在 \( F[X] \) 中形如
  \[
    f = ((X - u_1)\cdots (X - u_n))^{[K(u_1):K]_i},
  \]
  其中 \( u_i \) 两两不同.
  于是 \( (X - u_1) \cdots (X - u_n) \) 的系数的 \( [K(u_1): K]_i \) 次幂落于 \(
  K \) 中,
  结合\cref{theorem-purely-inseparable-extension-TFAE-conditions}以及\cref{corollary-finite-purely-inseparable-degree}知道这些系数落在
  \( P \) 中, 也就是说 \( u \) 在 \( P \) 的不可约多项式可分.
  而正规性是已知的.

  \( \operatorname{Gal}(F/P) \subseteq \operatorname{Aut}(F/K) \) 是明显的,
  反过来, 假设 \( \sigma \in \operatorname{Aut}(F/K), u \in P \)
  那么又由\cref{theorem-purely-inseparable-extension-TFAE-conditions}, 存在 \( n
  \in \mathbb{N} \) 使得 \( u^{p^n} \in K \) 于是
  \[
    \sigma(u)^{p^n} = \sigma(u^{p^n}) = u^{p^n},
  \]
  故 \( \sigma(u) = u \).
  同构由限制
  \[
    \operatorname{Gal}(F/P) \to \operatorname{Gal}(S/P \cap S),\quad \sigma
    \mapsto \left. \sigma \right\vert_{S}
  \]
  给出.
\end{proof}

\section{不变量}

\subsection{迹与行列式}

回顾在线性代数中, 给定一个 \( n \) 维线性空间的线性变换 \( \alpha \),
记其在一组基下的矩阵为 \( A = (a_{ij}) \),
\[
  \operatorname{Tr}(A) = \sum_i a_{ii},\quad
  \det (A) = \sum_{\sigma \in S_n}\operatorname{sgn}(\sigma)a_{1\sigma(1)}\cdots
  a_{n \sigma(n)},\quad
  c_A(X) = \det(XI_n - A)
\]
分别为 \( A \) 的迹, 行列式, 特征多项式.
它们与基选取无关, 即可以定义 \( \alpha \) 的迹, 行列式, 特征多项式
\[
  \operatorname{Tr}(\alpha) = \operatorname{Tr}(A),\quad
  \det(\alpha) = A\det(A),\quad
  c_{\alpha}(X) = C_A(X).
\]

假设 \( E/F \) 为 \( n \) 次域扩张, 那么 \( \alpha \in E \) 定义了 \( F
\)-线性映射
\[
  \alpha_L: E \to E,\quad x \mapsto \alpha x.
\]
称 \( \operatorname{Tr}_{E/F}(\alpha) := \operatorname{Tr}(\alpha_L) \) 为 \(
\alpha \) 的\emph{迹}, \( \operatorname{Norm}_{E/F}(\alpha) = \det (\alpha_L) \)
为 \( \alpha \) 的\emph{范数}, \( c_{\alpha, E/F}(X) := c_{\alpha_L}(X) \) 为 \(
\alpha \) 的\emph{特征多项式}.

\begin{proposition}
  假设 \( E/F \) 为有限域扩张, \( \alpha \in E \) , \( f \in F[X] \) 为 \(
  \alpha \) 在 \( F \) 的极小多项式.
  那么
  \[
    c_{\alpha, E/F}(X) = f^{[E:F[\alpha]]}.
  \]
\end{proposition}
\begin{proof}
  首先假设 \( E = F[\alpha] \), 这时由 Cayley-Hamilton 定理可以知道 \(
  c_{\alpha_L}(\alpha_L) = 0 \) 结合 \( E \to \operatorname{Aut}(E), \alpha
  \mapsto \alpha_L \) 是一个单态射知道 \( c_{\alpha_L}(\alpha) = 0 \).
  而 \( c_{\alpha_L} \) 和 \( f \) 次数相等且都首一, 于是 \( c_{\alpha, E/F}(X)
  = c_{\alpha_L}(X) = f(X) \).

  一般地, 假设 \( \beta_1, \ldots, \beta_n \) 为 \( F[\alpha] \) 的一组 \( F
  \)-基, \( \gamma_1, \ldots, \gamma_m \) 为一组 \( E \) 的 \( F[\alpha] \)-基.
  那么 \( \beta_i \gamma_j \) 为一组 \( E \) 的 \( F \)-基.
  于是 \( \alpha_L \) 在这组基下矩阵为主对角元全是 \( n \times n \)
  块的准对角阵, 结合 \( E = F[\alpha] \) 的情况即可.
\end{proof}

\begin{corollary}
  \label{corollary-trace-and-norm-and-roots}
  假设 \( \alpha \) 极小多项式 \( f \) 的所有不同的根为 \( \alpha_1, \ldots,
  \alpha_n \), 且 \( [E:F[\alpha]] = m \), 那么
  \[
    \operatorname{Tr}(\alpha) = m[F[\alpha]:F]_i \sum_{i=1}^n \alpha_i,\quad
    \operatorname{Norm}_{E/F}(\alpha) = \left( \prod_{i = 1}^n \alpha_i
    \right)^{m[F[\alpha]:F]_i}.
  \]
  特别地, 如果 \( \alpha \) 在 \( F \) 上不可分, 那么 \(
  \operatorname{Tr}(\alpha) = 0 \).
\end{corollary}

\begin{corollary}
  \label{corollary-generalized-trace-and-norm-and-roots}
  假设 \( E \) 是 \( F \) 的有限扩张, \( \Omega \) 是 \( F \) 的包含 \( E \)
  的代数闭包, 令 \( \Sigma \) 为所有 \( F \)-同态 \( E \to \Omega \), 那么对任意
  \( \alpha \in E \), 有
  \[
    \operatorname{Tr}(\alpha) = [E:F[\alpha]] \sum_{\sigma \in \Sigma} \sigma
    \alpha,\quad \operatorname{Norm}_{E/F}(\alpha) =  \prod_{\sigma \in \Sigma}
    (\sigma\alpha)^{[E:F[\alpha]]}.
  \]
\end{corollary}

\begin{proposition}
  假设 \( E/M, M/F \) 为有限扩张, 那么
  \[
    \operatorname{Tr}_{M/F} \circ \operatorname{Tr}_{E/M} =
    \operatorname{Tr}_{E/F},\quad \operatorname{Norm}_{M/F} \circ
    \operatorname{Norm}_{E/M} = \operatorname{Norm}_{E/F}
  \]
\end{proposition}
\begin{proof}
  分别考虑极小多项式知道, 如果 \( \alpha \in E \) 在 \( M \) 上不可分, 那么 \(
  \alpha \) 在 \( F \) 上不可分,
  于是这时由\cref{corollary-trace-and-norm-and-roots} \( \operatorname{Tr}_{M/F}
  \circ \operatorname{Tr}_{E/M}(\alpha) = \operatorname{Tr}_{E/F}(\alpha) = 0
  \).
  假设 \( \alpha \in E \) 在 \( M \) 上可分, \( \alpha_1, \ldots, \alpha_n \) 为
  \( \alpha \) 在 \( M \) 上极小多项式的所有不同根,
  那么再由\cref{corollary-trace-and-norm-and-roots}
  \[
    \begin{split}
      \operatorname{Tr}_{M/F} \circ \operatorname{Tr}_{E/M}(\alpha) &=
      [E:M[\alpha]] \operatorname{Tr}_{M/F}(\sum_i \alpha_i)\\ &= [E:M[\alpha]]
      \frac{\operatorname{Tr}_{E/F}(\sum_i \alpha_i)}{[E:M]}\\ &= [E:M[\alpha]]
      \frac{[M[\alpha]:M]\operatorname{Tr}_{E/F}(\alpha)}{[E:M]}\\ &=
      \operatorname{Tr}_{E/F}(\alpha)
    \end{split}.
  \]
  范数等式的证明是类似的.
\end{proof}

\subsection{判别式}

考虑一个域上的首一多项式
\[
  f(X) = X^n + a_1 X^{n - 1} + \cdots + a_n,
\]
设在 \( f \) 的某个分裂域上 \( f(X) = \prod_{i = 1}^n (X - \alpha_i) \).
置
\[
  \Delta (f) = \prod_{1 \leq i \leq j \leq n}(\alpha_i - \alpha_j),\quad
  \operatorname{Disc}(f) = \Delta(f)^2 = \prod_{1 \leq i \leq j \leq n}(\alpha_i
  - \alpha_j)^2.
\]
称 \( \operatorname{Disc}(f) \) 的判别式为 \( \operatorname{Disc}(f) \).

\begin{proposition}
  设 \( f \in F[X] \) 为首一不可约多项式, \( \alpha \) 为 \( f \)
  的某个分裂域的一个根, 那么
  \[
    \operatorname{Disc}f = (-1)^{m(m - 1)/2}
    \operatorname{Norm_{F[\alpha]/F}}f'(\alpha).
  \]
\end{proposition}
\begin{proof}
  \( f \) 不可分的情况是明显的, 下面考虑 \( f \) 可分的情况.
  由\cref{corollary-generalized-trace-and-norm-and-roots}
  \[
    \begin{split}
      \operatorname{Disc}f &= \prod_{i < j}(\alpha_i - \alpha_j)^2\\ &=
      (-1)^{m(m - 1)/2} \cdot \prod_i (\prod_{j \neq i} (\alpha_i - \alpha_j))\\
      &= (-1)^{m(m - 1)/2} \cdot \prod_{i}f'(\alpha_i)\\ &= (-1)^{m(m - 1)/2}
      \operatorname{Norm}_{F[\alpha]/F}(f'(\alpha))
    \end{split}
  \]
\end{proof}

\section{超越扩张}

\subsection{代数相关与代数无关}

假设给定域 \( \Omega \supseteq F \), 称 \( \alpha_1, \cdots, \alpha_n \subseteq
\Omega \) 是\emph{\( F \)-代数无关的}, 如果 \( F \)-态射
\[
  F[X_1, \cdots, X_n] \to \Omega,\quad f \mapsto f(\alpha_1, \cdots, \alpha_n)
\]
的核为 \( 0 \), 否则称 \( \alpha_1, \cdots, \alpha_n \) 是\emph{\( F
\)-代数相关的}.
一般地, 称任意集合 \( A \subseteq \Omega \) 是\emph{\( F \)-代数无关的}, 如果 \(
A \) 的任意有限集是代数无关的; 否则, 称 \( A \) 是\emph{\( F \)-代数相关的}.

如果 \( \alpha_1,\cdots, \alpha_n \) 为 \( F \) 上的代数无关元, 那么映射
\[
  F[X_1, \cdots, X_n] \to F[\alpha_1, \cdots, \alpha_n],\quad f(X_1, \cdots,
  X_n) \mapsto f(\alpha_1, \cdots, \alpha_n)
\]
为同构, 从而诱导同构 \( F(X_1, \cdots, X_n) \simeq F(\alpha_1, \cdots, \alpha_n) \).
\( F(\alpha_1,\cdots, \alpha_n) \) 称为 \emph{\( F \)的纯超越扩张}.

\begin{example}
  特别地, 一个元素 \( \alpha \in \Omega \) \( F \)-代数无关当且仅当 \( \alpha \) 在 \(
  F \) 上超越.
\end{example}

\begin{lemma}
  \label{lemma-algebraic-dependent-TFAE-conditions}
  假设 \( \gamma \in \Omega, A \subseteq \Omega \), 那么下面条件等价:
  \begin{enumerate}
    \item \( \gamma \) 在 \( F(A) \) 上代数.
    \item 存在 \( \beta_1, \ldots, \beta_n \in F(A) \) 使得 \( \gamma^n +
      \beta_1 \gamma^{n - 1} + \cdots + \beta_n = 0 \).
    \item 存在不全为零的 \( \beta_0, \beta_1, \ldots, \beta_n \in F(A) \) 使得
      \( \beta_0\gamma^n + \beta_1 \gamma^{n - 1} + \cdots + \beta_n = 0 \).
    \item 存在某个 \( f \in F[X_1, \ldots, X_m, Y] \) 和 \( \alpha_1, \ldots,
      \alpha_m \in A \) 使得 \( f(\alpha_1, \ldots, \alpha_m, Y) \neq 0 \) 但 \(
      f(\alpha_1, \ldots, \alpha_m, \gamma) = 0 \).
  \end{enumerate}
  称满足上面的等价条件的 \( \gamma \) 是\emph{\( A \)-代数相关的}.
  如果集合 \( B \subseteq \Omega \) 中元素都是 \( A \)-代数相关的, 那么称 \( B
  \) 是\emph{\( A \)-代数相关的}.
\end{lemma}
\begin{proof}
  考虑 \( (1) \implies (2) \implies (3) \implies (1), (4) \iff (3) \) 即可.
\end{proof}

\subsection{超越基}

\begin{lemma}[交换性质]
  \label{lemma-exchange-algebraically-denpent}
  假设 \( \left\lbrace \alpha_1, \cdots, \alpha_m \right\rbrace \subseteq \Omega
  \).
  如果 \( \beta \) 在 \( \left\lbrace \alpha_1, \ldots, \alpha_m
  \right\rbrace \) 上代数相关, 但在 \( \left\lbrace \alpha_1, \ldots, \alpha_{m
  - 1} \right\rbrace \) 上代数无关, 那么 \( \alpha_m \) 在 \( \left\lbrace
  \alpha_1, \ldots, \alpha_{m - 1}, \beta \right\rbrace \) 上代数相关.
\end{lemma}
\begin{proof}
  由\cref{lemma-algebraic-dependent-TFAE-conditions}, 存在 \( f \in F[X_1,
  \ldots, X_m, Y] \) 使得
  \[
    f(\alpha_1, \ldots, \alpha_m, Y) \neq 0,\quad f(\alpha_1, \ldots, \alpha_m,
    \beta)= 0.
  \]
  将 \( f \) 写作 \( X_m \) 的多项式
  \[
    f(X_1, \ldots, X_m, Y) = \sum_i a_i(X_1, \ldots, X_{m - 1}, Y) X^{n - i}_m,
  \]
  由于 \( f(\alpha_1, \ldots, \alpha_m, Y) \neq 0 \), 存在某个 \( a_i(\alpha_1,
  \ldots, \alpha_{m - 1}, Y) \neq 0 \).
  因为 \( \beta \) 不在 \( \left\lbrace \alpha_1, \ldots, \alpha_{m - 1}
  \right\rbrace \) 上代数相关,
  由\cref{lemma-algebraic-dependent-TFAE-conditions}, \( a_i(\alpha_1, \ldots,
  \alpha_{m - 1}, \beta) \neq 0 \).
  换句话说, \( f(\alpha_1, \ldots, \alpha_{m - 1}, X_m, \beta) \neq 0 \).
  但 \( f(\alpha_1, \ldots, \alpha_{m}, \beta) = 0 \),
  再由\cref{lemma-algebraic-dependent-TFAE-conditions}即得.
\end{proof}
\begin{lemma}[代数相关的传递性]
  \label{lemma-transitivity-of-algebraic-dependence}
  如果 \( C \) 是一个 \( B \) 上的代数相关集, \( B \) 为 \( A \) 上的代数相关集,
  那么 \( C \) 在 \( A \) 为代数相关集.
\end{lemma}

\begin{theorem}[基本结果]
  \label{theorem-algebraic-dependent-fundamental-result}
  令 \( A = \left\lbrace \alpha_1, \cdots, \alpha_m \right\rbrace \) 和 \( B =
  \left\lbrace \beta_1, \cdots, \beta_n \right\rbrace \) 为 \( \Omega \)
  的两个子集.
  假设
  \begin{enumerate}
    \item \( A \) 在 \( F \) 上代数无关,
    \item \( A \) 在 \( B \) 上代数相关,
  \end{enumerate}
  那么, \( m \leq n \).
\end{theorem}
\begin{proof}
  考虑 \( A \setminus (A \cap B) \) 的个数 \( k \).
  如果 \( k = 0 \), 那么结论已经得到.

  如果 \( k > 0 \), 记 \( B = \left\lbrace \alpha_1, \cdots, \alpha_{m - k},
  \beta_{m - k + 1}, \cdots, \beta_n \right\rbrace \).
  设 \( m - k + 1 \leq j \leq n \) 是满足 \( \alpha_{m - k + 1} \) 在 \( \left\lbrace
  \alpha_1, \cdots, \alpha_{m - k}, \beta_{m - k + 1}, \cdots, \beta_j \right\rbrace \)
  代数相关的最小元.
  由\cref{lemma-exchange-algebraically-denpent}, \( \beta_j \) 在\( B_1 = B \cup
  \left\lbrace \alpha_{m - k + 1} \right\rbrace \setminus \left\lbrace \beta_{j}
  \right\rbrace \) 上代数相关.
  由 \( B_1 \) 构造, \( B \) 在 \( B_1 \) 上代数相关,
  又由\cref{lemma-transitivity-of-algebraic-dependence}, \( A \) 在 \( B_1 \)
  上代数相关.
  而 \( A \setminus (B_1 \cap A) = k - 1 \), 应用归纳假设即得.
\end{proof}

\paragraph{超越基}

\( \Omega \) 在 \( F \) 上的\emph{超越基}为 \( F \) 上的代数无关集 \( A \)
使得 \( \Omega \) 在 \( F(A) \) 上代数.
% 由下面的定理,
% 我们知道超越基有限时(实际上, 无穷亦如此)超越基的基数由 \( \Omega \) 和 \( F \)
% 自身决定, 称其为 \emph{超越次数}.
% 下面我们要证明利用前面所得到的工具证明超越基是存在, 且其基数可定义的(i.e.
% 超越基的基数与超越基本身的选取无关).

\begin{lemma}
  \label{lemma-making-algebraic-minimal-set-as-transcendence-basis}
  如果 \( A \) 为 \( \Omega \) 的满足 \( \Omega \) 在 \( F(A) \) 代数的最小子集,
  那么 \( A \) 为 \( \Omega \) 在 \( F \) 上的超越基.
\end{lemma}
\begin{proof}
  如果 \( A \) 代数相关, 那么可以找到 \( \beta \in A \) 使得 \( \beta \) 在 \( A
  \setminus \left\lbrace  \beta \right\rbrace \) 中代数相关, 因此 \( A \) 在 \(
  A \setminus \left\lbrace \beta \right\rbrace\) 中代数相关.
  由\cref{lemma-transitivity-of-algebraic-dependence}, \( \Omega \) 在 \( F(A
  \backslash \left\lbrace \beta \right\rbrace) \) 上代数, 这与 \( A \)
  极小性矛盾.
\end{proof}

\begin{theorem}
  \label{theorem-finite-transcendence-basis}
  如果存在有限子集 \( A \subseteq \Omega \) 使得 \( \Omega \) 在 \( F(A) \)
  上代数, 那么 \( \Omega \) 有一个\( F \)上的有限超越基.
  另外, 每个超越基是有限的, 且它们有相同的元素个数.
\end{theorem}
\begin{proof}
  有限超越基的存在性由\cref{lemma-making-algebraic-minimal-set-as-transcendence-basis}保证,
  个数相等由\cref{theorem-algebraic-dependent-fundamental-result}保证.
\end{proof}

\begin{lemma}
  \label{lemma-maximal-algebraic-independent-set-as-transcendence-basis}
  设 \( A \) 是代数无关的, 但 \( A \cup \left\lbrace \beta \right\rbrace \)
  是代数相关的. 那么 \( \beta \) 在 \( F(A) \) 上代数.
\end{lemma}
\begin{proof}
  由假设存在\( \alpha_1, \ldots, \alpha_n \in A \) 和非零 \( f \in F[X_1,
  \ldots, X_n, Y] \) 使得
  \[
    f(\alpha_1, \ldots, \alpha_n, \beta) = 0.
  \]
  将 \( f \) 写为 \( Y \) 的多项式
  \[
    f = g_0 Y^m + g_1 Y^{m - 1} + \cdots + g_m,\quad g_i \in F[X_1, \ldots,
    X_n],\quad g_0 \neq 0,\quad m \geq 1.
  \]
  由于 \( g_0 \neq 0 \) 且 \( \alpha_i \) 代数独立, 所以 \( g_0(\alpha_1,
  \ldots, \alpha_n) \neq 0 \).
  于是 \( f(\alpha_1, \ldots, \alpha_n, Y) \neq 0 \),
  由\cref{lemma-algebraic-dependent-TFAE-conditions}, \( \beta \) 在 \( A \)
  上代数.
\end{proof}

\begin{proposition}
  \label{proposition-maximal-algebraic-independent-set-as-transcendence-basis}
  \( \Omega \) 的任意最大代数无关子集 \( A \) 都是 \( \Omega \) 在 \( F \)
  上的超越基.
\end{proposition}
\begin{proof}
  对每个 \( \beta \in \Omega \setminus A \), 由极大性 \( A \cup \left\lbrace
  \beta \right\rbrace \) 代数独立,
  因此由\cref{lemma-maximal-algebraic-independent-set-as-transcendence-basis} \(
  \beta\) 在 \( A \) 上代数.
\end{proof}

\begin{theorem}
  每个代数无关子集 \( S \subseteq \Omega \) 都包含于 \( \Omega \) 的一个超越基.
  特别地, 超越基存在.
\end{theorem}
\begin{proof}
  假设 \( \mathcal{S} \) 为包含 \( S \) 的代数无关集, 由包含赋予偏序关系.
  设 \( S_1 \subseteq S_2 \subseteq \cdots \) 为 \( \mathcal{S} \) 的一个上链,
  下面证明 \( \bigcup S_i \) 为此链在 \( \mathcal{S} \) 中的一个上界.
  假设不然, 存在一个有限集 \( B' \subseteq \bigcup S_i \) 代数相关, 而 \( B' \)
  包含于某个 \( S_i \) 这是不可能的.
  于是由 Zorn 引理可以得到 \( \mathcal{S} \) 的一个极大元 \( S' \).
  于是由\cref{proposition-maximal-algebraic-independent-set-as-transcendence-basis},
  \( S' \) 是一个超越基.
\end{proof}

\paragraph{超越次数}


\begin{proposition}
  \( \Omega/F \) 的超越基基数相等, 称此基数为 \( \Omega/F \) 的\emph{超越次数},
  记作 \( \operatorname{trdeg}_F(\Omega) \).
\end{proposition}
\begin{proof}
  有限情况由\cref{theorem-finite-transcendence-basis}保证.
  假设 \( B, B' \) 是两个基数都是无穷的超越基.
  由选择公理, 可以假设 \( \left\lvert B' \right\rvert < \left\lvert B
  \right\rvert \).
  对任意 \( \alpha \in B' \), 存在有限集 \( B_{\alpha} \subseteq B \) 使得 \(
  \alpha \) 在 \( B_{\alpha} \) 上代数.
  设 \( B^* = \bigcup_{\alpha \in B'} B_{\alpha} \), 于是 \( B^* \subseteq B \).
  下面证明 \( B^* = B \).
  假设不然, 那么存在 \( \beta \in B \setminus B^* \).
  \( \beta \) 在 \( B' \) 上代数, \( B' \) 在 \( B^* \) 上代数,
  由\cref{lemma-transitivity-of-algebraic-dependence}, \( \beta \) 在 \( B^* \)
  上代数.
  也就是 \( \beta \) 在 \( B^* \) 上代数代数, 这与超越基定义矛盾.
  于是 \( \left\lvert B \right\rvert = \left\lvert B^* \right\rvert \leq \aleph_0
  \left\lvert B' \right\rvert = \left\lvert B' \right\rvert \).
\end{proof}

\begin{proposition}
  假设 \( F/K, E/F \) 为域扩张, 那么
  \[
    \operatorname{trdeg}_K E = \operatorname{trdeg}_K F +
    \operatorname{trdeg}_F E.
  \]
\end{proposition}
\begin{proof}
  取 \( B, C \) 分别为 \( F/K, E/F \) 的超越基, 可以知道 \( B \cup C \) 是 \(
  E/K \) 的超越基.
\end{proof}

\begin{proposition}
  任意两个有相同超越次数的 \( F \) 上的代数闭域是 \( F \)-同构的.
\end{proposition}

\paragraph{L\"{u}roth 定理}

\begin{theorem}[L\"{u}roth]
  令 \( L = F(x) \) 其中 \( x \) 在 \( F \) 上超越.
  每个 \( L \) 的真包含 \( F \) 的子域 \( E \) 具有形式 \( E = F(u) \), 其中 \(
  u \in L \) 在 \( F \) 上超越.
\end{theorem}
